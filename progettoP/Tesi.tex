\documentclass[11pt]{article}
\usepackage{fullpage}
\usepackage[utf8]{inputenc}
\usepackage[T1]{fontenc}
\usepackage[italian]{babel}
\addto\captionsitalian{%
    \renewcommand{\figurename}{Fig.}%
    \renewcommand{\contentsname}{Tab.}%
}
\usepackage{empheq}
\usepackage{color}
\usepackage{listings}
\usepackage{amsmath}
\usepackage{hyperref}
\usepackage{xcolor}
\hypersetup{
    colorlinks,
    linkcolor={red!50!black},
    citecolor={blue!50!black},
    urlcolor={blue!80!black}
}
\usepackage[all]{hypcap}
\usepackage{longtable}
\usepackage{array}
%\usepackage[scaled]{beramono}
\usepackage[T1]{fontenc}
\usepackage{tabularx}
\usepackage{caption}
\usepackage{amssymb}
\usepackage{enumitem}
\usepackage{booktabs}
\usepackage{multirow}
\usepackage{tipa}
\usepackage{pbox}
\usepackage{placeins}
\usepackage{adjustbox}

%\setmonofont{Consolas} %to be used with XeLaTeX or LuaLaTeX
\definecolor{bluekeywords}{rgb}{0,0,1}
\definecolor{blue(munsell)}{rgb}{0.0, 0.5, 0.69}
\definecolor{greencomments}{rgb}{0,0.5,0}
\definecolor{redstrings}{rgb}{0.64,0.08,0.08}
\definecolor{xmlcomments}{rgb}{0.5,0.5,0.5}
\definecolor{types}{rgb}{0.17,0.57,0.68}
\setlength\parindent{0pt}

\lstset{language=[Sharp]C,
captionpos=b,
%numbers=left, %Nummerierung
%numberstyle=\tiny, % kleine Zeilennummern
frame=lines, % Oberhalb und unterhalb des Listings ist eine Linie
showspaces=false,
showtabs=false,
breaklines=true,
showstringspaces=false,
breakatwhitespace=true,
escapeinside={(*@}{@*)},
commentstyle=\color{greencomments},
morekeywords={partial, var, get, set,string,false,true},
keywordstyle=\color{bluekeywords},
stringstyle=\color{redstrings},
basicstyle=\ttfamily\small,
extendedchars=true,
literate={à}{{\`a}}1 {è}{{\`e}}1 {ò}{{\`o}}1 {ù}{{\`u}}1 {é}{{\'e}}1 {ì}{{\`e}}1,
}
\lstset%
{%
    emph=[1]%
    {%
        Process,
        CPLEX,
        Program,
        IntParam,
        NumVarType,
        INumVar,
        SystemInformation,
        Param,
        Stopwatch,
        Instance,
        ProcessStartInfo,
        Point,
        ILinearNumExpr,
        List,
        PathGenetic,
        Random,
        ILinearNumExpr,
        TSPLazyConsCallback,
        StreamWriter,
        IRange,
        itemList,
        DoubleParam,
        Convert,
        StringSplitOptions
    },
    emphstyle=[1]{\color{blue(munsell)}},
}

\begin{document}

\section*{SEZIONE UNO}
\label{sec:SezioneUnoS}

Questa tesi tratta lo studio di molteplici metodi risolutivi per il "Problema del commesso viaggiatore", indicato in lingua inglese con la sigla \textbf{TSP}\footnote{Travelling Salesman Problem.}. Il testo segue di pari passo lo sviluppo di un software risolutivo da noi realizzato e pertanto, oltre a nozioni puramente teoriche, vengono anche presentati in maniera dettagliata sia l'ambiente di sviluppo utilizzato che il codice di programmazione.

A seguire è possibile trovare i risultati in entrambe le forme tabellari che grafiche di svariati test eseguiti con lo scopo di verificare la bontà di quanto prodotto.

Le caratteristiche del codice di programmazione, che risultano di minore interesse ai fini della tesi, sono riportate nella sezione finale \hyperref[sec:AppendiceS]{Appendice} del testo. D'altro canto ogni altra nozione algoritmica viene commentata nel dettaglio solamente durante il suo primo utilizzo così da non appesantire eccessivamente la lettura.

\subsection*{INTRODUZIONE}
\label{sec:IntroduzionS}

Utilizziamo questa sezione introduttiva per parlare della storia, le applicazioni e correnti sfide riguardanti il \textit{TSP}, da sempre uno dei problemi più discussi e studiati nella disciplina della \textbf{Ricerca Operativa} soprattutto grazie alla sua rilevanza in numerose applicazioni pratiche.

Il particolare nome \textit{TSP} deriva dalla sua più tipica rappresentazione: data una rete di città e strade che le collegano, ci si chiede quale sia il percorso \textbf{migliore} che un commesso viaggiatore dovrebbe seguire nel caso in cui voglia visitare ogni città una ed una sola volta ritornando infine al punto di partenza. Per quanto detto, risulta naturale modellare questo problema attraverso un grafo \textbf{pesato} i cui nodi rappresentano le città mentre ed i lati corrispondono alle vie di accesso che le collegano. Per quanto riguarda il \textbf{peso} da attribuire a queste ultime è possibile utilizzare qualsivoglia tecnica: una semplice lunghezza fisica, tempi di percorrenza o la presenza di pedaggi sono solo alcuni esempi.

Data questa situazione possiamo notare come sia molto facile mutarla in infinite altre, sia di uso pratico che teorico. Fondamentali sono le applicazioni nell'ambito della logistica distributiva più comunemente note come problemi di routing. Non ci riferisce solamente al classico smistamento di pacchetti in una rete internet ma anche l’organizzazione di sistemi di distribuzione per beni e servizi: movimentazione di pezzi o semilavorati tra reparti di produzione, raccolta e distribuzione di materiali, smistamento di merci da centri di produzione a quelli di distribuzione e molto altro ancora.

Sebbene le applicazioni nel contesto dei trasporti rimangano comunque le più naturali, la semplicità del \textit{modello} ha portato negli anni allo sviluppo di innumerevoli applicazioni nelle più svariate aree di interesse. Un esempio può essere la programmazione di una macchina per eseguire fori in un circuito\footnote{In questo esempio i fori da forare sono le città, le strade sono definite dai possibili movimenti della macchina ed il loro peso è il tempo necessario per spostare la testa del trapano.}

Per inquadrare l'evoluzione temporale del \textit{TSP} è necessario risalire fino all'Ottocento e quindi ai matematici Sir William Rowan Hamilton e Thomas Penyngton. Il primo, ad esempio, durante l'anno 1857 nella città di Dublino e più specificatamente nel corso di una riunione della British Association for the Advancement of Science, descrisse il così detto  "Icosian game": questo consisteva nel trovare un percorso che toccasse tutti i vertici di un icosaedro, passando lungo gli spigoli, ma senza mai percorrere due volte lo stesso spigolo. L'icosaedro era originariamente composto da 12 vertici, 30 spigoli e 20 facce identiche a forma di triangolo equilatero.

Questo problema è in realtà un particolarissimo TSP nel quale gli archi che collegano vertici adiacenti, e quindi corrispondono a spigoli dell'icosaedro, sono consentiti e gli altri no (si può pensare che richiedano moltissimo tempo e quindi vadano sicuramente scartati). Per veder nascere la formulazione più generale del TSP, proposta all'inizio di questa sezione, è necessario attendere il secolo successivo, negli anni Venti e Trenta per la precisione, con il matematico ed economista Karl Menger.

Ciò non significa che furuno contemporaneamente proposti anche metodi risolutivi \textit{intelligenti}: per molto tempo non si ebbe altra idea che quella di generare e valutare tutte le soluzioni possibili, mantenendo il problema praticamente insolubile. Il numero totale dei differenti percorsi possibili attraverso $n$ città è infatti esorbitante: dato un qualsiasi punto di partenza, ci sono a disposizione $(n - 1)$ scelte per il successivo, $(n - 2)$ per quello dopo ancora e così via, per un totale di $(n - 1)!$ percorsi anche se per simmetria quelli tra loro distinti risultano \textit{solamente} la metà $\frac{(n-1)!}{2}$.

Solo nel 1954, George Dantzig, Ray Fulkerson e Selmer Johnson proposero un metodo più raffinato per risolvere il TSP anche se allora limitato ad un campione di $n = 49$ città: queste rappresentavano le capitali degli Stati Uniti e il costo del percorso era calcolato in base alle distanze stradali.

Nel 1962, \textit{Procter and Gamble} bandì un concorso per 33 città mentre nel 1977 si miriva a collegare nel modo più efficiente possibile le 120 principali città della Germania Federale. La vittoria andò a Martin Gr\"otschel oggi Presidente del Konrad-Zuse-Zentrum f\"ur Informarionstechnik Berlin(ZIB) e docente presso la Technische Universit\"at Berlin(TUB).

Nel 1987 Padberg e Rinaldi riuscirono a completare il giro degli Stati Uniti attraverso 532 città ed in contemporanea Groetschel e Holland trovarono il TSP ottimale riguardante un giro del mondo passante per 666 mete di interesse.

Nel 2001, Applegate, Bixby, Chvátal, and Cook trovarono la soluzione esatta ad un problema di 15.112 città tedesche, usando il metodo \textbf{cutting plane}, originariamente proposto nel 1954 da George Dantzig, Delbert Ray Fulkerson e Selmer Johnson. Il calcolo fu eseguito da una rete di 110 processori della Rice University e della Princeton University. Il tempo di elaborazione totale fu equivalente a 22,6 anni su un singolo processore Alpha a 500 MHz.

Sempre Applegate, Bixby, Chv\a`tal, Cook, e Helsgaun trovarono nel Maggio del 2004 il percorso ottimale di 24,978 città della Svezia.

Nel marzo 2005, il TSP riguardante la visita di tutti i 33.810 punti in una scheda di circuito fu risolto usando \textbf{CONCORDE}: fu trovato un percorso di 66.048.945 unità, e provato che non poteva esisterne uno migliore. L'esecuzione richiese approssimativamente 15,7 anni CPU.

Questa carrellata storica mostra come il semplice passaggio da un decennio al successivo portava il superamento di ostacoli via via sempre maggiori a testimonianza dell'enorme studio svoltosi a livello globale, agevolato dallo sviluppo di hardware sempre migliore, portando infine alla comparsa dei primi \textbf{software solver}: tra i più noti troviamo \textbf{IBM ILOG CPLEX Optimization Studio} o più semplicemente \textbf{CPLEX} ed il già citato \textbf{Concorde}, entrambi utilizzati all'interno di questa tesi.

\textit{CPLEX} è stato sviluppato da Robert E. Bixby /footnote{Docente presso la Rice University, Texas} e commercializzato a partire dal 1988 da \textbf{CPLEX Optimization Inc.}. Attualmente la licenza è proprietà di \textbf{IBM} la quale concede, per fini accademici e riveste un ruolo centrale in gran parte degli argomenti trattati in questa tesi.

D'altro canto Concorde è stato utilizzato solo marginalmente e vi è dedicata una sezione apposita \hyperref[sec:ConcordeS]{Concorde}, per il momento basti pensare che ne è stata realizzata anche una versione per smartphone /footnote{Al giorno d' oggi solo per ambiente iOS, scaricabile gratuitamente dell' App Store} capace anch'essa di risolvere i più comuni problemi \textit{TSP} in frazioni di secondo.

Tali risultati sembrerebbero andare teoricamente contro alla natura stessa dei \textit{TSP} in quanto è dimostrabile la loro appartenenza alla categoria di problemi \textbf{NP-hard}: ciò significa che, al momento, non è noto in letteratura un algoritmo che risolva una sua qualunque istanza in tempo \textbf{polinomiale}.

Con la nascita di sempre nuove applicazioni aventi alla loro base problemi \textit{NP-hard}, è sorta la necessità di poter usufruire di algoritmi che valorizzassero maggiormente la velocità di risoluzione anche a discapito del ritrovamento della soluzione ottima, accontentandosi di assicurarne una solamente il più \textit{vicino} ad essa possibile. Tali algoritmi sono detti \textbf{euristici} e naturalmente ne sono presenti diversi anche per il \textit{TSP}: è proprio grazie ad un loro utilizzo combinato a tecniche esatte\footnote{Oltre ad hardware molto performante rispetto al passato.} che i \textit{solver} moderni possono raggiungere risultati apparentemente impensabili. Nel corso di questa tesi viene dato largo spazio ad entrambe le categorie algoritmiche \textit{esatte} ed \textit{euristiche}.


\subsection*{AMBIENTE DI SVILUPPO: CPLEX, Visual Studio e C\#}
\label{sec:AmbienteSviluppoS}

Il progetto è stato sviluppato in ambiente Windows, in particolare il sistema operativo scelto è Windows 10.

\textbf{CPLEX} è una componente centrale per progetto e quindi, prima di procedere, mostriamo i passi principali da eseguire per assicursi di potervi interagire dall'IDE\footnote{\textbf{Integrated Development Environment}: in altre parole è l'acronimo per \textit{ambiente di sviluppo (integrato)}.} scelto. Tra i diversi linguaggi di programmazione compatibili con \textit{CPLEX} si è scelto di utilizzare il \textbf{C\#}, successore del \textbf{C++} ed anch'esso orientato agli oggetti.

L'\textit{IDE} più comune per chi desidera utilizzare tale linguaggio è senza dubbio \textbf{Visual Studio}, sviluppato dalla stessa \textbf{Microsoft} e distribuito gratuitamente. La versione utilizzata in questo progetto, e dunque quella a cui fa riferimento questa guida, è \textbf{Visual Studio Community 2017}. Una volta aperto l'installer reperibile al seguente \href{https://www.visualstudio.com/it/thank-you-downloading-visual-studio/?sku=Community&rel=15#}{indirizzo} è sufficiente selezionare i pacchetti \textbf{Sviluppo per desktop .NET} e \textbf{Sviluppo di applicazioni desktop con C++} (vedremo in seguito perché sono necessari pacchetti C++).

Terminato questo processo, assicuriamoci che la versione di \textit{CPLEX} presente nella macchina sia \textbf{almeno} la \textbf{12.7.0}\footnote{Versioni precedenti non sono compatibili con \textit{Visual Studio Community 2017}.} utilizzata anche per questo progetto.

Ogni progetto \textit{Visual Studio} che desidera utilizzare liberie esterne deve necessariamente settare le sue proprietà indicando le directory in cui è possibile reperirle. Creiamo un progetto \textbf{C\#} selezionando dal menu a tendina \textbf{Visual C\#} e quindi \textbf{App console (.NET Framework)}, forniamo il nome e percorso che desideriamo come mostrato in figura \hyperref[ProgettoCSharpImg]{(1)}.

\begin{figure}[htbp]
    \centering
    \scalebox{0.65}{\includegraphics{Immagini/"Progetto C Sharp".png}}
    \caption{Creazione progetto C\#}
    \label{ProgettoCSharpImg}
\end{figure}

La connessione alle liberie di \textit{CPLEX} viene stabilita seguendo i passaggi seguenti:

\begin{itemize}
    \item Selezionare la voce \textbf{Progetto} dai menù e quindi \textbf{Aggiungi riferimento...};
    \item Premere il pulsante \textbf{Sfoglia} e dopo essersi recati nella propria cartella di installazione di \textit{CPLEX}, in genere "C:\textbackslash Program Files\textbackslash IBM\textbackslash ILOG\textbackslash CPLEX\_StudioXXXX\textbackslash cplex" accedere alla sotto directory "\textbackslash bin\textbackslash x64\_win64" e selezionare i file \textbf{ILOG.CPLEX.dll} e \textbf{ILOG.Concert.dll};
    \item Selezionare la voce \textbf{Compilazione} dai menù e quindi \textbf{Gestione configurazione..}, nella nuova finestra inserire all'interno di \textbf{Piattaforma} un nuovo campo e selezionare \textbf{x64}\footnote{Il progetto utilizza 64 bit.}. Per quanto riguarda il campo \textbf{Configurazione} è indifferente selezionare \textbf{Debug} oppure \textbf{Release}: come è facilmente intuibile nella prima modalità offre tool di \textit{debug} aggiuntivi a discapito di una minima perdita prestazionale. Nel nostro caso la modalità \textbf{Release} è quella utilizzata per ottenere i risultati mostrati nella sezione \hyperref[sec:TestRisultatiS]{Test e Risultati};
\end{itemize}

La completa interazione con \textit{CPLEX} deve essere eseguita attraverso gli oggetti e le funzionalità offerta dai due pacchetti \textbf{ILOG.CPLEX} che \textbf{ILOG.Concert}, ora accessibili attraverso la direttiva \textbf{using}.

\subsection*{Creazione ed utilizzo DLL C/C++ }
\label{sec:CreazioneDLL}

In questa sezione viene spiegato come sia possibile utilizzare codice esterno, compilato anche in linguaggi differenti dal \textbf{C\#}, sotto forma di \textbf{DLL}. Nel nostro caso si giungerà ad un punto in cui è necessario servirsi del linguaggio di programmazione \textbf{C} per l'interazione con librerie di \textbf{Concorde} software solver dedicato ai \textit{TSPs}.

La problematica principale con cui ci si scontra in questi casi è l'incompatibilità dei tipi, qui ancora più accentuata in quanto \textit{C}, al contrario di \textit{C\#}, non è un linguaggio orientato agli oggetti e l'interfaccia con \textit{CPLEX} segue un differente approccio.

La soluzione migliore è quella di comunicare alla \textit{DLL} solamente le informazioni fornite in input dall'utente e cioè il nome del file contenente i dati ed il time limit, così che possa gestire autonomamente l'intera procedura. In altre parole il codice \textit{C\#} diventa in questo caso solamente\footnote{In realtà mantiene anche un cronometro per il tempo di risoluzione.} una interfaccia per richiamare la \textit{DLL}\footnote{L'output standard della \textit{DLL} richiamata viene automaticamente settato a quello del progetto chiamante senza la necessità di eseguire alcun settaggio.}.

Entriamo ora nel dettaglio della procedura da seguire:

\begin{itemize}
    \item Dall'interno di \textit{Visual Studio} creare un nuovo progetto selezionando la voce \textbf{Visual C++} e quindi \textbf{Progetto Win32}. Nel caso in cui queste opzioni siano assenti significa che durante l'installazione dell'\textit{IDE} non sono stati selezionati i pacchetti textit{C++}, per maggiori dettagli andare alla sezione \hyperref[sec:AmbienteSviluppoS]{Ambiente di sviluppo};
    \item Nelle schermate successive è necessario selezionare l'opzione \textbf{DLL} come \textit{tipo di applicazione} e \textbf{Progetto vuoto} come \textit{opzione aggiuntiva};
    \item Creatosi il nuovo e progetto vuoto, dal menu \textbf{Esplora soluzioni} nella cartella \textbf{File di origine} premiamo il tasto destro ed aggiungiamo un nuovo elemento. Selezioniamo dal menu \textbf{File di C++ (.cpp)}, assegniamo il nome che preferiamo ed infine il tasto \textbf{Aggiungi}.
    \item Il file appena creato è compilato come codice \textbf{C++}, racchiudendolo però dentro
    \begin{lstlisting}
    extern "C"
    {
        \\Codice..
    }
    \end{lstlisting}
    viene automaticamente interpretato come linguaggio \textbf{C}.
    \item Definiamo il metodo \textit{entry point} per la DLL inserendo il prefisso \textbf{\_\_declspec(dllexport)}. Per semplificae la programmazione, è consigliabile sfruttare il punto di ingresso come una semplice interfaccia per la chiamata ad una seconda funzione che avvia effettivamente la risoluzione della istanza \textit{TSP}:
    
    \begin{lstlisting}
    __declspec(dllexport) int Concorde(char *fileName, int timeLimit)
    {
        return ExeMain(strtok(fileName, "\0"), timeLimit);
    }
    \end{lstlisting}
    
    \item In modo analogo alla creazione del file corrente, selezionando invece l'opzione per aggiungere file esistenti, importiamo tutte le componenti con estensione \textbf{.C} necessarie da Concorde. Una lista dettagliata è fornita nella sezione \hyperref[sec:ConcordeS]{Concorde}.
    \item Settiamo ora le proprietà del progetto in modo tale che sia possibile utilizzare CPLEX, diverse guide sono già disponibili e la procedura non viene qui riportata. In aggiunta è necessario selezionare nel sottomenù \textbf{C/C++ $\rightarrow$ Generale $\rightarrow$ Directory di inclusione aggiuntive} la cartella dove sono presenti i file \textbf{.h} di Concorde.
    \item Completata la stesura del codice, la effettiva creazione della \textit{DLL} è avviabile dal menu \textbf{Compilazione} selezionando la voce \textbf{Compila soluzione} \footnote{Assicurarsi che nella barra degli strumenti sia selezionata la modalità \textbf{release} a \textbf{64 bit}, in realtà è solamente sufficiente che i bit coincidano con quelli adottati nel progetto originale \textbf{C\#}.}. A compilazione terminata, la \textit{DLL} si trova all'interno della sottocartella \textbf{/x64/Release}.
    \item Copiare la \textit{DLL} ottenuta nella directory del \textbf{progetto C\#}, in particolare su \textbf{/bin/x64/Release} e \textbf{/bin/x64/Debug}, per utilizzarla nelle due modalità.
    \item L'importazione della \textit{DLL} all'interno di una classe \textit{C\#} del progetto avviene attraverso:
    
    \begin{lstlisting}
    [DllImport("NOMEDLL.dll")]
    public static extern int ENTRYPOINTDLL(StringBuilder fileName, int timeLimit);
    \end{lstlisting}
    
    Il metodo \textit{entry point} è quindi richiamabile come qualsiasi altra funzione.
\end{itemize}
    
Prima di concludere la sezione, si specifica che la classe \textbf{StringBuilder} è una interfaccia naturale offerta da \textit{C\#} per la conversione automatica del proprio tipo \textbf{String} a \textbf{char*} del linguaggio \textit{C}.

\newpage

\section*{SEZIONE DUE}
\label{sec:SezioneDueS}

Questa breve sezione viene utilizzata per introdurre formalmente i \textit{TSP} ed in particolare il modello matematico generale che li descrive, evidenziando per quale motivo tali problemi appartengono alla categoria \textbf{NP-Hard}.

\subsection*{MODELLO MATEMATICO}
\label{sec:ModelloMatematicoS}

Nella sua formalizzazione più generale, il Problema del Commesso Viaggiatore consiste nell'individuare un circuito \textbf{hamiltoniano} di costo minimo per un dato grafo orientato $G=(V,A)$, dove $V = $\{$ {v_1,\dots,v_n} $\}$ $ è un insieme di $n$ nodi e $A = $\{$ {(i,j): i, j \in V} $\}$ $ è un insieme di $m$ archi\footnote{Chiaramente sia $n$ che $m$ sono interi positivi.}.

Senza perdita di generalità, si suppone che il grafo $G$ sia completo e che il costo associato all'arco $[i,j]$, indicato con $c_{ij}$, sia non negativo. Si osservi che aver imposto $c_{ij} \ge 0$ non è limitativo poichè è sempre possibile sommare a tutti i costi una valore costante, sufficientemente elevato, in modo tale da renderli positivi senza alterarne l'ordinamento delle soluzioni.

A differenza di quanto detto in precedenza, per tutto il proseguimento della tesi supporremmo il grafo $G$ non orientato. Tale scelta deriva dal fatto che $c_{ij}$, nel nostro contesto, rappresenta sempre la distanza, tipicamente euclidea, fra i vertici $i$ e $j$ pertanto:

\begin{eqnarray}
c_{ij} = c_{ji}
\end{eqnarray}

Quando il grafo è non orientato, la famiglia di coppie non ordinate di elementi appartenenti a $V$ viene indicata per convenzione con la lettera $E$.

Definita la nomenclatura del problema, si considera come variabili decisionali proprio i lati del grafo nel seguente modo:

\begin{eqnarray}
&\displaystyle x_{e}=
\begin{cases}
1 & \text{se il lato $e \in E$ viene scelto nella soluzione trovata} \\
0 & \text{altrimenti}
\end{cases}
\end{eqnarray}

Di conseguenza la formulazione del modello matematico diviene:

\begin{eqnarray}
\label{eqModel1}
&\displaystyle \text{min}\displaystyle\underbrace{\sum_{e \in E} c_e x_e}_{\text{costo circuito}} \\[1.5ex]
\label{eqModel2}
& \displaystyle\underbrace{\sum_{e \in \delta(V)} x_e = 2}_{\text{due lati incidenti in }v}\text{,  }\forall v \in V \\[1.5ex]
\label{eqModel3}
&\displaystyle 0\leq x_e \leq 1 \text{ intera , }\forall e \in E
\end{eqnarray}

L'insieme definito da (\ref{eqModel2}) vengono chiamati \textbf{vincoli di grado} e impongono che in ogni vertice incidano esattamente due lati. Questa formulazione è si \textbf{compatta} in quanto il numero di vincoli è \textbf{polinomiale} rispetto ad $n$ ma contemporaneamente \textbf{non completa}: nessuna equazione impedisce la formazione di \textit{subtour} e quindi di ottenere una soluzione dove il grafo non risulti connesso.

\begin{figure}[htbp]
    \centering
    \scalebox{0.5}{\includegraphics[width=\textwidth]{Immagini/"subtour".jpg}}
    \caption{Soluzione con due subtour}
\end{figure}

Una possibile formulazione di vincoli atti ad impedire la formazione di subtour, detta appunto \textbf{subtour elimination}, risulta essere:

\begin{eqnarray}
\label{eqSubtourElimination}
&\displaystyle{\sum_{e \in \delta(S)}x_e \geq 1} \text{ , } \forall S \subsetneq V \text{ : } 1 \in S \text{ , } |S| \geq 2
\end{eqnarray}

Il vincolo (\ref{eqSubtourElimination}) indica che se si considera un \textbf{qualunque} sottoinsieme $S \subsetneq V$, che includa il vertice numerato con il simbolo $1$, allora il taglio di G indotto da S:

\begin{eqnarray}
&\displaystyle\delta(S) = \{ {[i,j]\in E : i \in S, j \notin S} \}
\end{eqnarray}

deve contenere almeno un lato appartenente ad $E$: poichè tutti i subtour violano tale vincolo la soluzione ottima non potrà contenerne al suo interno.

Analizzando il numero dei possibili sottoinsiemi $S$ distinti si nota che tale quantità risulta \textbf{esponenziale} rispetto ad $n$, pertanto è proprio la (\ref{eqSubtourElimination}) a rendere i \textit{TSPs} \textbf{NP-hard}.

In particolare la dimensione di $S$, dato un numero $n$ di nodi, è $2^n$: immaginando di associare un bit ad ogni vertice (il cui valore definisce se appartiene o meno al sottoinsieme), e quindi di rappresentare un qualsiasi sottoinsieme attraverso una sequenza di $n$ bit se ne possono definire $2^n$ distinti. In realtà avendo imposto che il vertice $1$ appartenga sempre ad ogni $S$ e che $S \subsetneq V$, la quantità trovata va ridimensionata a $2^{n-1} - 1$, di grado comunque esponenziale rispetto ad $n$.

Per completezza la (\ref{eqSubtourEliminationAlternativa}) mostra una formulazione alternativa della (\ref{eqSubtourElimination}) caratterizzata dal medesimo numero di disequazione:

\begin{eqnarray}
\label{eqSubtourEliminationAlternativa}
&\displaystyle{\sum_{e \in E(S)}x_e \leq |S| - 1} \text{ , } \forall S \subsetneq V \text{ , } |S| \geq 2
\end{eqnarray}

La gestione \textbf{contemporanea} di un numero esponenziale di vincoli implica in genere tempi di risoluzione troppo elevati. Nella pratica però non è necessario inserirli tutti nel modello matematico, è sufficiente considerarne un numero molto più ridotto e cioè solamente quelli \textbf{utili} all'ottenimento della soluzione ottima. Non esiste alcun metodo che permetta di conoscere tale informazione in anticipo ed è compito degli algoritmi risolutori definire un opportuno \textbf{separatore}: una funzione che fornita in ingresso una soluzione $x^*$ ottima per il modello corrente generi tutti i vincoli violati da essa.

L'aggiunta di tali disequazioni al modello matematico impone alla successiva risoluzione di trovare una soluzione ottima diversa dalla precedente $x^*$. Nel momento in cui il separatore non fornisce più alcun vincolo aggiuntivo significa che siamo in presenza dell'ottimo globale.

\newpage

\section*{SEZIONE TRE}
\label{sec:SezioneTreS}

Questa sezione introduce la struttura del progetto per quanto riguarda la sua parte puramente di programmazione così da rendere la prosecuzione del testo più comprensibile e semplice.

Sono presentate le principali classi e strutture dati realizzate utili a livello globale per tutti gli algoritmi implementati mentre altre di interesse più \textit{locale} vengono introdotte solamente più avanti nel momento più opportuno.

\subsection*{STRUTTURA DEL PROGETTO VISUAL STUDIO}
\label{sec:StrutturaProgettoVSS}

Per prima cosa è importante chiarire l'organizzazione delle cartelle e files, una struttura ordinata agevola enormemente ogni fase del progetto: progettazione, stesura del codice, debugging, simulazione della release e versione finale.

All'interno della cartella radice, da chiamata TSPCsharp, si sono create le seguenti sottocartelle:

\begin{itemize}
    \item \textbf{Src}: contiene il progetto della applicazione e quindi tutti i file sorgente;
    \item \textbf{Data}: include le istanze \textit{TSP} utilizzabili dal software;
    \item \textbf{Concorde}: contiene la porzione dei file sorgente, in linguaggio di programmazione \textit{C}, del software solver \textbf{Concorde} necessarie per il progetto;
    \item \textbf{Output}: la directory dove l'applicazione pone i file di output prodotti;
\end{itemize}

Il software sviluppato è composto dalle seguenti dieci classi:

\begin{itemize}
    \item \textbf{Instance}
    \item \textbf{ItemList}
    \item \textbf{PathGenetic}
    \item \textbf{PathStandard}
    \item \textbf{Point}
    \item \textbf{Program}
    \item \textbf{Tabu}
    \item \textbf{TSP}
    \item \textbf{TSPLazyConsCallback}
    \item \textbf{Utility}
\end{itemize}

Per le classi \textbf{Point}, \textbf{Instance}, \textbf{Program}, \textbf{TSP} e \textbf{Utility} viene fornita una descrizione in questo sezione essendo di interesse globale per tutto il progetto.

Si specifica che le variabili globali di qualsiasi classe, presenti solamente in quelle che offrono strutture dati utili al programma, sono dichiarate con metodo di accesso \textbf{internal}\footnote{Membri interni sono accessibili solo all'interno di file nello stesso assembly.} e possiedono i propri metodi \textbf{Getter}\footnote{Utilizzato per leggere la variabile.} e \textbf{Setter}\footnote{Utilizzato per modificare la variabili.} che ne ereditano il livello di accessibilità. Il linguaggio di programmazione \textit{C\#} offre la seguente sintassi semplificata:

\begin{lstlisting}
    protected var variabile { get; set; }
\end{lstlisting}

\subsection*{CLASSE POINT}
\label{sec:ClassePointS}

La classe \textbf{Point} offre una struttura dati in grado di memorizzare le coordinate bidimensionali di un singolo nodo $n$.
Presenta a tal fine le variabili globali \textbf{x} e \textbf{y} di tipo \textbf{double}. Il costruttore della classe non fa altro che ricevere in input i valori da assegnare a queste ultime. La classe presenta inoltre un ulteriore metodo pubblico e statico chiamato \textbf{Distance} che permette il calcolo della distanza tra due nodi:

\begin{lstlisting}
    public static double Distance(Point p1, Point p2, string pointType)
\end{lstlisting}

Dove:

\begin{itemize}
\item \textbf{p1}: riferimento ad un oggetto della classe \textbf{Point} che rappresenta il primo nodo;
\item \textbf{p2}: riferimento ad un oggetto della classe \textbf{Point} che rappresenta il primo nodo;
\item \textbf{pointType}: stringa che descrive quale tipo di formula matematica si debba utilizzare per calcolare il costo del lato delimitato da \textbf{p1} e \textbf{p2}. Può assumere i seguenti valori:
    \begin{itemize}
    \item EUC\char`\_2D
    \item ATT
    \item MAN\char`\_2D
    \item GEO
    \item MAX\char`\_2D
    \item CEIL\char`\_2D
    \end{itemize}
\end{itemize}

Di seguito è riportato il codice del metodo \textbf{Distance} dove le formule matematiche utilizzate seguono le direttive indicate dai creatori delle istanze \textit{TSP} compatibili con l'applicazione\footnote{Il documento è reperibile al seguente \href{https://www.iwr.uni-heidelberg.de/groups/comopt/software/TSPLIB95/tsp95.pdf}{indirizzo}.}.

\begin{lstlisting}
double xD = p1.x - p2.x;
double yD = p1.y - p2.y;

if (pointType == "EUC_2D")
{
    return (int)(Math.Sqrt(xD * xD + yD * yD) + 0.5);
}
else if (pointType == "MAN_2D")
{
    xD = Math.Abs(xD);
    yD = Math.Abs(yD);
return (int)(xD + yD + 0.5);
}
    else if (pointType == "MAX_2D")
{
    xD = Math.Abs(xD);
    yD = Math.Abs(yD);

    int fI = Convert.ToInt32(xD + 0.5);
    int sI = Convert.ToInt32(yD + 0.5);
    if (fI >= sI)
        return fI;
    else
        return sI;
}
else if (pointType == "GEO")
{
    double PI = Math.PI;

    int deg = (int)(p1.x + 0.5);
    double min = p1.x - deg;
    double latitude1 = PI * (deg + 5 * min / 3) / 180;

    deg = (int)(p1.y + 0.5);
    min = p1.y - deg;
    double longitude1 = PI * (deg + 5 * min / 3.0) / 180;

    deg = (int)(p2.x + 0.5);
    min = p2.x - deg;
    double latitude2 = PI * (deg + 5 * min / 3.0) / 180;
    
    deg = (int)(p2.y + 0.5);
    min = p2.y - deg;
    double longitude2 = PI * (deg + 5 * min / 3.0) / 180;
    
    double RRR = 6378.388;
    double q1 = Math.Cos(longitude1 - longitude2);
    double q2 = Math.Cos(latitude1 - latitude2);
    double q3 = Math.Cos(latitude1 + latitude2);
    
    return (int)((RRR * Math.Acos(0.5 * ((1 + q1) * q2 - (1 - q1) * q3))) + 1);
}
else if (pointType == "ATT")
{
    double rij = Math.Sqrt((xD * xD + yD * yD) / 10.0);
    int tij = Convert.ToInt32(rij);

    if (tij < rij)
        return tij + 1;
    else
        return tij;
}
else if (pointType == "CEIL_2D")
{
    return Math.Ceiling(Math.Sqrt(xD * xD + yD * yD) + 0.5);
}

//Nel caso in cui 
throw new Exception("Bad input format");
\end{lstlisting}

\subsection*{CLASSE INSTANCE}
\label{sec:ClasseInstanceS}

La classe \textbf{Instance} memorizza tutti i dati caratterizzanti l'istanza del Problema del Commesso Viaggiatore corrente. Essendo le variabili globali di questa classe numerose viene riportato il loro codice dichiarativo commentato:

\begin{lstlisting}
//Dati input utente

//Nome file
internal string inputFile { get; set; }
//Tempo limite di esecuzione
internal double timeLimit { get; set; }


//Dati ricavati dal file di input

//Numero di nodi
internal int nNodes { get; set; }
//Coordinate (x,y) di ogni nodo
internal Point[] coord { get; set; }

//Determina come calcolare la distanza tra due punti
internal string edgeType { get; set; }


//Parametri da definire durante la risoluzione

//Costo della migliore soluzione intera trovata
internal double xBest { get; set; }
//Valore di ogni lato nella migliore soluzione intera trovata
internal double[] bestSol { get; set; }
//Migliore lower bound trovato, utilizzato in alcuni algoritmi
internal double bestLb { get; set; }
\end{lstlisting}

All'interno di questa classe è presente l'unico metodo pubblico e statico, esclusi i vari \textit{get} e \textit{set}, \textbf{Print} utilizzato per stampare all'interno della \textit{console} tutte le coordinate memorizzate in un oggetto di tipo \textbf{Instance} --> \textbf{coord}:

\begin{lstlisting}
static public void Print(Instance inst)
\end{lstlisting}

Dove:

\begin{itemize}
\item \textbf{inst}: riferimento ad un oggetto della classe \textbf{Instance} contente tutti dati che descrivono l'istanza del Problema del Commesso Viaggiatore fornita in ingresso dall'utente;
\end{itemize}

Il contenuto di \textbf{Print} è per definizione molto banale. L'unico aspetto degno di nota è che i reali indici identificativi dei nodi sono compresi tra i valori $1$ ed $n$ mentre, dato che nei linguaggi di programmazione le strutture dati \textit{contenitori} come ad esempio i \textbf{vettori} o le \textbf{liste} partono da $0$, il loro indice di memorizzazione è sempre scalato di uno. \textit{Print} è realizzato in modo tale da mostrare il reale indice di ogni nodo:

\begin{lstlisting}
for (int i = 0; i < inst.NNodes; i++)
    Console.WriteLine("Point #" + (i + 1) + "= (" + inst.coord[i].x + ";" + inst.coord[i].y + ")");
    
/*
* Esempio di stampa:
* Point #1= (x_1;y_1)
* Point #2= (x_2;y_2)
* Point #n= (x_n;y_n)
*/
\end{lstlisting}

\subsection*{CLASSE PROGRAM}
\label{sec:ClasseProgramS}

\textbf{Program} è la classe \textit{entry point} del progetto creata automaticamente dall'\textit{IDE}. Contiene pertanto il metodo di \textit{ingresso} \textbf{Main} che per convenzione, all'interno del linguaggio \textit{C\#}, è sempre il primo ad essere eseguito: riceve quindi come parametro di ingresso un vettore di stringhe, chiamato \textbf{argv}, al cui interno si trova la riga di comando fornita dall'utente\footnote{\textit{Visual Studio} offre la possibilità di simulare l'utilizzo della riga di comando anche quando l'applicazione è avviata dall'interno dell'\textit{IDE} stesso.}.

L'applicazione si aspetta  di trovarvi memorizzate le informazioni riguardanti il \textbf{tempo limite} di esecuzione e soprattutto il \textbf{nome} dell'istanza \textit{TSP} da risolvere\footnote{Reperibile all'interno dell'apposita cartella \textit{Data}.}.

La prima azione eseguita quindi dalla classe \textit{Program} è l'analisi del vettore \textit{argv}, operazione delagata al metodo \textbf{ParseInput} descritto alla seguente sezione \hyperref[sec:LetturaInputS]{Lettura Input File}.

Successivamente è necessario effettuare l'accesso in lettura al file, indicato dall'utente, contenente le informazioni dell'istanza \textit{TSP} da risolvere. Anche questa operazione viene delegata ad un secondo metodo \textbf{Populate} analizzato nella sezione \hyperref[sec:MetodoPopulateS]{Metodo Populate}.

Dichiarato e parzialmente popolato un oggetto di tipo \textbf{Instance} grazie ai metodi sopra citati, è invocato il metodo \textbf{Solve} contenuto nella \textbf{TSP}, la prossima ad essere descritta. Per ora basti sapere che \textit{Solve} chiede all'utente di indicare l'algoritmo risolutore che desidera applicare e provvede alla sua esecuzione.

L'ultima operazione effettuata dal metodo \textit{Main} consiste nel richiedere all'utente se desidera eliminare i file con estensione \textbf{.dat} e \textbf{.lp} creatisi durante l'esecuzione del programma. I dettagli riguardanti i file generati con queste estensioni sono riportati nel seguito della tesi.

\begin{lstlisting}
foreach (string file in Directory.GetFiles("..\\..\\..\\..\\Output\\", "*.dat").Where(item => item.EndsWith(".dat")))
{
    File.Delete(file);
}

foreach (string file in Directory.GetFiles("..\\..\\..\\..\\Output\\", "*.lp").Where(item => item.EndsWith(".lp")))
{
    File.Delete(file);
}
\end{lstlisting}

La classe \textit{Program} definisce l'unica costante globale pubblica di tutto il progetto:

\begin{lstlisting}
public const int VERBOSE = 5;
\end{lstlisting}

Le varie stampe a video eseguite tramite \textit{console} nella applicazione, sono classificate secondo una scala numerica di importanza in modo tale che: solo nel caso in cui il corrispettivo valore risulti infiore o pari al \textbf{VERBOSE} attuale l'istruzione sia realmente eseguita.

\begin{lstlisting}
if (VERBOSE >= 5)
    Console.WriteLine("Esempio VERBOSE");
\end{lstlisting}

\subsection*{CLASSE TSP}
\label{sec:ClasseTSPS}

La classe \textbf{TSP} è pensata come il cuore del programma in quanto la struttura principale di tutti i metodi risolutivi viene riportata al suo interno. Vi è presente un unico metodo pubblico \textbf{Solve}:

\begin{lstlisting}
static public void Solve(Instance instance)
\end{lstlisting}

Dove:

\begin{itemize}
\item \textbf{instance}: riferimento ad un oggetto della classe \textbf{Instance} contente tutti dati che descrivono l'istanza del Problema del Commesso Viaggiatore fornita in ingresso dall'utente;
\end{itemize}

Le operazioni eseguite da questo metodo sono raggruppabili in:

\begin{itemize}
    \item Inizializzazione degli oggetti comunemente utilizzati dalla maggior parte degli algoritmi risolutori;
    \item Viene richiesto all'utente di indicare quale metodo di risoluzione desidera che venga applicato tra quelli disponibili. Se necessario richiede anche parametri aggiuntivi caratterizzanti la scelta effettuata;
    \item Viene invocato il metodo correto per gestire la richiesta espressa dall'utente;
    \item Nel caso in cui l'algoritmo abbia prodotto una soluzione valida, il suo costo ed il tempo di esecuzione trascorso sono stampati nella \textit{console};
\end{itemize}

Per quanto riguarda gli oggetti da inizializzare troviamo: un oggetto della classe \textbf{Stopwatch} utilizzato come cronometro per gestire il tempo di esecuzione trascorso (viene anche avviato), un oggetto della classe \textbf{Process} necessario per disegnare una qualsiasi soluzione attraverso il programma \textbf{GNUPlot} ed infine un oggetto di tipo \textbf{Cplex} utilizzato per interfacciarsi con l'omonimo software solver. Maggiori dettagli riguardanti la classe \textit{Stopwatch} sono forniti nella sezione \hyperref[sec:StopwatchS]{Classe Stopwatch}, per quanto riguarda l'inizializzazione dell'oggetto di tipo \textit{Process} ed il software \textit{GNUPlot} maggiori dettagli sono disponibili rispettivamente in \hyperref[sec:InitProcessS]{Metodo InitProcess} e \hyperref[sec:GNUPlotS]{GNUPlot}.

\begin{lstlisting}

//Inizializzazione oggetto di tipo Cplex
Cplex cplex = new Cplex();
//Inizializzazione oggetto di tipo Stopwatch
Stopwatch clock = new Stopwatch();
//Avvio del cronometro
clock.Start();
/*
* Il metodo InitProcess della classe Utility fornisce un oggetto process
* inizializzato pronto per essere utilizzato per la comunicazione con
* GNUPlot per disegnare il grafico di una qualsiasi soluzione trovata
*/
Process process = Utility.InitProcess(instance);

\end{lstlisting}

La comunicazione con l'utente per conoscere quale metodo di risoluzione desidera utilizzare è molto semplice ed avviene attraverso la scrittura e lettura di stringhe nella \textit{console C\#}, pertanto non viene qui riportata.

L'avvio dello specifico algoritmo risolutivo viene riportato nella sezione ed esso dedicata in quanto al momento risulterebbe prematuro discuterne.

Infine la stampa riguardante l'eventuale costo della migliore soluzione trovata ed il relativo tempo di esecuzione richiesto dall'algoritmo selezionato, più lo stop del cronometro, avviene nel seguente modo:

\begin{lstlisting}

/*
* Program.VERBOSE >= 0 implica che la stampa venga sempre eseguita
* Se l'algoritmo avviato non ha modificato il valore di instance.xBest
* significa che nessuna soluzione è stata trovata
*/
if(Program.VERBOSE >= 0 && instance.xBest != 0)
    Console.WriteLine("The best solution found in " + clock.ElapsedMilliseconds / 1000 + " seconds has cost: " + instance.xBest);
else
    Console.WriteLine("No solution found in " + clock.ElapsedMilliseconds / 1000 + " seconds");

//Il cronometro viene fermato
clock.Stop();

\end{lstlisting}

\subsection*{CLASSE UTILITY}
\label{sec:ClasseUtilityS}

La classe Utility può essere considerata come una libreria: contiene al suo interno solamente metodi \textbf{statici}, utilizzati dagli algoritmi risolutori implementati, che si è deciso di raggruppare al suo interno così da rendere il codice il più compatto e leggibile possibile. Questi saranno presenti durante il corso della tesi oppure inseriti nella sezione \hyperref[sec:AppendiceS]{Appendice}.

\subsection*{CLASSE PATHGENETIC}
\label{sec:ClassePathGeneticS}

La classe \textbf{PathGenetic} utilizzata per memorizzare i dati di una soluzione generica, estende \textbf{PathStandard} già discussa nel paragrafo X.Y. aggiungendo due campi utili solamente per gli algoritmi genetici: il primo di tipo \textit{double} memorizza la fitness associata alla soluzione, il secondo di tipo intero identifica il circuito all'atto dell'estrazione dei percorsi che formeranno la generazione successiva\footnote{I suddetti parametri prendono nome \textbf{fittness} e \textbf{nRoulette}}.
La classe è dotata del metodo privato \textbf{CalculateFitness} il quale semplicemente setta la variabile fitness come descritto in precedenza:


\begin{lstlisting}

private void CalculateFitness()
{
fitness = 1 / cost;
}

\end{lstlisting}

La variabile \textbf{cost} e l’array \textbf{path} sono ereditati da PathStandard e vengono settati utilizzando uno dei tre costruttori a disposizione

\begin{lstlisting}

public PathGenetic(int[] path, double cost) : base()
{
this.path = path;
this.cost = cost;
CalculateFitness();
nRoulette = -1;
}

public PathGenetic(int[] path, Instance inst) : base(path, inst)
{
CalculateFitness();
nRoulette = -1;
}


public PathGenetic(): base()
{
fitness = -1;
nRoulette = -1;
}

ANDREBBE UN COMMENTINO SU OGNUNO

\end{lstlisting}

\newpage

\section*{SEZIONE QUATTRO}
\label{sec:SezioneQuattroS}

In questa sezione dedichiamo spazio alla struttura e sintassi con cui i file di input, contenenti le informazione della istanza \textit{TSP} che si desidera risolvere tramite l'applicativo, devono essere organizzati.

In particolare sono analizzati i metodi \hyperref[sec:LetturaInputS]{ParseInput} e \hyperref[sec:MetodoPopulateS]{Populate} utilizzati dalla classe \hyperref[sec:ClasseProgramS]{Program} per, rispettivamente, ottenere il nome del file descrittore dell'istanza \textit{TSP} da risolvere e l'effettiva estrapolazione delle informazione ivi contenute così da memorizzarle all'interno di un oggetto di tipo \hyperref[sec:ClasseInstanceS]{Instance}.

Prima di procedere, si segnala l'indirizzo

\begin{center}
    \href{http://elib.zib.de/pub/mp-testdata/tsp/tsplib/tsp/index.html}{http://elib.zib.de/pub/mp-testdata/tsp/tsplib/tsp/index.html}
\end{center}

contenente una serie di istanze \textit{TSP} compatibili, alcune delle quali utilizzate per i \hyperref[sec:TestRisultatiS]{test} riportati a fine tesi.

\subsection*{METODO PARSEINPUT}
\label{sec:LetturaInputS}

Le informazioni che l'utente deve \textbf{necessariamente} fornire tramite \textbf{riga di comando} sono semplicemente due: il nome del file contenente le caratteristiche dell'istanza \textit{TSP} che desidera risolvere ed il tempo limite di esecuzione\footnote{Questo corrisponde alla massima durata applicabile agli algoritmi di risoluzione.}.

\textit{Visual Studio} offre la possibilità di avviare un proprio progetto, in modalità \textit{debug} o \textit{release}, direttamente dall'interno dell'\textit{IDE} omettendo quindi la classica riga di comando. I parametri che però si desidererebbe passare alla applicazione attraverso questa modalità, sono comunque indicabili selezionando dal menù la voce \textbf{Progetto} e quindi \textbf{Proprietà di "nome\_applicazione"...}: entrare nella scheda \textbf{Debug} ed utilizzare la \textit{text box} \textbf{Argomenti della riga di comando:}.

Il contenuto degli argomenti della riga di comando viene automaticamente \textit{splittato} nel vettore di stringhe \textbf{argv} ricevuto come parametro di ingresso dal metodo \textbf{Main} del progetto.

Il criterio di \textit{splitting} è molto banale e delimita le varie stringhe utilizzando il carattere \textbf{' '}, ad esempio "test test1 test2 123" diviene \textbf{argv[0] = test; argv[1] = test1; argv[2] = test2; argv[3] = 123}.

Per convenzione l'applicazione richiede che gli argomenti della riga di comando utilizzino la seguente sintassi: \textbf{"-timelimit xxx -file yyy"} oppure l'opposto \textbf{" -file yyy -timelimit xxx"} dove \textbf{xxx} e \textbf{yyy} sono rispettivamente i \textbf{secondi} limite ed il \textbf{nome} del file comprendente la sua \textbf{estensione}.

Il metodo che trasferisce tali informazioni all'interno di un oggetto di tipo \textbf{Instance} è \textbf{ParseInput}:

\begin{lstlisting}
static void ParseInput(Instance inst, string[] input)
\end{lstlisting}

Dove:

\begin{itemize}
\item \textbf{inst}: riferimento ad un oggetto della classe \textbf{Instance} contente tutti dati che descrivono l'istanza del Problema del Commesso Viaggiatore fornita in ingresso dall'utente;
\item \textbf{input}: vettore di \textbf{stringhe} da cui recuperare le informazioni da inserire all'interno di \textbf{inst};
\end{itemize}

Seguendo la regola precedentemente esposta, il contenuto di \textit{ParseInput} è semplicemente:

\begin{lstlisting}

for (int i = 0; i < input.Length; i++)
{
    /*
    * Cerco una delle due parole chiavi "-file" e "-timelimit"
    * all'indice successiva è contenuta la relativa informazione
    */
   if (input[i] == "-file")
   {
        /*
        * Si aspetta di trovare il nome del file input
        * (compreso di estensione)
        */
        inst.inputFile = input[++i];
        continue;
    }
    if (input[i] == "-timelimit")
    {
        //Si aspetta di trovare il tempo limite espresso in secondi)
        inst.timeLimit = Convert.ToDouble(input[++i]);
        continue;
    }
}
\end{lstlisting}

Nel caso in cui una qualsiasi delle due informazioni richieste non viene fornita scatta una eccezione:

\begin{lstlisting}

if (inst.InputFile == null || inst.TimeLimit == 0)
     throw new Exception("Missing information or bad format inside the command line");

\end{lstlisting}

\subsection*{METODO POPULATE}
\label{sec:MetodoPopulateS}

Inserita l'informazione riguardante il nome del file, contenente l'istanza \textit{TSP} che si desidera studiare, all'interno di un oggetto della classe \textbf{Instance}), questo viene ulteriormente \textit{popolato} con le informazioni contenute all'interno del file stesso grazie al metodo \textbf{Populate}.

La struttura che il file di input, posto all'interno della directory \textbf{Data}, deve assumere è del tipo

\begin{center}
<parolaChiave> : <valore>
\end{center}

dove le parole chiavi possibili sono le seguenti:

\begin{itemize}
    \item \textbf{NAME}:<stringa>
    \begin{itemize}
    \item nome con cui l'istanza è nota in letteratura. Non memorizzato;
    \end{itemize}

    \item \textbf{TYPE}:<stringa>
        \begin{itemize}
    \item indica il tipo dell'istanza. Non memorizzato;
    \end{itemize}
     
    \item \textbf{COMMENT}:<stringa>
    \begin{itemize}
    \item include informazioni aggiuntive, solitamente il nome deli gli autori che hanno proposto l'istanza. Non memorizzato;
    \end{itemize}
    
    \item \textbf{DIMENSION}:<integer>
    \begin{itemize}
    \item indica il numero di nodi. Memorizzabile nel campo \textbf{Instance.nNodes};
    \end{itemize}
    
    \item \textbf{EDGE WEIGHT TYPE}:<string>
    \begin{itemize}
    \item Definisce il modo con cui il costo del lato deve essere calcolato, maggiori dettagli nella sezione \hyperref[sec:ClassePointS]{Classe Point}. Memorizzabile nel campo \textbf{Instance.edgeType};
    \end{itemize}
    
    \item \textbf{NODE COORD SECTION}:
    \begin{itemize}
    
    \item Il contenuto di questa sezione si sviluppa in più righe, tante quante i nodi caratterizzanti l'istanza. In ognuna di esse devono trovarsi nel seguente ordine:
    \begin{itemize}
    \item L'indice \textit{reale} $i$ che rappresenta il nodo di cui si stanno per fornire le coordinate. Tali valori sono inseriti attraverso un oggetto di tipo \hyperref[sec:ClassePointS]{Point} all'interno di 
    \begin{center}
        \textbf{Instance.coord[i-1]}
    \end{center}
    Di conseguenza $i$ non può assumere valori inferiori ad \textbf{$1$} o superiori ad \textbf{$n-1$};
    \item Un numero reale positivo che definisce la coordinata \textbf{x} del nodo.
    \item Un numero reale positivo che identifica la coordinata \textbf{y} del nodo.
    \end{itemize}
    \end{itemize}
\end{itemize}

Il file di testo deve terminare sempre con la stringa \textbf{EOF}.\\

All'interno del linguaggio \textit{C\#} sono definite diverse classe in grado di estrapolare il contenuto di un file testo utilizzando specifici encoding. Tra le più utilizzate si trova il tipo \textbf{StreamReader} provvisto tra gli altri anche di un costruttore che accetta solamente il \textit{path} del file su cui creare un flusso di lettura di byte mentre l'encoding è settato automaticamente ad \textbf{UTF8}\footnote{\textit{UTF8} è perfettamente compatibili con i semplici file di interesse.}.

Il metodo \textbf{Populate}, ricevuto in ingresso l'oggetto \textbf{Instance inst}, presenta al suo interno la seguente struttura:

\begin{lstlisting}

//Stringa sulla quale vengono inserite le varie righe lette dal file
string line;
/*
* Variabile booleana che indica se la lettura del file
* è giunta alla sezione contenente le coordinate dei nodi del grafo
*/
bool readingCoordinates = false;
//Rappresenta il numero di nodi per cui sono state lette le coordinate
int cntNodes = 0;

/*
* Quando si tenta una lettura di un file è buona norma inserirla
* dentro un blocco try-catch in quanto questa può fallire per
* diversi motivi (file aperto da un altra applicazione,
* path errato, encoding non compatibile, file corrotto...)
* In questo modo si cattura l'eventuale eccezione evitando
* un blocco brusco dell'applicazione
*/
try
{
    /*
    * La direttiva using permette una autogestione della chiusura
    * (obbligatoria) di un oggetto StreamReader ed il flusso
    * sottostante creatosi. Ogni .. all'interno del path 
    * specificato corrisponde alla directory superiore
    * alla attuale
    */
    using (StreamReader sr = new StreamReader("..\\..\\..\\..\\Data\\" + inst.inputFile))
    {
        /*
        * L'intero file viene letto, line contiene ad ogni iterazione
        *  una diversa linea del testo
        */
        while ((line = sr.ReadLine()) != null)
        {
            //Gestione di line
        }
    }
}
catch (System.Exception e)
{
    //L'errore accorso viene stampato
    Console.WriteLine("The file could not be read:");
    Console.WriteLine(e.Message);
}

\end{lstlisting}

Per rendere la gestione delle varie \textit{linee} più comprensibile, si anticipa ora una breve spiegazione dei metodi, non statici, appartenenti alla classe \textbf{string} di cui si fa utilizzo premettendo che una \textit{stringa} è vista anche come un array di \textit{caratteri}:

\begin{lstlisting}

public bool StartWith(string value)

\end{lstlisting}

Restituisce il valore \textbf{true} nel caso in cui la stringa sul quale viene invocato ha come prefisso il contenuto di \textbf{value}, \textbf{false} altrimenti.


\begin{lstlisting}

public int IndexOf(string value, int startIndex)

\end{lstlisting}

Restituisce l'indice in base zero della prima occorrenza del contenuto di \textbf{value}. La ricerca ha inizio alla posizione \textbf{startIndex}.


\begin{lstlisting}

public string Remove(int startIndex, int count)

\end{lstlisting}

Restituisce la stringa sulla quale viene invocato private di \textbf{count} caratteri a partire da quello di indice \textbf{startIndex}.


\begin{lstlisting}

public string[] Split(char[] separator, StringSplitOptions options)

\end{lstlisting}

Suddivide la stringa su cui viene invocato in sottostringhe in base ai caratteri contenuti in \textbf{separator}. È possibile specificare se le sottostringhe includono elementi della matrice vuota attraverso \textbf{options} che può valere \textbf{StringSplitOptions.RemoveEmptyEntrie} oppure \textbf{StringSplitOptions.None}.


\begin{lstlisting}

public string Replace(string oldValue, string newValue)

\end{lstlisting}

Restituisce la stringa su cui viene invocato sostituendo tutte le occorrenze di \textbf{oldValue} con \textbf{newValue}.


Si ricorda inoltre che nel linguaggio \textit{C\#} la conversione da tipo \textbf{string} ad altri non è immediata o realizzabile tramite un semplice cast: è necessario usufruire della classe \textbf{Convert} che mette a disposizione i vari metodi statici \textbf{ToDouble}, \textbf{ToInt32} e molti altri.

Viene infine mostrata la rimanente porzione di codice del metodo \textit{Populate}, presente all'interno del ciclo \textbf{while} lasciato sospeso precedentemente, che gestisce tutte le righe lette dal file descrittore dell'istanze \textit{TSP} da risolvere:

\begin{lstlisting}

/*
* Il contenuto delle righe con prefisso le seguenti parole chiavi
* viene solamente mostrato a video ma non memorizzato dentro
* l'oggetto di tipo Istance che si sta popolando
* in quanto tali informazioni non risultano rilevanti ai fini
* della risoluzione dell'istanza TSP
*/
if (line.StartsWith("NAME") || line.StartsWith("COMMENT") || line.StartsWith("TYPE") || line.StartsWith("EDGE_WEIGHT_FORMAT: FUNCTION") || line.StartsWith("DISPLAY_DATA_TYPE: COORD_DISPLAY")) 
{
    if (VERBOSE >= 5)
        Console.WriteLine(line);
    
    //Si passa alla riga successiva del file
    continue;
}

/*
* Il numero di nodi del grafo è un parametro essenziale
* viene memorizzato in Instance.nNodes e
* determina la grandezza del vettore Instance.coord
* nel quale sono memorizzate le coordinate (x,y)
* di ogni nodo
*/
if (line.StartsWith("DIMENSION"))
{
    //Memorizzazione del numero n di nodi del grafo
    inst.nNodes = Convert.ToInt32(line.Remove(0, line.IndexOf(':') + 2));
    //Alloco al vettore coord lo spazio di memoria di n oggetti Point
    inst.coord = new Point[inst.nNodes];
    
    if (VERBOSE >= 5)
        Console.WriteLine(line);
    
    //Si passa alla riga successiva del file
    continue;
}

/*
* Questo parametro determina la formula matematica
* da utilizzare per calcolare la distanza tra due oggetti Point
* Viene memorizzato in Instance.edgeType
*/
if (line.StartsWith("EDGE_WEIGHT_TYPE"))
{
    string tmp = line.Remove(0, line.IndexOf(':') + 2);
    //Solo questi tipi sono supportati dalla applicazione
    if (!(tmp == "EUC_2D" || tmp == "ATT" || tmp == "MAN_2D" || tmp == "GEO" || tmp == "MAX_2D" || tmp == "CEIL_2D"))
        throw new System.Exception("Format error:  only EDGE_WEIGHT_TYPE == {ATT, MAN_2D, GEO, MAX_2D and CEIL_2D} are implemented");
    
    //Se il tipo di peso di nodo è supportato dalla applicazione
    inst.edgeType = tmp;

    if (VERBOSE >= 5)
        Console.WriteLine(line);
    
    //Si passa alla riga successiva del file
    continue;
}

/*
* La parola chiave NODE_COORD_SECTION indica che le prossime n
* righe contengono le informazioni riguardanti le coordinate (x,y)
* degli n nodi del grafo
*/
if (line.StartsWith("NODE_COORD_SECTION"))
{
    //L'informazione riguardante il numero di nodi deve essere nota
    if (inst.nNodes <= 0)
        throw new System.Exception("DIMENSION section should be before NODE_COORD_SECTION section");
        
    if (VERBOSE >= 5)
        Console.WriteLine(line);
        
    //Viene settato a true il valore di readingCoordinates
    readingCoordinates = true;
    
    //Si passa alla riga successiva del file
    continue;
}

//Questa riga viene ignorata
if (line.StartsWith("EDGE_WEIGHT_FORMAT: FUNCTION "))
{
    //Si passa alla riga successiva del file
    continue;
}

//La parola chiave EOF indica la terminazione del file
if (line.StartsWith("EOF"))
{
    /*
    * Le informazioni memorizzate riguardanti le coordinate (x,y)
    * degli n nodi vengono stampate a video
    */
    Instance.Print(inst);
    
    if (VERBOSE >= 5)
        Console.WriteLine(line);
    
    //Il ciclo di lettura viene interrotto
    break;
}

/*
* Se il valore readingCoordinates è pari a true significa che
* la lettura del file è giunta alla n righe contenenti
* le coordinate (x,y) degli n nodi del grafo del problema TSP
*/
if (readingCoordinates)
{
    /*
    * elements viene settato nel seguente modo
    * elements[0] -> indice reale del nodo
    * elements[1] -> coordinata x del nodo
    * elements[2] -> coordinata y del nodo
    */
    string[] elements = line.Split(new[] { ' ' }, StringSplitOptions.RemoveEmptyEntries);

    int i = Convert.ToInt32(elements[0]);
    //Il valore dell'indice deve essere compreso tra 1 ed n
    if (i < 0 || i > inst.nNodes)
        throw new System.Exception("Unknown node in NODE_COORD_SECTION section");
    /*
    * Se l'indice reale i è valido
    * l'oggetto Point che descrive le sue coordinate bidimensionali
    * viene memorizzato alla posizione i-1 di Instance.coord
    */
    inst.coord[i - 1] = new Point(Convert.ToDouble(elements[1].Replace(".", ",")), Convert.ToDouble(elements[2].Replace(".",",")));
    
    //Il contatore dei nodi analizzati è aumentato
    cnt++;
    
    /*
    * Reperite le informazioni di tutti gli n nodi
    * readingCoordinates è settato a false
    */
    if(cnt == inst.nNodes)
        readingCoordinates = false;
    
    //Si passa alla riga successiva del file
    continue;
}

//Se la riga è priva di caratteri si passa alla successiva
if (line == "")
    continue;

/*
* Giunti a questo punto finale del ciclo while significa 
* che il file non rispetto lo standar prefissato
* la sua lettura non può essere continuata
*/
throw new System.Exception("The file bad format");

\end{lstlisting}

\newpage

\section*{SEZIONE CINQUE}
\label{sec:SezioneCinqueS}

Attraverso il paragrafo \hyperref[sec:ModelloMatematicoS]{Modello Matematico} è possibile notare come la struttura generale del modello matematico, con o senza i vincoli di \textit{subtour elimination}, di un \textit{TSP} rimane sempre la medesima: ciò che varia riguarda il numero di variabili utilizzate ed il loro costo. Ottenute queste due informazioni, seguendo quanto mostrato nella sezione \hyperref[sec:SezioneQuattroS]{precedente}, è possibile quindi procedere alla creazione del modello matematico utilizzando le apposite strutture dati messe a disposizione da \textbf{CPLEX}.

Essendo questa operazione necessaria ogniqualvolta si desidera utilizzare suddetto solver, in questa sezione vengono mostrati i passaggi da seguire per una generica istanza \textit{TSP} caratterizzata da $n$ nodi e quindi $(n-1)*n/2$ lati di costo generico $c_{i}: i \in \left [ 1,n \right ]$.

\subsection*{MODELLO MATEMATICO: LINGUAGGIO C}
\label{sec:ModelloCS}

Per istanziare un nuovo modello di \textbf{programmazione lineare} è necessario inizializzare un \textbf{environment} di CPLEX attraverso la funzione \textbf{CPXopenCPLEX} la quale ritorna un puntatore all'environment creato:


\begin{lstlisting}

CPXENVptr CPXopenCPLEX(int* status_p)

\end{lstlisting}

Dove:

\begin{itemize}
\item \textbf{status\char`\_p}: puntatore ad una variabile di tipo intero utilizzata per comunicare un eventuale codice di errore;
\end{itemize}

Ad un enviroment è possibile associare uno o più modelli matematici attraverso il comando \textbf{CPXcreateprob}:

\begin{lstlisting}

CPXLPptr CPXcreateprob(CPXCENVptr env, int * status_p, const char * probname_str)

\end{lstlisting}

Dove:

\begin{itemize}
    \item \textbf{env}: puntatore all'environment sul quale si è deciso di creare il modello;
    \item \textbf{status\char`\_p}: puntatore ad una variabile di tipo intero utilizzata per comunicare un eventuale codice di errore;
    \item \textbf{probname\char`\_str}: rappresenta un array di caratteri che definisce il nome del modello creato;
\end{itemize}

Anche in questo caso il valore di ritorno della funzione \textit{CPXcreateprob} è un puntatore, nello specifico al modello matematico, al momento ancora vuoto, aggiunto ad \textit{env}.

Si procede quindi al \textit{popolamento} del modello defininendo in principio la funzione obiettivo da minimizzare (o massimizzare) e quindi le variabili con i relativi costi. Esistono diverse modalità per eseguire tale operazione ma tutte sfruttana la funzione \textbf{CPXnewcols}:

\begin{lstlisting}

int CPXnewcols (CPXENVptr env,CPXLPptr lp,int ccnt,double *obj, double *lb, double *ub, char *ctype, char **colname);

\end{lstlisting}

Dove:

\begin{itemize}
\item \textbf{env} : puntatore all'enviroment di CPLEX nel quale è definito il modello matematico;
\item \textbf{lp} : puntatore ai dati che caratterizzano il modello matematico;
\item \textbf{ccnt} : intero che indica il numero delle nuove variabili che vengono aggiunte al problema;
\item \textbf{obj} : array di lunghezza \textbf{ccnt} contenente per ogni variabile il relativo coefficiente nella funzione obiettivo, in altre parole quindi il loro costo;
\item \textbf{lb} : array di lunghezza \textbf{ccnt} contenente il \textit{lower bound}\footnote{Valore \textbf{minimo} che può assumere.} di ogni variabile aggiunta;
\item \textbf{ub} : array di lunghezza \textbf{ccnt} contenente l'\textit{upper bound}\footnote{Valore \textbf{massimo} che può raggiungere.} di ogni variabile aggiunta;
\item \textbf{ctype} : array di lunghezza \textbf{ccnt} contenente il tipo di ogni variabile. I valori che ogni elemento di questo array può assumere sono:
\begin{itemize}
    \item \textbf{\textipa{"}C\textipa{"}}: variabile \textbf{continua};
    \item \textbf{\textipa{"}B\textipa{"}}: variabile \textbf{binaria};
    \item \textbf{\textipa{"}I\textipa{"}}: variabile \textbf{intera};
\end{itemize}
\item \textbf{colname} : array di lunghezza \textbf{ccnt} contenente i puntatori ad array di char. Sono qui presenti tutti i nomi identificativi delle variabili che vengono aggiunte al modello matematico;
\end{itemize}

Per motivi di semplicità non è consigliabile inserire tutte le $n$ variabili, e relativi costi, attraverso un'unica chiamata di \textit{CPXnewcols}: tale metodo però, come appena mostrato, si aspetta un utilizzo opposto. La soluzione in realtà è molto semplice e sfrutta una particolare peculiarità del linguaggio di programmazione \textit{C}, questo infatti permette di anteporre il simbolo \& ad una qualsiasi variabile per ottenerne un puntatore alla locazione di memoria. Questa può essere infine utilizzata come \textit{array} per tutti gli elementi dell'argomento della funzione che lo richiedono.

Discutiamo ora delle singole variabili che si vuole aggiungere al modello matematico tenendo presente che quanto viene detto rimarrà valido anche per il linguaggio di programmazione \textbf{C\#}.

\'E noto che per ogni coppia di nodi, identificati dagli indici $i$ e $j$\footnote{Si ricordi che tali valori devono essere tra loro diversi per evitare la presenza di cappi.}, esiste un unico lato, privo di direzione, che li collega. Vi è quindi la necessità di scegliere una convenzione per l'assegnazione del nome ai vari lati:

\begin{itemize}
    \item \textbf{$x(i,j)$} se e solo se $i < j$;
    \item \textbf{$x(j,i)$} se e solo se $i > j$;
\end{itemize}

Questa convenzione offre anche un importante spunto per definire l'indice univoco della posizione con cui identificare tutti i dati riguardanti uno specifico lato all'interno del relativo \textit{array} utilizzato da \textbf{Cplex}: presi due qualsiasi lati distinti $x(i,j)$ e $x(v,w)$, le informazioni riguardanti il primo vengono memorizzate all'interno del relativo \textit{array} ad un indice inferiore rispetto a quello del secondo se e solo se $(i<v) || (i==v \& j<w)$.

In altre parole, analizzando il contenuto degli \textit{array} iterativamente si incontrano i dati riguardanti i lati $x(1,2)$, $x(1,3)$, ..., $x(1,n)$, $x(2,3)$, $x(2,4)$, ... , $x(2,n)$, ..., $x(n-2,n)$, $x(n-1,n)$.

Di conseguenza è possibile definire un metodo \textbf{xPos} che dati gli indici $i$ e $j$ \textit{non reali}\footnote{Si ricordi che per convenzione se l'indice \textit{reale} di un nodo è $i$, questo viene memorizzato alla posizione $i-1$ dato che nei linguaggi di programmazione \textbf{C} e \textbf{C\#} il primo indici di un array, ad esempio, è pari a $0$ e non ad $1$.} di due nodi più il riferimento alla variabile \textbf{struct} \textit{Instance} in \textit{C}, oppure all'oggetto della classe \textbf{Instance} in \textit{C\#}, fornisce l'offset da utilizzare per reperire i dati relativi al lato $x(i,j)$\footnote{Oppure $x(j,i)$ in base ai valori di $i$ e $j$.}. Maggiori dettagli riguardanti tale funzione, per entrambi i linguaggi di programmazione, sono riportati nell'apposita sezione \hyperref[sec:XPos]{Metodo xPos}.

\begin{lstlisting}
double lb = 0.0;
double ub = 1.0;
char binary = 'B';

char **cname = (char **)calloc(1, sizeof(char *));
cname[0] = (char *)calloc(100, sizeof(char));

//Il doppio ciclo for tiene conto solo delle coppie di nodi
//(i,j) tali che (i < j)
for (int i = 0; i < inst->nNodes; i++)
{
    for (int j = i + 1; j < inst->nNodes; j++)//Mi interessano solo le coppie con i<j
    {
        //Il nome del lato utilizza gli indici reali dei nodi
        sprintf(cname[0], "x(%d,%d)", i + 1, j + 1);
        //Calcolo del costo del lato x(i,j)
        double obj = dist(inst->coord[i], inst->coord[j], inst->edgeType);
        //Si definisce la variabili del lato x(i,j) ed il suo costo
        //all'interno della funzione obiettivo attraverso CPXnewcols
        //se fallisce si stampa un messaggio di errore
        if (CPXnewcols(env, lp, 1, &obj, &zero, &ub, &binary, cname))
            printError(" ... errato CPXnewcols su x");
        //Viene controllato che il tutto sia avvenuto correttamente
        //in caso contrario si stampa un messaggio di errore
        if (CPXgetnumcols(env, lp) - 1 != xPos(i, j, inst))
            printError(" ... errata posizione per x");
    }
}
\end{lstlisting}

La restante definizione dei vincoli, esclusi quelli di \textit{subtour elimination}, avviene attraverso la funzione \textbf{CPXnewrows}:

\begin{lstlisting}
int CPXnewrows(CPXCENVptr env, CPXLPptr lp, int rcnt, const double * rhs, const char * sense, const double * rngval, char ** rowname)
\end{lstlisting}

Dove:

\begin{itemize}
\item \textbf{env} : puntatore all'enviroment di CPLEX nel quale è definito il modello matematico;
\item \textbf{lp} : puntatore ai dati che caratterizzano il modello matematico;
\item \textbf{rcnt}: intero che definisce il numero di nuovi vincoli da aggiungere al modello matematico;
\item \textbf{rhs}: array di lunghezza \textbf{rcnt} contenente il termine noto di ogni vincolo;
\item \textbf{sense}: array di lunghezza \textbf{rcnt} i cui elementi possono assumere i seguenti valori:

\begin{itemize}
\item \textbf{\textipa{"}L\textipa{"}}: indica che il vincolo è una disuguaglianza il cui segno è  $\leq$
\item \textbf{\textipa{"}E\textipa{"}}: indica che il vincolo è una uguaglianza
\item \textbf{\textipa{"}G\textipa{"}}: indica che il vincolo è una disuguaglianza il cui segno è $\geq$
\item \textbf{\textipa{"}R\textipa{"}} : indica che il vincolo è limitato 
\end{itemize}

\item \textbf{rngval}: array di lunghezza \textbf{rcnt} contenente i valori di range per i nuovi vincoli;
\item \textbf{rowname}:  array di lunghezza \textbf{rcnt} contenente i nomi dei nuovi vincoli;
\end{itemize}

Anche in questo caso anzichè aggiungere tutti i vincoli in una singola iterazione, risulta più semplice aggiungerne uno nuovo volta per volta:

\begin{lstlisting}
//Per nodo determina un proprio vincolo
for (int i = 0; i < inst->nNodes; i++)
{
    //Offset a cui inserire il nuovo vincolo
    int lastRow = CPXgetnumrows(env, lp);
    //Il termine noto di ogni vincolo è pari a 2
    double rhs = 2.0;
    //Indica che in vincoli sono tutte uguaglianze
    char sense = 'E';//L è <=, G è >=, e E è =
    //Il nome del vincolo
    sprintf(cname[0], "grado(%d)", i + 1);
    //Aggiunto un vincolo nuovo inizialmente vuoto
    if (CPXnewrows(env, lp, 1, &rhs, &sense, NULL, cname)) 
        printError(" errato CPXnewrows [z1]");
    //Modifica dei coefficienti delle singole variabili del nuovo vincolo
    for (int j = 0; j < inst->nNodes; j++)
    {
        if (i != j && CPXchgcoef(env, lp, lastRow, xPos(i, j, inst), 1.0))
            printError(" errato CPXchgcoef [x1]");
    }
}
\end{lstlisting}

\subsection*{Modello Matematico: Linguaggio C\#}
\label{sec:ModelloCSS}

Per una corretta comprensione di questa sezione è consigliato aver letto la precedente \hyperref[sec:ModelloCS]{Modello Matematico: Linguaggio C} in quanto alcune nozioni ivi fornite rimangono valide e pertanto qui non più riportate.

La definizione del modello matematico per \textit{CPLEX} in linguaggio di programmazione \textbf{C\#} risulta semplificata rispetto a quanto visto per il linguaggio \textit{C} grazie alla possibilità di utilizzare tipi di oggetti creati appositamente.

Il primo passo consiste nell'istanziare un oggetto di tipo \textbf{Cplex} utilizzato per sbalire una interfaccia di comunicazione con il solver stesso.

\begin{lstlisting}
Cplex cplex = new Cplex();
\end{lstlisting}

Ogni oggetto che rappresenta una \textbf{variabile} utilizzata nel modello matematico deve avere un tipo che implementi l'interfaccia \textbf{INumVar}. Non è necessario costruire manualmente una classe di questo genere in quanto è presente il metodo \textbf{NumVar} della classe \textbf{Cplex}, il quale ha come valore di ritorno un oggetto compatibile con \textit{INumVar}:

\begin{lstlisting}
public virtual INumVar NumVar(double lb, double ub, NumVarType type, string name)
\end{lstlisting}

Dove:

\begin{itemize}
\item \textbf{lb}: rappresenta il \textit{lower bound} della variabile;
\item \textbf{ub}: rappresenta l'\textit{upper bound} della variabile;
\item \textbf{type}: questo campo determina il tipo della variabile, può assumere i seguenti valori:
\begin{itemize}
\item \textbf{NumVarType.Int}: nel caso di variabile intera;
\item \textbf{NumVarType.Int}: nel caso di variabile binaria;
\item \textbf{NumVarType.Float}: nel caso di variabile continua;
\end{itemize}
\item \textbf{name}: nome identificativo assegnato alla nuova variabile;
\end{itemize}

In modo del tutto analogo, l'interfaccia \textbf{ILinearNumExpr} deve essere implementata dal tipo degli oggetti che si desidera utilizzare per definire una \textbf{espressione lineare}. Anche in questo caso viene messo a disposizione il metodo \textbf{LinearNumExpr} della classe \textbf{Cplex}:

\begin{lstlisting}
ILinearNumExpr expr = cplex.LinearNumExpr();
\end{lstlisting}

La variabile \textbf{expr} deve essere quindi modificata volta per volta in modo tale da costruire tutte le espressioni lineari necessarie. Per aggiungervi un nuovo elemento si utilizza il metodo non statico \textbf{AddTerm}:

\begin{lstlisting}
void AddTerm(INumVar var,double coef)
\end{lstlisting}

Dove:

\begin{itemize}
\item \textbf{var}: la variabile che si desidera aggiungere alla espressione;
\item  \textbf{coeff}: il coefficiente della variabile che si desidera aggiungere alla espressione;
\end{itemize}

Partendo con la creazione della funzione obiettivo, il codice realizzato è il seguente:

\begin{lstlisting}
//Il vettore contenente tutte le variabili viene creato
INumVar[] x = new INumVar[(instance.nNodes - 1) * instance.nNodes / 2];

//La variabili su cui viene definita una espressione lineare
ILinearNumExpr expr = cplex.LinearNumExpr();

//Solamente i lati delimitati dai noi (i,j) per cui i < j sono considerati
for (int i = 0; i < instance.nNodes; i++)
{
    for (int j = i + 1; j < instance.nNodes; j++)
    {
        /*
        * La funzione xPos determina il corretto indice
        * da utilizzare per identificare il lato x(i+1, j+1)
        * all'interno del vettore di variabili x
        */
        int position = xPos(i, j, instance.nNodes);
        
        //Creazione della variabile
        x[position] = cplex.NumVar(0, 1, NumVarType.Bool, "x(" + (i + 1) + "," + (j + 1) + ")");
        /*
        * Aggiunta del termine relativo al lato x(i+1, j+1)
        * ad expr, per la funzione obiettivo il coefficiente
        * è equivalente al costo del lato in questione
        */
        expr.AddTerm(x[position], Point.Distance(instance.coord[i], instance.coord[j], instance.edgeType));
    }
}

\end{lstlisting}

Specificare ad un oggetto di tipo \textit{Cplex} che una espressione, correttamente definita, rappresenta la funzione obiettivo del modello matematico viene eseguito attraverso i metodi \textbf{AddMinimize} oppure \textbf{AddMaximize}: rispettivamente da utilizzare nel caso in cui l'espressione indicata sia da minimizzare o massimizzare.

\begin{lstlisting}
cplex.AddMinimize(expr);
\end{lstlisting}

L'operazione successiva di creazione dell'espressione di un vincolo, eccetto per quelli di \textit{subtour elimination}, avviene nel seguente modo:

\begin{lstlisting}
expr = cplex.LinearNumExpr();

/*
* Ogni nodo del grafo determina un proprio vincolo
* che coinvolge tutti i lati in esso incidenti
*/
for (int i = 0; i < instance.nNodes; i++)
{
    for (int j = 0; j < instance.nNodes; j++)
    {
        /*
        * L'espressione è la semplice somma di tutti i lati
        * incidenti nel nodo quindi tutti quelli del tipo
        * x(i+1,..) oppure x(..,i+1) validi
        */
        if (i != j)
            expr.AddTerm(x[xPos(i, j, instance.nNodes)], 1);
    }
    
    //Aggiunta del vincolo al modello
    cplex.AddEq(expr, 2, "degree(" + (i + 1) + ")");
}
\end{lstlisting}

L'aggiunta di un vincolo avviene tramite l'utilizzo di uno tra i seguenti metodi \textbf{AddEq}, \textbf{AddLe}, \textbf{AddGe} che rispettivamente identificano una una equazione, una disequazione avente segno $\leq$, una disequazione avente segno $\geq$.

Si riportano i dettagli del solo \textbf{AddEq} in quanto l'unico di interesse al momento:

\begin{lstlisting}
public virtual IRange AddEq(INumExpr e, double v, string name)
\end{lstlisting}

Dove:

\begin{itemize}
    \item \textbf{e}: espressione contenente le variabili del vincolo;
    \item \textbf{v}: termine noto del vincolo;
    \item \textbf{name}: nome identificativo del vincolo;
\end{itemize}

Il codice proposto, dopo minime modifiche dovute alla presenza della variabile \textbf{nEdges}, viene inserito all'interno di un apposito metodo \textbf{BuilModel} appartenente alla classe \textbf{Utility}, che oltre alla definizione del modello matematico restituisce il riferimento al vettore di variabili:

\begin{lstlisting}
public static INumVar[] BuildModel(CPLEX cplex, Instance instance, int nEdges)
\end{lstlisting}

Dove:

\begin{itemize}
\item \textbf{cplex}: oggetto sul quale si definisce il modello matematico;
\item \textbf{inst}: riferimento ad un oggetto della classe \textbf{Instance} contente tutti dati che descrivono l'istanza del Problema del Commesso Viaggiatore fornita in ingresso dall'utente;
\item \textbf{nEdges}: questo parametro viene utilizzato solo per creare \textit{particolari} modelli matematici in alcuni metodi risolutivi specifici.
\end{itemize}

Concludiamo questa sezione mostrando il metodo che avvia la risoluzione del modello matematico costruito e come sia possibile reperire il costo ed i valori della variabili relativi alla miglior soluzione trovata da \textbf{CPLEX} entro un tempo limite (se specificato).

\begin{lstlisting}
cplex.Solve();
\end{lstlisting}

Il metodo \textbf{Solve} permette di avviare la risoluzione del modello matematico definito all'interno della variabile su cui è invocato. \textit{CPLEX}, se non specificato altrimenti, fornisce continuamente aggiornamenti riguardanti lo stato del processo attraverso stampe nello \textit{standard output} del progetto\footnote{Di default la classica \textit{console} del \textit{C\#}.}.

\begin{lstlisting}
cplex.ObjValue;
\end{lstlisting}

Terminata l'esecuzione del metodo \textbf{Solve()} il miglior valore trovato per quanto riguarda la funzione obiettivo specificata si trova all'interno della variabile \textbf{double ObjValue}. Infine il valore assunto da una singola variabile nella soluzione \textit{incumbent} è recuperabile attraverso:

\begin{lstlisting}
public virtual double GetValue(INumVar var)
\end{lstlisting}

Dove:

\begin{itemize}
\item \textbf{var}: riferimento alla variabile di cui si desidera conoscere il valore nella soluzione \textit{incumbent};
\end{itemize}

Come alternativa è possibile utilizzare il metodo:

\begin{lstlisting}
public virtual double GetValues(INumVar[] var)
\end{lstlisting}

Dove:

\begin{itemize}
    \item \textbf{var}: riferimento al vettore di variabili di cui si desidera conoscere i valori nella soluzione \textit{incumbent};
\end{itemize}

Seppur multiple esecuzioni di \textbf{GetValue} possano produrre il medesimo risultato della singola invocazione di \textbf{GetValues}, il primo approccio è estremamente sconsigliato in quanto multipli accessi alla memoria di \textit{Cplex} comportano tempi di esecuzione maggiori non trascurabili.

\newpage

\section*{SEZIONE SEI}
\label{sec:SezioneSeiS}

La risoluzione del modello matematico per come è stato costruito nella precedente sezione \hyperref[sec:SezioneCinqueS]{Sezione Cinque} come noto, essendo privo dei vincoli di \textit{subtour elimination}, non può fornire risultati accettabili indipendentemente da quale tipologia di algoritmi venga applicata.

All'interno di questa sezione viene quindi introddoto il primo metodo di risoluzione \textbf{esatto}, denominato \textbf{LOOP}, che fornisce una tecnica molto semplice per aggiungere ad ogni sua iterazione una serie di vincoli necessari per ottenere la soluzione \textbf{ottima} del \textbf{TSP} preso in analisi.

Successivamente sono proposte delle varianti nel complesso sempre \textbf{esatte} di questo algoritmo che, sfruttando una \textbf{iniziale} fase \textbf{euristica}, mirano a ridurre i tempi di esecuzione globali.

\subsection*{METODO LOOP}
\label{sec:MetodoLoopS}

Il primo algoritmo risolutivo sperimentato e quindi introdotto in questo testo prende il nome di \textbf{LOOP}, più nello specifico si tratta della sua variante \textbf{priva} di fasi \textit{euristiche}.

L'idea alla base è molto semplice: inizialmente il modello matematico creato, anche se privo di alcun vincolo di \textit{subtour elimination}, viene risolto. La soluzione \textit{ottima} così ottenuta viene processata individuando eventuali componenti connesse al suo interno. Una loro presenza in quantità superiore ad \textbf{uno} implica inevitabilmente un grafo \textbf{non connesso} e di conseguenza la soluzione attuale deve essere dichiarata \textbf{non valida}. A questo punto si deve procedere con l'individuazione dei vincoli che avrebbero impedito al \textit{solver} di proporre tale soluzione e conseguente loro aggiunta al modello motematico.

Eseguito l'aggiornamento, il procedimento viene ripetuto dal principio partendo con una nuova risoluzione. Proprio da questo ripetersi degli stessi identici passi sistematicamente, creando così un \textbf{loop}, dà il nome alla tecnica.

Naturalmente, trattandosi di una algoritmo, il ciclo \textit{loop} deve avere una o più condizioni di arresto. In questo caso specifico si tratta dell'individuazione di una sola componente connessa e pertanto l'ultima soluzione trovata è \textbf{valida} e anche l'\textbf{ottimo globale} desiderato.

Il fatto di partire con un modello matematico privo di vincoli aggiuntivi, combinato ad una loro aggiunta mirata e molto graduale, permette alle prime iterazione del metodo \textbf{LOOP} di essere estremamente veloci garantendo nel complesso dei tempi di esecuzione molto validi\footnote{Si noti che ogni vincolo, una volta aggiunto al modello matematico, non deve essere rimosso durante le fasi successive dell'algoritmo. In questo modo i tempi di esecuzione tendono a crescere con il progredire delle iterazioni eseguite.}.

Per poter implementare il metodo Loop risulta quindi evidente la necessità di sviluppare un'opportuna funzione in grado di individuare la presenza di subtour all'interno di una generica soluzione proposta e di generarne gli opportuni vincoli per eliminarli.\\
In letteratura esistono molteplici modi per eseguire tali operazioni, quella da noi adottata si rifà all'algoritmo di \text{Kruskal} per trovare un albero a costo minimo in un grafo connesso con lati non orientati\footnote{Nello specifico la parte di nostro interesse è quella che impedisce la formazione di più componenti connesse}.\\
La tecnica da noi adottata è stata quella di creare due metodi chiamati \textbf{InitCC} e \textbf{UpdateCC}: il primo serve solamente come inizializzazione per le strutture dati utilizzate dal secondo il quale, se invocato una volta per ogni lato appartenente alla soluzione attuale ne trova tutte le componenti connesse indicando anche quali lati sono a loro appartenenti. I dettagli riguardo le loro implementazioni sono visibili nella appendice di questo testo, per ora specifichiamo solamente che al termine dell'utilizzo del metodo \textbf{UpdateCC} i seguenti oggetti:

\begin{lstlisting}

        List<ILinearNumExpr> rcExpr = new List<ILinearNumExpr>();
        List<int> bufferCoeffRC bufferCoeffRC = new List<int>();

\end{lstlisting}

risultano essere costruiti, in particolare \textbf{rcExpr} contiene le espressioni dei subtour elimination mentre invece \textbf{bufferCoeffRC} contiene il numero di lati appartenenti ad ogni subtour e quindi il termine noto delle precedenti espressioni\footnote{Il codice assume che l'espressione di indice $i$ presente all'interno di \textbf{rcExpr} abbia il proprio termine noto nella posizione di indice $i$ dentro \textbf{bufferCoeffRC}}.\\
Se all'interno di \textbf{rcExpr} è presente una espressione sola significa che la soluzione attuale è valida e quindi ottima per il problema, al contrario si deve procedere all'inserimento dei vincoli con un semplice ciclo for:

\begin{lstlisting}
                
for (int i = 0; i < rcExpr.Count; i++)
  cplex.AddLe(rcExpr[i], bufferCoeffRC[i] - 1);
              
\end{lstlisting}

\subsection*{METODI UpdateCC e InitCC}
\label{sec:UCCICCS}

Come già specificato nella sezione riguardo il metodo \textbf{LOOP} questi due metodi di supporto appartenenti alla classe \textbf{Utility} hanno il compito di individuare tutte le componenti connesse (che da ora in avanti abbrevieremo con \textbf{cc}) di una generica soluzione proposta.\\
Prima di passare alla implementazione vera e propria introduciamoli ad alto livello: inizialmente si vuole assumere l'esistenza di $n$ \textbf{cc} distinte, ognuna di esse contenente un nodo della soluzione. Questo è il compito della dalla funzione \textbf{InitCC}.\\
Successivamente per ogni lato della soluzione si vuole analizzare a quali \textbf{cc} sono assegnati i due nodi che lo caratterizzano. Se queste sono differenti vanno unificate in modo tale che tutti i nodi appartenenti, ad esempio
, alla seconda ora appartengano tutti alla prima. Nel caso in cui invece le due \textbf{cc} coincidano significa che abbiamo trovato un subtour e il relativo vincolo di eliminazione deve essere definito. Tutte questo è invece compito del metodo \textbf{UpdateCC}.

Iniziamo quindi l'analisi del codice necessario. Per prima cosa si necessita di un vettore di interi che contenga all'indice $i-esimo$ l'identificativo della \textbf{cc} alla quale appartiene il nodo $i$\footnote{Per semplicità si è deciso di identificare ogni \textbf{cc} con un valore intero univoco}. L'inizializzazione di questo vettore viene fornita da \textbf{InitCC}:

\begin{lstlisting}

public static void InitCC(int[] cc)
{
    for (int i = 0; i < cc.Length; i++)
    {
        cc[i] = i;
    }
}

\end{lstlisting}

Passiamo ora al metodo \textbf{UpdateCC} che presenta la seguente firma:

\begin{lstlisting}

public static void UpdateCC(CPLEX cplex, INumVar[] z, List<ILinearNumExpr> rcExpr, List<int> bufferCoeffRC, int[] relatedComponents, int i, int j)

\end{lstlisting}

dove:

\begin{itemize}
    \item \textbf{cplex}: oggetto contenente il modello matematico corrente, eventualmente necessario per la creazione del vincolo di subtour elimination;
    \item \textbf{z}: vettore contenente le variabili del modello, eventualmente necessario per la creazione del vincolo di subtour elimination;
    \item \textbf{rcExpr}: Lista all' interno della quale vengono memorizzate le espressioni contenenti le variabili del vincolo di subtour;
    \item \textbf{bufferCoeffRC}: Lista contenente i termini noti dei vincoli di subtour;
    \item \textbf{relatedComponents}: vettore che definisce per ogni nodo del grafo la relativa componente connessa;
    \item \textbf{i}: Nodo che con il parametro j forma il lato (i,j);
    \item \textbf{j}: Nodo che con il parametro i forma il lato (i,j).
\end{itemize}

La funzione UpdateCC viene invocata dal metodo Loop $n$ volte, alla k-esima invocazione riceve in ingresso il k-esimo lato appartenente alla soluzione ottima del modello corrente. Per verificare se il lato ricevuto genera un subtour nel grafo G=(V,T*), dove T* contiene i precedenti $k-1$ lati controllati, si verifica se i vertici del lato appartengono alla medesima componente connessa. Nel caso in cui i due vertici non appartengono alla medesima componente connessa, è necessario aggiornare le componenti connesse dei vertici per l' invocazione successiva del metodo, viceversa si è  individuato un subtour caratterizzato dai nodi aventi come componente connessa la medesima dei nodi i e j.

A livello implementativo si è utilizzato un array di interi chiamato relatedComponents, di dimensione pari al numero di vertici del grafo, come struttura dati necessaria per fotografare le componenti connesse del grafo G=(V,T*); relatedComponents contiene all' indice j la componente connessa del nodo j. La funzione InitCC, invocata ad ogni iterazione del metodo Loop, ha il compito di inizializzare relatedComponents associando ad ogni nodo una componente connessa diversa: in particolare si è  scelto di associare al nodo j la componente connessa j. Passiamo ora ad analizzare come è  stato nella pratica implementato il metodo UpdateCC, la sua intestazione è la seguente:

\begin{lstlisting}

public static void UpdateCC(CPLEX cplex, INumVar[] z, List<ILinearNumExpr> rcExpr, List<int> bufferCoeffRC,int[] relatedComponents, int i, int j)

\end{lstlisting}

dove:
\begin{itemize}
\item cplex: oggetto contenente il modello matematico corrente;
\item z: vettore contenente le variabili del modello;
\item rcExpr: Lista all' interno della quale vengono memorizzate le espressioni contenenti le variabili del vincolo di subtour;
\item bufferCoeffRC: Lista contenente i termini noti dei vincoli di subtour;
\item relatedComponents: vettore che definisce per ongi nodo del grafo la relativa componente connessa;
\item i: Nodo che con il parametro j forma il lato [i,j];
\item j: Nodo che con il parametro i forma il lato [i,j].
\end{itemize}

Il caso in cui non si crei un subtour è gestito molto semplicemente in questo modo:

\begin{lstlisting}

if (relatedComponents[i] != relatedComponents[j])
{
    for (int k = 0; k < relatedComponents.Length; k++)
    {
        if ((k != j) && (relatedComponents[k] == relatedComponents[j]))
        {
            //Same as Kruskal
            relatedComponents[k] = relatedComponents[i];
        }
    }
    //Finally also the vallue relative to the Point i are updated
    relatedComponents[j] = relatedComponents[i];
}

\end{lstlisting}

Dove per convenzione si è deciso di inglobare la \textbf{cc} del nodo $j$ in quella del nodo $i$.\\
Il secondo caso è invece gestito nel seguente modo:

\begin{lstlisting}

else
{
    ILinearNumExpr expr = cplex.LinearNumExpr();

    //cnt stores the # of nodes of the current related components
    int cnt = 0;

    for (int h = 0; h < relatedComponents.Length; h++)
    {
        //Only nodes of the current related components are considered
        if (relatedComponents[h] == relatedComponents[i])
        {
            //Each link involving the node with index h is analized
            for (int k = h + 1; k < relatedComponents.Length; k++)
            {
                //Testing if the link is valid
                if (relatedComponents[k] == relatedComponents[i])
                {
                    //Adding the link to the expression with coefficient 1
                    expr.AddTerm(z[zPos(h, k, relatedComponents.Length)], 1);
                }
            }
            cnt++;
        }
    }
    //Adding the objects to the buffers
    rcExpr.Add(expr);
    bufferCoeffRC.Add(cnt);
}

\end{lstlisting}

Ripetere il metodo \textbf{UpdateCC} una ed una sola volta per ogni lato appartenente alla soluzione corrente ci assicura che le due liste \textbf{rcExpr} e \textbf{bufferCoeffRC} contengano tutti i dati per implementare i subtour elimination desiderati.

\subsection*{METODO LOOP CON PRIMA FASE EURISTICA}
\label{sec:LoopEuristicoS}

Il metodo Loop, indipendente dalla implementazione che si decide di utilizzare, vuole essere un algoritmo esatto. In altre parole è necessario assicurarsi che il risultato finale da esso prodotto sia \textbf{sempre} il migliore possibile. Come vedremo nei paragrafi successivi, sono stati pensati molteplici algoritmi, detti euristici, che al contrario cercano solamente di avvicinarsi al risultato ottimo limitando al contempo i loro tempi di esecuzione.
\'E infatti quest'ultimo fattore a risultare cruciale per molti problemi di programmazione lineare, a maggior ragione per quelli che, come il \textit{commesso viaggiatore}, vogliono studiare situazioni \textbf{np-difficili}\footnote{In letteratura è noto infatti che i problemi appartenenti a quest'ultima categoria sono caratterizzati da tempi di risoluzione, al caso peggiore, esponenziali rispetto al numero di variabili che li caratterizzano. Nel caso in cui siano definiti nel tipico linguaggio della programmazione lineare, la caratteristica appena esposta è riscontrabile da un numero di vincoli anch'esso esponenziale.}.
Per quanto appena esposto si è pensato di progettare una variante del metodo \textbf{Loop} caratterizzata da una fase iniziale \textbf{euristica} i cui risultati vengono poi sfruttati da una seconda fase finale \textbf{esatta}. Durante la sua progettazione ci si accorge fin da subito che, come per ogni algoritmo euristico, non esistono specifici paletti che se fissati assicurano al \textbf{100\%} il raggiungimento dei propri obiettivi. Nel nostro caso ciò che desideriamo è chiaramente una fase \textit{euristica} più veloce di quella \textit{esatta} ma che produca anche risultati utili a quest'ultima. Abbiamo quindi deciso che, all'interno della nostra applicazione, sia l'utente stesso a poter settare alcuni parametri che rendono le due fasi più o meno differenti tra loro. In questo modo, basandosi sulle proprie esperienze e test, è possibile ottenere i risultati migliori per qualsiasi istanza del problema che si desidera risolvere.
Entriamo ora nel dettaglio delle due fasi chiarendo fin da subito che tutte e due devono sempre fornire soluzioni \textbf{valide} per il problema che andiamo a risolvere. In realtà ciò che è interessante discutere riguarda quasi solamente la fase \textit{euristica} in quanto quella \textit{esatta} è in tutto e per tutto il classico metodo \textbf{Loop} già esposto in precedenza\footnote{Ciò che varia è solamente che il modello matematico iniziale dato in pasto alla fase \textit{esatta} presenta già dei vincoli di \textbf{subtour elimination} individuati dalla fase \textit{euristica} dove però il modello matematico di partenza presenta già alcuni vincoli di \textit{subtour elimination}.}.
Esistono innumerevoli modi per rendere euristico il metodo \textbf{Loop}, possiamo suddividerli in tre grandi categorie: della prima fanno parte le tecniche che rendono la risoluzione stessa da parte di \textbf{CPLEX} euristica, nella seconda ricadono i metodi che introducono nuovi vincoli al modello matematico ed infine la terza categoria è una semplice combinazione delle due precedenti. All'interno del nostro progetto sono state sviluppate tre varianti della fase \textit{euristica}, una per ogni categoria appena esposta:

\begin{itemize}
    \item \textbf{prima categoria}: durante la risoluzione del problema attraverso la tecnica del \textbf{Branch\&Cut}, CPLEX, oltre al banale calcolo della soluzione ottima per ogni nodo dell'albero decisionale, sfrutta internamente algoritmi euristici per agevolare il processo. Durante quest'ultimo si hanno quindi a disposizione due parametri, il primo è il costo della soluzione euristica migliore (\textbf{$C_{eu}$}), mentre il secondo è il classico costo \textbf{lower bound}\footnote{Nel nostro caso sarà il costo della migliore soluzione intera trovato.} (\textbf{$L_{b}$}). \'E importate far notare che questi valori mutano mano a mano che si procede alla costruzione dell'albero decisione, in particolare \textbf{$C_{eu}$} cresce mentre \textbf{$L_{b}$} scende fino a che, teoricamente, non coincidano.
    La distanza \textbf{relativa} tra i due valori viene costantemente monitorata da CPLEX e, se questa scende sotto la soglia minima del suo parametro interno \textbf{EpGap}, il processo di risoluzione viene considerato terminato e la miglior soluzione valida trovata viene restituita. Maggiori dettagli riguarda \textit{EpGap} sono forniti nel paragrafo ad esso dedicato LINK!!!!!!!, per ora ci basta dire che se di default è settato ad un valore vicino allo $0$, la sua variazione è proprio ciò che viene utilizzato nel nostro programma all'interno della fase \textit{euristica} del metodo \textit{Loop}.
    Risulta estremamente intuitivo e facilmente verificabile che per una specifica istanza non è possibile a priori determinare quanto velocemente verrà raggiunto un certo gap durante la fase di \textit{Branch\&Cut}. E\' quindi consigliabile eseguire inizialmente il normale metodo \textbf{Loop}, analizzare l'output fornito da CPLEX durante la risoluzione ed osservare per quale gap soluzioni successive dell'albero decisionale non producono più miglioramenti sostanziali e\\o richiedono tempi di esecuzione eccessivi. Per quanto detto il valore di \textbf{EpGap} viene lasciato a discrezione dell'utente;
    \item \textbf{seconda categoria}: come precedentemente indicato ad inizio paragrafo, i tempi di risoluzione risultano esponenziali rispetto al numero di variabili che caratterizzano il problema in questione. Limitarne il numero ha pertanto un impatto notevole nella risoluzione del modello matematico ed a tale scopo è possibile decidere di settare a priori il valore assunto da alcune variabili. Tanto migliore risulta essere la previsione così introdotta, migliori saranno i tempi di risoluzione ottenibili sia per la fase \textit{euristica} che quella \textit{esatta} del metodo \textit{Loop}. La scelta da noi effettuata è di lasciare selezionabili per ogni nodo gli $m$ collegamenti a costo minore che lo vedono come un loro vertice. Il numero $m$ è lasciato selezionabile dall'utente e da risultati sperimentali non è consigliabile che si discosti troppo dal valore $10$;
    \item \textbf{terza categoria}: vengono semplicemente combinate le due precedenti;
\end{itemize}

Nei due paragrafi successivi vengono mostrati alcuni dettagli, tra cui quelli realizzativi, per l'utilizzo delle varianti euristiche esposte del metodo \textbf{Loop}.

\subsection*{EpGap}

EpGap risulta essere un parametro interno di CPLEX, il cui valore di default è $1e^{-06}$. Indicando con \textbf{bestnode} il miglior valore della funzione obiettivo calcolata ad un nodo dell'albero decisionale attraverso metodi euristici propri di CPLEX, e con \textbf{bestInteger} il costo della miglior soluzione intera trovata fino a quel momento, ossia il costo dell'incumbent, qualora la seguente quantità:

$$|bestNode - bestInteger|/(1e-10 + |bestInteger|)$$

risulti inferiore ad \textit{EpGap} il solver si arresta.

Il settaggio del parametro in questione è molto semplice, è sufficiente invocare il metodo \textbf{SetParam} sull'oggetto della classe \textbf{CPLEX} che si sta utilizzando come riportato di seguito:

\begin{lstlisting}

cplex.SetParam(CPLEX.DoubleParam.EpGap, newValue);

\end{lstlisting}

Essendo \textit{EpGap} un valore di distanza \textbf{relativo} e non \textbf{assoluto}, la variabile \textbf{newValue} deve essere compresa tra i valori $0$\footnote{La risoluzione termina fornendo sempre la soluzione ottima.} e $1$\footnote{La risoluzione termina alla prima soluzione ammissibile individuata.}.

Completato il settaggio di \textit{EpGap} è sufficiente procedere con il normale metodo risolutivo \textbf{Loop}. Al termine di quest'ultimo otteniamo una soluzione euristica ma soprattutto un insieme di vincoli di \textit{subtour elimination}. Entriamo quindi nella fase \textit{esatta} dell'algoritmo riportando al valore di default \textit{EpGap} e ripetiamo il metodo risolutivo \textbf{Loop} sul modello matematico originale ampliato dai nuovi vincoli appena citati.

\begin{figure}[htbp]
    \centering
    \includegraphics[width=\textwidth]{Immagini/"InterfacciaGrafica".jpg}
    \caption{Output}
\end{figure}

\subsection*{TITOLO DA CAMBIARE}
\label{sec:TitoloS}

Nel paragrafo LINK!!!!!!!!!!!!!!!, è stato presentato il metodo \textbf{BuildModel} della classe \textbf{Utility} il cui compito è di creare il modello matematico, privo dei vincoli di subtour elimination, risolubile da CPLEX. Durante la presentazione della sua firma, era stato lasciato in sospeso l'utilizzo del parametro \textit{nEdges} in quanto allora del tutto prematura.

Come specificato nel paragrafo LINK!!!!!!!!!! una variante euristica del metodo \textbf{Loop}, prevede di utilizzare solamente un numero massimo $m$ di lati incidenti in ogni nodo del problema in questione del commesso viaggiatore. Tale parametro è inoltre richiesto all'utente per i motivi già specificati e viene comunicato al metodo \textbf{BuildModel} proprio grazie alla variabile di ingresso \textbf{nEdges}\footnote{Come dettaglio implementativo specifichiamo inoltre che per convenzione si è deciso di settare $nEdges = -1$ nel caso in cui si voglia coistruire un modello con tutti i lati possibili abilitati}.

A livello implementativo, per rendere inutilizzabili certi collegamenti, è sufficiente non inserirli nel modello matematico oppure farlo ma fissandone sia il \textbf{lower} che l'\textbf{upper bound} a 0. Quest'ultima opzione è quella più comoda da utilizzare in quanto nella successiva fase \textit{esatta} dell'algoritmo il modello matematico deve comunque aver disponibili al suo interno tutte le variabili.

L'individuazione algoritmica di quali collegamenti debbano essere abilitati risulta l'operazione più complessa da un punto di vista computazionale. Come vedremo nel corso di questa tesi, diversi algoritmi necessitano di conoscere per ogni nodo l'ordinamento completo dei lati ad esso incidenti basato sul loro costo e per tanto si è definita una unica funzione \textbf{BuildSL} adibita a tale scopo\footnote{In realtà il costo computazionale per trovare gli $m$ lati meno costosi incidenti in un nodo, ripetuto per tutti gli $n$ nodi disponibili, è il medesimo del metodo \textit{BuildSL} e quindi l'utilizzo di quest'ultimo non porta alcuno svantaggio.}.

I dettagli riguardanti al metodo \textbf{BuildSL} sono riportati al seguente paragrafo della appendice LINK!!!!!!!!!!!!!!!, in questo contesto ci basta indicare che essa restituisce una lista di vettori di interi, chiamata \textbf{listArray}, in cui $i-esimo$ elemento contiene, in ordine crescente, la sequenza dei restanti $n-1$ vertici basandosi sulla loro distanza rispetto al nodo $i$. In altre parole, una volta invocato \textit{BuildSL}, i lati il cui \textbf{upper bound} deve essere posto pari ad $1$ sono individuabili dagli estremi $[0,(listArray[0])[0]], [0,(listArray[0])[1]], ..., [0,(listArray[0])[m]], ..., [i,(listArray[i])[0]], ..., [i,(listArray[i])[m]], ..., [n-1,(listArray[n-1])[0]], ..., [n-1,(listArray[n-1])[m]]$.

Di seguito è riportato per completezza il codice all'interno della funzione \textbf{BuildModel} che nel caso di $nEdges \geqslant 1$ sfrutta le informazioni presenti in \textit{listArray} per il settaggio delle variabili del modello matematico:

\begin{lstlisting}

if (nEdges > 0)
{
    List<int>[] listArray = BuildSLComplete(instance);

    for (int i = 0; i < instance.NNodes; i++)
    {
        for (int j = 0; j < nEdges; j++)
        {
            int position = xPos(i, listArray[i][j], instance.NNodes);

            x[position].UB = 1;
        }
    }
}

\end{lstlisting}

Dove chiaramente in precedenza era stato necessario inizializzare tutte le variabili contenute in \textbf{x} come:

\begin{lstlisting}

ILinearNumExpr expr = cplex.LinearNumExpr();


//Populating objective function
for (int i = 0; i < instance.NNodes; i++)
{
    for (int j = i + 1; j < instance.NNodes; j++)
    {
        ////xPos return the correct position where to store the variable corresponding to the actual link (i,i)
        int position = xPos(i, j, instance.NNodes);

        if (nEdges > 0)
            x[position] = cplex.NumVar(0, 0, NumVarType.Bool, "x(" + (i + 1) + "," + (j + 1) + ")");
        else
            ...

        expr.AddTerm(x[position], Point.Distance(instance.Coord[i], instance.Coord[j], instance.EdgeType));
        }
    }
}

\end{lstlisting}

Terminata la costruzione del modello matematico si procede alla sua risoluzione attraverso il classico metodo \textbf{Loop}. Al suo termine, otteniamo una soluzione euristica e soprattutto una lista di vincoli di \textit{subtour elimination}. Per passare alla risoluzione esatta del problema in analisi, manteniamo questi ultimi e settiamo l'\textbf{upper bound} di ogni variabile ad $1$. Questa ultima operazione è ottenibile invocando il metdo \textbf{ResetVariables} appartenente alla classe Utility:


\begin{lstlisting}

public static void ResetVariables(INumVar[] x)
{
    for (int i = 0; i < x.Length; i++)
        x[i].UB = 1;
}

\end{lstlisting}

Dove chiaramente \textbf{x} è il vettore contenente i riferimenti alle variabili utilizzate dal modello matematico da noi definito.

\section*{SEZIONE SETTE}
\label{sec:SezioneSetteS}

In questa sezione esponiamo una tecnica alternativa per l'inserimento delle espressioni di subtour elimination all'interno di un sistema. Ciò che varia rispetto al metodo \textbf{Loop} presentato in precedenza è il \textbf{momento} in cui tali espressioni vengono definite.\\
L'idea è di sfruttare il fatto che CPLEX come metodo di risoluzione per i problemi di \textbf{PLI} utilizza la tecnica del \textbf{Branch\&Cut}\footnote{Da ora in avanti sarà abbreviato con la sigla \textbf{B\&C}}. Viene inoltre offerta la possibilità di conoscere la soluzione trovata, sia essa frazionaria o intera, per ogni nodo dell'albero decisione ma soprattuto la possibilità di ampliare il modello matematico come meglio crediamo.\\
Quello che vogliamo fare risulta a questo punto molto chiaro: se alla analisi della soluzione di un nodo sono presenti subtour il modello matematico deve essere modificato per eliminarli.\\
A livello pratico CPLEX permette l'implementazione distinta di callback che vengono eseguite nel momento in cui viene trovata una soluzione intera oppure frazionaria\footnote{In informatica una callback è una funzione definita dall'utente che viene eseguita in automatico dal sistema ogniqualvolta scatta un particolare evento.}: nel primo caso è necessario implementare una \textbf{"lazy constraint callback"} mentre nel secondo caso una \textbf{"user cut callback"}. Da notare che in realtà solo soluzioni valide per i criteri di fathoming possono far scattare una callback, ciò non avviene ad esempio se il valore della soluzione di un nodo risulta maggiore rispetto a quello dell'incumbent\footnote{Per incumbent si intende il valore migliore trovato fino a questo momento relativo ad un soluzione accettabile.}.\\
In generale i tagli possono essere definiti \textbf{locali} o \textbf{globali}: mentre i primi hanno validità esclusiva all'interno del sottoalbero avente come radice il nodo per la quale sono stati generati, i secondi hanno validità per tutti i nodi dell'albero decisionale e vengono memorizzati in una struttura globale detta \textbf{pool di tagli}. CPLEX inoltre fornisce la possibilità di definire un taglio \textbf{purgeable} o meno: nel primo caso significa che può essere rimosso in un secondo momento poiché ritenuto inefficace. Durante il proseguo della tesi i tagli saranno da considerarsi sempre globali e non purgeable.

\begin{figure}[htbp]
    \centering
    \includegraphics[width=\textwidth]{Immagini/"Callback".jpg}
    \caption{Soluzione frazionaria}
\end{figure}

Prima di procedere con l'esposizione dei dattagli riguardanti l'implementare di tali procedure, è possibile effettuare le seguenti considerazioni:\\

\begin{itemize}
    \item La soluzione ottima fornita da CPLEX risulta per costruzione priva di subtour: non è quindi più necessario, al contrario del metodo \textbf{Loop}, lanciare molteplici risoluzioni. A livello pratico è sufficiente invocare solo una volta il metodo \textbf{CPLEX.Solve()}.
    \item Maggiore è il numero di vincoli che andiamo ad inserire, maggiore diventa il tempo di risoluzione per i vari nodi successivi dell'albero decisionale.
    \item Il numero di nodi che forniscono soluzioni frazionarie risulta di molto superiore rispetto a quelli con soluzione intera. Tenendo conto di quanto detto al punto precedente, non è quindi saggio andare ad analizzare tutte le soluzioni frazionarie ma solo una loro minima percentuale. Senza questo accorgimento si andrebbe inevitabilmente ad inserire innumerevoli vincoli superflui per l'ottenenimento della soluzione ottima con conseguenti tempi di risoluzione eccessivamente elevati.
    \item I moderni processori hanno a disposizione molteplici core sia reali che virtuali e quindi sfruttare tecniche di multi-threading. In particolare CPLEX permette di settare il numero di thread utilizzabili, in modo tale che ognuno di essi si occupi dalla risoluzione di un nodo dell'albero decisionale: in questo modo, in linea teorica, si drovrebbe ottenere un boost delle prestazioni con conseguente riduzione dei tempi di calcolo. D'altro canto, come per qualsiasi applicazione informatica, l'utilizzo del multi-threading risulta rischioso in quanto l'accesso contemporaneo ai medesimi dati può portare ad una loro inconsistenza. Nel nostro caso può capitare che più callback eseguite contemporaneamente vadano a modificare variabili condivise andando così incontro ad eccezioni o anomalie tali da non garantire più la correttezza della soluzione prodotta da CPLEX.\\
    Per evitare queste problematiche, i progettisti di CPLEX hanno preferito settare il numero di thread al valore \textbf{1} dopo l' installazione di una callback. \'E quindi nostro compito modificare tale parametro così da renderlo pari al numero di processori virtuali a nostra disposizione e di conseguenza assicurarci che le callback risultino \textbf{thread-safe}. Maggiori dettagli sono riportati nei successivi paragrafi.
\end{itemize}

\subsection*{LAZYCONSTRAINT CALLBACK C\#}
\label{sec:LazyS}

Per poter utilizzare una \textbf{lazy constraint callback} in \textbf{C\#} CPLEX fornisce all'interno delle proprie librerie la classe astratta \textbf{LazyConstraintCallback} che a sua volta estende \textbf{ControlCallback}. \'E quindi necessario creare una propria classe che estenda quest'ultima, nel nostro caso è stato deciso di chiamarla \textbf{TSPLazyConsCallback}, in questo modo è necessario definire al suo interno il metodo \textbf{Main} che verrà invocato automaticamente dal sistema ogniqualvolta scatta la callback\footnote{Rocordiamo che tutti i metodi astratti presenti all'interno di una classe astratta devono essere obbligatoriamente definiti da tutte le classi che estendono quest'ultima}.\\
Una volta terminato questo processo l'installazione della callback viene eseguita nel seguente modo:

\begin{lstlisting}

cplex.Use(new TSPLazyConsCallback(...));

\end{lstlisting}

Dove, come al solito, cplex è l'instanza della classe \textbf{CPLEX} sulla quale definiamo il modello matematico privo dei vincoli di subtour elimination.\\
Mostriamo ora la firma del costruttore della classe \textbf{TSPLazyConsCallback} riportando una breve descrizione dei parametri di ingresso:

\begin{lstlisting}

public TSPLazyConsCallback(CPLEX cplex, INumVar[] z, Instance instance, Process process, bool BlockPrint)

\end{lstlisting}

\begin{itemize}
    \item \textbf{cplex}: necessario per l'individuazione dei vincoli di subtour elimination, contiene i dati del modello matematico utilizzato;
    \item \textbf{z}: identico al punto precedente, contiene i riferimenti alla variabili del modello matematico;
    \item \textbf{instance}: necessario nel caso in cui si desideri stampare attraverso GNUPlot le soluzioni intere che hanno fatto scattare la callback;
    \item \textbf{process}: identico al punto precedente;
    \item \textbf{BlockPrint}: è il parametro booleano che determina se procedere o meno con le stampe delle soluzioni intere (se \textbf{true} si procede con la stampa);
\end{itemize}

Come accennato nel paragrafo precedente, l'installazione di una callback setta automaticamente il numero di thread ad uno. Per modificare tale valore, ponendolo pari al numero logico di cores messi a disposizione dal processo in uso, è sufficiente eseguire la seguente riga di codice:

\begin{lstlisting}

cplex.SetParam(CPLEX.Param.Threads, cplex.GetNumCores());

\end{lstlisting}

Come sarà possibile vedere più avanti, la tecnica da noi utilizzata per l'individuazione di subtour risulta thread-safe in quanto non vengono utilizzate variabili condivise da più threads se non nella sola modalità di lettura. L'aggiunta di eventuali tagli, d'altro canto, viene gestita in modo automatico da CPLEX assicurandoci, anche in questo caso, una procedura thread-safe. Un discorso apparte deve invece essere fatto nel caso in cui la variabile \textbf{BlockPrint} descritta in precedenza sia stata posta a \textbf{true}. Come era logico aspettarsi, abbiamo verificato che spesso la procedura di stampa attraverso GNUPlot di una qualsiasi soluzione richiede un tempo di esecuzione maggiore rispetto la frequenza con cui le callback sono effettuate. Ricordando inoltre che, il metodo da noi utilizzato per comunicare a GNUPlot le coordinate cartesiane dei punti del grafo cartesiano prevede la scrittura di queste ultime in un apposito file di testo, è stato necessario individuare un modo per evitare problemi riguardanti il multi-threading: utilizzare sempre lo stesso file di testo causa infatti errori nella stampa dei grafi, in particolare la lettura delle coordinate da parte di GNUPlot risulta troppo lenta e durante questo processo più thread rischiano di modificare il file con le proprie coordinate.\\
Per evitare questo problema è stato quindi necessario stampare le coordinate prodotte dai vari nodi dell'albero decisionale in differenti files. A tal proposito la tecnica da noi scelta è stata quella di inserire nel nome di questi ultimi anche l'id numerico del nodo a loro associato che viene fornito dirretamente da CPLEX:


\begin{lstlisting}

string nodeId = GetNodeId().ToString();

...

string fileName = instance.InputFile + "_" + nodeId;

\end{lstlisting}

La funzione \textbf{GetNodeId} risulta disponibile in quanto ereditata dalla classe \textbf{ControlCallback}.\\
Dopo i dovuti chiarimenti riguardanti il multi-threading passiamo ora a descrivere nei dettagli come è stata realizzato il metodo Main della classe TSPLazyConsCallback. Prima di tutto per verificare l'eventuale presenza di subtour bisogna naturalmente accedere alla soluzione fornita per il nodo dell'albero decisionale in questione: a tal fine si possono utilizzare i metodi \textbf{GetValues} e \textbf{GetValue}, ereditati entrambi dalla classe ControlCallback, che ricevono in input rispettivamente un vettore di riferimenti per variabili del modello matematico e un singolo riferimento ad una variabile di quest'ultimo. Dopo pochi test ci si accorge immediatamente che invocare più volte il metodo \textbf{GetValue}, ad esempio dentro un ciclo for, risulta molto più oneroso in termini temporali rispetto una singola evocazione del metodo \textbf{GetValues}: si può quindi dedurre che è molto più dispendioso effettuare multiplici accessi all'interfaccia fornita da CPLEX rispetto alla quantità di dati che ad essa richiediamo.\\
Per quanto appena detto la nostra scelta è ricaduta nel metodo \textbf{GetValues} che restituisce un vettori di \textbf{double} contenente il valore dell variabili (il cui riferimento è ricevuto come ingresso) nella soluzione corrente del modello matematico. Da notare che anche in questo caso anche se ci aspettiamo tutti valori interi, in particolare pari a 0 oppure 1, è possibile che ci siano in realtà discostamenti infinitesimi pertanto quando controlliamo il valore di una variabile verifichiamo semplicemente se è maggiore o minore del valore 0,5.\\
I metodi utilizzati per l'individuazione di eventuali subtour e la eventuale stampa del grafo attraverso GNUPlot sono identici a quelli utilizzati per il metodo \textbf{Loop} pertanto non sono qui riportarti.\\
Una volta ottenute tutte le informazioni riguardanti i subtour, al contrario di quanto viene fatto nel metodo Loop non è richiesto di ampliare direttamente il modello con nuovi vincoli ma, come era già stato accennato in precedenza, deve essere popolato il pool di tagli associato al modello matematico:

\begin{lstlisting}

IRange[] cuts = new IRange[ccExprLC.Count];

//if cuts.Length is 1 the graph has only one tour then cuts aren't needed
if (cuts.Length > 1)
{
    for (int i = 0; i < cuts.Length; i++)
    {
        cuts[i] = cplex.Le(ccExprLC[i], bufferCoeffCCLC[i] - 1);
        Add(cuts[i], 1);
    }
}

\end{lstlisting}

Dove:

\begin{itemize}
    \item \textbf{cuts}: è un vettore di \textbf{IRange} che sono la struttura di dati base fornita da CPLEX per memorizzare espressioni lineari;
    \item \textbf{ccExprLC[i]}: analogamente per quanto avviene nel metodo Loop, contiene i dati delle variabili dell'i-esimo taglio memorizzati come \textbf{ILinearNumExpr} (struttura dati fornita da CPLEX);
    \item \textbf{bufferCoeffCCLC[i]}: analogamente per quanto avviene nel metodo Loop, contiene il numero di variabili che definiscono l'i-esimo taglio;
    \item \textbf{cplex.Le}: è la funzione definita da CPLEX che restituisce una espressione lineare che impone le variabili, ricevute come primo parametro, minori oppure uguali del secondo parametro ricevuto;
    \item \textbf{Add}: è la funzione eraditata dalla classe \textbf{LazyConstraintCallback} che permette di aggiungere un taglio \textbf{globale}. Come parametri riceve quindi il taglio stesso ed un valore intero che indica a CPLEX come debba gestire quest'ultimo:
    \begin{itemize}
        \item \textbf{0}: il taglio è aggiunto al pool in maniera permanente;
        \item \textbf{1}: il taglio è definito come \textbf{purgeable} quindi eliminabile nel caso in cui non risulti più efficiente;
        \item \textbf{2}: il taglio viene trattato come se fosse stato generato da CPLEX, quindi ad esempio prima di essere aggiunto al pool viene analizzata la sua efficacia e di conseguenza l'operazione va quindi a buon fino o meno;
    \end{itemize}

\end{itemize}

\subsection*{CONCORDE}
\label{sec:ConcordeS}

L'utilizzo di una \textbf{user cut callback} risulta molto più complicato rispetto a quanto appena visto per la \textbf{lazy constraint callback}: dato che tali callback scattano nel momento in cui viene trovata una soluzione frazionaria, l'individuazione di eventuali subtour non può essere eseguita con le tecniche esposte fino ad ora.\\
Per effetturare tale operazione si necessita di un separatore che, ricevendo in ingresso la soluzione x* con almeno una componente frazionaria, fornisca in uscita un insieme $S\subsetneq{V}$, $|S|\geq{2}$ tale per cui:

\begin{equation}\label{eq:concorde1}
\displaystyle{\sum_{\left ( i,j \right )\in E\left ( S \right )} x_{i,j}^{*}\nleqslant \left | \widehat{S} \right |-1}
\end{equation}

Tale sottoinsieme non è facilmente individuabile come nel caso di soluzione intere in cui era sufficiente individuare le componenti connesse. Per esempio in Figura 3 si è riportato il supporto di una soluzione x* frazionaria ove i lati colorati di rosso, blu e grigio indicano che la corrispondente variabile assumme rispettivamente i valori 1, 0.5 , 1.5.

\begin{figure}[htbp]
    \centering
    \includegraphics[width=\textwidth]{Immagini/"SoluzioneFrazionaria".jpg}
    \caption{Soluzione frazionaria}
\end{figure}

Tale grafo risulta connesso; tuttavia è presente un sottoinsieme S = {1,2,3} per cui vale \eqref{eq:concorde1}.\\
Una seconda formulazione equivalente alla \eqref{eq:concorde1} risulta essere la seguente:

\begin{equation}\label{eq:concorde2}
\displaystyle{\sum_{\left ( i,j \right )\in \delta \left ( S \right )} x_{i,j}^{*}\ngeqslant 2}
\end{equation}

Si osserva che il primo membro di \eqref{eq:concorde2} può essere visto come la capacità di una sezione di una rete di flusso se si interpretano le x* come le capacità della rete.
E' possibile calcolare una sezione di capacità minima risolvendo un problema di max flow che sappiamo essere di programmazione lineare e quindi risolubile attraverso un algoritmo polinomiale.\\
Poichè però la sezione di capacità minima dipende dal nodo sorgente \textit{s} e dal nodo di destinazione \textit{t}, si devono in realtà risolvere \textbf{n-1} problemi di max flow: per tale ragione è stato preferito utilizzare una porzione del software \textbf{Concorde} che offre, attraverso le proprie librerie, la possibilità di risolvere tale problema con tempi di esecuzione molto brevi ed allo stesso tempo di alleggerire il nostro carico di lavoro che in ogni caso non avrebbe prodotto risultati migliori.\\
Concorde è un software, sviluppano in linguaggio \textbf{C} da David Applegate, Robert E. Bixby, Vašek Chvátal, e William J. Cook, specializzato nella risoluzione ottimizzata di istanze del problema del commesso viaggiatore. Per fini accademici la distribuzione e l'utilizzo è fornita in modo gratuito e le librerei possono essere scaricate direttamente dal seguente indirizzo:
\begin{center}
\href{http://www.math.uwaterloo.ca/tsp/concorde/downloads/downloads.htm}{http://www.math.uwaterloo.ca/tsp/concorde/downloads/downloads.htm}
\end{center}

Come accennato poco fa il linguaggio utilizzato da Concorde non è \textbf{C\#} bensì \textbf{C} pertanto una implementazione diretta del software non è possibile. La soluzione da noi adottata è la seguente: abbiamo creato un nuovo progetto Visual Studio in linguaggio \textbf{C/C++} ed al suo interno abbiamo creato un codice che soddisfacesse unicamente alla funzionalità ambedue i tipi di callback proposti interfacciandosi alle librerie di Concorde per trovare i vincoli di subtour elimination ed aggiungerli al modello matematico. Successivamente il tutto è stato impachettato all'interno di una \textbf{DLL} compatibile con il linguaggio \textbf{C\#}. Il processo di creazione della DLL è stato spiegato nel seguente paragrafo LINK!!!!!!!!!!!!!!!!!!!!, d'ora in avanti quindi ci focalizzeremo unicamente nel contenuto della libreria dinamica da noi creata.\\

Per poter utilizzare Concorde in ambiente Windows è necessario importare ogni singolo file .c e .h che appartiene alla distribuzione: nel nostro caso però solo una minima parte delle sue funzionalità è di nostro interesse e pertanto solamente i seguenti file sono stati da noi utilizzati:


\begin{itemize}
    \item \textbf{allocrus.c}
    \item \textbf{connect.c}
    \item \textbf{cut\_st.c}
    \item \textbf{mincut.c}
    \item \textbf{shrink.c}
    \item \textbf{sortrus.c}
    \item \textbf{urandom.c}
    \item \textbf{cut.h}
    \item \textbf{machdefs.h}
    \item \textbf{macrorus.h}
    \item \textbf{util.h}
    \item \textbf{end}
\end{itemize}

Dove affinché il programma compili correttamente, è necessario effettuare le seguenti modifiche:

\begin{itemize}
    \item All'interno dei file \textbf{allocrus.c} e \textbf{util.h} è necessario importare tramite il comando \textbf{import} l'header \textbf{malloc.h}.
    \item All' interno del file \textbf{machdefs.h} è necessario eliminare l'inclusione di \textbf{config.h}.
\end{itemize}

\subsection*{LAZYCONSTRAINTCALLBACK IN C}
\label{sec:LazyDueS}

In questa mostreremo solamente i dettagli per l'installazione e l'utilizzo delle \textbf{lazyconstraint callback} in linguaggio \textbf{C} in quanto un discorso più ampio è stato precedentemente in LINK!!!!!!!!!.
L'installazione di questo tipo di callback avviene attraverso l'invocazione della routine \textbf{CPXsetlazyconstraintcallbackfunc} la cui firma è:

\begin{lstlisting}

CPXsetlazyconstraintcallbackfunc(CPXENVptr env,int(*)(CALLBACK\_CUT\_ARGS) lazyconcallback, void * cbhandle)

\end{lstlisting}

Dove:

\begin{itemize}
    \item \textbf{env}: Espressione contenente una combinazione lineare delle variabili del vincolo; SIAMO SICURI!!!!???!??!
    \item \textbf{lazyconcallback}: Rappresenta il nome, attribuito dal programmatore esterno, della funzione che viene invocata da CPLEX qualora la soluzione del rilassamento continuo di un nodo abbia valore intero ed inferiore all'incumbent. Nel nostro caso il nome assunto è \textbf{myLazyCallBack};
    \item \textbf{cbhandle}: Puntatore ad una struttura dati passata dall'utente contenente le informazioni che devono essere visibili all'interno della callback \textit{myLazyCallBack}. Come parametro si è passato il puntatore all'istanza;
\end{itemize}

In modo del tutto analogo a quanto fatto per \textbf{C\#}, per impostare il numero di thread pari ai core virtuali offerti dalla macchina in utilizzo si sono utilizzati i metodi \textbf{CPXsetintparam} e \textbf{CPXgetnumcores}:

\begin{lstlisting}

CPXgetnumcores(env, int * nCore);
CPXsetintparam(env, CPXPARAM\_Threads, nCore);


\end{lstlisting}

Passiamo ora a descrivere la funzione \textit{myLazyCallBack} che identifica i subtour ed aggiunge i relativi vincoli al modello, ricordando che la sua firma deve rispettare specifici parametri definiti da CPLEX stesso\footnote{Essendo tali funzioni invocate automaticamente da CPLEX i tipi di parametri che esse ricevono sono stati definiti a priori e non risultano modificabili}:

\begin{lstlisting}

static int CPXPUBLIC myLazyCallBack(CPXCENVptr env, void *cbdata, int wherefrom, void *cbhandle, int *useraction\_p)

\end{lstlisting}

Dove:

\begin{itemize}
    \item \textbf{env}: rappresenta l’istanza dell’ enviroment con il quale stiamo lavorando. DIVERSO DA SOPRA!!!!!!!!
    \item \textbf{cbdata}: come accennato poco fa questo parametro è quello specificato dall'utente durante l'installazione della callback, non essendo noto a priori il tipo di dato che l'utente desidera ricevere si utilizza \textbf{void};
    \item \textbf{wherefrom}: definisce da che punto del \textit{B\&C} è stata invocata la funzione, ai fini pratici tale parametro, per la lazy callback è risultato irrilevante;
    \item \textbf{cbhandle}: puntatore a dati privati utilizzato da CPLEX;
    \item \textbf{useraction\_p}: puntatore ad un intero utilizzato dall'utente per comunicare a CPLEX diverse informazioni. Tale parametro può assumere i seguenti tre valori:
    \begin{itemize}
    \item \textbf{0}: avente come costante simbolica CPX\_CALLBACK\_DEFAULT comunica a CPLEX che fino a quel punto la callback non ha aggiunto tagli al modello;
    \item \textbf{1}: avente come costante simbolica CPX\_CALLBACK\_FAI impone a CPLEX di uscire dall'ottimizzazione;
    \item \textbf{2}: avente come costante simbolica CPX\_CALLBACK\_SET comunica a CPLEX che sono stati aggiunti tagli;
    \end{itemize}

\end{itemize}

La prima operazione da compiere consiste nell effettuare un cast al puntatore \textbf{cbhandle} il cui tipo è noto solo al programmatore che ha installato la callback: nel nostro caso il puntatore è di tipo \textbf{instance} per cui:


\begin{lstlisting}

instance *inst = (instance*)cbhandle;

\end{lstlisting}

Successivamente è necessario assegnare al parametro \textbf{*useraction\_p} il valore \textbf{CPX\_CALLBACK\_DEFAULT}.

Per ottenere la soluzione del rilassamento continuo è necessario utilizzare il metodo \textbf{CPXgetcallbacknodex} avente come intestazione:

\begin{lstlisting}

int CPXgetcallbacknodex(CPXCENVptr env, void * cbdata, int wherefrom, double * x, int begin, int end) 

\end{lstlisting}

Dove:

\begin{itemize}
    \item Per quanto riguarda env, cbdata, wherefrom vale la descrizione vista per il metodo myLazyCallBack;
    \item \textbf{x}: array che al termine del metodo conterrà la soluzione intera del rilassamento continuo;
    \item \textbf{begin}: indica l’indice della prima variabile di cui si vuole conoscere il valore;
    \item \textbf{end}: indica l’indice dell’ultima variabile di cui si vuole conoscere il valore;
\end{itemize}

Nel nostro caso dato che vogliamo conoscere tutte le variabili, assegniamo i valori \textbf{$0$} e \textbf{$[n*(n-1)/2] – 1$} rispettivamente ai parametri \textbf{begin} ed \textbf{end}.

All'atto dell'invocazione del metodo \textit{CPXgetcallbacknodex} non viene passato come parametro l'array \textbf{bestLb} contenuto in \textit{inst} ma viene creato un opportuno array chiamato \textbf{xstar}:

\begin{lstlisting}

double *xstar = (double*)malloc(inst->nCols * sizeof(double));

\end{lstlisting}

Questa operazione risulta necessaria al fine di realizzare un codice che risulti thread-safety: poiché \textit{inst} è un puntatore accessibile da tutti i thread esiste il rischio di accessi multipli sia in modalità di lettura popolando così \textit{bestLb} con valori appartenti a soluzioni differenti.
A questo punto entra in gioco Concorde per l'individuazione e l'introduzione dei vincoli di subtour elination, dato che il metodo da utilizzare è il medesimo che vedremo per le \textit{usercut callback} rimandiamo al paragrafo seguente per maggiori dettagli.
Una volta completato tale passaggio non rimane altro che impostare il parametro \textbf{*useraction\_p} al valore \textbf{CPX\_CALLBACK\_SET} al fine di comunicare a CPLEX che sono stati aggiunti tagli.

\subsection*{USERCUT CALLBACK IN C}
\label{sec:UserS}

Come già anticipato nei precedenti paragrafi questo tipo di callback sono utilizzate per gestire soluzioni frazionarie ottenute per i vari nodidell'albero decisionale durante una risoluzione di tipo \textit{B\&C} per problemi di programmazione lineare da parte di CPLEX. A livello concettuale sono del tutto simili a quanto visto nel paragrafo precedente per le \textit{lazy callback} in linguaggio \textbf{C}, è raccomandata una lettura del paragrafo precedente a loro dedicato in quanto di seguito saranno esposti estensivamente solamente i dettagli riguardanti la gestione dei tagli.\footnote{Notiamo in realtà che l'implementazione delle \textit{usercut callback} avviene sempre in concomitanza all'implementazione delle \textit{lazy callback} per tanto il settaggio del numero di thread è necessario solamente una volta}.

L'installazione delle callback avviene tramite la funzione \textbf{CPXsetusercutcallbackfunc}:

\begin{lstlisting}

int CPXsetusercutcallbackfunc (CPXENVptr env, int(*)(CALLBACK\_CUT\_ARGS) lazyconcallback, void * cbhandle)

\end{lstlisting}

La funzione invocata da CPLEX in corrispondenza di una soluzione frazionaria è stata da noi chiamata \textbf{myUserCutCallBack} la cui firma, che anche in questo caso viene imposta dai progettisti di CPLEX, risulta essere:

\begin{lstlisting}

int CPXPUBLIC myUserCutCallBack(CPXCENVptr env, void *cbdata, int wherefrom, void *cbhandle, int *useraction\_p)

\end{lstlisting}

CPLEX, una volta calcolata una soluzione frazionara, genera in automatico dei propri tagli\footnote{Ad esempio taglio di \textit{Gomory}}. Quando il parametro \textit{wherefrom} risulta pari a \textbf{CPX\_CALLBACK\_MIP\_CUT\_LAST} significa che l'iterazione successiva da parte di CPLEX consisterebbe nell'operazione di branching sul nodo in questione: solo in questa condizione risulta conveniente generare i propri vincoli caratteristici del problema che si sta risolvendo. Qualora il parametro \textit{wherefrom} assuma invece altri valori, si effettua una semplice \textbf{return 0} senza eseguire alcuna operazione, altrimenti come già discusso per le lazy, è necessario recuperare il puntate all’istanza.
Riprendendo quanto detto nel paragrafo LINK PARAGRAFO CALLBACK||||| è sconsigliato aggiungere \textit{manualmente} ad ogni nodo dell’albero decisionale dei tagli in quanto il loro numero complessivo risulterebbe troppo elevato andando quindi a \textbf{peggiorare} le prestazioni di CPLEX. Per tale ragione, dopo alcuni test e secondo le linee guida discusse durante il corso, si è deciso che solamente con una probabilità del \textbf{$10\%$} la \textit{usercut callback} da noi definita entra in gioco. Dato che l'id numerico assegnato ai nodi dell'albero decisionale, ottenuto attraverso al funzione \textbf{CPXgetcallbacknodeinfo}, non ha alcuna relazione diretta alla probabilità che venga generata una soluzione intera oppure frazionaria, è sufficiente effettuare una operazione di modulo dieci a tale valore: nel caso in cui il risultato sia pari a zero si procede con il calcolo dei tagli. Successivamente, invocando la nota funzione \textbf{CPXgetcallbacknodex} si ottiene la soluzione frazionaria. Prima di procedere con la parte principale di questo metodo, per ragioni di chiarezza riportiamo il codice che esegue quanto finora descritto:


\begin{lstlisting}

*useraction\_p = CPX\_CALLBACK\_DEFAULT;
    
int nodecount = 0;
CPXgetcallbacknodeinfo(env, cbdata, wherefrom, 0, CPX\_CALLBACK\_INFO\_NODE\_DEPTH, &nodecount);
    
if (wherefrom == CPX\_CALLBACK\_MIP\_CUT\_LAST)
{
    instance *inst = (instance*)cbhandle;
        
    double *xstar = (double*)malloc(inst->nCols * sizeof(double));

     if ((nodecount % 10) != 0)
        return 0;

    if (CPXgetcallbacknodex(env, cbdata, wherefrom, xstar, 0, inst->nCols - 1))
    {
        free(comps);
        free(compscount);
        free(xstar);
        free(elist);
        return 1; 
    }
}

\end{lstlisting}


Da questo momento inizieremo ad utilizzare le funzionalità offerta da \textit{Concorde}, tutte le funzioni il cuo nome inizia con la sigla \textbf{"CC"} sono importate da quest'ultimo.
L'aggiunta di eventuali vincoli di \textit{subtour elimination} avviene tramite l'invocazione in un primo momento della funzione \textbf{CCcut\_connect\_components} la quale identifica le componenti connesse della soluzione ricevuta come parametro indipendentemente dal fatto che sia intera o frazionaria.

Di seguito sono riportati nel dettaglio tutti i parametri che tale funzione vuole ricevere in ingresso, si osserva che mentre i primi 4 costituiscono l'effettivo input della funzione i rimanenti 3 sono in realtà settata al suo interno e quindi possono essere visti come parametri di output:

\begin{itemize}
    \item \textbf{ncount}: rappresenta il numero di nodi del grafo;
    \item \textbf{econut}: rappresenta il numero di lati del grafo, ossia ncount*(ncount-1)/2;
    \item \textbf{*elist}: vettore di dimensione $2*econut$, contiene al suo interno tutti i lati del grafo caratterizzati dai nodi sul quale esso incide memorizzati in locazioni consecutive dell’array, è stato da noi realizzato nel seguente modo:
    
    \begin{lstlisting}
    
    int loader = 0;
    for (int i = 0; i < inst->nNodes;  i++)
    {
        for (int j = i + 1; j < inst->nNodes;  j++) 
        {
            elist[loader++] = i;
            elist[loader++] = j;
        }
    }
    
    \end{lstlisting}
    
    \item \textbf{*x}: soluzione per la quale si desiderano individuare le componenti connesse;
    \item \textbf{*ncomp}: rappresenta il numero di componenti connesse;
    \item \textbf{**compscount}: vettore di vettori contenenti il numero di nodi per ciascuna componente connessa, è strutturato in modo che compscount[i] contenga il numero di nodi presente nell’i-esima componente connessa;
    \item \textbf{**comps}: vettore di vettori contenenti gli indici dei nodi presenti all'interno delle componenti;
\end{itemize}

Nonostante non risulti necessario, al fine di rendere il codice maggiormente leggibile, si è deciso di assegnare sia alle variabili che ai puntatori il medesimo nome che assumono all’interno di \textit{CCcut\_connect\_components}.

\begin{lstlisting}

int *compscount = (int*)malloc(inst->nMaxCuts * sizeof(int));
int *comps = (int*)malloc(inst->nNodes * sizeof(int));
int nLati = ((inst->nNodes - 1)*inst->nNodes / 2);
int *elist = (int*)malloc((nLati * 2) * sizeof(int));
int ncomp; 

\end{lstlisting}

La chiamata alla funzione risulta quindi essere:

\begin{lstlisting}
if (CCcut\_connect\_components(inst->nNodes, nLati, elist, xstar, &ncomp, &compscount, &comps))
    printError(" error in CCcut\_connect\_components() inside fractcutusercallback");
\end{lstlisting}

Al suo termine, in modo del tutto trasparente, otteniamo le tre variabili \textbf{ncomp}, \textbf{compscount} e \textbf{comps} che forniscono tutte le informazioni necessarie all'aggiunta dei tagli all'interno dell'apposito pool fornito da CPLEX.
Completiamo quest'ultima operazione tramite la routine fornita da CPLEX \textbf{CPXcutcallbackadd} la cui firma è:

\begin{lstlisting}

CPXcutcallbackadd(CPXCENVptr env,void * cbdata,int  wherefrom,int  nzcnt, double  rhs,int sense,int cutind,double const *  cutval, int purgeable);

\end{lstlisting}

Dove:

\begin{itemize}
    \item \textbf{env,cbdata,wherefrom}: parametri noti già discussi nella callback myLazyCallback;
    \item \textbf{nzcnt}: numero di coefficienti diversi da zero del vincolo;
    \item \textbf{rhs}: definisce il termine noto del vincolo;
    \item \textbf{sense}: può assumere i seguenti valori:
    
    \begin{itemize}
    \item \textbf{cutind}: array di \textit{nzcnt} elementi contenenti gli indici delle variabili presenti nel vincolo;
    \item \textbf{cutval}: array di \textit{nzcnt} elementi contenenti i corrispondenti valori dei coefficienti;
    \item \textbf{purgeable}: valore intero che specifica come CPLEX deve trattare il taglio:
    
        \begin{itemize}
        \item \textbf{CPX\_USECUT\_FORCE}: il taglio una volta aggiunto al rilassamento non può essere più rimosso;
        \item \textbf{CPX\_USECUT\_PURGE}: il taglio è aggiunto al rilassamento ma può essere eliminato in un secondo momento se giudicato inefficiente;
        \item \textbf{CPX\_USECUT\_FILTER}: il taglio deve essere trattato come se generato da CPLEX il quale prima di aggiungerlo al rilassamento lo analizza e può quindi decidere di abortire l'operazione di aggiunta.
        nel rilassamento(per esempio è già presente un taglio più efficiente);
        \end{itemize}
    \end{itemize}
\end{itemize}

Per aggiungere un taglio per ogni componente connessa è necessario popolare i vettori \textbf{cutval}, \textbf{cutind} e la variabile \textbf{nzcnt} opportunamente sfruttando le informazioni fornite da \textit{Concorde}. Per stabilire quali nodi appartengono alla t-esima componente connessa si sono dichiarate due variabili intere \textbf{k1} e \textbf{k2} che contengono sistematicamente l'indice del \textbf{primo} e dell'\textbf{ultimo} nodo tra quelli appartenenti alla t-esima componente connessa memorizzata in \textit{comps}. Si osserva che \textit{k2} è inizializzato al valore \textbf{$-1$} in quanto gli indici di un qualsiasi vettore partono da $0$.

\begin{lstlisting}

    if (ncomp > 1)
        {
            int k1 = 0;
            int k2 = -1;

            for (int c = 0; c < ncomp; c++)
            {
                int dimIndexValue = compscount[c] * (compscount[c] - 1) / 2; 
                int *cutind = (int*)malloc(dimIndexValue * sizeof(int));
                double *cutval = (double*)malloc(dimIndexValue * sizeof(double));
                int nzcnt = 0;

                
                k2 += compscount[c];

                for (int i = k1; i < k2; i++)
                {
                    for (int j = i + 1; j <= k2; j++)
                    {
                        cutval[nzcnt] = 1.0;
                        cutind[nzcnt] = xPos(comps[i], comps[j], inst);
                        nzcnt++;
                    }
                }

                k1 = k2 + 1;
                
                CPXcutcallbackadd(env, cbdata, wherefrom, nzcnt, compscount[c] - 1, 'L', cutind, cutval, CPX\_USECUT\_FORCE);

                *useraction\_p = CPX\_CALLBACK\_SET; 
                free(cutind);
                free(cutval);
            }

            free(elist);
            free(comps);
            free(compscount);
            free(xstar);


            return 0;
        }

\end{lstlisting}

Nel caso in cui la soluzione presenti una sola componente connessa, come in Fig. X, invocando la funzione \textbf{CCcut\_violated\_cuts} di \textit{Concorde} è possibile individuare gli insiemi $S$ che soddisfino la disuguaglianza \eqref{eq:concorde2}: noto $S$ risulta poi banale inserire il relativo vincolo di subtour. In particolare \textit{CCcut\_violated\_cuts} è una funzione in grado di individuare sezioni di capacità inferiori ad una certa soglia. Descriviamo quindi i 7 parametri che tale funzione riceve in input:

\begin{itemize}
    \item \textbf{int ncount}, \textbf{int ecount}, \textbf{int *elist}: il loro significato è già stato descritto per la funzione \textit{CCcut\_connect\_components};
    \item \textbf{dlen}: vettore contenente la capacità di ogni lato;
    \item \textbf{cutoff}:[Questo è il termine noto della disequazione f2, non ho capito perchè devo togliere a 2 un EPSILON, così è scritto nel pdf condiviso dal prof che si chiama         RO2\_TSPutilities]
    \item \textbf{(*(doit\_fn)(double, int, int *, void *) }: è una funzione creata da noi che risulta essere una vera e propria callback: ogniqualvolta \textit{Concorde} individua un insieme $S$ cercato tale funzione viene invocata. Al suo interno, grazie ai parametri forniti\footnote{Maggiori dettagli riguardo i quattro parametri di ingresso saranno forniti a breve durante la descrizione di come l'implementazione di tale funzione è stata da noi realizzata.} è nostro compito procedere all'ampliamento del pool di tagli di \textit{CPLEX}.
    \item \textbf{pass\_param}: puntatore ad una struttura dati contenente variabili e puntatori che devono essere accessibili all’interno della callback;
\end{itemize}

Nel nostro caso l'invocazione di tale metodo avviene nel seguente modo:

\begin{lstlisting}

CCcut_violated_cuts(inst->nNodes, inst->nCols, elist, xstar, 2.0 - cutThreshold, doitFuncConcorde, (void*)&in)

\end{lstlisting}

Dove occorre solamente far notare che la funzione callback da noi definita prende il nome \textbf{doitFuncConcorde} mentre l'ultimo parametro è una \textbf{struct} da noi creata contenente al suo interno tutte le informazioni occorrenti per invocare il metodo \textbf{CPXcutcallbackadd}, descritto in precedenza, all'interno della callback.

\begin{lstlisting}

typedef struct 
{
    instance *inst;
    CPXCENVptr env;
    void *cbdata;
    int wherefrom;
    int *useraction\_p;
} inputCC;

\end{lstlisting}

Per concludere questo paragrafo non rimane altro che parlare più in dettagli riguardo la realizzazione della callback \textbf{doitFuncConcorde}, per prima cosa forniamo la sua firma:

\begin{lstlisting}

int doitFuncConcorde(double cutValue, int cutcount, int *cut, void *inParam)

\end{lstlisting}

Dove:

\begin{itemize}
    \item \textbf{cutValue}: rappresenta il valore del taglio;
    \item \textbf{cutcount}: rappresenta il numero dei nodi;
    \item \textbf{cut}: array contenente l'indice associato ai nodi;
    \item \textbf{inParam}: struttura dati appena descritta;
\end{itemize}

Dopo aver effettuato la classica operazione di recupero del puntatore alla struttura dati fornita in ingresso alla callback

\begin{lstlisting}

inputCC *in = (inputCC*)inParam;

\end{lstlisting}

possiamo procedere con l'aggiunta del taglio  attraverso \textit{CPXcutcallbackadd}: si sono così definiti due array di interi \textbf{cutind} e \textbf{cutval} contenenti rispettivamente gli indici delle variabili che costituiscono il taglio ed il relativo coefficiente. Si è inoltre dichiarata una variabile intera \textbf{nzcnt} che contiene il numero di variabili caratterizzanti il taglio:

\begin{lstlisting}

int dimIndexValue = inst->nNodes * (inst->nNodes - 1) / 2;
int *cutind = (int*)malloc(dimIndexValue * sizeof(int));
double *cutval = (double*)malloc(dimIndexValue * sizeof(double));
int nzcnt = 0;

    for (int i = 0; i < cutcount - 1; i++)
    {
        for (int j = i + 1; j <= cutcount - 1; j++)
        {
            int n1 = cut[i];
            int n2 = cut[j];
            !!!!!!!!!!!!!!!!!!!!!!!!!!!!!!!!!!!!!!!!!!!!!!!!!!!!!!!!!!!!!!!!!!!!!!!!!!!!!!!!!!!!!!!!!!!!!!!!!!!!!!!!!!
            METTI QUA IL COMMENTO A xPos che avevi fatto... il codice va commentato !!!!!!!!!!!!!!!!!!!!!!!!!!!!!!!!!!
            !!!!!!!!!!!!!!!!!!!!!!!!!!!!!!!!!!!!!!!!!!!!!!!!!!!!!!!!!!!!!!!!!!!!!!!!!!!!!!!!!!!!!!!!!!!!!!!!!!!!!!!!!!
            cutind[nzcnt] = xPos(n1, n2, inst);
            cutval[nzcnt] = 1.0;
            nzcnt++;
        }
    }
CPXcutcallbackadd(in->env, in->cbdata, in->wherefrom, nzcnt, cutcount - 1, 'L', cutind, cutval, CPX\_USECUT\_FORCE);

*in->useraction\_p = CPX\_CALLBACK\_SET;

free(cutind);
free(cutval);

return 0;

\end{lstlisting}

\section*{SEZIONE OTTO}
\label{sec:SezioneOttoS}

Fino a questo momento sono stati presentati algoritmi che, alla loro naturale terminazione, garantiscono di risolvere una generica istanza del problema del commesso viaggiatore in modo esatto, ovvero restituendo sempre un ottimo globale come soluzione.
L'applicazione di metodi esatti non è sempre possibile per due motivi principali: il primo è che si necessita di un programma di calcolo molto potente ed in genere costoso come può essere \textbf{CPLEX}, in secondo luogo, quando si procede all'analisi di problemi \textbf{NP-hard}, indipendentemente da quali accorgimenti introduciamo, non è mai garantito di trovare la soluzione migliore in tempi relativamente brevi.
Per tali ragioni, nelle applicazioni reali, capita spesso che l'unica strada percorribile sia il ricorrere a metodi \textit{non esatti}, che, come abbiamo già potuto vedere nelle varianti del metodo \textbf{Loop} presentate, prendono il nome di algoritmi euristici: la soluzione che offrono sarà sempre ammissibile ma non viene garantità la sua ottimilità, al contrario nella maggior parte dei casi, soprattutto per istanze complesse, questa non viene quasi mai raggiunta. \'E inoltre tenere ben presente che esistono molteplici algoritmi euristici più o meno potenti, essendo inoltre tecniche non esatte, è possibile trovarne infinite varianti per ognuno di essi. In generale, come meglio vedremo nel seguito del testo, possiamo classificarli in \textbf{costruttivi}, \textbf{migliorativi}, \textbf{metaeuristici} e \textbf{matheuristics}.
Quanto esposto fino ad ora ci porta intuitivamente a pensare che la bontà di un metodo proposto può subire enormi variazioni in base a quali istanze su cui viene applicato.

\subsection*{ALGORITMI COSTRUTTIVI GREEDY}
\label{sec:GreedyS}

Gli algoritmi costruttivi hanno la caratteristica di determinare una soluzione ammissibile partendo da una \textit{vuota}. Quest'ultima, durante tutto il corso dell'agoritmo, seguendo il criterio di espansione, viene continuamente aggiornata ed ampliata attraverso l'introduzione di nuove componenti fintanto che non diviene completa e quindi ammissibile.
Una sottocategoria molto importante degli algoritmi costruttivi viene definita come \textbf{greedy}: la soluzione euristica al problema è ottenuta attraverso una sequenza \textit{finita} di decisioni "localmente ottime". In altre parole, attraverso una struttura ricorsiva, ad ogni sua iterazione, la soluzione parziale viene aggiornata con l'aggiunta dell'elemento migliore disponibile in quel momento. La correttezza di queste operazione non deve però mai essere verificata runtime dell'algoritmo ma solamente a livello teorico durante la sua fase di progettazione. |'E proprio questa caratteristica che motiva il nome \textit{greedy} e soprattutto rende tali algoritmi estremamente velociti.

\begin{figure}[htbp]
    \centering
    \includegraphics[width=\textwidth]{Immagini/"ProceduraGreedy".jpg}
    \caption{Algoritmo Greedy}
\end{figure}

Nella procedura, \textbf{S} è l'insieme degli elementi di \textbf{E} che sono stati inseriti nella soluzione parziale corrente mentre \textbf{Q} è l'insieme degli elementi appartenenti ad \textbf{E} ancora da esaminare.
La procedura fa uso della sottoprocedura \textbf{Best} la quale fornisce il miglior elemento di \textbf{E} tra quelli ancora in \textbf{Q} sulla base di un prefissato criterio euristico.
Come esempio della categoria di algoritmi euristici costruttivi greedy presentiamo nel paragrafo seguente la tecnica del \textbf{nearest neighbour}.

\subsection*{ALGORITMO NEAREST NEIGHBOR}
\label{sec:NearestNeighborS}

L'algoritmo \textbf{nearest neighbour} è stato uno dei primi algoritmi utilizzati per risolvere istanze del problema del commesso viaggiatore. La sua dimostrazione di correttezza non viene qui riportata in quanto non risulta di interesse ai fine del nostro progetto ed è inoltre ampiamente discussa in letteratura. Limitiamoci quindi ad una descrizione del concetto fondamentale alla base dell'algoritmo: dato un qualsiasi percorso (soluzione) parziale, il modo migliore per ampliarlo è banalmente attraverso l'aggiunta di un nuovo arco a costo \textbf{minore} avente come estremi uno dei due nodi liberi\footnote{Nodi attraversati del percorso parziale ma aventi un solo lato incidente.} del percorso stesso mentre il secondo sia uno ancora disponibile\footnote{Non ancora attraversato dal circuito parziale.}.

\begin{figure}[htbp]
    \centering
    \includegraphics[width=\textwidth]{Immagini/"ng".jpg}
    \caption{Esempio di algoritmo Nearest Neighbour}
\end{figure}

Entriamo ora più nel dettaglio nella realizzazione di questo algoritmo descrivendone i passaggi fondamentali. Noteremo che sono state fatte delle scelte di programmazione che in un primo momento potrebbero apparire inmotivate se considerate limitatamente a qanto visto fino ad ora.
L'algoritmo preso in analisi produce una soluzione valida molto velocemente ma che risulta essere scadente per quanto riguarda il suo costo, nel nostro progetto di conseguenza il \textit{nearest neighbour} viene unicamente utilizzato per fornire una o più soluzioni di partenza per altri algoritmi euristici. Risulta pertanto di maggior importanza che al suo interno sia introdotta una certa randomicità nelle scelte che effettua in modo tale che multipli utilizzi sulla stessa instanza del TSP producano soluzioni tra loro scorrelate. Algoritmi che presentano tale caratteristica possono essere trovati in letteratura con la dicitura \textbf{GRASP}\footnote{Greedy Randomly Adaptive Search Procedure.}
\begin{itemize}
    \item Innanzi tutto, partendo da una soluzione vuota è necessaria una operazione preliminare prima di innescare la ricorsività dell'algoritmo. In altre parole dobbiamo fornire un nodo iniziale il cui lato incidente a costo minore diventa il la prima vera componente della soluzione che quindi da vuota diviene parziale.
    La scelta di quale debba essere il nodo di partenza avviene in modo casuale;
    \item Come sarà più chiaro in seguito, l'algoritmo \textit{nearest neighbour} non produce mai soluzioni soddisfacenti ed i solamente utilizzato come punto di partenza per algoritmi più complessi. Questi ultimi necessitano in genere di una struttura dati che veda il circuito prodotto come un percorso orientato. Seguendo questa linea di pensiero, supponendo che l'ultima iterazione abbia introdotto il nodo $j$ nel percorso parzile, il lato successivo che andremo a selezionare dovrà sempre essere incidente in $j$. In questo modo risulta molto più semplice tenere traccia della soluzione come un vero e proprio percorso orientato, essendo questo utilizzato solamente come input per altri metodo più complessi la nostra scelta non risulta in alcun modo limitante;
    \item Per introdurre un ulteriore livello di casualità nell'algoritmo, la scelta di quale lato entra a far parte della soluzione parziale non ricade sempre in quello a costo minore. In particolare dopo varie prove si è deciso che quest'ultima opzione avviene con una percentuale del $90\%$, mentre con il $9\%$ la scelta ricade nel secondo miglior lato e con il restate $1\%$ nel terzo miglior lato;
    \item Spesso può capitare che il lato designito per essere aggiunto al percorso causerebbe la presenza di un cappio al suo interno. Naturalmente la soluzione finale risulterebbe non valida e quindi tale situazione deve essere evitata. Banalmente, nel caso in cui il lato in questione sia la $i-esima$ scelta migliore, questo viene sostituito dalla $(i+1)-esima$ miglior scelta. Naturalmente il tutto viene ripetuto iterativamente fino a che non si trova un lato accettabile;
\end{itemize}

Di seguito è riportata la realizzazione del codice tenendo presente che è richiesta in input una struttura dati che permetta di ottenere l'ordine crescente, per ogni nodo, dei lati in esso incidenti basandosi chiaramente sul loro costo. Tale informazione è fornita dal metodo \textbf{BuildSLComplete} già descritto brevemente LINK!!!!!!!!!!!!! (loop euristico) ed in dettaglio nell'apposito paragrafo della appendice LINK!!!!!!!!!!!!!!!!!!!.

\begin{lstlisting}

public static PathGenetic NearestNeighbor(Instance instance, Random rnd, List<int>[] listArray)
{
// heuristicSolution is the path of the current heuristic solution to generate
    int[] heuristicSolution = new int[instance.NNodes];
    double distHeuristic = 0;

    int currentIndex = rnd.Next(instance.NNodes);
    int startindex = currentIndex;

    bool[] availableIndexes = new bool[instance.NNodes];

    availableIndexes[currentIndex] = true;

    for (int i = 0; i < instance.NNodes - 1; i++)
    {
        bool found = false;

        int plus = RndPlus(rnd);

        int nextIndex = listArray[currentIndex][0 + plus];

        do
        {
            if (availableIndexes[nextIndex] == false)
            {
                heuristicSolution[currentIndex] = nextIndex;
                distHeuristic += Point.Distance(instance.Coord[currentIndex], instance.Coord[nextIndex], instance.EdgeType);
                availableIndexes[nextIndex] = true;
                currentIndex = nextIndex;
                found = true;
            }
            else
            {
                plus++;
                if (plus >= instance.NNodes - 1)
                {
                    nextIndex = listArray[currentIndex][0];
                    plus = 0;
                }
                else
                    nextIndex = listArray[currentIndex][0 + plus];
            }

        } while (!found);
    }

    heuristicSolution[currentIndex] = startindex;
    distHeuristic += Point.Distance(instance.Coord[currentIndex], instance.Coord[startindex], instance.EdgeType);

    return new PathGenetic(heuristicSolution, distHeuristic);
}

\end{lstlisting}

\subsection*{ALGORITMI MIGLIORATIVI}
\label{sec:MigliorativiS}

Gli algoritmi euristici migliorativi si basano su un'idea estremamente semplice ed intuitiva: data una soluzione ammissibile \textbf{x}, relativa ad un problema di ottimizzazione, viene esaminato se attraverso minime variazioni questa risulta migliorabile in termini di funzione obiettivo.
In gergo più tecnico si parla di ricercare soluzioni \textit{vicine} a quella attuale ma migliorative. Per poter definire il concetto di "vicinanza" è necessario discutere quello di \textbf{mossa}. Questa è una operazione di modifica (caratteristica dell'algoritmo migliorativo) che viene eseguita su \textbf{x} e che ha come conseguenza la generazione di un \textbf{insieme} di soluzioni ammissibili le quali costituiscono un intorno di \textbf{x}, indicato con \textbf{N(x)}. Si parla allora di \textbf{y} vicina ad \textbf{x} se e solo se differiscono tra loro per una sola mossa e quindi $y \in N(x)$.

Una volta definito $N(x)$ questo viene esplorato secondo due possibili strategie che sono \textbf{first improvement} e \textbf{steepest descent}. Nel primo caso l'esplorazione dell'intorno termina non appena si trova una soluzione migliore di quella corrente. Nel secondo caso, invece, l'esplorazione è completa e viene trovato il miglioramento più consistente.

Qualora esista una soluzione \textbf{y} migliore di \textbf{x}, il procedimento viene iterato esplorando N(y); viceversa  l'algoritmo si arresta. Giunti a questo punto il risultato finale può essere una soluzione \textit{localmente ottima} oppure, più raramente, globalmente ottima\footnote{Da notare che un ottimo è globale se lo è anche localmente}. Poiché da un punto di vista matematico il processo di ricerca analizza, ad ogni interazione, un intorno della soluzione corrente, gli algoritmi migliorativi vengono anche chiamati \emph{algoritmi di ricerca locale}.

Tranne alcuni casi particolari in cui la funzione obiettivo ha determinate caratteristiche di convessità, nella maggior parte dei problemi reali questa presenta un grande numero di minimi locali che spesso si discostano totalmente dell'ottimo globale. In effetti, una delle fortunate eccezioni è il metodo del simplesso per la programmazione lineare che si pone alla base degli studi in questo settore: esso fornisce sia un metodo per analizzare in un numero finito di passi tutti gli ottimi locali del problema e soprattutto se questi sono anche globali.

Tra i più famosi algoritmi migliorativi applicabili al problema del commesso viaggiatore troviamo i \textbf{K-Opt} dove \textbf{K} è in genere un numero intero superiore a $2$. Procediamo quindi ad una loro descrizione generale seguita da una implementazione particolare della tecnica del \textbf{2-Opt}.

\subsection*{ALGORITMO K-OPT}
\label{sec:KOptS}

Gli algoritmi \textbf{K-Opt}, sigla inglese per K-Ottimalità, fanno parte della categoria degli algoritmi migliorativi e sono caratterizzati da una mossa, applicata ad un circuito hamiltoniano, consistente nello \textbf{scambio} di \textbf{K} archi con altrettanti non facenti parte del percorso producendone uno \textit{vicino}, migliore e chiaramente valido.
Come specificato nel paragrafo precedente \textbf{K} può assumere qualsiasi valore superiore a $2$\footnote{Chiaramente \textbf{K} non può superare il numero di archi che costituiscono la soluzione.} ma in genere è proprio $K = 2$ l'unica variante realmente utilizzata. Risulta infatti facilmente verificabile che più il suo valore è alto, più salgono sia la complessità computazione che di scrittura\\progettazione dell'algoritmo\footnote{Vedremo in seguito che questa affermazione è valida solo per algoritmi che applicano direttamente la definizione di K-Ottimalità. Esistono infatti metodi che la ottengono indirettamente se sono caratterizzati da tempi di esecuzione molto buoni.} perdendo così i vantaggi offerti dall'utilizzo di tecniche euristiche.
Concentriamoci quindi unicamente nella variante $2-Opt$: presa una qualsiasi coppia di archi distinti $([i,j];[h,k])$\footnote{Notiamo che questi non possono mai essere presi consecutivi e cioè con un vertice in comune in quanto non ci sarebbe modo di produrre una soluzione \textit{vicina} valida.} è possibile sostituirli correttamente con una sola delle combinazioni $([i,k];[h,j])$ e $([i,h];[j,k])$. Una di queste due, infatti, trasforma la soluzione in una seconda contenente due subtour.
A discapito quindi di un concetto molto basilare, l'algoritmo $2-opt$\footnote{In generale lo stesso discorso può applicarsi anche a tutto il resto della famiglia di algoritmi.} introduce la difficoltà di verificare quale delle due possibili sostituzioni è valida. L'approccio migliore in questi casi è di introdurre un fittizio ordinamento nel circuito attraverso una struttura dati di supporto apposita. Come è possibile infatti vedere dalla figura X, se gli archi sono ordinati, e quindi lo è anche il circuito, la difficoltà appena discussa è facilmente risolvibile: 

\begin{figure}[htbp]
    \centering
    \includegraphics[width=\textwidth]{Immagini/"2-opt".jpeg}
    \caption{Esempio di azione nell'algoritmo 2-opt}
\end{figure}

Viene da sé che applicare l'approccio $2-Opt$ a qualsiasi coppia di archi non garantisce un miglioramento del costo della soluzione. Questo però avviene nel $100\%$ dei casi se i due archi in questione visualmente si incrociano come mostrato nella figura X, tale affermazione è facilmente verificabile matematicamente.
Nel complesso, nel caso in cui si voglia trovare l'operazione di due ottimalità migliore\footnote{Cioè trovare la soluzione vicina a costo più basso} è necessario confrontare tutte le coppie possibili ottenendo quindi una complessità computazione pari a $O(n^2)$ rispetto al numero di nodi mentre nel caso di un generico valore di $K$ si passa ovviamente a $O(n^3)$.
Nel caso del \textit{TSP}, alcuni esperimenti svolti versoo la fine degli anni '50 hanno mostrato che il passaggio da due ottimalità a tre ottimalità porta un miglioramento sensibile della qualità della soluzione trovata, che giustifica pienamente l'aumento di carico computazionale ma non il costo di scrittura\\progettazione dell'algoritmo se paragonato ad altri metodi euristici che mostreremo nel seguito del testo. Miglioramenti praticamente trascurabili si hanno invece per $K > 3$. 
Concludiamo le nozioni riguardanti i metodi migliorativi di K ottimalità nel seguente paragrafo dove vengono mostrati i dettagli implementativi dell'algoritmo $2-Opt$ da noi progettato

\subsection*{METODO TwoOpt}
\label{sec:DueOptS}

Il metodo TwoOpt implementa l' euristico algoritmo $2-Opt$ applicato al problema del commesso viaggiatore la cui discussione teorica è stata presentata nel paragrafo precedente.

\begin{lstlisting}
public static void TwoOpt(Instance instance, PathStandard pathG)
\end{lstlisting}

Dove:

\begin{itemize}
    \item \textbf{instance}: oggetto dove sono memorizzati i dati relativi alla istanza del problema TSP;
    \item \textbf{pathG}: rappresenta il percorso sul quale l'algoritmo viene eseguito;
\end{itemize}

La classe \textbf{PathStandard}, descritta in dettagli in questo capitolo della appendice LINK!!!!!!!!!!!!!!!!, permette di memorizzare una soluzione del problema come un percorso ordinato. Molto semplicemente al suo interno si mantiene aggiornato un vettore di interi di nome \textbf{path} dove all'indice $i-esimo$ troviamo il nodo successivo da visitare, seguendo un il percorso attuale, trovandosi nel vertice di indice $i$.
Di seguito è riportato il contenuto del metodo \textbf{TwoOpt} che nel nostro caso utilizza la tecnica \textit{first improvement}, già discussa nei paragrafi precedenti.

\begin{lstlisting}

int indexStart = 0;
int cnt = 0;
bool found = false;

do
{
   found = false;
   int a = indexStart;
   int b = pathG.path[a];
   int c = pathG.path[b];
   int d = pathG.path[c];

   for (int i = 0; i < instance.NNodes - 3; i++)
   {
      double distAC = Point.Distance(instance.Coord[a], instance.Coord[c], instance.EdgeType);
      double distBD = Point.Distance(instance.Coord[b], instance.Coord[d], instance.EdgeType);
      double distAD = Point.Distance(instance.Coord[a], instance.Coord[d], instance.EdgeType);
      double distBC = Point.Distance(instance.Coord[b], instance.Coord[c], instance.EdgeType);

      double distTotABCD = Point.Distance(instance.Coord[a], instance.Coord[b], instance.EdgeType) +
      Point.Distance(instance.Coord[c], instance.Coord[d], instance.EdgeType);

      if (distAC + distBD < distTotABCD)
      {
            Utility.SwapRoute(c, b, pathG);
            pathG.path[a] = c;
            pathG.path[b] = d;
            pathG.cost = pathG.cost - distTotABCD + distAC + distBD;
            indexStart = 0;
            cnt = 0;
            found = true;
            //"break" = "first improvement" technique
            break;
      }
      
      c = d;
      d = pathG.path[c];
    }

    if (!found)
    {
        indexStart = b;
        cnt++;
    }

} while (cnt < instance.NNodes);

\end{lstlisting}

\'E utile fare infine le seguenti annotazioni: le assegnazioni degli indici possono ad un primo sguardo sembrare errate in quanto non tutte le possibili coppie di lati vengono analizzate, in realtà questo non avviene solamente per quelli tra loro consecutivi. Infine, dato che come vedremo a breve la tecnica della due ottimalità è utilizzata in un contesto più ampio, il percorso modificato viene poi riutilizzato e quindi è necessario mantenere aggiornato anche il suo ordinamento fittizzio. Questa funzionalità è offerta dal metodo \textbf{SwapRoute} descritto nel dettaglio nella appendice LINK!!!!!!!.

\subsection*{MULTISTART}
\label{sec:MultiStartS}

La tecnica del \textit{multi start} è un modo molto semplice per combinare i due algoritmi euristici proposti, \textit{nearest neighbour} e \textit{2-Opt}, sfruttando appieno i loro punti di forza.
L'idea alla base è la seguente: \textit{nearest neighbour}, secondo l'implementazione presentata LINK!!!!!!!, offre la possibilità di produrre molto velocemente un numero soluzioni valide al problema TSP in esame tutto diverse tra loro; Il suo principale svantaggio è che queste presentano risultati molto scadenti per quanto riguarda la funzione obiettivo da minimizzare. A questo punto è logico pensare di introdurre la tecnica del \textit{2-Opt}, l'algoritmo infatti necessità di una soluzione iniziale su cui essere applicato e produce velocemente risultati accettabili indipendentemente dalla bontà del punto di partenza.
Nel complesso quindi, l'algoritmo \textit{multi start} ripeto la combinazione appena proposta memorizzando solamente la soluzione migliore torvata durante il processo. Il tutto naturalmente fino allo scadere del classico timelimit ricevuto in ingresso dalla applicazione.
Riportiamo quindi di seguito il codice commentato senza fornire ulteriori indicazioni in quanto la sua comprensione dovrebbe a questo punto risultare facile:

\begin{lstlisting}

//It stores the current best path found
PathStandard incumbentSol = new PathStandard();
//It stores the latest path found
PathStandard heuristicSol;
//Used by nearest neighbour, it orders the links accident in a generic node based on their cost
List<int>[] listArray = Utility.BuildSLComplete(instance);

//At least one time the combo nearest neighbour and 2-Opt is used to produce a valide solution
do
{
//Using the nearest neighbour technique
heuristicSol = Utility.NearestNeighbor(instance, rnd, listArray);

//Using the 2-Opt technique on the nearest neighbour solution produced
TwoOpt(instance, heuristicSol);

//Confronting the best solution so far with the latest
if (incumbentSol.cost > heuristicSol.cost)
{
incumbentSol = heuristicSol;

Console.WriteLine("Incumbed changed");
}
else
Console.WriteLine("Incumbed not changed");

} while (clock.ElapsedMilliseconds / 1000.0 < instance.TimeLimit); //Cicle is repeated until the time limit is over

\end{lstlisting}

\subsection*{GENERAZIONE NUMERI CASUALI - SEMERANDOM DA SPOSTARE!!!!!!!!!!!!!!! INIZIO O APPENDICE}
\label{sec:SemiRandomS}

Per generare un numero casuale è sufficiente istanziare la classe Random ed invocare sull' istanza creata il metodo \textbf{Next} o \textbf{NextDouble}. Per esempio nel caso in cui si voglia generare un numero casuale intero tra 1 e 99 è necessario scrivere le seguenti righe di codice:

\begin{lstlisting}

Random random = new Random();
int nun = random.Next(1,100);

\end{lstlisting}

I numeri random sono generati, a partire da un valore d' inizializzazione chiamato \textbf{seme}, da un algoritmo matematico. Per sua natura l' algoritmo è deterministico: se si fornisce in input lo stesso seme genererà sempre la medesima sequenza di numeri. Per tale ragione il valore del seme viene derivato dall'orologio di sistema all' atto della  creazione dell' istanza della classe Random qualora si utilizza il costruttore di default. In questo modo, non essendo predicibile il valore del seme, la sequenza di numeri generati dall' algoritmo risulta essere sistematicamente casuale. 

L' orologio di sistema non sempre risulta un buon valore da utilizzare per settare il seme. Si supponga di avere la necessità di creare due oggetti diversi della classe Random: qualora quest’ ultimi siano creati uno di seguito all’ altro avranno entrambi il medesimo seme poichè l' orologio di sistema risulta lo stesso nel lasso di tempo che il processore impiega ad eseguire le due istruzioni.

Per constatare ciò si è realizzato il seguente programma:

\begin{lstlisting}

Random rdn1 = new Random();
Random rdn2 = new Random();

for (int i = 0; i < 10; i++)
{
    Console.WriteLine(rdn1.Next(1, 10) + "-" + rdn2.Next(1, 10));
}
Console.ReadLine();

\end{lstlisting}

Il cui output, come previsto, risulta mostrato in Fig C.

\begin{figure}[htbp]
    \centering
    \includegraphics[width=\textwidth]{Immagini/"ng".jpg}
    \caption{Output}
\end{figure}

Per ovviare a questa problematica la classe Random dispone di un secondo costruttore che riceve come parametro un intero che setta il valore del seme: sarà a questo punto compito del programmatore passare valori casuali e  differenti all' atto della creazione delle due istanze della classe Random.


\section*{METAEURISTICI}
\label{sec:MetaEuristiciS}

\subsection*{TABU}
\label{sec:TabuS}

\subsection*{VNS}
\label{sec:VNSS}

\subsection*{ALGORITMI GENETICI}
\label{sec:GeneticoS}

Gli Algoritmi Genetici (\textit{AG}), proposti nel 1975 da J.H. Holland, sono un modello computazionale idealizzato dall'evoluzione naturale darwinista. L'aggettivo "genetico" deriva dal fatto che il modello evolutivo darwiniano trova spiegazioni nella branca della biologia detta genetica e dal fatto che tali algoritmi attuano meccaniche concettualmente simili a quelli dei processi biochimici scoperti da questa scienza. I principi fondamentali che consentono la nascita e lo sviluppo di un processo evolutivo che porta all'evoluzione di una specie sono la \textbf{selezione naturale} e la \textbf{varietà del genotipo}\footnote{Il termine \textit{genotipo} indica la costituzione genetica di un organismo o di un gruppo di individui} della popolazione.
La selezione naturale è il meccanismo grazie al quale si ha un progressivo e cumulativo aumento della frequenza degli individui aventi caratteristiche ottimali per l'ambiente in cui essi vivono poiché solo quelli che meglio si adattano ad un certo habitat riescono a sopravvivere e a riprodursi.
I meccanismi generatori della variazione del genotipo della popolazione sono sostanzialmente due:

\begin{itemize}
    \item Un processo di \textbf{riproduzione} nel quale gli individui, detti genitori, si accoppiano producendo di nuovi, detti figli, il cui patrimonio genetico risulta pertanto una combinazione di quello dei genitori;
    \item Un processo di \textbf{mutazione} che colpisce i figli i quali subiscono una modifica del patrimonio genetico ereditato dai genitori per effetto dell'ambiente che li circonda;
\end{itemize}

I cambiamenti che si verificano da una generazione all'altra risultano essere molto piccoli ma, dato che sopravvivono soprattutto quelli positivi, un loro accumulo porta nel tempo a grandi cambiamenti.
La ricerca parte da una popolazione iniziale di individui, detti cromosomi, che rappresentano ipotetiche soluzioni al problema dato. Ogni individuo della popolazione viene codificato da un vettore\footnote{Oltre alla codifica vettoriale in letteratura è possibile trovare anche quella ad albero. Tuttavia essa viene utilizzata per codificare gli individui della popolazione nell'ambito della programmazione genetica (che è?????????).} i cui elementi contengono simboli appartenenti ad un alfabeto finito, detti geni. Ad ogni soluzione è associato un valore determinato da una funzione chiamata \textbf{Fitness} il cui scopo è di determinare la bontà di un individuo nel risolvere il problema in questione.
Così come nella natura solamente gli individui che meglio si adattano all' ambiente sono in grado di sopravvivere e riprodursi, anche negli algoritmi genetici le soluzioni migliori sono quelle che hanno la maggiore probabilità di trasmettere i propri geni alle generazioni future.
Come vedremo in seguito sono fondamentalmente tre le caratteristiche determinanti per un algoritmo genetico: determinare quale funzione di fitness si andrà ad utilizzare, partendo dalla attuale generazione decidere come creare un pull di possibili candidati per quella successiva ed infine come selezionare tra questi ultimi quelli che sopravviveranno.
Essendo la definizione delle funzione di fitness direttamente dipendente da quale tipo di problema si desidera studiare, concludiamo questa introduzione elencando solamente quali operatori genetici è possibile applicare per definire le restanti due caratteristiche di un algoritmo.

\subsection*{OPERATORI GENETICI}
\label{sec:OperatoriGeneticiS}

In questo paragrafo vengono trattati i principali operatori genetici applicabili ai cromosomi. Per ogni operatore vengono inoltre descritte le principali varianti che si possono trovare in letteratura.

\subsubsection*{OPERATORE DI CROSSOVER}
\label{sec:OpCrossoverS}

Il crossover è una metafora della riproduzione in cui il materiale genetico dei discendenti è una combinazione di quello dei genitori. Di seguito sono indicati alcuni dei metodi più comuni per creare un \textit{figlio} partendo da due \textit{genitori}, le instanze così ottenute vanno a far parte di quelle candidate alla sopravvivenza per la generazione successiva:

\begin{itemize}
    \item \textbf{Crossover ad un punto}: date due soluzioni si tagliano i loro vettori di codifica in un punto casuale o predefinito per ottenere due teste \{$H_a, H_b$\} e due code\{$T_a, T_b$\}, si possono costruire quindi altrettante soluzioni distinte combinando la testa di un genitore con la coda dell'altro $S_1 = H_a \cup T_b , S_2 = H_b \cup T_1$;
    
    \item \textbf{Crossover a due punti}: date due soluzioni si tagliano i loro vettori di codifica in due punti predefiniti o casuali al fine di ottenere una coppia di teste\{$H_a, H_b$\}, parti centrale \{$I_a, I_b$\} ed code \{$T_a, T_b$\}. Le due soluzioni sono ottenute scambiando le due parti centrali nei genitori $S_1 = H_a \cup I_b \cup T_a , S_2 = H_b \cup I_a \cup T_b$;
    
    \item \textbf{Crossover uniforme}: consiste nello scambiare casualmente elementi tra le soluzioni candidate all'evoluzione;
    
    \item \textbf{Crossover aritmetico}: consiste nell'utilizzare un'operazione aritmetica per creare la nuova soluzione, ad esempio eseguendo una \textit{XOR} o una \textit{AND} tra elementi dei genitori se interpretati come una sequenza binaria;
\end{itemize}

\begin{figure}[htbp]
    \centering
    \includegraphics[width=\textwidth]{Immagini/"selezioneSP".jpg}
    \caption{Esempio di operatore di crossover}
\end{figure}

\subsubsection*{OPERATORI DI SELEZIONE}
\label{sec:OpSelezioneS}

A causa di complessi fenomeni di interazione non lineare, non è sempre vero che da due soluzioni promettenti ne nasca una terza \textit{migliore} né che da due soluzioni con valori di fitness basso venga generato un figlio \textit{peggiore}. Pertanto non è statisticamente conveniente utilizzare i soli elementi con valori di fitness elevata sia durante la scelta dei genitori che durante la scelta di quali elementi faranno parte della generazione successiva. Per quanto riguarda quest'ultimo caso, oltre al semplice valore di fitness, vengono prese in considerazione particolari tecniche di \textit{selezione}. Le più comuni sono:

\begin{itemize}
    \item \textbf{Selezione a roulette}: la probabilità che una soluzione venga scelta per far parte della successiva generazione è direttamente proporzionale al valore restituito dalla funzione di fitness. Immaginiamo quindi di avere a disposizione una roulette la cui ruota viene divisa in sezioni tutte assegnate ai vari candidati, la loro grandezza è quindi proporzionale all'idoneità dell'individuo. La selezione è banalamente ottenuta con molteplici rotazioni della roulette tenendo conto che un individuo non può essere selezionato più volte. Questa tecnica presenta dei problemi nel caso in cui le sezioni della ruota risultino tra loro eccessivamente sbilanciate in ampiezza, le soluzioni peggiori vengono selezionate troppo raramente e questo per quanto già esposto non è necessariamente un bene;
    
    \item \textbf{Selezione di Boltzmann}: le soluzioni vengono scelte con un grado di probabilità che, agli inizi dell'algoritmo, favorisce \textit{l'esplorazione} mentre più avanti tende a stabilizzarsi. Questa tecnica ritiene utile, in un primo momento, consentire agli individui meno idonei di riprodursi quasi quanto quelli migliori, e far procedere lentamente la selezione così da mantenere una certa diversità all'interno della popolazione. In seguito si rafforza la selezione per favorire maggiormente gli individui ad alta idoneità, presumendo che la fase iniziale, con grande diversità e poca selezione, abbia consentito alla popolazione di individuare la zona giusta nello spazio di ricerca;
    
    \item \textbf{Selezione a torneo}: da un pool di possibili soluzioni, nel caso più comune vengono scelti in modo del tutto casuale sia due individui che un numero $c \in [0, 1]$. Se quest'ultimo risulta minore di un parametro $k \in [0, 1]$ fissato, si seleziona il più idoneo tra i due candidati, altrimenti la scelta ricade sul peggiore. Naturalmente si procede fino a quando non ho tutti gli elementi per la generazione successiva.
\end{itemize}

Non esiste in assoluto un metodo migliore tra quelli proposti, molto dipende direttamente da come questi sono implementati e soprattutto sia dalla dimensione del problema che dalla quantità di vincoli imposti: ad esempio, nel caso in cui sia richiesto di trovare nel minor tempo possibile una \textbf{buona} soluzione è sconsigliato utilizzare la selezione di Boltzmann.

\subsubsection*{OPERATORI DI MUTAZIONE}

L'operatore di mutazione prevede che in funzione di una prefissata e usualmente piccola probabilità $p_{mutation}$, il valore di un bit del figlio venga cambiato: questo serve per simulare quanto avviene in natura dove, anche se raramente, è possibile che vi sia una variazione del genotipo durante l' evoluzione di un essere vivente.
La figura x.y illustra un esempio di mutazione.

\begin{figure}[htbp]
    \centering
    \includegraphics[width=\textwidth]{Immagini/"Mutazione".jpg}
    \caption{Esempio di mutazione}
\end{figure}

\subsection{ALGORITMO GENETICO TSP}
\label{sec:GeneticoTSPS}

Gli \textit{AG} risolvono un determinato problema generando sempre nuove \textbf{popolazioni} di soluzioni dove in genere troviamo una fitness media piuttosto bassa, giungendo solamente dopo diverse generazioni a valori più elevati. Per poter applicare un algoritmo genetico, occorre anzitutto codificare numericamente le soluzioni e individuare una opportuna funzione di fitness. La codifica vettoriale dei cromosomi più adatta per i problemi di TSP risulta essere un vettore di interi dove ogni elemento identifica in maniera univoca una delle città da visitare mentre il suo posizionamento identifica l'ordine di visita.
La funzione di fitness realizzata riceve in ingresso una soluzione \textbf{ammissibile}\footnote{Per qualsiasi generazione non sono quindi accettabili elementi non validi per il problema in questione. La funzione XXXX LINK!!! è stata realizzata a a tale scopo.} e restituisce un valore reale pari al reciproco del suo costo PERCHE'?????? LO DICIAMO DOPO???????.
La prima generazione viene ottenuta attraverso il metodo \textbf{NearestNeightborGenetic} dell classe \textbf{Utility}. Come facilmente intuibile genera soluzioni che tendono a collegare nodi tra loro vicini, per maggiori dettagli si consulti l'apposita sezione ad esso dedicata nella appendice LINKK!!!!!!!!!!!!!!!!!.
La generazione di un \textit{figlio} a partire da due \textit{genitori} avviene attraverso il crossover ad un punto già presentato dove però solamente una delle due soluzioni ottenute entra a far parte del pool di candidati per la successiva generazione. Il crossover viene fornito dal metodo \textbf{GenerateChild}, sempre appartenente alla classe \textbf{Utility}, che viene descritto nell'apposito paragrafo LINK!!!!!!!, mentre i restanti condidati alla nuova generazione sono gli elementi stessi della generazione precedente (motivo?!?!?!).
Come operatore di selezione, si è deciso di utilizzare la \textit{selezione a roulette} dato che le soluzioni hanno lo stesso ordine di grandezza per quanto riguarda i loro costi QUI NON E' CHIARO CHE SIGNIFICA---INTENDI CHE LE FETTE CHE VENGONO FUORI NON SONO TROPPO DIVERSE??????. Infine la mutazione avviene con probabilità $p_{mutex}$ pari all'1\%.
Quanto descritto viene per la maggior parte gestito attraverso il metodo NOME!!!!!!!!!!!!! con l'ausilio della classe \textbf{pathgenetic} descritta nella apposita sezione LINK!!!!!!!!!!!!!!!!!!!!!!!!!!!!.

\subsection*{FUNZIONE GENETICALGORITHM}
\label{sec:FunzioneGenAlgS}

Questa classe ha il compito di amministrare tutto tutte le fasi dell'algoritmo genetico. Per prima cosa si procede con una dichiariazione ed inizializzazione delle varie strutture dati necessarie. Tra queste troviamo due liste di \textit{PathGenetic} chiamate \textbf{OriginallyPopulated} e \textbf{ChildPoulation} che, durante tutto il processo, contengono rispettivamente l'insieme dei circuiti hamiltoniani che compongono la generazione $i-esima$ ed i figli da loro generati. La prima generazione viene ottenuta attraverso il metodo \textbf{NearestNeightborGenetic}, discusso in maggiore dettagli nella appendice LINK!!!!!, che, ricevendo la lista completa per ogni nodo di quali sono ad esso più vicini\footnote{Funzione già descritta qui LINK!!!!!!!!!!!!!!.}, costruisce le varie instanze tendendo a collegare i nodi più vicini tra loro\footnote{Il numero di componenti per ogni generazione viene chiesto in input all'utente e passato come parametro di ingresso alla funzione \textit{GeneticAlgorithm}.}.
Durante la generazione dei nuovi figli, poiché l'indice in cui è memorizzato un circuito all'interno di \textit{OriginallyPopulated}, dovuto in genere all'estrazione della rouellette, è casuale\footnote{Con casuale si intende che non è presente alcuna forma di correlazione fra l'indice e il costo della soluzione.} si è deciso di accoppiare circuiti memorizzati in celle adiacenti; la casualità di OriginallyPopulated consente di combinare fra loro soluzioni buone con altre meno buone e ciò statisticamente risulta particolarmente vantaggioso.
Una volta prodotti i figli è necessario procedere con la creazione della nuova generazione padre: questo viene eseguito dal metodo \textbf{NextPopulation} descritto in seguito. Per identificare il miglior circuito della generazione corrente si è realizzato il metodo \textbf{BestSolution}, qualora il valore ottenuto risulti minore dell’incumbent\footnote{Con incumbent si intende la miglio soluzione fin ora calcolata.} quest'ultimo viene aggiornato. L'algoritmo termina quanto scade il time limit fornito dall' utente.  Riportiamo di seguito il codice realizzato:

\begin{lstlisting}

PathGenetic incumbentSol = new PathGenetic();
PathGenetic currentBestPath = null;

List<PathGenetic> OriginallyPopulated = new List<PathGenetic>();
List<PathGenetic> ChildPoulation = new List<PathGenetic>();

List<int>[] listArray = Utility.BuildSLComplete(instance);

//Generate the first population
for (int i = 0; i < sizePopulation; i++)
OriginallyPopulated.Add(Utility.NearestNeightborGenetic(instance, rnd, true, listArray));
do
{
//Generate the children
for (int i = 0; i < sizePopulation; i++)
{
if (i % 2 != 0)
ChildPoulation.Add(Utility.GenerateChild(instance, rnd, OriginallyPopulated[i], OriginallyPopulated[i - 1], listArray));
}

OriginallyPopulated = Utility.NextPopulation(instance, sizePopulation, OriginallyPopulated, ChildPoulation);

//currentBestPath contains the best path of the current population
currentBestPath = Utility.BestSolution(OriginallyPopulated, incumbentSol);

if (currentBestPath.cost < incumbentSol.cost)
{
incumbentSol = (PathGenetic)currentBestPath.Clone();
Utility.PrintGeneticSolution(instance, process, incumbentSol);
}

// We empty the list that contain the child
ChildPoulation.RemoveRange(0, ChildPoulation.Count);

} while (clock.ElapsedMilliseconds / 1000.0 < instance.TimeLimit);

Console.WriteLine("Best distance found within the timelit is: " + incumbentSol.cost);

\end{lstlisting}

\subsection*{NEARESTNEIGHTBORGENETIC}
\label{sec:NearestNGS}

Il corrente metodo è simile alla funzione NearestNeight discussa nel paragrafo X.Y ma con due differenze significative:

\begin{itemize}
    \item La sequenza degli elementi nell array che codifica il percorso creato dal metodo indicano l'ordine con cui il circuito visita i nodi. Si ricorda invece che il metodo NearestNeight produce percorsi codificati in array in cui alla generica posizione \textbf{i} è collocato il nodo successivo al nodo \textbf{i}. Tale modifica è dettata solamente da una agevolazione nell'utilizzo successivo di queste informazioni da parte dell'algoritmo genetico nella creazione della prima generazione.
    
    \item Poiché un algoritmo genetico è tanto migliore quanto gli individui che formano la popolazione di partenza hanno caratteristiche dissimili fra di loro, si è fatto in modo che i circuiti fossero il più possibili diversi gli uni dagli altri.
\end{itemize}


L' intestazione del metodo risulta essere:

\begin{lstlisting}

public static PathGenetic NearestNeightborGenetic(Instance instance, Random rnd, bool rndStartPoint, List<int>[] listArray)

\end{lstlisting}


Dove:


\begin{itemize}
    \item \textbf{instance}: oggetto della classe \textit{Instance} contenente tutti i dati che descrivono l'istanza del problema del Commesso Viaggiatore fornita in ingresso dall' utente;
    \item \textbf{rnd}: istanza della classe \textit{Random} precedentemente inizializzato con un seme random diverso per ogni iterazione del programma;
    \item \textbf{rndStartPoint}: variabile booleana che determina se il nodo di partenza sul quale viene applicato l' algoritmo nearest neightbor risulta essere casuale (in tal caso assume il valore true) oppure sia il nodo di default 0;
    \item \textbf{listArray}: lista in cui all'indice \textbf{i} è presente un vettore di dimensione instance.Nnodes al cui interno sono, in ordine crescente rispetto alla distanza assunta dal nodo \textbf{i}, presenti gli indici associati ai nodi del grafo.
\end{itemize}

Il circuito prodotto dal metodo viene memorizzato all'interno del vettore \textbf{heuristicSolution} avente una dimensione pari al numero di nodi del grafo. Poiché si vogliono soluzioni che siano il più possibile dissimili tra loro, è consigliabile fare in modo che \textit{rndStartPoint} sia posta a \textit{true}\footnote{Da notare che tale parametro non è settata runtime ma solamente via hardcode.}. Il vettore \textbf{VisitedNodes} di tipo bool è un vettore di supporto che memorizza all'indice \textbf{i} il valore logico true se il nodo \textbf{i} è già stato visitato, false altrimenti.

\begin{lstlisting}

// heuristicSolution is the path of the current heuristic solution generate
int[] heuristicSolution = new int[instance.NNodes];

bool[] VisitedNodes = new bool[instance.NNodes];

int firstNode = 0;

//rndStartPoint define if the starting point is random or always the node 0 
if (rndStartPoint)
firstNode = rnd.Next(0, instance.NNodes);

heuristicSolution[0] = firstNode;
VisitedNodes[firstNode] = true;

\end{lstlisting}

Una volta definito il nodo di partenza i restanti nodi, che se visitati rispettando il loro ordine compongono un circuito hamiltoniano, son ottenuti attraverso un ciclo \textit{for}: alla generica iterazione \textbf{i} del ciclo, sfruttando la struttura dati listArray e la funzione \textbf{RndGenetic} si memorizza all'interno della variabile \textbf{nextNode} il nodo successivo visitato dal percorso sempre che questo sia ancora disponibile. Per verificarne la disponibilità si utilizza l'array \textit{VisitedNodes}, qualora non sia possibile utilizzare tale nodo si passa al successivo più vicino.


(la variabile contatore de ciclo for è inizializzata al valore 1, quindi il heuristicSolution[i-1] è memorizzato l' ultimo nodo visitato)  QUESTO VA COME COMMENTO NEL CODICE

\begin{lstlisting}

for (int i = 1; i < instance.NNodes; i++)
{
bool found = false;
int candPos = RndGenetic(rnd);
int nextNode = listArray[heuristicSolution[i - 1]][candPos];
do
{
//We control that the selected node has never been visited
if (VisitedNodes[nextNode] == false)
{
VisitedNodes[nextNode] = true;
heuristicSolution[i] = nextNode;
found = true;
}
else
{
candPos++;
if (candPos >= instance.NNodes - 1)
{
nextNode = listArray[heuristicSolution[i - 1]][0];
candPos = 0;
}
else
nextNode = listArray[heuristicSolution[i - 1]][candPos];
}

} while (!found);
}

\end{lstlisting}

\subsection*{GENERATECHILD}
\label{sec:GenerateChildS}

Per generare un figlio si è realizzato il metodo \textbf{GenerateChild}, appartenente alla classe Utility, avente la seguente intestazione:

\begin{lstlisting}

public static PathGenetic GenerateChild(Instance instance, Random rnd, PathGenetic mother, PathGenetic father, List<int>[] listArray)

\end{lstlisting}

Dove:

\begin{itemize}
    \item \textbf{instance}: oggetto della classe Instance contenente tutti i dati che descrivono l'istanza del problema del Commesso Viaggiatore fornita in ingresso dall' utente;
    \item \textbf{rnd}: istanza della classe \textit{Random} precedentemente inizializzato con un seme random diverso per ogni iterazione del programma;
    \item \textbf{father}: circuito hamiltoniamo che sarà accopiato con il parametro mather;
    \item \textbf{mother}: hamiltoniamo che sarà accopiato con il parametro father;
    \item \textbf{listArray}: lista in cui all’ indice i è presente un vettore di dimensione instance.Nnodes al cui interno sono, in ordine crescente rispetto alla distanza assunta dal nodo i, presenti gli indici associati ai nodi del grafo.
\end{itemize}

Come precedentemente accennato il seguente metodo produce un figlio utilizzando l'operatore di crossover a singolo punto.

\begin{lstlisting}

int crossover = (rnd.Next(0, instance.NNodes));

for (int i = 0; i < instance.NNodes; i++)
{
if (i > crossover)
pathChild[i] = mother.path[i];
else
pathChild[i] = father.path[i];
}

\end{lstlisting}


Una volta creato il figlio, con una probabilità p = 0.01 viene effettuata su di esso una mutazione utilizzando il metodo \textbf{Mutation} LINK?!!?!?.


\begin{lstlisting}

if (rnd.Next(0, 101) == 100)
Mutation(instance, rnd, pathChild);

\end{lstlisting}

Il figlio ottenuto quasi certamente non risulta essere un circuito ammissibile, per tale motivo si è progettato il metodo \textbf{Repair}. \'E interessante far notare che quest'ultimo non cerca di modificare i circuiti non modo da abbassarne il più possibile il costo ma al contrario è stato costruito in modo tale da risultare il più veloce possibile: soprattutto per popolazioni numerose e time limit alti tale scelta risulta essenziale. Di conseguenza i figli così prodotti sono tipicamente caratterizzati da un costo che con bassa probabilità risulta migliore rispetto a quello dei genitori e ciò comporta una saturazione dell'algoritmo dopo poche iterazioni (PERCHè?!?!?!?! METTERLO COME FOOTNOTE). Come operazione finale si è quindi deciso di applicare su di essi l'algoritmo \textbf{TwoOpt} ma solamente con una probabilità $p = 1 / (n/2)$, dove n è il numero dei nodi. La ragione per cui $p$ assume tale valore è dovuta al fatto che l'operazione di due ottimalità ha complessità computazionale O()[quando faccio il multistart controllo meglio quanto è la sua complessità] ed una sua applicazione più frequente rallenterebbe troppo l'algoritmo.

\begin{lstlisting}

if (ProbabilityTwoOpt(instance, rnd) == 1)
{
child.path = InterfaceForTwoOpt(child.path);
TSP.TwoOpt(instance, child);
child.path = Reverse(child.path);
}

\end{lstlisting}

Da notare che la funzione per ottenere la due ottimalità è utilizzata da più metodi di risoluzione (multi-start ecc..) e necessita che il percorso della soluzione hamiltoniana sia memorizzato con un apposito formato diverso da quello presentato per l'algoritmo genetico: sono quindi necessarie due semplici interfacce \textbf{InterfaceForTwoOpt, Reverse}, i cui tempi di esecuzione sono $O(n)$.

\subsection*{NEXTPOPULATION}
\label{sec:NextPopulationS}

La funzione \textbf{NextPopulation}, appartenente alla classe \textbf{Utility}, consente di definire la nuova generazione scegliendone gli elementi tra la vecchia generazione ed i suoi figli attraverso una estrazione a roulette. La firma di tale funzione risulta essere:

\begin{lstlisting}

public static List<PathGenetic> NextPopulation(Instance instance, int sizePopulation, List<PathGenetic> FatherGeneration, List<PathGenetic> ChildGeneration)

\end{lstlisting}

Dove:

\begin{itemize}
    
    \item \textbf{instance}: oggetto della classe Instance contenente tutti i dati che descrivono l'istanza del problema del Commesso Viaggiatore fornita in ingresso dall' utente;
    \item \textbf{sizePopulation}: parametro che indica di quanti elementi deve essere la nuova generazione, per come è stato costruito il nostro programma questo parametro non varia mai ed è richiesto una sola volta all'utente;
    \item \textbf{FatherGeneration}: Lista contenente i circuiti che definiscono la generazione corrente;
    \item \textbf{ChildGeneration}: Lista contenente i circuiti figli generati da FatherGeneration utilizzando la funzione GenerateChild.
    
\end{itemize}

Per prima cosa uniamo le due liste di circuiti in quanto si è deciso che l'unico metro di giudizio durante la selezione deve essere il valore di fitness attribuito ad ogni soluzione indipendentemente dalla loro provenienza.

\begin{lstlisting}

for (int i = 0; i < ChildGeneration.Count; i++)
FatherGeneration.Add(ChildGeneration[i]);

\end{lstlisting}

Passiamo ora a descrivere come è gestita la selezione a roulette, chiaramente esistono molteplici metodi e quello da noi scelto non ha alcun vantaggio significativo rispetto agli altri. L'idea alla base dell'algoritmo è di assegnare un valore univoco ad ogni circuito estraibile e di utilizzare tali valori molteplici volte come caselle della roulette, implementata come una lista di interi. Il numero di inserimenti per ogni valore è direttamente proporzionale alla fitness del circuito a cui è associato. Tutto questo viene per la maggior parte gestito all'interno del metodo \textbf{FillRoulette} LINK!!!!!!!!!!!!!!!!!, sempre definito nella classe \textbf{Utility}, che restituisce inoltre la grandezza della roulette così creata e poi memorizzato nella variabile \textbf{upperExtremity}:

\begin{lstlisting}

List<PathGenetic> nextGeneration = new List<PathGenetic>();
List<int> roulette = new List<int>();
Random rouletteValue = new Random();
int upperExtremity = FillRoulette(roulette, FatherGeneration);

\end{lstlisting}

La creazione della nuova generazione avviene estraendo valori random non superiori ad \textit{upperExtremity}, questi forniscono gli indici della lista-roulette le cui posizioni indicano indirettamente quale circuito deve far parte della nuova generazione. Naturalmente è possibile estrarre più volte la stessa soluzione, in questo caso la cosa va ignorata ripetendo nuovamente il processo. Nel complesso si dovranno estrarre (\textit{sizePopulation}) circuiti hamiltoniani.

\begin{lstlisting}

List<int> NumbersExtracted = new List<int>();
bool find = false;
int numberExtracted;

for (int i = 0; i < instance.SizePopulation; i++)
{
do
{
find = true;
numberExtracted = rouletteValue.Next(0, upperExtremity);
//A path can't be extracted more than one time
if (NumbersExtracted.Contains(roulette[numberExtracted]) == false)
{
find = false;
NumbersExtracted.Add(roulette[numberExtracted]);
nextGeneration.Add(FatherGeneration.Find(x => x.NRoulette == roulette[numberExtracted]));
}
} while (find);
}
return nextGeneration;

\end{lstlisting}

\section*{MATHEURISTICS}
\label{sec:MathEuristiciS}

Gli algoritmi \textbf{matheuristics} nascono con l'obbiettivo di migliorare una soluzione di partenza ammissibile \textbf{x} sfruttando il \textbf{modello matematico} del problema (ogni tanto parli di MIP, ma si intende modelli misti interi e frazionari, è giusto usarlo???) che si vuole risolvere. Analogamente a quanto visto per gli algoritmi di ricerca locale e metaeuristici, i matheuristici tentano di individuare una soluzione migliore \textbf{x*} all' interno di un intorno N(x) ottenuto, in questo ambito, modificando opportunamente il modello matematico di partenza. 

L'esplorazione di N(x) non avviene per enumerazione come visto in precedenza ma tramite appositi solver come ad esempio CPLEX: grazie alla loro sempre maggiore ottimizzazione, questa operazione risulta quindi essere molto veloce. 

I matheuristici possono essere applicati a qualunque soluzione ammissibile x; ciò nonostante per apprezzarne davvero la potenza è consigliabile utilizzare un punto di partenza già buono. A tale scopo si può quindi pensare di concatenarli  all'esecuzione di un algoritmo migliorativo o, meglio ancora, al termine del multistart o di un metaeuristico. Questi infatti arrivano molto frequentemente a saturare in prossimità di una soluzione molto buona \textbf{$\overline{x}$} ma non ottima.

Sfruttando questa combinazione di algoritmi è stato dimostrato che già in un breve lasso di tempo si riescono ad ottenere sostanziali miglioramenti ed in alcuni casi è possibile raggiungere anche l'ottimo globale. Si osserva che esiste una notevole differenza fra l'ottenere quest'ultimo risultato grazie ad un algoritmo esatto e seguendo il procedimento precedentemente discusso: mentre nel primo caso è sempre certificato, nel secondo questo non avviene\footnote{Il solver utilizzato garantisce solamente l'ottimalità del modello da noi fornitogli che chiaramente non può essere quello originale del problema.} e non si ha nemmeno modo di certificarlo a meno che non lo si conosca a priori.

A questo punto è chiaro come il fattore caratterizzante di un algoritmo matheuristico sia come questo definisce l'intorno della soluzione di partenza sul quale si suppone possano essere localizzati dei miglioramenti. 

Uno schema comunemente utilizzato prende il nome di \textbf{hard variable fixing} e prevede di fissare il valore di un certo numero di variabili, rendendole di fatto costanti, mentre le restanti vengono lasciate libere. Lo svantaggio principale derivante da questa tecnica è che a priori non esistono indicazioni su cosa è più opportuno fissare e cosa lo è meno. Di conseguenza esiste una sua variante detta \textbf{soft variable fixing} che lascia al solver questa responsabilità indicandogli solamente con quale proporzione deve avvenire la cosa.

Nei prossimi paragrafi sono riportanti alcuni esempi di algoritmi matheuristici che si basano sui concetti appena esposti ricordando che come per qualsiasi metodo euristico non esistono varianti più o meno efficaci in assoluto ma molto dipende dal problema che si sta analizzando e da fattori casuali non calcolabili.

\subsection*{HARD FIXING}
\label{sec:HardFixingS}

L'hard fixing è una tecnica euristica che, partendo da una qualsiasi soluzione ammissibile \textbf{$x_H$}, tenta di trovarne una migliore all'interno di un suo intorno. Quest'ultimo viene definito dall'algoritmo stesso attraverso il fissaggio a priori di alcune variabili del modello matematico, rendendole così di fatto costanti, utilizzando i valori che queste assumo in $x_H$. 

Come già discusso nel paragrafo precedente, una volta definito l'intorno e quindi un nuovo modello matematico, la sua risoluzione viene affidata ad un solver esterno pertanto l'unica fase saliente dell'algoritmo è la scelta stessa di quale porzione di $x_H$ mantenere valida e quale no. 

Esistono chiaramente molteplici approcci ammissibili e nessuno di questi risulta migliore in assoluto. Come spesso accade negli algoritmi euristici la scelta più semplice produce risultati molto buoni e quindi alcune varianti dell'\textit{hard fixing} semplicemente fissano un qualunque lato con probabilità $p$ o meno con probabilità $1-p$\footnote{Ovviamente $p$ è direttamente proporzionale a quanti lati sul loro totale vogliamo fissare.}.

Altri approcci invece tentano di definire un sistema di ranking come ad esempio dare più valore agli elementi con costo minore.

Indipendentemente da quale tecnica si decida di utilizzare maggiore è il numero dei lati fissati, minore è il carico di lavoro per risolvere il rilassamento continuo derivante ma allo stesso tempo diviene minore anche l'ampiezza dell'intorno che si va a sondare.

Nel complesso quindi è buona norma tentare sì molteplici fissaggi ma anche variarne il numero cercando di trovare la miglior combinazione per il problema in questione.

Nel nostro caso si è deciso di optare per la prima variante andando progressivamente a diminuire il valore di $p$. Il motivo di tale scelta è che il nostro algoritmo non ha un target specifico di istanze aventi una struttura ben definita dove diventa quindi evidente un sistema di ranking efficiente.

Preferiamo quindi ampliare il range di possibili intorni da visitae dando però sempre la possibilità all'algoritmo di trovare la soluzione ottima. Di conseguenza più è alto il tempo limite che gli andiamo a concedere maggiori sono le probabilità di avvicinarsi al risultato desiderato.

\subsection*{WARM-START}
\label{sec:WarmStartS}

Tra le svariate funzionalità offerta da CPLEX possiamo trovarne una in particolare che è sempre consigliato utilizzare quando possibile e nello specifico risulta essere una naturale aggiunta alla tecnica dell'\textbf{hard-fixing}. Stiamo parlando della possibilità di fornire a CPLEX una soluzione al problema che dovrà andare a risolvere in modo tale che, una volta controllata la sua validità durante la fase di prepocessing, possa essere utilizzata come incumbent. Questa soluzione prende il nome di \textbf{warm-start} o \textbf{MIP-start} e chiaramente quando si procede al fissaggio di alcune variabili in accordo ad $x_H$, questa rimane ovviamente valida per il problema e quindi è una naturale candidata per questo ruolo.

CPLEX trae diversi potenziali vantaggi grazie ad un \textbf{warm-start}: prima di tutto sfruttando il criterio di bounding, avere a disposizione un buon incumbent fin dai primi momenti della tecnica del \textit{branch\&cut} può velocizzare la costruzione dell'albero decisionale dichiarandone alcuni nodi come sondati; In secondo luogo permette di applicare particolare tecniche euristiche che necessitano di questa condizione di partenza, parliamo ad esempio degli algoritmi \textbf{RINS} e \textbf{polishing} che saranno discussi successivamente anche in questa tesi.

In linguaggio \textbf{C\#} l'operazione di warm-start avviene attraverso il metodo non statico \textbf{AddMIPStart} della classe \textit{CPLEX} la cui firma è:

\begin{lstlisting}

public virtual int AddMIPStart(INumVar[] vars,double[] values)

\end{lstlisting}

Dove:

\begin{itemize}
    \item \textbf{vars}: Vettore contenente nella posizione $i$ la $i-esima$ variabile del modello;
    \item \textbf{values}: Vettore contenente nella posizione $i$ il valore della $i-esima$ variabile del modello;
\end{itemize}

Si osserva inoltre che nel caso in cui si attuino consecutive esecuzione dell'\textit{hard-fixing} non è più necessario aggiornare manualmente il valore di \textit{warm-start}\footnote{Se non alla prima run.}: variare solamente i fissaggi delle variabili, senza modificare ulteriormente il modello matematico, non viene considerata da CPLEX una operazione \textit{drastica} al punto di abbandonare i risultati ottenuti fino a quel momento.

In altre parole, nel caso in cui non sia trovata una soluzione migliore rispetto al \textit{warm-start} attuale la validità di quest'ultimo viene automaticamente ricontrollata durante la successiva fase di preprocessing\footnote{La sua validità è sempre garantita in quanto siamo nel caso in cui sì applichiamo un fissaggio diverso ma rispetto la stessa soluzione di partenza dato che non ne sono state trovate di migliori.} effettuata dal solver. Altrimenti nel caso in cui l'ultimo rilassamento continuo abbia individuato un nuovo valore per l'incumbent, e quindi un nuovo \textit{warm-start}, CPLEX provvede autonomamente al suo aggiornamento.

\subsection*{PREPROCESSINGTSP}
\label{sec:PreProcessingTSPS}

Una volta defenito l'intorno \textbf{$N(x_I)$} e quindi aver fissato il \textit{lower bound} di alcune variabili a $1$ è possibile procedere immediatamente alla risoluzione del problema modificato attraverso il solver, nel nostro caso CPLEX. Come ben noto, nel caso più generico, il pool di tagli è inizialmente vuoto e solo quelli necessari, non determinabili a priori, sono individuati attraverso le callback installate.

Nel caso dell'\textbf{hard-fixing} questa affermazione non è del tutto esatta in quanto \textbf{alcuni} dei tagli \textbf{necessari}\footnote{Per necessari si intende che sicuramente } possono essere trovati molto velocemente attraverso una analisi del fissaggio effettuato. Una loro aggiunta preventiva risulta quindi molto vantaggiosa in quanto si evitano certamente delle iterazioni al solver risparmiando nel complesso diverso tempo.
In \hyperref[fig:Preprocessiong1]{FiguraX} troviamo una generica situazione dove gli archi in \textit{blu} sono stati \textit{fissati} mentre quelli in \textit{azzurro} no. Proprio questi ultimi però verrebbero immediatamente selezionati dal solver in quanto la soluzione con i due subtour indicati risulta banalmente la meno costosa. Il compito del preprocessingTSP (\textbf{ppTSP}) da noi progettato è quindi proprio questo, trovare tutti quei lati che, se anche selezionati singolarmente, generano un subtour\footnote{Dato che non ha senso fissare $n-1$ lati su un totale di $n$, se la singola selezione di un lato genera un subtour, qualsiasi siano le restanti selezioni, la soluzione così prodotta avrà sempre almeno un secondo subtour e quindi risulta invalida.} e di conseguenza forzarli a valore $0$.

\begin{figure}[htbp]
    \centering
    \scalebox{.75}{\includegraphics{Immagini/"PreProcessing".jpg}}
    \label{fig:Preprocessiong1}
    \caption{Preprocessing}
\end{figure}

A livello algoritmo l'operazione di \textit{ppTSP} è molto semplice da realizzare. Sfruttiamo la stessa tecnica utilizzata nel corso del progetto per individuare le componenti connesse di una \textbf{soluzione} proposta. Queste nel nostro caso non risulteranno mai essere un ciclo completo per costruzione e quindi si possono sempre individuare le coppie di nodi posti alle loro estremetà: i lati che collegano i due elementi appartenenti alla stessa coppia sono esattamente quelli da forzare a valore nullo\footnote{Questo discorso è sempre valido a meno che la componente connessa non ciclica trovata sia composta da un solo lato, in quel caso non è richiesta nessuna operazione.}.

Una definizione equivalente di quanto stiamo cercando sono tutti quei nodi posti alla estremi di un solo lato che è stato fissato. Proponiamo ora il codice commentato che sfrutta quest'ultima affermazione e permette di attuare la completa fase di \textit{ppTSP}:

\begin{lstlisting}

public static void PreProcessingTSP(Instance instance, INumVar[] x)

\end{lstlisting}

Dove:

\begin{itemize}
    \item \textbf{instance}: riferimento all' oggetto contenente tutte le informazioni relative all' istanza del Problema del Commesso Viaggiatore corrente;
    \item \textbf{x}: vettore contente le variabili del modello matematico;
\end{itemize}

\begin{lstlisting}

//Questo vettore tiene traccia all'indice i-esimo di quante volte il nodo i è estremo di un lato fissato, opzioni valide sono 0, 1, 2
int[] cntNode = new int[instance.NNodes];

//Questo vettore tiene traccia di quale componente connessa fa parte ogni nodo
int[] compConn = new int[instance.NNodes];

//Questo vettore di liste di interi è la struttura dati di supporto dove infine ogni lista contenente due elementi sarà relativa ad una singola
//componente connessa e immagazzinerà al suo interno gli indici dei nodi posti alle sue estremità. Chiaramente tale informazione è facilmente
//ricavabile dai due precedenti vettori cntNode e compConn ma con un costo complessivo più alto.
List<int>[] externalNodes = new List<int>[instance.NNodes];

//Classica inizializzazione delle componenti connesse, ad ogni nodo è assegnato ne è assegnata una propria
Utility.InitCC(compConn);

//Inizializzazione di externalNodes
for (int i = 0; i < instance.NNodes; i++)
    externalNodes[i] = new List<int>();

for (int i = 0; i < instance.NNodes; i++)
{
    for (int j = i + 1; j < instance.NNodes; j++)
    {
        //Trovo indice corretto del lato con estremi i nodi di indice i e j
        int position = Utility.xPos(i, j, instance.NNodes);

        //Nel caso sia stato fissato dall'hard-fixing lo si analizza
        if (x[position].LB == 1)
        {
            //Aggiornamento del contatore relativo ai nodi i e j
            cntNode[i] += 1;
            cntNode[j] += 1;

            //Aggiorno componenti connesse seguendo la tecnica di Kruskal
            for (int k = 0; k < instance.NNodes; k++)
            {
                if ((k != j) && (compConn[k] == compConn[j]))
                    compConn[k] = compConn[i];
            }

            compConn[j] = compConn[i];
        }
    }
}

//Popoliamo correttamente externalNodes in modo tale che solamente i due nodi estremi della stessa componente connessa
//siano inserite all'interno della stessa lista
for(int i = 0; i< instance.NNodes; i++)
{
    if (cntNode[i] == 1)
        externalNodes[compConn[i]].Add(i);
}

//Controlliamo quali elementi del vettore di liste appena popolato risultano avere esattamente due elementi ed eseguiamo
//l'operazione fondamentale del preprocessing discussa
for (int i = 0; i < instance.NNodes; i++)
{
    if (externalNodes[i].Count == 2)
    {
        //Calcolo l'indice del lato di cui voglio forzare l'upper bound a 0
        int pos = xPos(externalNodes[i][0], externalNodes[i][1], instance.NNodes);

        //Se il lato analizzato era in precedenza stato fissato ad uno dall'algoritmo di hard fixing, quindi il suo lower bound
        //è pari ad 1, non lo andiamo a toccare in quanto significa che la componente connesse non ha altri elementi
        if (x[pos].LB == 0)
            x[pos].UB = 0;
    }
}

\end{lstlisting}

\subsection*{IMPLEMENTAZIONE HARD FIXING}
\label{sec:ImplHardFixingS}

\begin{lstlisting}

static void HardFixing(CPLEX cplex, Instance instance, Process process, Random rnd, Stopwatch clock)

\end{lstlisting}

Dove:

\begin{itemize}
    \item \textbf{cplex}: riferimento all'oggetto contenente il modello da risolvere;
    \item \textbf{instance}: riferimento all'oggetto contenente tutte le informazioni relative all'istanza del Problema del Commesso Viaggiatore corrente;
    \item \textbf{process}: riferimento all'oggetto necessario per la stampa dei dati attraverso \textit{GNUPlot};
    \item \textbf{rdn}: istanza della classe Random utilizzata per la generazione dei numeri casuali;
    \item \textbf{clock}: riferimento all'oggetto utilizzato come cronometro, deve già essere stato avviato;;
\end{itemize}

Di seguito troviamo il codice del metodo \textbf{commentato}. Facciamo però presente che l'\textit{hard fixing} può ricevere in ingresso una qualsiasi soluzione di partenza, preferibilmente già buona, sulla quale vogliamo definire un intorno di ricerca. Nella versione base dell'algoritmo riportato viene utilizzato il modo più veloce per ottenere tale risultato e quindi attraverso la funzione \textbf{NearestNeightbor} LINK!!!!!!!!!!!!!!! con conseguente algoritmo di due ottimalità.

Nei test a fine capitolo vedremo invece una variante che sfrutta soluzioni già precedentemente ottenute da altri algoritmi euristici attraverso loro tempi di esecuzione infinitamente più alti.

\begin{lstlisting}
//Oggetto utilizzato per la scrittura su file dei vari incumbent trovati in modo tale che GNUPlot possa stamparne il ciclo a video
StreamWriter file;

//Vettore che vuole essere sempre aggiornato al miglior ciclo trovato
double[] currentIncumbentSol = new double[(instance.NNodes - 1) * instance.NNodes / 2];

//Variabile sempre aggiornata al costo del miglior ciclo trovato
double currentIncumbentCost = Double.MaxValue;

//All'indice i-esimo troviamo in ordine crescente quali sono i nodi più vicini all'i-esimo nodo
List<int>[] listArray = Utility.BuildSLComplete(instance);

//Variabile booleana che comunicherà alla lazy callback utilizzata di non stampare a video tutte gli aggiornamenti dell'incumbent che trova
bool BlockPrint = false;

//Indica dopo quante consecutive esplorazioni di diversi intorni sia necessario ampliare l'ampiezza dei successivi
const int VALUECONSITENOTIMPROV = 3;

//Variabile costantemente aggiornata a quante esplorazioni consecutive non hanno portato ad una soluzione migliore, inizialmente si parte da VALUECONSITENOTIMPROV
int consecutiveiterationNotImprov = VALUECONSITENOTIMPROV;

//La percentuale inizile p che ha ogni lato di essere fissato
double percentageFixing = 0.8;

//Semplice inizializzazione del vettore che andrà a contenere la selezione o meno di ogni lato nella soluzione migliore finale
instance.BestSol = new double[(instance.NNodes - 1) * instance.NNodes / 2];

//Creo il modello matematico di partenza
INumVar[] x = Utility.BuildModel(cplex, instance, -1);

//Utilizzo del NearestNeightbor per iniziare a costruire la soluzione di partenza
PathStandard heuristicSol = Utility.NearestNeighbor(instance, rnd, listArray);

//Applicazione del 2-Opt per completare la soluzione di partenza
TwoOpt(instance, heuristicSol);

//La soluzione di partenza è settata come attuale incumbent
for (int i = 0; i < instance.NNodes; i++)
{
    int position = Utility.xPos(i, heuristicSol.path[i], instance.NNodes);
    currentIncumbentSol[position] = 1;
}
//Il costo della soluzione di partenza è settato come il miglior costo attuale
currentIncumbentCost = heuristicSol.cost;

//Installazione lazy callback per la risoluzione del modello
TSPLazyConsCallback tspLazy = new TSPLazyConsCallback(cplex, x, instance, process, BlockPrint);
cplex.Use(tspLazy);

//Forniamo a CPLEX il warm-start
cplex.AddMIPStart(x, currentIncumbentSol);

//Setto i thread utilizzati da CPLEX pari al numero di core virtuali della macchina utilizzata, il codice della Lazy è thread-safe
cplex.SetParam(CPLEX.Param.Threads, cplex.GetNumCores());

//La polita da noi scelta prevede che al primo miglioramento ottenuto durante l'esplorazione dell'intorno attuale questa venga fermata in modo da ripetere il procedimento
//in modo da ripetere il procedimentocon il nuovo valore di incumbent. Per ottenere questo risultato comunichiamo a CPLEX che alla seconda soluzione intera trovata fermi
//la propria risoluzione. Notiamo che la prima soluzione intera è il warm-start
cplex.SetParam(CPLEX.LongParam.IntSolLim, 2);

do
{
    //Modifichiamo il modello introducento il fissaggio di alcune variabili creando in questo modo un intorno della soluzione migliore attuale
    Utility.ModifyModel(instance, x, rnd, percentageFixing, currentIncumbentSol);

    //Risolviamo il modello attuale
    cplex.Solve();

    //Se otteniamo un miglioramento
    if (currentIncumbentCost > cplex.GetObjValue(CPLEX.IncumbentId))
    {
        //Preparo il file
        file = new StreamWriter(instance.InputFile + ".dat", false);

        //Aggiorno le variabili con l'attuale percorso e costo incumbent
        currentIncumbentCost = cplex.GetObjValue(CPLEX.IncumbentId);
        currentIncumbentSol = cplex.GetValues(x, CPLEX.IncumbentId);

        //Stampa del nuovo percorso incumbent su file
        for (int i = 0; i < instance.NNodes; i++)
        {
            for (int j = i + 1; j < instance.NNodes; j++)
            {
                int position = Utility.xPos(i, j, instance.NNodes);

                if (currentIncumbentSol[position] >= 0.5)
                {
                    file.WriteLine(instance.Coord[i].X + " " + instance.Coord[i].Y + " " + (i + 1));
                    file.WriteLine(instance.Coord[j].X + " " + instance.Coord[j].Y + " " + (j + 1) + "\n");
                }
            }
        }
        //Chiusura del file di tipo StreamWriter, il flush dal buffer viene eseguito
        file.Close();

        //Stampa attraverso GNUPLot del nuovo incumbet path
        Utility.PrintGNUPlot(process, instance.InputFile, 1, currentIncumbentCost, -1);

        //Resettiamo il numero di successivi non miglioramenti
        consecutiveiterationNotImprov = VALUECONSITENOTIMPROV;
    }
    else
    {
        //Il numero di consecutivi non miglioramenti è decrementato
        consecutiveiterationNotImprov--;
    }

    //Se ho raggiunto il limite di consecutivi non miglioramenti
    if (consecutiveiterationNotImprov == 0)
    {
        //Se la percentuale di fissaggio di un nodo è ancora superiore al 20% la diminuisco
        if (percentageFixing > 0.2)
        {
            //Diminuzione della percentuale di fissammento di un nodo di un 10%
            percentageFixing -= 0.1;
            //Resetto il numero di consecutivi non miglioramenti in seguito alla variazione dell'ampiezza dell'intorno da analizzare
            consecutiveiterationNotImprov = VALUECONSITENOTIMPROV;
        }
    }

    //Elimino l'ultimo fissaggio delle variabili in preparazione al prossimo 
    for (int i = 0; i < x.Length; i++)
    {
        x[i].LB = 0;
        x[i].UB = 1;
    }

} while (clock.ElapsedMilliseconds / 1000.0 < instance.TimeLimit);
//Ripeto il tutto fino a che non scade il timelimit, l'ultima iterazione può sforarlo fino a che non è completa

//Aggiorno la variabili instance con la soluzione migliore finale ed il relativo costo
instance.XBest = currentIncumbentCost;
instance.BestSol = currentIncumbentSol;

//Stampo righe vuota nell'output standard di CPLEX
cplex.Output().WriteLine();
cplex.Output().WriteLine();
//Stampo il costo della miglior soluzione trovata nell'output standard di CPLEX
cplex.Output().WriteLine("x = " + instance.XBest + "\n");
\end{lstlisting}

\subsection*{LOCAL BRANCH}
\label{sec:LocalBranchS}

Terminata l'analisi di una possibile variante della tecnica \textit{hard-fixing} proseguiamo con la tipologia gemella del \textbf{soft-fixing}: si ricordi che la principale differenza tra i due approcci è che il secondo lascia al solver, nel nostro caso CPLEX, anche l'operazione di fissaggio indicandogli solamente la forma che questo deve assuemere.

Nello specifico prendiamo in considerazione una particolare tipologia di algoritmi facente parte del ramo \textit{soft-fixing} e cioè il \textbf{Local Branch}: proposto nell'anno 2002 dai docenti universitari \textbf{Matteo Fischetti} \footnote{Docente presso l' Università di Padova, Dipartimento di Ingegneria dell’ Informazione.} e \textbf{Andrea Lodi} \footnote{Docente presso l' Università di Bologna, Dipartimento di Ingegneria dell'Energia Elettrica e dell'Informazione.}, questa variante mira ad ottenere l'esplorazione di un intorno $N(x,r)$ di una soluzione $x$ con un costo computazionale estremamente inferiore a $O(n^r)$ che invece abbiamo visto essere necessario per un classico algoritmo di \textit{r-ottimalità}.

Il \textit{Local Branch} non utilizza la classica tecnica di eplorazione per enumerazione di $N(x,r)$ ma introduce una specifica disequanzione al modello matematico del problema \textit{TSP}:


\begin{equation}\label{eq:LocalBranchVincoloAsimmetrico}
\displaystyle\sum_{e \text{ } : \text{ } x_e = 1} x_e^{*} \ge n - r
\end{equation}

Attraverso questo vincolo è il solver stesso a generare un intorno $N(x,r)$ sul quale cercare la soluzione migliore al suo interno. In altre parole, nel nostro esempio, sfruttiamo la velocità offerta da CPLEX nel risolvere un modello matematico e costruiamo quest'ultimo così che ammetta tutte le soluzioni a loro volta valide per un algoritmo di \textit{r-ottimalità}.

Prendendo più in esame \eqref{eq:LocalBranchVincoloAsimmetrico}, è possibile notare come la sua elevata potenza sia in pieno contrasto alla sua estrema semplicità. Tra tutte le variabili disponibili, sono considerate solamente quelle facenti parte della soluzione $x$ attuale\footnote{Il cui numero totale pari al numero di nodi indicato con $n$.} e si impone che almeno $n-r$ dovranno far parte della nuova soluzione prodotta dal solver. Sono quindi ammessi fino a $r$ scambi tra variabili in soluzione e non, producendo di fatto \textit{gratuitamente} un'operazione di $r-ottimalità$.

La forma espressa dalla \eqref{eq:LocalBranchVincoloAsimmetrico} prende il nome di formulazione \textbf{asimentrica} in quanto non appaiono direttamente le variabili non selezionate da $x$. \'E presente anche una sua variante \textbf{simmetrica} che risulta essere sia più esplicita che maggiormente \textit{complessa}:

\begin{equation}\label{eq:LocalBranchVincoloSimmetrico}
\displaystyle\underbrace{\sum_{j\text{ } : \text{ } x_j = 0} x_j^{*}}_{\text{\# variabili cha passano da 0 a 1}} + \underbrace{\sum_{j\text{ }:\text{ } x_j = 1}(1 - x_j^{*})}_{\text{\# variabili cha passano da 1 a 0}} \le r
\end{equation}

Come si può vedere da una breve analisi le due forme \eqref{eq:LocalBranchVincoloAsimmetrico} e \eqref{eq:LocalBranchVincoloSimmetrico} sono del tutto equivalenti, nello specifico \eqref{eq:LocalBranchVincoloSimmetrico} definisce la massima distanza di \textbf{Hamming}\footnote{La distanza di \textit{Hamming} fra due vettori di pari dimensione, corrisponde al numero di posizioni aventi simboli corrispondenti diversi.} che può sussistere fra \textbf{$x^*$} e \textbf{$x$}\footnote, rispettivamente la nuova e la attuale soluzione.

Esponiamo due concetti riguardanti le due formulazioni proposto. Come prima cosa \eqref{eq:LocalBranchVincoloSimmetrico}, al contrario di \eqref{eq:LocalBranchVincoloAsimmetrico}, nel caso in cui voglia simulare una operazione $k-Opt$ deve porre $r = 2*k$ in quando devo indicare sia i lati che aggiungo rispetto a $x$ ma anche quelli che tolgo da quest'ultima.

In secondo luogo la decisione di quale delle due formulazioni adottare dipende dal tipo di problema: notiamo infatti che in \eqref{eq:LocalBranchVincoloSimmetrico} appaiono tutte le variabili del problema mentre in \eqref{eq:LocalBranchVincoloAsimmetrico} solamente quelle facenti parte della soluzione attuale. Nel caso in cui queste due quantità risultino paragonabili allora le due formulazioni sono entrambe valide. In caso contrario, come ad esempio i modelli di TSP analizzati in questa tesi dove la differenza è di un ordine di grandezza, \eqref{eq:LocalBranchVincoloSimmetrico} risulta molto più densa di \eqref{eq:LocalBranchVincoloAsimmetrico} e quindi più difficilmente gestibile dal solver.

Concludiamo questa introduzione teorica con una considerazione sui valori assegnabili ad $r$ nella \eqref{eq:LocalBranchVincoloAsimmetrico}. Sappiamo che una operazione di \textit{k ottimalità} comprende anche tutte le possibili $m-Opt$ dove $k>m$ pertanto più alto viene posto il valore di $r$ più diventa ampio l'intorno $N(x,r)$ dove andiamo a cercare la nuova soluzione e quindi le possibilità di ottenere un risultato migliore.
Naturale conseguenza è anche una maggiore complessità computazionale sia nel caso di una banale esplorazione per enumerazione di $N(x,r)$ sia attraverso l'uso del \textit{Local Branching}. Non è mai pertanto consigliabile oltreppassare certi limiti, direttamente dipendenti dal tipo di macchina sulla quale vengono fatti eseguire gli algoritmi, in quanto si viene a parte la fondamentale caratteristica degli euristici: la loro velocità.

\subsection*{IMPLEMENTAZIONE LOCAL BRANCHING}
\label{sec:ImplementazioneLocalBranchS}

Presentiamo ora il codice \textbf{commentato} che realizza la tecnica \textit{Local Branching} in versione \textbf{asimmetrica} per le motivazioni esposte nel precedente paragrafo.

Come per il metodo \textbf{hard fixing} qui viene riportato un versione basilare del codice che crea una soluzione di partenza attraverso la applicazione consecutiva del NN LINK!!!!!! e $2-Opt$ LINK!!!!. Nella parte finale della tesi dove sono presentati vari test, saranno invece utilizzate soluzioni migliore ottenute in precedenza dall'applicazione di altri metodi euristici.

Infine riprendendo quanto detto alla fine dell'ultimo paragrafo specifichiamo che i valori di $r$ utilizzati\footnote{Dove $r$ determina quale operazione di $r-Opt$ si ottiene.} sono solamente ${3, 5, 7, 10}$, raggiungibili in progressione una volta che i precedenti falliscono\footnote{Cioè se non si rova una soluzione migliore di quella già disponibile all'interno del suo intorno $N(x,r)$.}.

\begin{lstlisting}

static void LocalBranching(CPLEX cplex, Instance instance, Process process, Random rnd, Stopwatch clock)

\end{lstlisting}

Dove:

\begin{itemize}
    \item \textbf{cplex}: riferimento all'oggetto contenente il modello da risolvere;
    \item \textbf{instance}: riferimento all'oggetto contenente tutte le informazioni relative all'istanza del Problema del Commesso Viaggiatore corrente;
    \item \textbf{process}: riferimento all'oggetto necessario per la stampa dei dati attraverso \textit{GNUPlot};
    \item \textbf{rdn}: istanza della classe Random utilizzata per la generazione dei numeri casuali;
    \item \textbf{clock}: riferimento all'oggetto utilizzato come cronometro, deve già essere stato avviato;
\end{itemize}

\begin{lstlisting}

//Vettore contenente i possibili valori del raggio r che definisce l'intorno nel quale il metodo cerca
//una soluzione migliore della attuale
int[] possibleRadius = {3, 5, 7, 10};

//Variabile che memorizza l'indice dove trovare l'attuale raggio in possibleRadius, inizializzata a 0
int currentRange = 0;

//Variabile booleana utilizzare per comunicare alla Lazy Callback di CPLEX se deve stampare la soluzione
//intera trovata nel caso migliore quella incumbent
bool BlockPrint = false;

//Inizializzazione del vettore contenente la soluzione incumbent
double[] incumbentSol = new double[(instance.NNodes - 1) * instance.NNodes / 2];

//Inizializzazione variabile contenente il costo della soluzione incumbent
double incumbentCost = double.MaxValue;

//Inizializzazione del vettore presente in istance sul quale viene memorizzata la soluzione finale migliore
instance.BestSol = new double[(instance.NNodes - 1) * instance.NNodes / 2];

//Creo il modello iniziale e assegnao ad x il riferimento alle sua variabili
INumVar[] x = Utility.BuildModel(cplex, instance, -1);

//All'indice i-esimo troviamo in ordine crescente quali sono i nodi più vicini all'i-esimo nodo
List<int>[] listArray = Utility.BuildSLComplete(instance);

//Prima parte della creazione della soluzione di partenza, si utilizza la tecnica del nearest neigh
PathStandard heuristicSol = Utility.NearestNeighbor(instance, rnd, listArray);

///Seconda parte della creazione della soluzione di partenza, si applica un semplice 2-Opt
TwoOpt(instance, heuristicSol);

//Aggiornamento del vettore incumbentSol secondo la soluzione iniziale appena prodotta
for (int i = 0; i < instance.NNodes; i++)
{
    int position = Utility.xPos(i, heuristicSol.path[i], instance.NNodes);

    incumbentSol[position] = 1;
}

//Installazione Lazy Callback
cplex.Use(new TSPLazyConsCallback(cplex, x, instance, process, BlockPrint));

//Settaggio per il multi-thread di CPLEX, si utilizzano tanti thread quanti i core virtuali della macchina
cplex.SetParam(CPLEX.Param.Threads, cplex.GetNumCores());

//Aggiunta di un warm-start per CPLEX
cplex.AddMIPStart(x, incumbentSol);

//Inizio creazione del vincolo asincrono caratterizzante il Local Branching

//Creao il contenitore per l'espressione del vincolo di Local Branching
ILinearNumExpr expr = cplex.LinearNumExpr();

//Popolo il contenitore secondo le variabili utilizzate dalla soluzione di partenza appena creata
for (int i = 0; i < instance.NNodes; i++)
    expr.AddTerm(x[Utility.xPos(i, heuristicSol.path[i], instance.NNodes)], 1);

//Dovendo ad ogni applicazione del Local Branching sostituire il vincolo caratterizzante, si memorizza
//quello attuale in una variabile apposita in modo tale da facilitare la sua rimozione futura
IAddable localBranchConstraint  = cplex.Ge(expr, instance.NNodes - possibleRadius[currentRange]);

//Aggiungo il vincolo creato non come un qualsiasi taglio ma come una vero e proprio vincolo
//del modello matematico, la risoluzione da parte di CPLEX è nettamente migliorata
cplex.Add(localBranchConstraint);


//Inizio ciclo do-while che ripete la tecnica del Local Branching fino al termine del tempo limite
//indicato dall'utente o quando il raggio r=10 non produce più miglioramenti
do
{
    //CPLEX risolve l'attuale modello
    cplex.Solve();

    //Se trovo un miglioramento rispetto alla soluzione incumbent
    if (incumbentCost > cplex.GetObjValue())
    {
        //Sostituisco l'incumbent attuale con i valori appena trovati da CPLEX
        incumbentCost = cplex.ObjValue;
        incumbentSol = cplex.GetValues(x);

        //Rimozione del vincolo caratterizzante attualmente utilizzato nel modello
        cplex.Remove(localBranchConstraint);

        //Preparo il nuovo vincolo caratterizzante sfruttando l'esplorazione della nuova soluzione
        //per la sua stampa a video attraverso GNUPlot

        expr = cplex.LinearNumExpr();

        StreamWriter file = new StreamWriter(instance.InputFile + ".dat", false);

        //Scandisco ogni variabile per vedere se fa parte della nuova soluzione
        for (int i = 0; i < instance.NNodes; i++)
        {
            for (int j = i + 1; j < instance.NNodes; j++)
            {
                int position = Utility.xPos(i, j, instance.NNodes);

                //Testo se l'attuale variabile è stata selezionata nella nuova soluzione
                if (incumbentSol[position] >= 0.5)
                {
                    //Stampo su file il percorso della nuova soluzione per GNUPlot
                    file.WriteLine(instance.Coord[i].X + " " + instance.Coord[i].Y + " " + (i + 1));
                    file.WriteLine(instance.Coord[j].X + " " + instance.Coord[j].Y + " " + (j + 1) + "\n");

                    //In contemporanea aggiungo il termine all'espressione per il nuovo vincolo caratterizzante
                    expr.AddTerm(x[position], 1);
                }
            }
        }

        //Lancio la stampa attraverso GNUPlot e chiudo il flusso del file utilizzato
        file.Close();
        Utility.PrintGNUPlot(process, instance.InputFile, 1, incumbentCost, -1);

        //Completo la creazione del nuovo vincolo caratterizzante
        localBranchConstraint = cplex.Ge(expr, instance.NNodes - possibleRadius[currentRange]);

        //Aggiungo il nuovo vincolo caratterizzante al modello matematico
        cplex.Add(localBranchConstraint);
    }
    else
    {
        //Nel caso in cui non si sia trovata una soluzione migliore testo se posso aumentare il range r
        if (possibleRadius[currentRange] != 10)
        {
            //Aumento l'indice di possibleRadius sui cui trovare il nuovo raggio da utilizzare
            currentRange++;

            //Rimuovo il precedente vincolo
            cplex.Remove(localBranchConstraint);

            //Aggiorno il range r del vincolo caratterizzante mantenendo chiaramente le variabili
            //coinvolte dato che la soluzione incumbent rimane la medesima della precedente iterazione
            localBranchConstraint = cplex.Ge(expr, instance.NNodes - possibleRadius[currentRange]);

            //Aggiungo il vincolo caratterizzante aggiornato al modello matematico
            cplex.Add(localBranchConstraint);
        }
        else
        {
            //Nel caso non sia possibile aumentare ulteriormente il raggio r termino il ciclo do-while
            break;
        }
    }
} while (clock.ElapsedMilliseconds / 1000.0 < instance.TimeLimit);

//Memorizzo nelle apposite variabili di instance la soluzione finale trovata ed il relativo costo
instance.BestSol = incumbentSol;
instance.BestLb = incumbentCost;

\end{lstlisting}

\subsection*{POLISHING}
\label{sec:polishingS}

L'algoritmo \textbf{polishing}\footnote{Polishing è utilizzato anche all'interno di CPLEX stesso ed è inoltre brevettato da quest'ultimo, la realizzazione è da attribuirsi ad Edward Rothberg} è di fatto l'ultima tecnica presentata in questa tesi e fa parte degli algoritmi \textbf{matheuristici}.

\textit{Polishing} non introduce di per sè novità ma bensì combina le migliori caratteristiche degli algoritmi \textbf{genetici} LINK!!!!! e di \textbf{Relaxation Induced Neighborhood Search} conosciuto più comunemente con la sigla \textbf{RINS}. Quest'ultimo nasce nel 2004 per opera di \textbf{Emilie Danna}, \textbf{Edward Rothberg} \footnote{Uno dei fondatori di \textbf{GuRoBi}, software solver concorrente a \textbf{CPLEX}.} e \textbf{Claude Le Pape}. \'E stato ampiente discusso nei precedenti paragrafi come nei metodi \textbf{matheuristici} il fattore caratterizzante sia la realizzazione del \textit{fissaggio} di alcune variabili del modello matematico in modo tale da limitare le possibili soluzioni e quindi i tempi di esecuzione da parte del solver. \textit{RINS} definisce una regola molto semplice a tale scopo: date due soluzioni $x_1$ e $x_2$, valide per il \textbf{MIP} preso in analisi, ogni variabili che assume lo stesso valore in entrambe deve essere fissata nel nuovo modello a tale valore. Naturalmente più $x_1$ ed $x_2$ sono \textit{simili} tra loro, più variabili fisso e quindi minore diventa la dimensione dell'intorno che il solver esplora.

\textit{RINS} sembra quindi un buon metodo da applicare in presenza di due soluzioni molto simili a quella ottima $x_{opt}$, è plausibile assumere che in tal caso il fissaggio vada a riguardare in gran parte valori concordi ad $x_{opt}$. La proprietà più importante è da vedersi nella estrema velocità di questa tecnica nel caso di problemi come il \textbf{TSP}, dove con un ordine del numero di variabili pari a \textbf{$O(n^2)$} solamente \textbf{n} assumo valore $1$ in una qualsiasi soluzione mentre le restanti $0$\footnote{Naturalmente anche il caso opposto ha pari valenza}. Come conseguenza diretta si ha che indipendentemente da come sono state ottenute $x_1$ e $x_2$ la probabilità che una qualsiasi variabili assuma valore $0$ in entrambe è estremamente elevata.

Ed è proprio in questo punto che entrano in gioco i metodi euristici \textbf{genetici}, durante la produzione di ogni nuova generazione si presenta la necessità di creare un gran numero di nuove soluzioni a partire da quelle già presenti. Questi algoritmi affrono quindi la possibilità di applicare \textit{RINS} in modo continuo su soluzioni via via più \textit{vicine} a quella ottima. Quanto appena esposto si pone alla base degli algoritmi \textbf{polishing}.

\subsection*{PROBLEMATICHE DELLA TECNICA POLISHING}
\label{sec:ProblematichePolishingS}

\textit{Polishing} eredita una problematica fondamentale dagli algoritmi \textit{genetici} dai noi volutamente non risolta in questi ultimi in modo tale evidenziarne l'importanza: stiamo parlando della possibilità dell'algoritmo di saturare dopo un certo periodo inversamente proporzionale alla grandezza dela popolazione scelta. Mano a mano che nuove generazioni di popolazione vengono create, la probabilità che queste contengano soluzioni sempre più simili tra loro cresce portando così ad una saturazione per quanto riguarda la ricerca di un miglioramento\footnote{Verrà indicata come \textbf{saturazione di primo tipo} nei commenti al codice.}.

Naturalmente questo fattore non è sempre un male in quanto se la saturazione dovesse avvenire nelle vicinanze del valore ottimo lo scopo dell'algoritmo potrebbe essere considerato raggiunto in quato metodo euristico. Questo fatto diventa però più probabile all'aumento della grandezza della popolazione e quindi del numero di soluzioni facenti parte di ogni generazione con l'inevitabile aumento dei tempi di calcolo.

Come evidenziato dai test riguardanti l'algoritmo genetico da noi implementato LINK!!!!!, la problematica esposta è molto evidente già dalle istanze con meno nodi vista necessità di utilizzare una popolazione molto piccola. La situazione per quanto riguarda l'algoritmo \textit{polishing} sarebbe stata la medesima indipendentemente dal fatto che quest'ultimo ci permettesse nella media di triplicare la grandezza di ogni generazione grazie all'uso di \textit{RINS}.

La necessità di creare un metodo \textbf{anti saturazione} ci ha portato a ragione sul perchè gli algoritmi genetici offrono potenzilamente ottimi risultati: combinare soluzioni \textit{buone} con altre \textit{peggiori} può portare, attraverso varie generazioni, a risultati molto migliori di quelli raggiungibili con il solo uso di \textit{buone} soluzioni.

Abbiamo quindi pensato di riapplicare questo principio in caso di saturazione e cioè sostituire una parte della generazione attuale con nuovi elementi, nel nostro caso generati nello stesso modo di quelli appartenenti alla prima generazione e quindi attraverso l'uso del \textbf{Nearestneighbour}\footnote{Vedremo come in realtà è stato effettuato del leggero ulteriore tuning di questo metodo in base al numero di saturazioni consecutive senza alcun miglioramento dell'incumbent.} LINK!!!!!!!!. Ipoteticamente la nostra idea è che in caso di saturazione le soluzioni facenti parte della attuale generazione abbiano ormai raggiunto un discreto fissaggio delle variabili al loro interno. Una loro combinazione con soluzioni potenzialmente peggiori attraverso \textit{RINS} dovrebbe portare ad una ulteriore selezione interna al fissaggio appena citato mantenendo valida così solo la sua porzione migliore.

Indipendentemente da quale istanza sia stata da noi testata i risultati sono stati sorprendenti, con il progredire delle varie risoluzioni la probabilità di ottenere un miglioramento solamente dopo una operazione di anti saturazione continuava ad aumentare.

Il successivo nodo da discutere è naturalmente come capire quando ci si trova in una situazione di saturazione. I due fattori da noi considerati sono stati i seguenti: 

\begin{itemize}
\item \textbf{Numero di generazioni successive senza miglioramenti dell'incumbent}: il valore da noi considerato è stato di $\textbf{10}$;
\item \textbf{Tempo impiegato per definire una nuova generazione}: questo parametro è più complesso del precedente in quanto dipende da diversi fattori quali l'hardware utilizzato e l'istanza utilizzata. Abbiamo quindi svolto diversi test e la formula che per noi ha prodotto il risultato desiderato per le istanze testate è

\begin{equation}\label{eq:tGen}
\displaystyle t_{GenTot}[s] = 0.085 * (\#Nodi / 10) * grandezzaPopolazione)
\end{equation}
\end{itemize}

Esiste una seconda problematica molto meno evidente e che può essere ricondotta ad un diverso tipo di saturazione, questa volta interno ad ogni generazione\footnote{Verrà indicata come \textbf{saturazione di secondo} tipo nei commenti al codice.}. Al crescere della complessità dell'istanza analizzata, in genere determinata dal numero di nodi $n$ e quindi da quello dei lati/variabili, può capitare che pochi \textit{figli} vengano generati in tempi molto più lunghi rispetto ai restati della stessa generazione. La conseguenza è quindi un \textit{ritardo} dell'algoritmo nel passaggio alla iterazione successiva senza però nessuna garanzia di un guadagno finale in termini di bontà dell'incumbent.

La nostra scelta è stata quindi di limitare questa situazione preferendo produrre più generazioni possibili. Le modalità di attuazione per questa idea sono multiple e variano nella loro intrusività con la normale prosecuzione dell'algoritmo. Ad esempio, una volta calcolati un certo numero di figli, si potrebbe iniziare a tenere traccia dei loro tempi di generazione medi e limitare i successivi di conseguenza.

La tecnica proposta è leggermente diversa da quest'ultima, meno lmitante e sfrutta la nozione fondamentale che la creazione dei nuovi figli, in ogni generazione esclusa la prima, avviene in multi threading: il calcolatore mette a disposizione un numero di thread pari ai propri processori virtuali, i quali generano in modo autonomo un nuovo \textit{figlio}. Questo richiede di attivare multiple istanze di CPLEX contemporaneamente, tra loro scorrelate e che utilizzino un solo thread.

Fatta questa premessa, indicando con $m$ il numero di \textit{figli} da creare ad ogni generazione e con $z$ il numero di thread a disposizione, la nostra tecnica prevede di tenere traccia di quanto tempo impiega l'algoritmo per ottenere $m-z$ figli. Tale intervallo è pesato proprio su $m-z$ ed in base al valore finale ottenuto si attuano limitazioni per i restanti $z$ \textit{figli}.

\begin{equation}\label{eq:tMZ}
\displaystyle t_{mzP}[s] = t_{mz}/(m-z)
\end{equation}

Entrando più nel dettaglio si limita il tempo per completare i restanti figli, per costruzione ora calcolabili tutti in parallelo, a :

\begin{itemize}
    \item \textbf{$10$ secondi}: se $t_{mzP}$ è inferiore a tale valore;
    \item \textbf{$t_{mzP}$}: se questo è compreso tra $10$ e $300$ secondi;
    \item \textbf{$300$ secondi}: se  $t_{mzP}$ è superiore a tale valore;
\end{itemize}

La descrizione di come questa seconda tecnica anti saturazione è stata implementata viene rimandata alla appendice in quanto risulta particolarmente complessa ma di relativo interesse. Prima di passare alla sezione successiva dove viene discusso come attivare thread multipli ed assegnargli compiti specifici facciamo solamente notare che per come è stato costruito, $t_{mzP}$ è pari, ma in genere \textbf{superiore} al tempo medio necessario per calcolare $z$ figli parallelamente.

\subsection*{UTILIZZARE MULTIPLE ISTANZE DI CPLEX IN PARALLELO}
\label{sec:ParCPLEXS}

L'utilizzo di multiple istanze di cplex, tra loro indipendenti ed attive contemporaneamente su appositi threads viene introdotto solamente ora in quanto tutti gli algoritmi discussi precedentemente al \textbf{polishing} sono stati presentati nella loro versione più base dove era l'unica istanza di CPLEX a lavorare in più threads. In realtà i test che verranno presentati in questa tesi fanno tutti uso di una versione modificata delle tecniche proposte che appunto sfrutta più istanze di CPLEX attive contemporaneamente.

Il motivo di tale scelta è che permette di ottenere multipli output ma con prestazioni chiaramente limitate. Teniamo sempre ben presente che lo scopo delle varie prove proposte è il confronto delle tecniche studiate e non una loro analisi in termini di prestazioni assolute in quanto queste sono molto dipendenti dall'hardware utilizzato.

Questo ragionamento non riguarda gli algoritmi \textit{genetico} e \textit{polishing} dato che si è scelto di elaborare in parallelo solamente i \textit{figli} di ogni generazione e quindi l'output prodotto è comunque unico.

Per quanto riguarda il \textit{genetico} il motivo è molto semplice: il calcolo delle soluzione \textit{figlie} a partire da due soluzioni \textit{genitori} è molto semplice e non impiega grandi risorse hardware. \'E quindi motivato un approccio multi-threading e dato che la bontà di tali algoritmi dipende molto dal numero di generazioni che si creano è preferibile velocizzare tale processo piuttosto che attivare più algoritmi genetici contemporaneamente.

D'altro canto l'algoritmo \textit{polishing}, tenendo comunque valida la motivazione appena esposta per il \textit{genetico}, si deve considerare che si utilizza CPLEX per generare un nuovo \textit{figlio} e che tale solver non garantisce risultati migliori se gli si condece più threads\footnote{Visitare contemporaneamente più nodi dell'albero decisionale durante la fase di \textbf{Branch\&Cut} non garantisce di terminare la risoluzione più velocemente rispetto all'analisi di un solo nodo ma a pieno potenziale hardware.}.

Vediamo quindi la struttura generale per attivare multipli threads in linguaggio \textbf{C\#}. \'E possibile trovare una applicazione pratica nell'esposizione del codice relativo all'algoritmo \textit{polishing} nella sezione successiva, mentre maggiori dettagli per tutti gli altri metodi risolutivi verranno forniti assieme ai test eseguiti.

Esistono molteplici metodi per ottenere il risultato desiderato, quello che esponiamo prevede l'uso della classe \textbf{ThreadPool}\footnote{Appartenente allo spazio dei nomi \textbf{System.Threading}.} la quale fornisce il metodo statico fondamentale \textbf{QueueUserWorkItem} che delaga al sistema la creazione di un nuovo thread il quale esegue una qualsivoglia porzione di codice con a disposizione tutte le variabili del thread chiamante:

\begin{lstlisting}
ThreadPool.QueueUserWorkItem(o =>
{
        \\Codice generico
});
\end{lstlisting}

Due semplici considerazioni: per prima cosa si nota che non è necessario istanziare un oggetto della classe \textbf{ThreadPool} ed in secondo luogo il nuovo thread viene gestito in background dal sistema ed attivato il prima possibile. Questo significa che il guadagno ottenuto nell'evitare qualsivoglia tipo di attesa, comporta la prosecuzione immediata del thread originale e soprattuto la mancanza di metodo diretto per conoscere quando le operazioni eseguite background terminano.

Per i nostri scopi queste mancanze devono essere obbligatoriamente risolte per entrambi gli algoritmi \textbf{genetico} e \textbf{polishing} che necessitano di sincronizzazione per il calcolo delle varie generazioni.

Introduciamo quindi le due classi \textbf{ManualResetEvent} e \textbf{WaitHandle} che una volta combinate rispondono esattamente alle nostre esigenze: un oggetto di tipo \textit{ManualResetEvent} una volta istanziato può essere \textit{gestito} dalla classe \textit{WaitHandle} la quale dispone del metodo statico \textbf{WaitAll}. Fintanto che tutti gli oggetti da lei gestita non chiamano il metodo \textbf{Set()} il thread su cui viene eseguita la chiamata \textit{WaitAll(...)} è posto in attesa.

La generazione di multipli thread deve quindi essere gestita nel seguente modo:

\begin{lstlisting}
var resetEventList = new List<ManualResetEvent>();

for (int i = 0; i < contatore; i++)
{
    resetEventList.Add(new ManualResetEvent(false));
    
    ManualResetEvent mRE = resetEventList[i];
    
    ThreadPool.QueueUserWorkItem(o =>
    {
        Generico codice
        mRE.Set();
    });
}

WaitHandle.WaitAll(resetEventList.ToArray());
\end{lstlisting}

Concludiamo infine con un commento tecnico al codice appena presentato. Come si può notare nel caso in cui si chiami \textbf{ThreadPool.QueueUserWorkItem} all'interno di un ciclo e, nello specifico, se il codice al suo interno dipende da quale iterazione genera la chiamata, è necessario creare un nuovo riferimento per tutte le variabili dipendenti da quanto appena detto. Come accennato in precedenza, l'operazione \textit{QueueUserWorkItem} programma la creazione di un nuovo thread ed i riferimenti alle variabili utilizzate sono assegnati unicamente al suo effettivo avvio. Se quest'ultima operazione viene quindi posticipata durante successive iterazione del ciclo si incorre inevitabilmente in errori multipli.

Apparentemente la soluzione proposta non dovrebbe risolvere tale problema: si tratta però di un \textit{pattern} automaticamente riconosciuto dal compilatore presente nelle ultime versioni del framework C\#, introdotto appositamente per tale scopo.

\subsection*{IMPLEMENTAZIONE POLISHING}
\label{sec:ImplPolishingS}

In questa sezione vengono presentate le parti più importanti del codice di programmazione che realizza il metodo \textit{matheuristico polishing} discusso. Data la natura di quest'ultimo, la struttura generica del codice è molto simile a quella del metodo genetico LINK!!!!!!!!, pertanto non sono discussi nel dettagli gli aspetti comuni tra i due algoritmi.

Iniziamo con la classica firma del metodo:

\begin{lstlisting} 
static void Polishing(Instance instance, Random rnd, Process process, int sizePopulation, Stopwatch clock)
\end{lstlisting}

Dove:

\begin{itemize}
    \item \textbf{instance}: riferimento all'oggetto contenente tutte le informazioni relative all' istanza del Problema del Commesso Viaggiatore corrente;
    \item \textbf{rnd}: istanza della classe Random utilizzata per la generazione dei numeri casuali, un eventuale seme random deve essere già stato assegnatogli;
    \item \textbf{process}: istanza della classe Process, utilizzata per la stampa su file e a video tramite GNUPlot delle soluzioni trovate, deve essere già stata inizializzata dal metodo chiamante;
    \item \textbf{sizePopulation}: Dimensione della popolazione di ogni generazione, in altre parole il numero di soluzioni facenti parte di queste ultime;
    \item \textbf{clock}: Oggetto che funge da cronometro;
\end{itemize}

Come prima operazione, trattandosi di un metodo matheuristco è necessario preparare i riferimenti a CPLEX e quindi costruire il modello di partenza ovviamente privo di alcun taglio. Ricordiamo inoltre che il codice presentato utilizza il multithreading per genere più \textit{figli} contemporaneamente e pertanto sono necessarie multiple istanze di CPLEX e relativi modelli matematici:

\begin{lstlisting}
//Lista contenente i riferimenti alle istanze degli oggetti di tipo CPLEX
List<CPLEX> CPLEXList = new List<CPLEX>();

//Lista contenente i riferimenti alle variabili utilizzate dagli oggetti di tipo CPLEX nel proprio modello matematico
List<INumVar[]> xList = new List<INumVar[]>();

//Data una popolazione di n elementi si creano n/2 figli 
for (int i = 0; i < sizePopulation / 2; i++)
{
    //Inizializzazione degli oggetti di tipo CPLEX
    CPLEXList.Add(new CPLEX());
    //Settaggio di un seme random basato sui ticks di sistema rappresentanti l'istante attuale
    CPLEXList[i].SetParam(CPLEX.Param.RandomSeed, (int)(DateTime.Now.Ticks & 0x0000FFFF));

    //Per ogni oggetto CPLEX inizializzato se ne definice il modello matematico, viene inoltre aggiunto 
    //il riferimento delle variabili utilizzate alla apposita lista in modo tale che il generico j-esimo
    //oggetto CPLEX in CPLEXList abbia le proprie nella j-esima posizione di xList
    xList.Add(Utility.BuildModel(CPLEXList[i], instance, -1));

    //Installazione lazyconstraint callback
    CPLEXList[i].Use(new TSPLazyConsCallback(CPLEXList[i], xList[i], instance, false));

    //Inizializzazione di un aborter per ogni oggetti CPLEX, utilizzato per gestire la saturazione di 
    //secondo tipo
    CPLEXList[i].Use(new CPLEX.Aborter());
}
\end{lstlisting}

Devono essere inizializzati anche gli oggetti che andranno a contenere i \textit{percorsi} delle varie soluzioni trovate. Viene utilizzata a tale scopo la classe \textbf{PathGenetic} LINK!!!!!! dove in questo caso il percorso è memorizzato sempre all'interno della variabile \textbf{path} ma rispettando la struttura delle variabili di CPLEX: molto semplicemente $path[i]$ assume valore $1$ se e solo se la variabile di indice $i-esimo$ della soluzione a cui si fa riferimento vale anch'essa $1$, in caso contrario deve valere $0$.

\begin{lstlisting}
//Definisco l'oggetto che codifica la soluzione incumbent (miglior soluzione in assoluto fra tutte le generazioni)
PathGenetic incumbentSol = new PathGenetic();

//Miglior soluzione all'interno della nuova generazione calcolata ad ogni iterazione
PathGenetic currentBestPath = null;

//Popolazione della generazione attuale
List<PathGenetic> CurrentPopulation = new List<PathGenetic>();

//Figli della generazione attuale
List<PathGenetic> ChildPoulation = new List<PathGenetic>();
\end{lstlisting}

La prima generazione, come anche l'anti saturazione di primo tipo, sfrutta una nuova versione del metodo \textbf{Nearest Neighbour} visto LINK!!!!!!! dove sono introdotte tre variazioni fondamentali:

\begin{itemize}
    \item Il circuito deve essere memorizzato all'interno degli oggetti \textit{PathGenetic} seguento la struttra descritta in precedenza;
    \item Il nodo iniziale per la costruzione del circuito è sempre quello di indice $0$ così che l'algoritmo \textit{RINS} abbia più possibilità di fissare anche variabili comuni di valore $1$;
    \item La probabilità di utilizzare il lato più corto disponibile oppure il successivo, dipende dal numero di nodi della istanza analizzata e da quante saturazioni di primo tipo successive avvengono;
    In particolare si ha che la probabilità di selezione il lato migliore cresce al crescere dei nodi della istanza in quanto creare soluzioni di partenza più dissimili tra loro aumenta ulteriormente i tempi di generazione dei \textit{figli}. Infine questa probabilità è abbassata del $5\%$ ad ogni saturazione di primo tipo consecutiva (\textit{malus} resettato al primo miglioramento trovato) così da variare progressivamente le soluzioni introdotte;
\end{itemize}

I dettagli implementativi riguardanti il metodo \textbf{NearestNeightborPolishing} sono presenti LINK!!!!! all'interno della appendice.

\begin{lstlisting}
//listArray[i][y] = z ; implica che il nodo di indice z è l'y-esimo più vicino a quello di indice i
List<int>[] listArray = Utility.BuildSLComplete(instance);

for (int i = 0; i < sizePopulation; i++)
{
    //Generazione di una soluzione e conseguente aggiunta del suo riferimento nella apposita lista
    CurrentPopulation.Add(Utility.NearestNeightborPolishing(instance, rnd, listArray, 0));
    
    //Aggiornamento valore incumbent
    if (currentBestPath == null || currentBestPath.Cost > CurrentPopulation[i].Cost)
        currentBestPath = CurrentPopulation[i];
}
\end{lstlisting}

Per poter cominciare la generazione ciclica delle varie generazioni sono necessarie inizializzazioni di variabili di \textit{servizio}, il settaggio del numero di \textbf{Threads} massimosi utilizzabile dalla classe \textit{ThreadPool} e lo start del cronometro. I dettagli riguardanti l'anti saturazione di primo tipo sono riportati nella apposita sezione LINK!!!! della appendice.

\begin{lstlisting}
//ThreadPool può programmare un numero massimo di threads pari ai core virtuali della macchina per la //generazione di nuovi figli, ed un thread aggiuntivo di servizio utilizzato per evitare la saturazione
//di secondi tipo e cioè evitare che alcuni figli impieghino troppo tempo a generarsi rispetto agli
//altri della stessa generazione. Questo thread rimane per la maggior parte del tempo in attesa
ThreadPool.SetMaxThreads(CPLEXList[0].GetNumCores() + 1, CPLEXList[0].GetNumCores() + 1);

//Questa variabile memorizza il tempo impiegato per generare i figli della attuale generazione
double timeC;

//Contatore di generazioni successive che non migliorano l'incumbent e generate in tempi validi per
//contribuire alla saturazione di primo tipo
int cntSat = 0;

//Contatore di saturazioni di primo tipo consecutive
int cntSatProgr = 0;

//Oggetti necessari per l'anti saturazione di secondo tipo, utilizzata per impedire che alcuni figli
//impieghino troppo tempo per la loro creazione rispetto ad altri della stessa generazione
object lockPol;
EventWaitHandle ewhCnt;
EventWaitHandle ewhCntComplete;

//Eventuale start del cronometro
if (!clock.IsRunning)
    clock.Start();
\end{lstlisting}

La creazione delle varie generazioni di soluzioni per il \textbf{TSP} corrente avviene all'interno di un ciclo \textbf{do-while} di durata temporale pari al tempo limite indicato da parte dell'utente. Di seguito è riportato il codice commentato che gestisce la creazione della nuova generazione. Il metodo \textbf{GenerateChildPolish} è analizzato a fine sezione.

\begin{lstlisting}
//Inizializzazione variabili che gestiscono la saturazione di secondo tipo
lockPol = new object();
EventWaitHandle ewhCnt = new EventWaitHandle(false, EventResetMode.AutoReset);
EventWaitHandle ewhCntComplete = new EventWaitHandle(false, EventResetMode.AutoReset);

//Aggiornamento dei time limit di CPLEX in modo tale che venga sempre lasciato almeno un secondo e mezzo
//per la generazione di ogni figlio
for (int i = 0; i < sizePopulation / 2; i++)
{
    if (instance.TimeLimit - (clock.ElapsedMilliseconds / 1000.0) > 1.5)
    {
        CPLEXList[i].SetParam(CPLEX.Param.TimeLimit, instance.TimeLimit - (clock.ElapsedMilliseconds / 1000.0));
        
        CPLEXList[i].SetParam(CPLEX.Param.DetTimeLimit, Program.TICKS_PER_SECOND * CPLEXList[i].GetParam(CPLEX.Param.TimeLimit));
    }
    else
    {
        CPLEXList[i].SetParam(CPLEX.Param.TimeLimit, 1.5);
        
        CPLEXList[i].SetParam(CPLEX.Param.DetTimeLimit, Program.TICKS_PER_SECOND * CPLEXList[i].GetParam(CPLEX.Param.TimeLimit));
    }
}

//resetEventC contiene la lista di ManualResetEvent utilizzati per segnalare quando un figlio è stato
//creato
List<ManualResetEvent> resetEventC = new List<ManualResetEvent>();

//resetEventSat è utilizzato per segnalare quando il thread di supporto per gestire la durata dei figli
//termina
ManualResetEvent resetEventSat = new ManualResetEvent(false);

//resetEventList viene popolata di tutti i riferimenti di resetEventC e resetEventSat ed questo oggetto ad
//essere gestito dalla classe WaitHandle
List<ManualResetEvent> resetEventList = new List<ManualResetEvent>();

//index è una variabile di supporto utilizzata per accoppiare correttamente i genitori dei figli
int[] index = new int[sizePopulation / 2];
//Popolamento di index e di resetEventList, un oggetto ManualResetEvent per ogni figlio
for (int i = 0; i < sizePopulation/2; i++)
{
    index[i] = i * 2 + 1;
    resetEventList.Add(new ManualResetEvent(false));
}


//Richiesta di attivavazione del thread di servizio che monitora e gestisce il tempo con cui vengono 
//generati i figli, è compito del metodo MonitorCPLEX gestire quindi la saturazione di secondo tipo
ThreadPool.QueueUserWorkItem(o =>
{
    MonitorCPLEX(CPLEXList[0].GetNumCores(), CPLEXList, sizePopulation, ewhCnt, ewhCntComplete);
    resetEventSat.Set();
});

//resetEventSat è aggiunto alla lista resetEvent
resetEvent.Add(resetEventSat);

//Prima di proseguire è necessario che il thread di servizio venga creato e MonitorCPLEX abbia completato 
//l'inizializzazione locale. Viene conferito un tempo di attesa limite di 50 ms
ewhCntComplete.WaitOne(50);

//startG tiene traccia dell'istante iniziale relativo alla creazione della nuova generazione
double startG = clock.ElapsedMilliseconds / 1000.0;


//ThreadPool setta la coda di thread per la creazione in parallelo di tutti i figli, le variabili
//ManualResetEvent utilizzate per segnalare il completamente di questi ultimi sono aggiunte a resetEvent
foreach(int i in index)
{
    int y = i;
    ManualResetEvent mRE = resetEventList[(y - 1) / 2];

    ThreadPool.QueueUserWorkItem(o =>
    {
        ChildPoulation.Add(Utility.GenerateChildPolish(CPLEXList[(y - 1) / 2], (Instance)instance.Clone(), xList[(y - 1) / 2], CurrentPopulation[y - 1], CurrentPopulation[y], lockPol, ewhCnt, ewhCntComplete));
        mRE.Set();
    });

    resetEvent.Add(mRE);
}

//Fino a che tutti i thread richiesti da ThreadPool, e quindi la generazione di tutti i figli ed il thread
//di servizio, terminano non si prosegue
WaitHandle.WaitAll(resetEvent.ToArray());

//Reset degli oggetti CPLEX.Aborter per ogni oggetto CPLEX
for (int i = 0; i < sizePopulation / 2; i++)
{
    CPLEXList[i].Use(new CPLEX.Aborter());
}

//La effettiva nuova generazione viene creata sfruttando la precedente ed i suoi figli nello stesso modo
//del metodo genetico
CurrentPopulation = Utility.NextPopulation(instance, sizePopulation, CurrentPopulation, ChildPoulation);

//Tempo totale misurato in secondi impiegato per la creazione della nuova generazione
timeC = clock.ElapsedMilliseconds / 1000.0 - startG;

//Aggiorno il riferimento alla soluzione migliore della nuova generazione
currentBestPath = Utility.BestSolution(CurrentPopulation);
\end{lstlisting}

La corrente iterazione del ciclo \textit{do-while} termina con la verifica del miglioramento dell'incumbent, in tal caso si procede anche alla stampa della nuova soluzione ed al reset dei contatori per la saturazione di primo tipo.

\begin{lstlisting} 
//Aggiornamento della soluzione incumbent
incumbentSol = (PathGenetic)currentBestPath.Clone();

//Stampa della nuova miglior soluzione
Utility.PrintGeneticSolution(instance, process, incumbentSol);

//Reset contatori saturazione di primo tipo
cntSat = 0;
cntSatProgr = 0;
\end{lstlisting}

Nel caso in cui non ci sia un miglioramento inizia il processo di anti saturazione di primo tipo.

\begin{lstlisting}
//Verifca se il tempo impiegato rientra nei parametri di saturazione
if (timeC < 0.085 * (instance.NNodes / 10) * sizePopulation)
    cntSat++;
else
    cntSat = 0;

//Nel caso in cui 10 generazioni consecutive non migliorative rientrano nei parametri di saturazione
//avvio il metodo anti saturazione di primo tipo
if (cntSat == 10)
{
    //Calcolo la diminuzione della probabilità che il metodo NearestNeighbor tenti di selezionare ad ogni
    //iterazione il lato meno costoso
    double minus = 0.05 * cntSatProgr;

    //Rimozione di un quarto delle soluzioni della corrente generazione, più precisamente sono sempre
    //rimosse quelle di indice 0, 4, 8, ecc...
    for (int i = 0; i < sizePopulation / 4; i++)
    {
        CurrentPopulation.RemoveAt(sizePopulation - (4 * (i + 1)));
    }

    //Rimpiazzo le soluzioni rimosse con delle nuove generate dal Nearest Neighbour
    for (int i = 0; i < sizePopulation / 4; i++)
    {
        CurrentPopulation.Insert(4 * i, Utility.NearestNeightborPolishing(instance, rnd, listArray, minus));
        
        //Nel caso improbabile che una delle nuove soluzioni migliori l'incumbent aggiorno tale valore
        if (CurrentPopulation[4 * i].Cost < currentBestPath.cost)
        {
            currentBestPath = CurrentPopulation[CurrentPopulation.Count - 1];
        }
    }

    //Aggiorno i contatori di saturazione di primo tipo
    cntSatProgr++;
    cntSat = 0;
}
\end{lstlisting}

Si conclude questa sezione con la dovuta analisi del metodo \textbf{GenerateChildPolish} che utilizza l'algoritmo \textit{RINS} per creare un \textit{figlio} a partire da due genitori.

\begin{lstlisting}
public static PathGenetic GenerateChildPolish(CPLEX cplex, Instance instance, INumVar[] x, PathGenetic mother, PathGenetic father, object lockPol, EventWaitHandle ewhCnt, EventWaitHandle ewhCntComplete)
\end{lstlisting}

\begin{itemize}
    \item \textbf{cplex}: oggetto della classe CPLEX utilizzato per risolvere il modello;
    \item \textbf{instance}: riferimento alla classe Instance, accesso solo in lettura;
    \item \textbf{x}: vettore contenente i riferimenti alla variabili del modello matematico di cplex;
    \item \textbf{mother}: primo dei due circuiti genitori;
    \item \textbf{father}: secondo dei due circuiti genitori;
    \item \textbf{lockPol, ewhCnt, ewhCntComplete}: oggetti necessari per segnalare in modo il completamento della generazione della soluzione \textit{figlio}, maggiori dettagli LINK!!!!!!;
\end{itemize}

Il contenuto di questa funzione è molto semplice, si procede inizialmente al fissaggio secondo la tecnica \textit{RINS} delle variabili a valore comune nei due circuiti \textit{genitori} ed inoltre il migliore tra questi viene settato come \textbf{warm-start}\footnote{La definizione di un warm-start è necessaria in quanto può capitare che CPLEX non generi una soluzione valida entro il tempo limite fissato causando successivamente una eccezione.}:

\begin{lstlisting}
//Il percorso della soluzione figlio viene inizializzzato
int[] path = new int[mother.Path.Length];

//Fissaggio delle variabili comuni tra i due genitori a valore 1
for (int i = 0; i < mother.path.Length; i++)
{
    if (mother.Path[i] == 1 && father.Path[i] == 1)
    x[i].LB = 1;
}

//Fissaggio delle variabili comuni tra i due genitori a valore 0
for (int i = 0; i < mother.path.Length; i++)
{
    if (mother.Path[i] == 0 && father.Path[i] == 0)
    x[i].UB = 0;
}

//cplex.AddMIPStart accetta unicamente un vettore di double come warm-start mentre all'interno
//degli oggetti di tipo PathGenetic è utilizzato un vettore di interi
if (mother.Cost > father.Cost)
    cplex.AddMIPStart(x, ConvertIntArrayToDoubleArray(father.Path));
else
    cplex.AddMIPStart(x, ConvertIntArrayToDoubleArray(mother.Path));
    
//Il modello matematico definito viene risolto da CPLEX
cplex.Solve();
\end{lstlisting}

Terminata la risoluzione del modello matematico, operazione che può anche essere stata forzata con un abort nel caso in cui si ricada nella saturazione di secondo tipo, è necessario risvegliare il thread di servizio in modo tale da comunicargli il completamento della generazione di un figlio. 

Tale thread può gestire una sola comunicazione per volta, è necessario creare quindi un sistema di serializzazione: i threads che gestiscono la creazione di un figlio devono essere certi che il thread di servizio non stia gestendo altre segnalazioni di completamento.

\begin{lstlisting}
//Se la terminazione non è avvenuta in seguito ad un abort dovuto a saturazione di secondo tipo
//si comunica al thread di servizio il completamento della generazione di un figlio
if (!cplex.GetAborter().IsAborted())
{
    //lockPol viene utilizzato per serializzare delle comunicazioni di completamento
    //se non è disponibile si attende la chiamata Monitor.Pulse(lockPol) da parte di un altro thread
    lock (lockPol)
    {
        //Viene comunicata il completamento del figlio ed atomicamente si attende una segnalazione
        //da parte del thread di servizio: solo dopo tale segnalazione è possibile liberare il lock su
        //lockPol
        WaitHandle.SignalAndWait(ewhCnt, ewhCntComplete, 1000, false);
        //Monitor.Pulse(lockPol) risveglia un qualsiasi thread in attesa su lockPol
        Monitor.Pulse(lockPol);
    }
}
\end{lstlisting}

Completata la fase di sincronizzazione non rimane altro che leggere la soluzione trovata da CPLEX, popolare un apposito oggetto \textbf{PathGentic} contenente il nuovo figlio e restituirlo, non prima di aver eliminato il fissaggio delle variabili secondo \textit{RINS}.

\begin{lstlisting}
//Si ottiene il valore delle variabili nella soluzione figlio trovata
double[] pathChild = cplex.GetValues(x);

//Popolamento del vettore path
for (int i = 0; i < mother.path.Length; i++)
{
    if (pathChild[i] >= 0.5)
        path[i] = 1;
    else
        path[i] = 0;
}

//Viene creato l'oggetto di tipo PathGenetic per il figlio generato
child = new PathGenetic(path, cplex.GetObjValue());


//Reset delle variabili modificate secondo la tecnica RINS
for (int i = 0; i < mother.path.Length; i++)
{
    x[i].LB = 0;
    x[i].UB = 1;
}

return child;
\end{lstlisting}

\section*{APPENDICE}
\label{sec:AppendiceS}

\subsection*{CLASEE STOPWATCH}
\label{sec:StopwatchS}

\subsection*{GNUPLOT}
\label{sec:GNUPlotS}

\subsection*{METODO INITPROCESS}
\label{sec:InitProcessS}

\subsection*{METODO XPOS}
\label{sec:XPos}

\subsection*{METODO MODIFYMODEL}
\label{sec:ModifyModelS}

Il metodo ModifyModel fissa in soluzione, con una certa percentuale p, lati che appartengono ad una soluzione ammissibile. La sua firma risulta essere:

\begin{lstlisting} 
public static void ModifyModel(Instance instance, INumVar[] x, Random rnd, double percentageFixing, double[] solution)
\end{lstlisting}

Dove:

\begin{itemize}
    \item \textbf{instance}: riferimento all' oggetto contenente tutte le informazioni relative all' istanza del Problema del Commesso Viaggiatore corrente;
    \item \textbf{x}: vettore contenente le variabili del modello;
    \item \textbf{rnd}: istanza della classe Random utilizzata per la generazione dei numeri casuali;
    \item \textbf{percentageFixing}: Probabilità con cui un lato viene fissato in soluzione;
    \item \textbf{solution}: Vettore che codifica la soluzione ammissibile su cui si fissano in soluzione i lati;
\end{itemize}

L' idea dell' algoritmo consiste nel scandire tutti i lati appartenenti alla soluzione ammissibile fornitagli in ingresso, invocare per ognuno lato il metodo \textbf{RandomSelect} e se ritorna 1 viene fissato il lato altrimenti no. Per fissarlo si pone semplicemente il LB della variabile ad esso associata a 1.  Il ciclo do - while serve per ripete il fissaggio qualora si fissino tutti i lati o non si vincoli solo un lato poichè non avrebbe senso a questo punto far partire cplex. Al termine del metodo viene effettuato il preprocessing discusso del paragrafo precedente.

\begin{lstlisting}

//Stored the number of variable fixed
int nVariabileFix = 0;

do
{
nVariabileFix = 0;

//Scan all edge that belong to the current heuristic solution
for (int i = 0; i < x.Length; i++)
{
if ((solution[i] == 1))
{
//Whit a percentageFixing probability fix a edge belong to the current solution
if (RandomSelect(rnd, percentageFixing) == 1)
{
x[i].LB = 1;
nVariabileFix++;
}
}
}

} while (nVariabileFix >= instance.NNodes - 1);

Utility.PreProcessingTSP(instance, x);

\end{lstlisting}

\subsection*{FILL ROULETTE}
\label{sec:FillRouletteS}

Il metodo FillRoulette ha il compito di popolare la roulette in modo tale che la selezione sia proporzionale alla fitness. Associa ad ogni circuito un numero intero, chiamato \textbf{NRoulette}, progressivo e inserisce all'interno della roulette tale valore un numero di volte proporzionale al valore della fitness del circuito, infine ritorna la dimensione della roulette. La sua firma risulta essere:

\begin{lstlisting}

static int FillRoulette(List<int> roulette, List<PathGenetic> CurrentGeneration)

\end{lstlisting}

\begin{itemize}
    \item \textbf{roulette}: Lista di interi che rappresenta la roulette e che viene popolata dal metodo;
    \item \textbf{CurrentGeneration}: Lista contenente i circuiti candidati a far parte della nuova generazione;
\end{itemize}

Utilizzando il metodo \textbf{Estimate} LINK!!!!!!!!!!!!! si ottiene una costante intera che viene memorizzata all'interno della variabile \textbf{proportionalityConstant}: moltiplicare questo valore per la fitness di un circuito ci dice quante volte il corrispondente \textit{NRoulette} associato vada inserito nella ruolette. Poiché all'interno della stessa generazione i valori della fitness non variano per ordini di grandezza, tale costante viene per convenzione calcolata utilizzando il circuito memorizzato all'indice $0$ in \textit{CurrentGeneration}.
\begin{lstlisting}

int sizeRoulette = 0;
            
int proportionalityConstant = Estimate(CurrentGeneration[0].Fitness);

for (int i = 0; i < CurrentGeneration.Count; i++)
{
   int prob = (int)(CurrentGeneration[i].Fitness * proportionalityConstant);
   CurrentGeneration[i].NRoulette = i;
   sizeRoulette += prob;
   
   for (int j = 0; j < prob; j++)
     roulette.Add(i);
}
return sizeRoulette;

\end{lstlisting}

\subsection*{ESTIMATE}
\label{sec:EstimateS}

Il metodo Estimate genera una costante di proporzionalità in modo tale che, eseguendo il prodotto fra tale costante ed il valore ricevuto come parametro di ingresso, si ottenga una quantità sempre maggiore di 100 PERCHE'??????? A COSA SERVE?????. La sua firma risulta essere:

\begin{lstlisting}

static int Estimate(double sample)
{
    int k = 1;
    while (sample*k < 100)
    {
        k = k * 10;
    }
    return k;
}

\end{lstlisting}

Dove:

\begin{itemize}
    \item \textbf{sample}: Valore della fitness presa come campione.
\end{itemize}

\subsection*{REPAIR}
\label{sec:RepairS}

Il metodo Repair è stato progettato per trasformare un percorso che non risulta essere un circuito hamiltoniano in un circuito hamiltoniano. Sappiamo che visitare più volte lo stesso nodo rende tale proprietà non vera così come la presenza di almeno un nodo isolato. L'algoritmo si sviluppa in due fasi: in un primo momento si eliminano dal vettore che codifica il percordo tutti i nodi duplicati, successivamente si fa in modo che quelli isolati vengano connessi al nodo ad essi più vicini. Un esempio di funzionamento dell'algoritmo è riportato nei tre grafici sottostanti.

\begin{figure}[htbp]
    \centering
    \includegraphics[width=\textwidth]{Immagini/"circuitoF".jpg}
    \caption{Circuito non hamiltoniano figlio}
\end{figure}

\begin{figure}[htbp]
    \centering
    \includegraphics[width=\textwidth]{Immagini/"circuitoH".jpeg}
    \caption{Circuito hamiltoniano incompleto}
\end{figure}

\begin{figure}[htbp]
    \centering
    \includegraphics[width=\textwidth]{Immagini/"circuitoC".jpg}
    \caption{Circuito hamiltoniano figlio}
\end{figure}

Discutiamo ora il codice prodotto. Costruiamo due liste di interi chiamate \textbf{isolatedNodes} e \textbf{nearlestIsolatedNodes} dove rispettivamente contengono all'i-esimo elemento l'indice dell'i-esimo nodo isolato e l' indice del nodo ad esso più vicino\footnote{Ci sono in realtà delle eccezzioni a quest'ultima affermazione che vedremo in seguito}.
Per popolare tali liste utilizziamo i metodi \textbf{FindIsolatedNodes} LINK!!!!!! e \textbf{FindNearestNode} LINK!!!!!!!!!!.
Dichiariamo le due liste \textbf{pathIncomplete} e \textbf{pathComplete}, nella prima andiamo ad inserire il percorso originale privo degli elementi che lo rendono non hamiltoniano mentre nella seconda costruiamo il percorso completo di tutti i nodi ed hamiltoniano. Per ottenere quest'ultimo risultato procediamo ciclicamente con la copia di ogni elemento appartente a \textit{pathIncomplete} in \textbf{pathComplete} facendo però in modo che, se per un qualsisi $i, j, m$ vale $pathIncomplete[i]=nearlestIsolatedNodes[j]$, allora nel caso andiamo a porre $pathComplete[m]=pathIncomplete[i]$ dovrà anche valere $pathComplete[m+1]=isolatedNodes[j]$ e $pathComplete[m+2]=pathIncomplete[i+1]$ simili a quanto avviene nella Figura 3!!!!!!!!!!!!!!!!!!!!!!!.

\begin{lstlisting}

int positionInsertNode = 0;

for (int i = 0; i < pathIncomplete.Count; i++)
{
    if (nearlestIsolatedNodes.Contains(pathIncomplete[i]))
    {
        pathComplete[positionInsertNode] = pathIncomplete[i];
        pathComplete[positionInsertNode + 1] = isolatedNodes[nearlestIsolatedNodes.IndexOf(pathIncomplete[i])];
        positionInsertNode++;
    }
    else
        pathComplete[positionInsertNode] = pathIncomplete[i];

    positionInsertNode++;
}
return new PathGenetic(pathComplete, instance);
\end{lstlisting}

\subsection*{FINDISOLATEDNODES}
\label{sec:FindIsolatedNodesS}

Tale funzione viene utilizzata per identificare tutti i nodi isolati presenti in un generico percorso.
La sua firma risulta essere:

\begin{lstlisting}
static void FindIsolatedNodes(Instance instance, int[] path, List<int> isolatedNodes)
\end{lstlisting}

\begin{itemize}
    \item \textbf{instance}: oggetto della classe Instance contenente tutti i dati che descrivono l'istanza del problema del Commesso Viaggiatore fornita in ingresso dall' utente;
    \item \textbf{isolatedNodes}: Lista contenente gli indici di tutti i nodi isolati;
\end{itemize}

Data la semplicità del metodo non si ritiene utile far nessuna considerazione, riportiamo direttamente il codice realizzato.

\begin{lstlisting}

bool nodeIsVisited = false;

for (int i = 0; i < instance.NNodes; i++)
{
    for (int j = 0; j < instance.NNodes; j++)
    {
         if (pathChild[j] == i)
         {
            nodeIsVisited = true;
            //If the node is visited can exit to for
            break;
         }
    }

    //If the node has nevere been visited is a isolated noode
    if (nodeIsVisited == false)
       isolatedNodes.Add(i);

    //Configure nodeIsVisited to its default value
    nodeIsVisited = false;
}

\end{lstlisting}

\subsection*{FINDNEARESTNODE}
\label{sec:FindNearestNodeS}

Dato un certo circuito ed una lista di nodi isolati in esso contenuti, il metodo FindNearestNode fornisce per ognuno di essi il nodo più vicino \textit{valido} ossia che rispetti le seguenti condizioni:

\begin{itemize}
    \item  Non deve essere un nodo isolato;
    \item  Non deve essere il nodo più vicino di un nodo isolato precedentemente analizzato. In altre parole se $n_3$ è un nodo non isolato e risulta il nodo più vicino dei nodi isolati $n_1$ e $n_2$ non è possibile avere: $nearestNeighIsolNode[indice_{n_1}] = n_3 \wedge nearestNeighIsolNode[indice{n_2}] = n_3$. Per convenzione, supponendo $indice_{n_1} < indice_{n_2}$ vale $nearestNeighIsolNode[indice_{n_1}] = n_3$ mentre a $nearestNeighIsolNode[indice_{n_2}]$ viene assegnato il successivo nodo valido più vicino disponibile.
\end{itemize}

Il codice commentato della funzione viene riportato di seguito.

COMMENTA CODICE!!!!!!!!!!!!!!!!!!!!!!! RIPORTA TUTTO IL METODO!!!!!!!!!!!!!!!!! PUBLIC ...

\begin{lstlisting}

int nextNode = 0;
int nearestNode = 0;
bool find = true;

for (int i = 0; i < isolatedNodes.Count; i++)
{
    find = false;
    nextNode = 0;
    do
    {
       nearestNode = listArray[isolatedNodes[i]][nextNode];

       if (((isolatedNodes.Contains(nearestNode)) == false) && (nearestNeighIsolNode.Contains(nearestNode) == false))
       {
          nearestNeighIsolNode.Add(nearestNode);
          find = false;
       }
       else
       {
           nextNode++;
           find = true;
       }
    } while (find);
}

\end{lstlisting}

\subsection*{BESTSOLUTION}
\label{sec:BestSolutionS}

Il metodo BestSolution riceve come input una serie di percorsi hamiltoniani memorizzati attraverso la classe \textbf{PathGenetic} LINK!!!!!!!!!!!!!!!!!!!! ed in output fornisce quello a costo minore, la sua intestazione risulta essere:

\begin{lstlisting}

public static PathGenetic BestSolution(List<PathGenetic> population)

\end{lstlisting}

Dove:

\begin{itemize}
    \item \textbf{population}: Insieme di cammini hamiltoniani;
\end{itemize}

Di seguito il codice:

\begin{lstlisting}

PathGenetic currentBestPath = population[0];

for (int i = 1; i < population.Count; i++)
{
   if (population[i].cost < currentBestPath.cost)
      currentBestPath = population[i];
}

return currentBestPath;

\end{lstlisting}

\subsection*{NEARESTNEIGHBORPOLISHING \& RNDPLUSPOLISHING}
\label{sec:NNPRNDPS}

Il metodo \textbf{NearestNeightborPolishing} applica una versione dell'algoritmo \textbf{NearestNeightbor} LINK!!!!! compatibile con il metodo matheuristico \textbf{Polishing}.
Viene memorizzato il circuito in modo compatibile alle variabili di CPLEX\footnote{Se la variabile di indice $i$ ha valore $x$ allora il vettore che descrive il circuito sarà $vect[i] = x$.}, utilizza come nodo di partenza sempre e solo quello di indice $0$ ed utilizza una propria versione del metodo \textbf{RndPlus}\footnote{\textbf{RndPlus} determina se la selezione del lato migliore possibile oppure uno dei successivi ad ogni iterazione dell'algoritmo \textbf{NearestNeighbor}.} chiamata \textbf{RndPlusPolishing}.

\begin{lstlisting}

public static PathGenetic NearestNeightborPolishing(Instance instance, Random rnd, List<int>[] listArray, double minus)

\end{lstlisting}

Dove:

\begin{itemize}
    \item \textbf{instance}: riferimento alla classe Instance contenente le informazioni riguardanti l'istanza presa in analisi;
    \item \textbf{rnd}: variabile utilizzata per ottenere valori random, precedentemente inizializzata;
    \item \textbf{listArray}: listArray[i][j] = z; implica che il nodo di indice $z$ è il $j-esimo$ più vicino a quello di indice $i$ nella attuale istanza;
    \item \textbf{minus}: valore necessario per il metodo \textbf{RndPlusPolishing};
\end{itemize}

Di seguito si riporta il codice commentato

\begin{lstlisting}

//Vettore in cui memorizzo il nuovo circuito, (instance.NNodes - 1) * instance.NNodes / 2 corrisponde
//al numero totale di lati presenti
int[] heuristicSolutionCPLEX = new int[(instance.NNodes - 1) * instance.NNodes / 2];

//Costo del circuito prodotto
double cost = 0;

//Memorizza i nodi che delimitano il lato preso in analisi 
int[] currentNodes = new int[2];

//Nodi facenti parte del circuito
List<int> notAvailableNodes = new List<int>();

//Il nodo di partenza è sempre quello di indice 0
currentNodes[0] = 0;

//Il nodo di indice 0 è non più disponibile per costruzione
notAvailableNodes.Add(currentNodes[0]);

//I lati da inserire nel circuito sono in numero pari al numero di nodi della istanza
for (int i = 1; i < instance.NNodes; i++)
{
    //Se falso significa che l'attuale iterazione del ciclo non ha ancora aggiunto un nuovo lato al circuito
    bool found = false;
    //Dal nodo corrente si tenta di passare al nodo plus-iesimo più vicino
    int plus = RndPlusPolishing(instance, rnd, minus);
    int nextNode = listArray[currentNodes[0]][plus];

    do
    {
        //Si controlla se il nextNode selezionato è ancora disponibile
        if (notAvailableNodes.Contains(nextNode) == false)
        {
            //Il lato da aggiungere ha estremi i nodi di indice currentNodes[0] e currentNodes[1]
            currentNodes[1] = nextNode;
            //Si trova l'indice utilizzato da CPLEX per memorizzare la variabile relativa al nodo
            //selezionato
            int pos = xPos(currentNodes[0], currentNodes[1], instance.NNodes);

            //Il costo corrente del circuito è aggiornato
            cost += (int)Point.Distance(instance.Coord[currentNodes[0]], instance.Coord[currentNodes[1]], instance.EdgeType);

            //Il vettore che descrive il circuito è aggiornato
            heuristicSolutionCPLEX[pos] = 1;

            //Aggiornamento dei nodi non disponibili e della nuova estremità del circuito
            notAvailableNodes.Add(nextNode);
            currentNodes[0] = nextNode;
            found = true;
        }
        else
        {
            //Nel caso il nodo selezionato non sia disponible si passa al successivo
            //meno distante
            plus++;
            if (plus >= instance.NNodes - 1)
            {
                nextNode = listArray[currentNodes[0]][0];
                plus = 0;
            }
            else
                nextNode = listArray[currentNodes[0]][0 + plus];
        }
    } while (!found);
}

//Il circuito viene effettivamente chiuso collegando l'ultimo nodo selezionato a quello di indice 0
heuristicSolutionCPLEX[xPos(0, currentNodes[1], instance.NNodes)] = 1;
//Il costo viene aggiornato
cost += (int)Point.Distance(instance.Coord[0], instance.Coord[currentNodes[1]], instance.EdgeType);
//Una variabile di tipo PathGenetic viene restituita, Path e costo sono settati di conseguenza
return new PathGenetic(heuristicSolutionCPLEX, cost);

\end{lstlisting}

Il metodo \textbf{RndPlusPolishing} agisce in base al numero di nodi della istanza considerata, i valori settati non seguono nessuna regola teorica ma sono il risultato di diversi tuning eseguiti durante la fase di test.

\begin{lstlisting}
//Valore random tra 0 (compreso) ed 1 (non compreso)
double tmp = rnd.NextDouble();

if (instance.NNodes < 400)
{
    if (tmp < 0.75 - minus)
        //Signica che si tenta di collegare il nodo più vicino a quello attuale
        return 0;
    else
        //Signica che si tenta di collegare il secondo nodo più vicino a quello attuale
        return 1;
}else if (instance.NNodes < 600)
{
    if (tmp < 0.8 - minus)
        return 0;
    else
        return 1;
}
else if (instance.NNodes < 800)
{
    if (tmp < 0.9 - minus)
        return 0;
    else
        return 1;
}
else
{
    if (tmp < 0.95 - minus)
        return 0;
    else
        return 1;
}

\end{lstlisting}

\subsection*{POLISHING - GESTIONE ALGORITMICA SATURAZIONE DI SECONDO TIPO}
\label{sec:PolishingSatDueS}

Ricordando che nell'algoritmo polishing, la saturazione di secondo tipo avviene quando gli ultimi $m$\footnote{Dove $m$ è il numero di \textit{figli} calcolati costantemente in modo parallelo da $m$ threads.} \textit{figli} non ancora determinati richiedono un tempo troppo elevato rispetto ai loro \textit{fratelli}.

L'idea per gestire questa problematica è molto semplice: una volta terminata la creazione di tutti i \textit{figli}, eccetto gli ulti $m$, si attende ancora un certo tempo $T_r$ LINK!!!!! dopodiché si lancia una operazione di abort su tutte le istanze di CPLEX. Quelle che hanno già terminato la propria risoluzione non subiscono alcun effetto mentre le altre la terminano in modo \textit{sicuro}.

Tutto questo viene gestito dal metodo \textbf{MonitorCPLEX} eseguito in background su un apposito thread. Le problematiche che incorrono sono principalmente due:
\begin{itemize}
    \item La comunicazione tra i thread che generano i nuovi \textit{figli} e \textit{MonitorCPLEX} deve essere serializzata.
    \item Fare in modo che, se durante la attesa del tempo $T_r$, tutti i figli sono completati e quindi \textbf{non} è presente saturazione di secondo tipo, \textit{MonitorCPLEX} termini immediatamente così che l'algoritmo di polishing possa proseguire senza inutili ritardi.
\end{itemize}

Per superare questi due ostacoli è necessario introdurre due classi \textbf{WaitHandle} e \textbf{EventWaitHandle}\footnote{Entrambe le classi \textit{WaitHandle} e \textit{EventWaitHandle} fanno parte dello spazio dei nomi System.Threading.}: la prima dispone dell'essenziale metodo \textbf{SignalAndWait(WaitHandle toSignal, WaitHandle toWaitOn)} il quale atomicamente risveglia un thread in attesa su \textbf{toSignal} e pone il chiamante esso stesso in atteso su \textbf{toWaitOn}.

La soluzione proposta per ottenere una completa sincronizzazione tra tutti i threads in gioco è descritta dalla seguente immagine:

\FloatBarrier

\begin{figure}[htbp]
    \centering
    \scalebox{0.5}{\includegraphics{Immagini/"DiagrammaPolishing".png}}
    \caption{Diagramma segnalazione serializzata completamento figlio}
\end{figure}

\FloatBarrier

Banalmente come già visto nel paragrafo LINK!!!!! il metodo \textbf{GenerateChildPolish} una volta terminato il comando \textbf{CPLEX.Solve()} usa il codice:

\begin{lstlisting}

if (!cplex.GetAborter().IsAborted())
{
    lock (lockPol)
    {
        WaitHandle.SignalAndWait(ewhCnt, ewhCntComplete, 1000, false);
        Monitor.Pulse(lockPol);
    }
}

\end{lstlisting}

Dove l'operazione \textit{WaitHandle.SignalAndWait} ha un timeout di un secondo\footnote{Questo intervallo di tempo è enormemente superiore ad una normale attesa e quindi assicura il corretto funzionamento della sincronizzazione. Essendo la situazione di timeout altrettanto rara non si hanno effetti reali per quanto riguarda un rallentamento dell'algoritmo \textit{polishing}} in quanto può capitare che l'operazione di abort sia avviata tra il test della clausola \textbf{if} e l'esecuzione di \textbf{WaitHandle.SignalAndWait} stesso creando così una attesa infinita.

Completata la creazione del proprio figlio, il thread che esegue \textbf{GenerateChildPolish} effettua la segnalazione, attende la sua totale gestione da parte di \textbf{MonitorCPLEX} e libera \textbf{lockPol}.

Di seguito è riportato infine il codice completo e commentato del metodo \textit{MonitorCPLEX}:

\begin{lstlisting}

public static void MonitorCPLEX(int numThreadsPar, List<CPLEX> cplexList, int sizePop, EventWaitHandle ewhCnt, EventWaitHandle ewhCntComplete)
{
    //Cronometro che determina se è necessario un abort o meno di CPLEX
    Stopwatch clock = new Stopwatch();
    clock.Start();
    //Dati sizePop si creano sempre sizePop/2 figli
    int sizeCh = sizePop / 2;

    //Quanti figli è necessario che si completino prima di avviare il controllo anti saturazione
    int maxWait = sizeCh - numThreadsPar;

    //Inizializzazione contatore figli completati
    int cnt = 0;
    
    //Inizializzazione tempo di attesa anti saturazione
    double tmpSingleThread = 0;

    //Si attendono maxWait segnalazioni di figli completati
    do
    {
        //Il thread corrente attende su ewhCnt e risveglia un thread, se presente,
        //in attesa su ewhCntComplete
        WaitHandle.SignalAndWait(ewhCntComplete, ewhCnt);
        //Una volta risvegliato si procede in modo serializzato all'incremento del contatore cnt
        cnt++;
        //Raggiunta la soglia fissata?
        if (cnt == maxWait)
        {
            //Si salva il tempo impiegato per la risoluzione completa di maxWait figli
            tmpSingleThread = clock.ElapsedMilliseconds / cnt;
            clock.Stop();
        }
    } while (cnt != maxWait);

    //Il valore di attesa finale è settato nei seguenti millisecondi
    if (tmpSingleThread < 10000) //se minore di 10 secondi lo impongo a 10 secondi
        tmpSingleThread = 10000;
    else if (tmpSingleThread > 300000) //se maggiore di 5 minuti secondi lo impongo a 5 minuti
        tmpSingleThread = 300000;

    //Il clock viene riavviato
    clock.Restart();

    do
    {
        //Si attendono le restanti segnalazioni aggiornado ad ognuna di esse il timeout
        if (WaitHandle.SignalAndWait(ewhCntComplete, ewhCnt, (int)(tmpSingleThread - clock.ElapsedMilliseconds), false))
        {
            cnt++;
        }
        else
        {
            clock.Stop();

            //Tutte le istanze di CPLEX subiscono un abort sicuro se ancora in fase risolutiva
            CPLEX.Aborter aborter;

            foreach (CPLEX cplex in cplexList)
            {
                aborter = new CPLEX.Aborter();
                cplex.Use(aborter);
                aborter.Abort();
            }
            //Gestita la saturazione il thread di supporto può essere terminato al più presto possibile
            return;
        }
    } while (cnt != sizeCh);//Tutti i figli sono stati generati senza saturazione, esco senza ulteriore
    //attesa
    
    //Prima di uscire si sblocca il thread che ha inoltrato l'ultima segnalazione
    //evito il timeout di un secondo
    ewhCntComplete.Set();
}

\end{lstlisting}

\section*{TEST E RISULTATI}
\label{sec:TestRisultatiS}

In questa sezione sono presentati i test eseguiti sugli algoritmi discussi nel testo. La presentazione è suddivisa in tre sottozezioni:
\begin{itemize}
    \item Instanze con numero di nodi inferiore a \textbf{$200$} $+$ algoritmi esatti;
    \item Instanze con numero di nodi compreso tra \textbf{$200$} e \textbf{$299$} $+$ algoritmi esatti;
    \item Instanze con numero di nodi compreso tra \textbf{$300$} e \textbf{$999$} $+$ algoritmi euristici;
\end{itemize}
Essendo il codice progettatto in Visual Studio utilizzando il linguaggio C\#, non è stato possibile utilizzare il cluster di calcolo offerto dal dipartimento di studio DEI dell'università di Padova. Pertanto la macchina su cui i test sono stati eseguiti è un normale PC di utilizzo quotidiano: da pretest eseguiti durante la fase di sviluppo, si era già notato come istanze di tagli superiore ai $300$ nodi richiedessero tempi troppo alti per essere risolte da algoritmi esatti; Tenendo presente inoltre che sarebbero state richiesti multipli tentativi di risoluzione per ogni coppia istanza/algoritmo, si è deciso di optare per la suddivisione appena descritta.

I dettagli hardware della macchina utilizzata sono riportati di seguito ma è necessario far notare che è stato imposto un limite massimo all'utilizzo della CPU pari al $75\%$: si è voluto sfruttare l'arco minimo di tempo per i test in modo tale che l'ambiente rimanesse il più omogeneo possibile, pertanto un uso superiore delle prestazioni del PC per 2/4 giorni di seguito\footnote{Attualemente ci troviamo in stagione estiva con alte temperature.} avrebbe potuto arrecarvi danni.

\begin{figure}[htbp]
    \centering
    \scalebox{1}{\includegraphics{Immagini/"Spec1".jpg}}
    \caption{CPU}
\end{figure}

\begin{figure}[htbp]
    \centering
    \scalebox{1}{\includegraphics{Immagini/"Spec2".jpg}}
    \caption{Caches}
\end{figure}

\begin{figure}[htbp]
    \centering
    \scalebox{1}{\includegraphics{Immagini/"Spec3".jpg}}
    \caption{Scheda madre}
\end{figure}

\begin{figure}[htbp]
    \centering
    \scalebox{1}{\includegraphics{Immagini/"Spec4".jpg}}
    \caption{RAM}
\end{figure}


Aggiungiamo solamente infine che il sistema operativo è Windows 10 Pro versione 1703 installato su una memoria a stato solido con velocità di lettura/scrittura fino a 535 MB sec/445 MB sec.

Concludiamo questa introduzione spendendo alcune righe descrivendo alcuni aspetti comuni a tutti i test.

Nelle tre successessive sottosezioni troviamo inizialmente una descrizione delle sigle utilizzate per identificare gli algoritmi utilizzati e l'indicazione del numero di run eseguite. Successivamente sono esposti i tempi medi di esecuzione, espressi in secondi, sotto forma tabellare utilizzando come tempo limite $30$ minuti. Infine troviamo una serie di immagini che mostrano visivamente un confronto dei vari algoritmi per ogni istanza.

Per nessun test sono riportati i circuiti hamiltoniani trovati in quanto le istanze utilizzate sono tutte note in litteratura e già risolte all'ottimo. L'obiettivo che si vuole raggiungere è, per quanto riguarda gli algoritmi esatti un confronto dei loro tempi di esecuzione, mentre invece per quelli euristici un confronto del costo delle soluzioni da loro trovate e il valore ottimo noto entro un tempo limite fissato.

In ogni algoritmo sono state quindi rimosse tutte le stampe sia visuali attraverso GNUPlot che su file del circuito trovato ed anche la stampa su file del modello matematico finale comprendente i tagli generati. Le prove per gli algoritmi esatti sono state automatizzate in modo tale che per la stessa istanza venissero eseguiti in serie tutte le run necessarie ogni algoritmo\footnote{Ogni run è stata caratterizzata da un seme diverso sia per quanto riguarda CPLEX sia per quanto riguarda gli oggetti utilizzati per generare valori random.}.

Nel caso in cui una o più run presentavano rallentamentamenti apparentemente non motivabili o improvvisi queste sono state ripetute per verificare che il problema non riguardasse fattori esterni determinati dalla macchina utilizzata. Tutti i valori riportati sono quindi verificati sotto questo punto di vista e dipendenti solamente dalla applicazione sviluppata.

\subsection*{Instanze con numero di nodi inferiore a \textbf{$200$} $+$ algoritmi esatti}

Gli algoritmi esatti a disposizione sono i seguenti: Loop completamente esatto più le sue due varianti a prima fase euristica (linguaggio C\#), utilizzo LazyConstraint Callback (linguaggio C\#), utilizzo LazyConstraint e UserCut Callback (linguaggio C).

Il numero di run per ogni algoritmo è stato di 5: i valori riportati sono quindi una media aritmetica.

Passiamo quindi alla descrizione delle sigle utilizzate:
\begin{itemize}
    \item \textbf{LOOP}: metodo Loop completamente esatto;
    \item \textbf{L EG\textit{X}}: metodo Loop con prima fase euristica dove EpGap è settato al $X\%$;
    \item \textbf{L LA\textit{X}}: metodo Loop con prima fase euristica dove sono abilitati i soli \textit{X} lati a costo minore incidenti in un qualsiasi nodo;
    \item \textbf{L EG\textit{X} LA\textit{Y}}: combinazione dei due punti precedenti;
    \item \textbf{LAZY}: utilizzo lazyconstraint callback in linguaggio\textit{ C\#};
    \item \textbf{USER}: utilizzo lazyconstraint callback e usercut callback in linguaggio \textit{C};
\end{itemize}

Notiamo che modificare un modello matematico attraverso l'eliminazione di alcune variabili può causare l'invalidazione di tutte le soluzioni originali: questi casi sono di seguito riconoscibili da un tempo di esecuzione medio pari a $0,00$ secondi.

\newpage

\begin{table}
\begin{adjustbox}{center}
\begin{tabular}{|c|c|c|c|c|c|c|c|c|c|}
    \cline{2-10} 
        \multicolumn{1}{c|}{} & \multirow{3}{*}{\textbf{LOOP}} & \multirow{3}{*}{\textbf{L EG5}} & \multirow{3}{*}{\textbf{L EG10}} & \multirow{3}{*}{\textbf{L LA5}} & \multirow{3}{*}{\textbf{L LA10}} & \multirow{3}{*}{\textbf{L EG5 LA5}} & \multirow{3}{*}{\textbf{L EG10 LA10}} & \multirow{3}{*}{\textbf{LAZY}} & \multirow{3}{*}{\textbf{USER}}\tabularnewline
        \multicolumn{1}{c|}{} &  &  &  &  &  &  &  &  & \tabularnewline
        \multicolumn{1}{c|}{} &  &  &  &  &  &  &  &  & \tabularnewline
        \hline 
        \textbf{\hyperref[fig:berlin52]{berlin52}} & 0,111 & 0,083 & 0,073 & 0,092 & 0,053 & 0,096 & 0,051 & 0,028 & 0,024\tabularnewline
        \hline 
        \textbf{\hyperref[fig:st70]{st70}} & 0,628 & 0,577 & 0,579 & 0,000 & 0,193 & 0,000 & 0,187 & 0,298 & 0,363\tabularnewline
        \hline 
        \textbf{\hyperref[fig:eil76]{eil76}} & 0,297 & 0,254 & 0,235 & 0,204 & 0,075 & 0,204 & 0,075 & 0,107 & 0,084\tabularnewline
        \hline 
        \textbf{\hyperref[fig:pr76]{pr76}} & 3,586 & 3,572 & 3,305 & 0,000 & 2,958 & 0,000 & 2,589 & 6,344 & 3,178\tabularnewline
        \hline 
        \textbf{\hyperref[fig:rat99]{rat99}} & 1,040 & 1,010 & 1,041 & 0,845 & 0,458 & 0,670 & 0,450 & 0,396 & 0,468\tabularnewline
        \hline 
        \textbf{\hyperref[fig:kroA100]{kroA100}} & 2,009 & 1,959 & 1,891 & 1,544 & 0,770 & 1,504 & 0,777 & 0,915 & 1,508\tabularnewline
        \hline 
        \textbf{\hyperref[fig:kroB100]{kroB100}} & 3,745 & 4,016 & 3,031 & 1,767 & 1,820 & 2,026 & 1,699 & 0,785 & 1,026\tabularnewline
        \hline 
        \textbf{\hyperref[fig:kroC100]{kroC100}} & 2,306 & 1,850 & 1,908 & 2,197 & 0,823 & 1,878 & 0,821 & 0,648 & 0,996\tabularnewline
        \hline 
        \textbf{\hyperref[fig:kroD100]{kroD100}} & 2,397 & 1,758 & 1,990 & 2,547 & 0,896 & 1,903 & 0,934 & 0,514 & 0,769\tabularnewline
        \hline 
        \textbf{\hyperref[fig:kroE100]{kroE100}} & 2,017 & 2,243 & 2,290 & 2,168 & 1,137 & 1,793 & 0,936 & 1,084 & 0,900\tabularnewline
        \hline 
        \textbf{\hyperref[fig:eil101]{eil101}} & 0,952 & 0,851 & 0,802 & 0,426 & 0,370 & 0,433 & 0,414 & 0,363 & 0,342\tabularnewline
        \hline 
        \textbf{\hyperref[fig:lin105]{lin105}} & 1,402 & 0,897 & 0,959 & 0,664 & 1,000 & 0,625 & 0,588 & 0,545 & 1,071\tabularnewline
        \hline 
        \textbf{\hyperref[fig:bier127]{bier127}} & 1,862 & 2,228 & 3,077 & 1,251 & 0,981 & 1,222 & 0,905 & 1,093 & 1,636\tabularnewline
        \hline 
        \textbf{\hyperref[fig:ch130]{ch130}} & 2,013 & 2,213 & 2,198 & 1,625 & 0,980 & 1,911 & 1,052 & 1,373 & 1,589\tabularnewline
        \hline 
        \textbf{\hyperref[fig:pr144]{pr144}} & 10,247 & 6,362 & 4,023 & 0,000 & 7,427 & 0,000 & 5,703 & 3,300 & 2,020\tabularnewline
        \hline 
        \textbf{\hyperref[fig:kroA150]{kroA150}} & 9,446 & 8,455 & 7,329 & 3,875 & 4,524 & 3,536 & 4,096 & 2,943 & 3,058\tabularnewline
        \hline 
        \textbf{\hyperref[fig:kroB150]{kroB150}} & 15,815 & 10,842 & 12,213 & 9,041 & 7,643 & 5,165 & 8,266 & 8,473 & 5,067\tabularnewline
        \hline 
        \textbf{\hyperref[fig:ch150]{ch150}} & 4,563 & 5,158 & 6,103 & 2,348 & 2,612 & 3,034 & 3,987 & 2,413 & 2,481\tabularnewline
        \hline 
        \textbf{\hyperref[fig:pr152]{pr152}} & 4,806 & 3,961 & 3,610 & 0,000 & 0,000 & 0,000 & 0,000 & 2,638 & 3,130\tabularnewline
        \hline 
        \textbf{\hyperref[fig:u159]{u159}} & 3,310 & 2,816 & 2,766 & 1,538 & 2,146 & 1,339 & 2,364 & 1,876 & 2,759\tabularnewline
        \hline 
        \textbf{\hyperref[fig:rat195]{rat195}} & 25,914 & 19,111 & 19,606 & 13,473 & 15,830 & 11,534 & 14,198 & 13,354 & 4,130\tabularnewline
        \hline 
\end{tabular}
\end{adjustbox}
\caption{Tabella risultati instanze con numero di nodi inferiore a \textbf{$200$} $+$ algoritmi esatti}
\end{table}

\begin{figure}[htbp]
    \centering
    \scalebox{.75}{\includegraphics{Immagini/GRAFI_ESATTI/"berlin52".png}}
    \label{fig:berlin52}
    \caption{berlin52}
\end{figure}

\begin{figure}[htbp]
    \centering
    \scalebox{.75}{\includegraphics{Immagini/GRAFI_ESATTI/"st70".png}}
    \label{fig:st70}
    \caption{st70}
\end{figure}

\begin{figure}[htbp]
    \centering
    \scalebox{.75}{\includegraphics{Immagini/GRAFI_ESATTI/"eil76".png}}
    \label{fig:eil76}
    \caption{eil76}
\end{figure}

\begin{figure}[htbp]
    \centering
    \scalebox{.75}{\includegraphics{Immagini/GRAFI_ESATTI/"pr76".png}}
    \label{fig:pr76}
    \caption{pr76}
\end{figure}

\begin{figure}[htbp]
\centering
\scalebox{.75}{\includegraphics{Immagini/GRAFI_ESATTI/"rat99".png}}
\label{fig:rat99}
\caption{rat99}
\end{figure}

\begin{figure}[htbp]
\centering
\scalebox{.75}{\includegraphics{Immagini/GRAFI_ESATTI/"kroA100".png}}
\label{fig:kroA100}
\caption{kroA100}
\end{figure}

\begin{figure}[htbp]
\centering
\scalebox{.75}{\includegraphics{Immagini/GRAFI_ESATTI/"kroB100".png}}
\label{fig:kroB100}
\caption{kroB100}
\end{figure}

\begin{figure}[htbp]
\centering
\scalebox{.75}{\includegraphics{Immagini/GRAFI_ESATTI/"kroC100".png}}
\label{fig:kroC100}
\caption{kroC100}
\end{figure}

\begin{figure}[htbp]
\centering
\scalebox{.75}{\includegraphics{Immagini/GRAFI_ESATTI/"kroD100".png}}
\label{fig:kroD100}
\caption{kroD100}
\end{figure}

\begin{figure}[htbp]
\centering
\scalebox{.75}{\includegraphics{Immagini/GRAFI_ESATTI/"kroE100".png}}
\label{fig:kroE100}
\caption{kroE100}
\end{figure}

\begin{figure}[htbp]
\centering
\scalebox{.75}{\includegraphics{Immagini/GRAFI_ESATTI/"eil101".png}}
\label{fig:eil101}
\caption{eil101}
\end{figure}

\begin{figure}[htbp]
\centering
\scalebox{.75}{\includegraphics{Immagini/GRAFI_ESATTI/"lin105".png}}
\label{fig:lin105}
\caption{lin105}
\end{figure}

\begin{figure}[htbp]
\centering
\scalebox{.75}{\includegraphics{Immagini/GRAFI_ESATTI/"bier127".png}}
\label{fig:bier127}
\caption{bier127}
\end{figure}

\begin{figure}[htbp]
\centering
\scalebox{.75}{\includegraphics{Immagini/GRAFI_ESATTI/"ch130".png}}
\label{fig:ch130}
\caption{ch130}
\end{figure}

\begin{figure}[htbp]
\centering
\scalebox{.75}{\includegraphics{Immagini/GRAFI_ESATTI/"pr144".png}}
\label{fig:pr144}
\caption{pr144}
\end{figure}

\begin{figure}[htbp]
\centering
\scalebox{.75}{\includegraphics{Immagini/GRAFI_ESATTI/"kroA150".png}}
\label{fig:kroA150}
\caption{kroA150}
\end{figure}

\begin{figure}[htbp]
\centering
\scalebox{.75}{\includegraphics{Immagini/GRAFI_ESATTI/"kroB150".png}}
\label{fig:kroB150}
\caption{kroB150}
\end{figure}

\begin{figure}[htbp]
\centering
\scalebox{.75}{\includegraphics{Immagini/GRAFI_ESATTI/"ch150".png}}
\label{fig:ch150}
\caption{ch150}
\end{figure}

\begin{figure}[htbp]
\centering
\scalebox{.75}{\includegraphics{Immagini/GRAFI_ESATTI/"pr152".png}}
\label{fig:pr152}
\caption{pr152}
\end{figure}

\begin{figure}[htbp]
\centering
\scalebox{.75}{\includegraphics{Immagini/GRAFI_ESATTI/"u159".png}}
\label{fig:u159}
\caption{u159}
\end{figure}

\begin{figure}[htbp]
\centering
\scalebox{.75}{\includegraphics{Immagini/GRAFI_ESATTI/"rat195".png}}
\label{fig:rat195}
\caption{rat195}
\end{figure}

\FloatBarrier

\subsection*{Instanze con numero di nodi compreso tra \textbf{$200$} e \textbf{$299$} $+$ algoritmi esatti}

Questo set di risultati ha le stesse identiche premesse indicate per il precedente se non che il numero di run eseguito per ogni coppia istanza/algoritmo è pari a $2$ e non $5$. Il motivo di tale scelta è puramente per una questione temporale.

Come più volte ripetuto è stato posto un tempo limite pari a $30$ minuti il quale è stato raggiunto dall'istanza \textbf{pr99} se risolta attraverso l'utilizzo della sola \textit{LazyConstraint Callback}: il valore medio comunicato da CPLEX per ottenere il risultato ottimo era in questo di circa lo $0,36\%$.

\vspace*{\fill}
{
\centering
\centerline{
\begin{tabular}{|c|c|c|c|c|c|c|c|c|c|}
        \cline{2-10} 
        \multicolumn{1}{c|}{} & \multirow{3}{*}{\textbf{LOOP}} & \multirow{3}{*}{\textbf{L EG5}} & \multirow{3}{*}{\textbf{L EG10}} & \multirow{3}{*}{\textbf{L LA5}} & \multirow{3}{*}{\textbf{L LA10}} & \multirow{3}{*}{\textbf{L EG5 LA5}} & \multirow{3}{*}{\textbf{L EG10 LA10}} & \multirow{3}{*}{\textbf{LAZY}} & \multirow{3}{*}{\textbf{USER}}\tabularnewline
        \multicolumn{1}{c|}{} &  &  &  &  &  &  &  &  & \tabularnewline
        \multicolumn{1}{c|}{} &  &  &  &  &  &  &  &  & \tabularnewline
        \hline 
        \textbf{\hyperref[fig:kroA200]{kroA200}} & 26,12 & 35,36 & 35,02 & 23,71 & 25,25 & 20,73 & 23,20 & 61,62 & 65,22\tabularnewline
        \hline 
        \textbf{\hyperref[fig:kroB200]{kroB200}} & 10,92 & 13,53 & 13,93 & 4,86 & 6,44 & 5,67 & 5,11 & 6,05 & 5,21\tabularnewline
        \hline 
        \textbf{\hyperref[fig:tsp225]{tsp225}} & 28,10 & 26,72 & 28,15 & 11,61 & 15,46 & 14,85 & 15,20 & 23,49 & 6,70\tabularnewline
        \hline 
        \textbf{\hyperref[fig:pr226]{pr226}} & 326,40 & 32,52 & 16,64 & 0,00 & 0,00 & 0,00 & 0,00 & 7,97 & 4,90\tabularnewline
        \hline 
        \textbf{\hyperref[fig:gil262]{gil262}} & 26,86 & 40,91 & 27,55 & 8,95 & 14,02 & 18,27 & 20,29 & 79,30 & 37,50\tabularnewline
        \hline 
        \textbf{\hyperref[fig:a280]{a280}} & 19,00 & 29,20 & 16,74 & 5,03 & 6,48 & 4,75 & 10,06 & 6,46 & 5,56\tabularnewline
        \hline 
        \textbf{\hyperref[fig:pr299]{pr299}} & 102,89 & 142,75 & 128,17 & 67,00 & 85,60 & 97,19 & 101,65 & 1800,00 & 27,52\tabularnewline
        \hline 
\end{tabular}
}
\captionof{table}{Tabella risultati instanze con numero di nodi compreso tra \textbf{$200$} e \textbf{$299$} $+$ algoritmi esatti}
}
\vspace*{\fill}

\begin{figure}[htbp]
    \centering
    \scalebox{.75}{\includegraphics{Immagini/GRAFI_ESATTI/"kroA200".png}}
    \label{fig:kroA200}
    \caption{kroA200}
\end{figure}

\begin{figure}[htbp]
    \centering
    \scalebox{.75}{\includegraphics{Immagini/GRAFI_ESATTI/"kroB200".png}}
    \label{fig:kroB200}
    \caption{kroB200}
\end{figure}

\begin{figure}[htbp]
    \centering
    \scalebox{.75}{\includegraphics{Immagini/GRAFI_ESATTI/"tsp225".png}}
    \label{fig:tsp225}
    \caption{tsp225}
\end{figure}

\begin{figure}[htbp]
    \centering
    \scalebox{.75}{\includegraphics{Immagini/GRAFI_ESATTI/"pr226".png}}
    \label{fig:pr226}
    \caption{pr226}
\end{figure}

\begin{figure}[htbp]
    \centering
    \scalebox{.75}{\includegraphics{Immagini/GRAFI_ESATTI/"gil262".png}}
    \label{fig:gil262}
    \caption{gil262}
\end{figure}

\begin{figure}[htbp]
    \centering
    \scalebox{.75}{\includegraphics{Immagini/GRAFI_ESATTI/"a280".png}}
    \label{fig:a280}
    \caption{a280}
\end{figure}

\begin{figure}[htbp]
    \centering
    \scalebox{.75}{\includegraphics{Immagini/GRAFI_ESATTI/"pr299".png}}
    \label{fig:pr299}
    \caption{pr299}
\end{figure}

\FloatBarrier

\subsection*{Instanze con numero di nodi compreso tra \textbf{$300$} e \textbf{$999$} $+$ algoritmi euristici}

\subsubsection*{MULTI START}

\vspace*{\fill}
{
    \centering
    \centerline{
        \begin{tabular}{|c|c|c|c|c|c|c|c|}
            \hline 
            \multicolumn{8}{|c|}{lin318 MULTI START (costo ottimo 42029)}\tabularnewline
            \hline 
            \hline 
            \multicolumn{2}{|c|}{Thread1} & \multicolumn{2}{c|}{Thread2} & \multicolumn{2}{c|}{Thread3} & \multicolumn{2}{c|}{Thread4}\tabularnewline
            \hline 
            Costo & Tempo (s) & Costo & Tempo (s) & Costo & Tempo (s) & Costo & Tempo (s)\tabularnewline
            \hline 
            44098 & 0,6 & 45374 & 0,6 & 45153 & 0,6 & 45442 & 0,6\tabularnewline
            \hline 
            43987 & 3,6 & 43904 & 1,7 & 44188 & 1,1 & 44054 & 1,5\tabularnewline
            \hline 
            43962 & 6,6 & 43663 & 4,1 & 43976 & 2,3 & 43695 & 1,9\tabularnewline
            \hline 
            43783 & 10,9 & 43620 & 14,9 & 43960 & 6,5 & 43139 & 40,6\tabularnewline
            \hline 
            43641 & 11,9 & 43549 & 20,7 & 43688 & 15,7 & 43094 & 684,7\tabularnewline
            \hline 
            43575 & 112,5 & 43231 & 319,1 & 43358 & 41,6 & 43094 & 1800,0\tabularnewline
            \hline 
            43286 & 145,3 & 43231 & 1800,0 & 43315 & 639,4 & \multicolumn{1}{c}{} & \multicolumn{1}{c}{}\tabularnewline
            \cline{1-6} 
            42935 & 872,0 & \multicolumn{1}{c}{} &  & 43230 & 833,4 & \multicolumn{1}{c}{} & \multicolumn{1}{c}{}\tabularnewline
            \cline{1-2} \cline{5-6} 
            42935 & 1800,0 & \multicolumn{1}{c}{} &  & 43020 & 1196,2 & \multicolumn{1}{c}{} & \multicolumn{1}{c}{}\tabularnewline
            \cline{1-2} \cline{5-6} 
            \multicolumn{1}{c}{} & \multicolumn{1}{c}{} & \multicolumn{1}{c}{} &  & 43020 & 1800,0 & \multicolumn{1}{c}{} & \multicolumn{1}{c}{}\tabularnewline
            \cline{5-6} 
        \end{tabular}
    }
    \captionof{table}{Tabella risultati instanze con numero di nodi compreso tra \textbf{$200$} e \textbf{$299$} $+$ algoritmi esatti}
}
\vspace*{\fill}

\vspace*{\fill}
{
    \centering
    \centerline{
\begin{tabular}{|c|c|c|c|c|c|c|c|}
    \hline 
    \multicolumn{8}{|c|}{pr439 MULTI START (costo ottimo 107217)}\tabularnewline
    \hline 
    \hline 
    \multicolumn{2}{|c|}{Thread1} & \multicolumn{2}{c|}{Thread2} & \multicolumn{2}{c|}{Thread3} & \multicolumn{2}{c|}{Thread4}\tabularnewline
    \hline 
    Costo & Tempo(s) & Costo & Tempo(s) & Costo & Tempo(s) & Costo & Tempo(s)\tabularnewline
    \hline 
    116111 & 1,2 & 112354 & 1,4 & 116794 & 0,9 & 116014 & 1,5\tabularnewline
    \hline 
    114306 & 2,4 & 112228 & 8,2 & 114316 & 1,7 & 111574 & 3,2\tabularnewline
    \hline 
    112609 & 6,7 & 111791 & 13,9 & 112781 & 7,9 & 109631 & 6,2\tabularnewline
    \hline 
    112481 & 11,7 & 110439 & 21,9 & 111045 & 18,0 & 109539 & 198,4\tabularnewline
    \hline 
    110785 & 19,8 & 110324 & 32,9 & 110899 & 34,3 & 109395 & 901,7\tabularnewline
    \hline 
    110182 & 146,5 & 109949 & 104,7 & 110584 & 78,9 & 109240 & 1158,3\tabularnewline
    \hline 
    110122 & 168,2 & 109641 & 294,3 & 110424 & 172,5 & 109240 & 1800,0\tabularnewline
    \hline 
    109928 & 366,4 & 109519 & 544,3 & 109705 & 213,4 & \multicolumn{1}{c}{} & \multicolumn{1}{c}{}\tabularnewline
    \cline{1-6} 
    109823 & 416,9 & 109270 & 1119,0 & 109659 & 526,0 & \multicolumn{1}{c}{} & \multicolumn{1}{c}{}\tabularnewline
    \cline{1-6} 
    109352 & 1066,4 & 109220 & 1431,2 & 109508 & 706,3 & \multicolumn{1}{c}{} & \multicolumn{1}{c}{}\tabularnewline
    \cline{1-6} 
    109335 & 1253,0 & 109220 & 1800,0 & 109508 & 1800,0 & \multicolumn{1}{c}{} & \multicolumn{1}{c}{}\tabularnewline
    \cline{1-6} 
    109335 & 1800,0 & \multicolumn{1}{c}{} & \multicolumn{1}{c}{} & \multicolumn{1}{c}{} & \multicolumn{1}{c}{} & \multicolumn{1}{c}{} & \multicolumn{1}{c}{}\tabularnewline
    \cline{1-2} 
\end{tabular}
}
\captionof{table}{Tabella risultati instanze con numero di nodi compreso tra \textbf{$200$} e \textbf{$299$} $+$ algoritmi esatti}
}
\vspace*{\fill}

\vspace*{\fill}
{
    \centering
    \centerline{
        \begin{tabular}{|c|c|c|c|c|c|c|c|}
            \hline 
            \multicolumn{8}{|c|}{d493 MULTI START (costo ottimo 35002)}\tabularnewline
            \hline 
            \hline 
            \multicolumn{2}{|c|}{Thread1} & \multicolumn{2}{c|}{Thread2} & \multicolumn{2}{c|}{Thread3} & \multicolumn{2}{c|}{Thread4}\tabularnewline
            \hline 
            Costo & Tempo(s) & Costo & Tempo(s) & Costo & Tempo(s) & Costo & Tempo(s)\tabularnewline
            \hline 
            36948 & 1,8 & 36912 & 2,4 & 36904 & 1,9 & 36459 & 1,6\tabularnewline
            \hline 
            36857 & 4,8 & 36895 & 4,9 & 36889 & 5,9 & 36409 & 37,6\tabularnewline
            \hline 
            36673 & 13,9 & 36821 & 12,1 & 36672 & 10,0 & 36256 & 91,0\tabularnewline
            \hline 
            36304 & 30,6 & 36771 & 20,6 & 36607 & 11,0 & 36175 & 1008,1\tabularnewline
            \hline 
            36257 & 191,1 & 36767 & 22,0 & 36393 & 15,8 & 36069 & 1059,5\tabularnewline
            \hline 
            36179 & 431,2 & 36633 & 31,5 & 36207 & 17,7 & 36069 & 1800,0\tabularnewline
            \hline 
            36157 & 1019,7 & 36588 & 32,3 & 36021 & 27,7 & \multicolumn{1}{c}{} & \multicolumn{1}{c}{}\tabularnewline
            \cline{1-6} 
            36157 & 1800,0 & 36534 & 68,8 & 36021 & 1800,0 & \multicolumn{1}{c}{} & \multicolumn{1}{c}{}\tabularnewline
            \cline{1-6} 
            \multicolumn{1}{c}{} &  & 36441 & 90,3 & \multicolumn{1}{c}{} & \multicolumn{1}{c}{} & \multicolumn{1}{c}{} & \multicolumn{1}{c}{}\tabularnewline
            \cline{3-4} 
            \multicolumn{1}{c}{} &  & 36284 & 111,9 & \multicolumn{1}{c}{} & \multicolumn{1}{c}{} & \multicolumn{1}{c}{} & \multicolumn{1}{c}{}\tabularnewline
            \cline{3-4} 
            \multicolumn{1}{c}{} &  & 36253 & 278,9 & \multicolumn{1}{c}{} & \multicolumn{1}{c}{} & \multicolumn{1}{c}{} & \multicolumn{1}{c}{}\tabularnewline
            \cline{3-4} 
            \multicolumn{1}{c}{} &  & 36026 & 556,5 & \multicolumn{1}{c}{} & \multicolumn{1}{c}{} & \multicolumn{1}{c}{} & \multicolumn{1}{c}{}\tabularnewline
            \cline{3-4} 
            \multicolumn{1}{c}{} &  & 36026 & 1800,0 & \multicolumn{1}{c}{} & \multicolumn{1}{c}{} & \multicolumn{1}{c}{} & \multicolumn{1}{c}{}\tabularnewline
            \cline{3-4} 
        \end{tabular}
    }
    \captionof{table}{Tabella risultati instanze con numero di nodi compreso tra \textbf{$200$} e \textbf{$299$} $+$ algoritmi esatti}
}
\vspace*{\fill}

\vspace*{\fill}
{
    \centering
    \centerline{
        \begin{tabular}{cc|c|c|c|c|c|c|}
            \hline 
            \multicolumn{8}{|c|}{rat575 MULTI START (costo ottimo 6773)}\tabularnewline
            \hline 
            \hline 
            \multicolumn{2}{|c|}{Thread1} & \multicolumn{2}{c|}{Thread2} & \multicolumn{2}{c|}{Thread3} & \multicolumn{2}{c|}{Thread4}\tabularnewline
            \hline 
            \multicolumn{1}{|c|}{Costo} & Tempo(s) & Costo & Tempo(s) & Costo & Tempo(s) & Costo & Tempo(s)\tabularnewline
            \hline 
            \multicolumn{1}{|c|}{7198} & 2,665 & 7284 & 2,226 & 7280 & 2,828 & 7230 & 2,307\tabularnewline
            \hline 
            \multicolumn{1}{|c|}{7077} & 33,559 & 7208 & 6,366 & 7193 & 7,738 & 7203 & 4,532\tabularnewline
            \hline 
            \multicolumn{1}{|c|}{7077} & 1800 & 7192 & 8,302 & 7166 & 9,767 & 7201 & 13,739\tabularnewline
            \hline 
            &  & 7186 & 10,762 & 7161 & 26,771 & 7149 & 15,732\tabularnewline
            \cline{3-8} 
            &  & 7164 & 12,977 & 7141 & 46,01 & 7141 & 47,113\tabularnewline
            \cline{3-8} 
            &  & 7153 & 18,606 & 7100 & 74,421 & 7118 & 356,143\tabularnewline
            \cline{3-8} 
            &  & 7152 & 26,139 & 7081 & 293,085 & 7060 & 613,947\tabularnewline
            \cline{3-8} 
            &  & 7122 & 83,84 & 7079 & 860,752 & 7055 & 870,445\tabularnewline
            \cline{3-8} 
            &  & 7084 & 295,355 & 7079 & 1800 & 7055 & 1800\tabularnewline
            \cline{3-8} 
            &  & 7084 & 1800 & \multicolumn{1}{c}{} & \multicolumn{1}{c}{} & \multicolumn{1}{c}{} & \multicolumn{1}{c}{}\tabularnewline
            \cline{3-4} 
        \end{tabular}
    }
    \captionof{table}{Tabella risultati instanze con numero di nodi compreso tra \textbf{$200$} e \textbf{$299$} $+$ algoritmi esatti}
}
\vspace*{\fill}

\vspace*{\fill}
{
    \centering
    \centerline{
        \begin{tabular}{|c|c|c|c|c|c|c|c|}
            \hline 
            \multicolumn{8}{|c|}{d657 MULTI START (costo ottimo 48912)}\tabularnewline
            \hline 
            \hline 
            \multicolumn{2}{|c|}{Thread1} & \multicolumn{2}{c|}{Thread2} & \multicolumn{2}{c|}{Thread3} & \multicolumn{2}{c|}{Thread4}\tabularnewline
            \hline 
            Costo & Tempo(s) & Costo & Tempo(s) & Costo & Tempo(s) & Costo & Tempo(s)\tabularnewline
            \hline 
            53830 & 3,2 & 52927 & 3,7 & 53398 & 3,2 & 52952 & 3,9\tabularnewline
            \hline 
            52167 & 6,4 & 52766 & 6,1 & 53368 & 6,3 & 52463 & 13,1\tabularnewline
            \hline 
            51703 & 12,8 & 52504 & 13,5 & 51720 & 10,5 & 51797 & 21,5\tabularnewline
            \hline 
            51680 & 27,4 & 52179 & 20,4 & 51550 & 58,2 & 51246 & 54,7\tabularnewline
            \hline 
            51561 & 47,3 & 51656 & 56,0 & 51512 & 317,0 & 50974 & 1030,6\tabularnewline
            \hline 
            51509 & 236,7 & 51430 & 85,7 & 51419 & 635,9 & 50974 & 1800,0\tabularnewline
            \hline 
            51406 & 329,1 & 51362 & 752,1 & 51230 & 895,8 & \multicolumn{1}{c}{} & \multicolumn{1}{c}{}\tabularnewline
            \cline{1-6} 
            51187 & 679,9 & 51109 & 941,9 & 51230 & 1800,0 & \multicolumn{1}{c}{} & \multicolumn{1}{c}{}\tabularnewline
            \cline{1-6} 
            51032 & 861,0 & 51109 & 1800,0 & \multicolumn{1}{c}{} & \multicolumn{1}{c}{} & \multicolumn{1}{c}{} & \multicolumn{1}{c}{}\tabularnewline
            \cline{1-4} 
            51032 & 1800,0 & \multicolumn{1}{c}{} & \multicolumn{1}{c}{} & \multicolumn{1}{c}{} & \multicolumn{1}{c}{} & \multicolumn{1}{c}{} & \multicolumn{1}{c}{}\tabularnewline
            \cline{1-2} 
        \end{tabular}
    }
    \captionof{table}{Tabella risultati instanze con numero di nodi compreso tra \textbf{$200$} e \textbf{$299$} $+$ algoritmi esatti}
}
\vspace*{\fill}

\vspace*{\fill}
{
    \centering
    \centerline{
        \begin{tabular}{|c|c|c|c|c|c|c|c|}
            \hline 
            \multicolumn{8}{|c|}{u724 MULTI START (costo ottimo 41910)}\tabularnewline
            \hline 
            \hline 
            \multicolumn{2}{|c|}{Thread1} & \multicolumn{2}{c|}{Thread2} & \multicolumn{2}{c|}{Thread3} & \multicolumn{2}{c|}{Thread4}\tabularnewline
            \hline 
            Costo & Tempo (s) & Costo & Tempo (s) & Costo & Tempo (s) & Costo & Tempo (s)\tabularnewline
            \hline 
            45295 & 6,1 & 45000 & 6,9 & 45065 & 6,4 & 44836 & 6,5\tabularnewline
            \hline 
            44761 & 10,6 & 44795 & 11,9 & 44891 & 17,9 & 44189 & 17,4\tabularnewline
            \hline 
            44535 & 19,3 & 44530 & 28,7 & 44528 & 22,9 & 44182 & 84,6\tabularnewline
            \hline 
            44079 & 48,2 & 44163 & 41,4 & 44460 & 149,9 & 44161 & 259,9\tabularnewline
            \hline 
            43864 & 1290,6 & 44151 & 72,2 & 44451 & 244,1 & 44054 & 386,8\tabularnewline
            \hline 
            43864 & 1800,0 & 44136 & 372,2 & 44439 & 337,9 & 43749 & 990,8\tabularnewline
            \hline 
            \multicolumn{1}{c}{} &  & 43979 & 603,3 & 43977 & 371,0 & 43711 & 1135,6\tabularnewline
            \cline{3-8} 
            \multicolumn{1}{c}{} &  & 43856 & 760,0 & 43971 & 1245,1 & 43711 & 1800,0\tabularnewline
            \cline{3-8} 
            \multicolumn{1}{c}{} &  & 43852 & 1420,2 & 43953 & 1303,0 & \multicolumn{1}{c}{} & \multicolumn{1}{c}{}\tabularnewline
            \cline{3-6} 
            \multicolumn{1}{c}{} &  & 43852 & 1800,0 & 43915 & 1380,6 & \multicolumn{1}{c}{} & \multicolumn{1}{c}{}\tabularnewline
            \cline{3-6} 
            \multicolumn{1}{c}{} & \multicolumn{1}{c}{} & \multicolumn{1}{c}{} &  & 43915 & 1800,0 & \multicolumn{1}{c}{} & \multicolumn{1}{c}{}\tabularnewline
            \cline{5-6} 
        \end{tabular}
    }
    \captionof{table}{Tabella risultati instanze con numero di nodi compreso tra \textbf{$200$} e \textbf{$299$} $+$ algoritmi esatti}
}
\vspace*{\fill}

\vspace*{\fill}
{
    \centering
    \centerline{
        \begin{tabular}{cccc|c|c|c|c|}
            \hline 
            \multicolumn{8}{|c|}{rat783 MULTI START (costo ottimo 8806)}\tabularnewline
            \hline 
            \hline 
            \multicolumn{2}{|c|}{Thread1} & \multicolumn{2}{c|}{Thread2} & \multicolumn{2}{c|}{Thread3} & \multicolumn{2}{c|}{Thread4}\tabularnewline
            \hline 
            \multicolumn{1}{|c|}{Costo} & \multicolumn{1}{c|}{Tempo (s)} & \multicolumn{1}{c|}{Costo} & Tempo (s) & Costo & Tempo (s) & Costo & Tempo (s)\tabularnewline
            \hline 
            \multicolumn{1}{|c|}{9485} & \multicolumn{1}{c|}{6,2} & \multicolumn{1}{c|}{9460} & 7,6 & 9356 & 4,9 & 9401 & 5,0\tabularnewline
            \hline 
            \multicolumn{1}{|c|}{9315} & \multicolumn{1}{c|}{11,2} & \multicolumn{1}{c|}{9317} & 19,9 & 9314 & 8,6 & 9399 & 31,7\tabularnewline
            \hline 
            \multicolumn{1}{|c|}{9284} & \multicolumn{1}{c|}{434,1} & \multicolumn{1}{c|}{9265} & 395,6 & 9310 & 128,0 & 9376 & 88,8\tabularnewline
            \hline 
            \multicolumn{1}{|c|}{9284} & \multicolumn{1}{c|}{1800,0} & \multicolumn{1}{c|}{9265} & 1800,0 & 9284 & 323,8 & 9369 & 127,7\tabularnewline
            \hline 
            &  &  &  & 9264 & 1071,2 & 9368 & 136,2\tabularnewline
            \cline{5-8} 
            &  &  &  & 9264 & 1800,0 & 9358 & 175,5\tabularnewline
            \cline{5-8} 
            &  &  & \multicolumn{1}{c}{} & \multicolumn{1}{c}{} &  & 9320 & 262,4\tabularnewline
            \cline{7-8} 
            &  &  & \multicolumn{1}{c}{} & \multicolumn{1}{c}{} &  & 9312 & 377,0\tabularnewline
            \cline{7-8} 
            &  &  & \multicolumn{1}{c}{} & \multicolumn{1}{c}{} &  & 9304 & 559,9\tabularnewline
            \cline{7-8} 
            &  &  & \multicolumn{1}{c}{} & \multicolumn{1}{c}{} &  & 9264 & 785,0\tabularnewline
            \cline{7-8} 
            &  &  & \multicolumn{1}{c}{} & \multicolumn{1}{c}{} &  & 9242 & 1207,1\tabularnewline
            \cline{7-8} 
            &  &  & \multicolumn{1}{c}{} & \multicolumn{1}{c}{} &  & 9242 & 1800,0\tabularnewline
            \cline{7-8} 
        \end{tabular}
    }
    \captionof{table}{Tabella risultati instanze con numero di nodi compreso tra \textbf{$200$} e \textbf{$299$} $+$ algoritmi esatti}
}
\vspace*{\fill}

\vspace*{\fill}
{
    \centering
    \centerline{
        \begin{tabular}{cc|c|c|c|c|c|c|}
            \hline 
            \multicolumn{8}{|c|}{dsj1000 MULTI START (costo ottimo 18659688 )}\tabularnewline
            \hline 
            \hline 
            \multicolumn{2}{|c|}{Thread1} & \multicolumn{2}{c|}{Thread2} & \multicolumn{2}{c|}{Thread3} & \multicolumn{2}{c|}{Thread4}\tabularnewline
            \hline 
            \multicolumn{1}{|c|}{Costo} & Tempo (s) & Costo & Tempo (s) & Costo & Tempo (s) & Costo & Tempo (s)\tabularnewline
            \hline 
            \multicolumn{1}{|c|}{20068535} & 19,4 & 20270696 & 13,6 & 20334609 & 14,3 & 19965496 & 21,9\tabularnewline
            \hline 
            \multicolumn{1}{|c|}{19852900} & 239,8 & 20138131 & 26,8 & 19978177 & 34,8 & 19960654 & 66,8\tabularnewline
            \hline 
            \multicolumn{1}{|c|}{19828957} & 279,7 & 19865783 & 79,5 & 19975814 & 213,2 & 19914229 & 165,1\tabularnewline
            \hline 
            \multicolumn{1}{|c|}{19828957} & 1800,0 & 19852720 & 1579,5 & 19946138 & 469,5 & 19818264 & 306,9\tabularnewline
            \hline 
            &  & 19741229 & 1720,5 & 19930787 & 607,3 & 19750658 & 838,6\tabularnewline
            \cline{3-8} 
            &  & 19741229 & 1800,0 & 19841513 & 718,0 & 19750658 & 1800,0\tabularnewline
            \cline{3-8} 
            & \multicolumn{1}{c}{} & \multicolumn{1}{c}{} &  & 19802820 & 1729,8 & \multicolumn{1}{c}{} & \multicolumn{1}{c}{}\tabularnewline
            \cline{5-6} 
            & \multicolumn{1}{c}{} & \multicolumn{1}{c}{} &  & 19802820 & 1800,0 & \multicolumn{1}{c}{} & \multicolumn{1}{c}{}\tabularnewline
            \cline{5-6} 
            & \multicolumn{1}{c}{} & \multicolumn{1}{c}{} & \multicolumn{1}{c}{} & \multicolumn{1}{c}{} & \multicolumn{1}{c}{} & \multicolumn{1}{c}{} & \multicolumn{1}{c}{}\tabularnewline
            & \multicolumn{1}{c}{} & \multicolumn{1}{c}{} & \multicolumn{1}{c}{} & \multicolumn{1}{c}{} & \multicolumn{1}{c}{} & \multicolumn{1}{c}{} & \multicolumn{1}{c}{}\tabularnewline
            & \multicolumn{1}{c}{} & \multicolumn{1}{c}{} & \multicolumn{1}{c}{} & \multicolumn{1}{c}{} & \multicolumn{1}{c}{} & \multicolumn{1}{c}{} & \multicolumn{1}{c}{}\tabularnewline
            & \multicolumn{1}{c}{} & \multicolumn{1}{c}{} & \multicolumn{1}{c}{} & \multicolumn{1}{c}{} & \multicolumn{1}{c}{} & \multicolumn{1}{c}{} & \multicolumn{1}{c}{}\tabularnewline
        \end{tabular}
    }
    \captionof{table}{Tabella risultati instanze con numero di nodi compreso tra \textbf{$200$} e \textbf{$299$} $+$ algoritmi esatti}
}
\vspace*{\fill}

\subsubsection*{TABU SEARCH}

\vspace*{\fill}
{
    \centering
    \centerline{\begin{tabular}{|cc|c|c|cccc|}
            \hline 
            \multicolumn{8}{|c|}{lin318 TABU SEARCH (costo ottimo 42029)}\tabularnewline
            \hline 
            \hline 
            \multicolumn{2}{|c|}{Thread1} & \multicolumn{2}{c|}{Thread2} & \multicolumn{2}{c|}{Thread3} & \multicolumn{2}{c|}{Thread4}\tabularnewline
            \hline 
            Costo & Tempo (s) & \multicolumn{1}{c|}{Costo} & Tempo (s) & Costo & Tempo (s) & \multicolumn{1}{c|}{Costo} & \multicolumn{1}{c|}{Tempo (s)}\tabularnewline
            \hline
            \multicolumn{1}{|c|}{42935} & 0 & 43231 & 0 & \multicolumn{1}{c|}{43020} & \multicolumn{1}{c|}{0} & \multicolumn{1}{c|}{43094} & 0\tabularnewline
            \hline 
            \multicolumn{1}{|c|}{42935} & 1800 & 43221 & 1240,856 & \multicolumn{1}{c|}{43020} & \multicolumn{1}{c|}{1800} & \multicolumn{1}{c|}{43094} & 1800\tabularnewline
            \hline 
            \multicolumn{1}{c}{} &  & 43194 & 1241,153 &  &  &  & \multicolumn{1}{c}{}\tabularnewline
            \cline{3-4} 
            \multicolumn{1}{c}{} &  & 43175 & 1241,67 &  &  &  & \multicolumn{1}{c}{}\tabularnewline
            \cline{3-4} 
            \multicolumn{1}{c}{} &  & 43156 & 1241,962 &  &  &  & \multicolumn{1}{c}{}\tabularnewline
            \cline{3-4} 
            \multicolumn{1}{c}{} &  & 43156 & 1800 &  &  &  & \multicolumn{1}{c}{}\tabularnewline
            \cline{3-4} 
        \end{tabular}
    }
    \captionof{table}{Tabella risultati instanze con numero di nodi compreso tra \textbf{$200$} e \textbf{$299$} $+$ algoritmi esatti}
}
\vspace*{\fill}

\vspace*{\fill}
{
    \centering
    \centerline{
        \begin{tabular}{|c|c|c|c|cccc}
            \hline 
            \multicolumn{8}{|c|}{pr439 TABU SEARCH (costo ottimo 107217)}\tabularnewline
            \hline 
            \hline 
            \multicolumn{2}{|c|}{Thread1} & \multicolumn{2}{c|}{Thread2} & \multicolumn{2}{c|}{Thread3} & \multicolumn{2}{c|}{Thread4}\tabularnewline
            \hline 
            Costo & Tempo (s) & \multicolumn{1}{c|}{Costo} & Tempo (s) & Costo & Tempo (s) & \multicolumn{1}{c|}{Costo} & \multicolumn{1}{c|}{Tempo (s)}\tabularnewline
            \hline 
            109335 & 0,0 & 109220 & 0,0 & \multicolumn{1}{c|}{109508} & \multicolumn{1}{c|}{0,0} & \multicolumn{1}{c|}{109240} & \multicolumn{1}{c|}{0,0}\tabularnewline
            \hline 
            109300 & 2,2 & 109158 & 1,2 & \multicolumn{1}{c|}{109499} & \multicolumn{1}{c|}{2,7} & \multicolumn{1}{c|}{109086} & \multicolumn{1}{c|}{4,1}\tabularnewline
            \hline 
            109245 & 3,7 & 108910 & 2,5 & \multicolumn{1}{c|}{109302} & \multicolumn{1}{c|}{4,2} & \multicolumn{1}{c|}{108967} & \multicolumn{1}{c|}{7,3}\tabularnewline
            \hline 
            109242 & 4,0 & 108906 & 252,5 & \multicolumn{1}{c|}{109070} & \multicolumn{1}{c|}{5,0} & \multicolumn{1}{c|}{108955} & \multicolumn{1}{c|}{7,8}\tabularnewline
            \hline 
            109188 & 4,4 & 108904 & 253,1 & \multicolumn{1}{c|}{109070} & \multicolumn{1}{c|}{1800,0} & \multicolumn{1}{c|}{108887} & \multicolumn{1}{c|}{8,3}\tabularnewline
            \hline 
            109185 & 293,8 & 108896 & 254,4 &  & \multicolumn{1}{c|}{} & \multicolumn{1}{c|}{108862} & \multicolumn{1}{c|}{1016,3}\tabularnewline
            \cline{1-4} \cline{7-8} 
            109175 & 294,7 & 108890 & 255,0 &  & \multicolumn{1}{c|}{} & \multicolumn{1}{c|}{108862} & \multicolumn{1}{c|}{1800,0}\tabularnewline
            \cline{1-4} \cline{7-8} 
            109173 & 295,6 & 108791 & 256,2 &  &  &  & \tabularnewline
            \cline{1-4} 
            109135 & 297,3 & 108766 & 256,8 &  &  &  & \tabularnewline
            \cline{1-4} 
            109110 & 298,1 & 108745 & 257,4 &  &  &  & \tabularnewline
            \cline{1-4} 
            109097 & 299,0 & 108742 & 257,9 &  &  &  & \tabularnewline
            \cline{1-4} 
            109051 & 300,6 & 108720 & 312,3 &  &  &  & \tabularnewline
            \cline{1-4} 
            109018 & 301,4 & 108715 & 312,8 &  &  &  & \tabularnewline
            \cline{1-4} 
            109001 & 302,9 & 108713 & 313,5 &  &  &  & \tabularnewline
            \cline{1-4} 
            108981 & 303,7 & 108654 & 353,6 &  &  &  & \tabularnewline
            \cline{1-4} 
            108977 & 304,4 & 108582 & 354,2 &  &  &  & \tabularnewline
            \cline{1-4} 
            108974 & 305,2 & 108571 & 354,8 &  &  &  & \tabularnewline
            \cline{1-4} 
            108927 & 306,5 & 108569 & 355,4 &  &  &  & \tabularnewline
            \cline{1-4} 
            108888 & 307,2 & 108530 & 356,6 &  &  &  & \tabularnewline
            \cline{1-4} 
            108884 & 307,9 & 108505 & 357,1 &  &  &  & \tabularnewline
            \cline{1-4} 
            108882 & 308,6 & 108480 & 357,7 &  &  &  & \tabularnewline
            \cline{1-4} 
            108837 & 310,5 & 108459 & 358,2 &  &  &  & \tabularnewline
            \cline{1-4} 
            108816 & 311,0 & 108449 & 452,2 &  &  &  & \tabularnewline
            \cline{1-4} 
            108814 & 311,6 & 108438 & 452,8 &  &  &  & \tabularnewline
            \cline{1-4} 
            108762 & 656,2 & 108432 & 453,5 &  &  &  & \tabularnewline
            \cline{1-4} 
            108737 & 656,8 & 108413 & 454,7 &  &  &  & \tabularnewline
            \cline{1-4} 
            108716 & 657,3 & 108392 & 455,2 &  &  &  & \tabularnewline
            \cline{1-4} 
            108696 & 657,9 & 108382 & 455,8 &  &  &  & \tabularnewline
            \cline{1-4} 
            108691 & 658,5 & 108379 & 456,4 &  &  &  & \tabularnewline
            \cline{1-4} 
            108691 & 1800,0 & 108379 & 1800,0 &  &  &  & \tabularnewline
            \cline{1-4} 
        \end{tabular}
    }
    \captionof{table}{Tabella risultati instanze con numero di nodi compreso tra \textbf{$200$} e \textbf{$299$} $+$ algoritmi esatti}
}
\vspace*{\fill}

\vspace*{\fill}
{
    \centering
    \centerline{
        \begin{tabular}{|c|c|c|c|cccc}
            \hline 
            \multicolumn{8}{|c|}{d493 TABU SEARCH (costo ottimo 35002)}\tabularnewline
            \hline 
            \hline 
            \multicolumn{2}{|c|}{Thread1} & \multicolumn{2}{c|}{Thread2} & \multicolumn{2}{c|}{Thread3} & \multicolumn{2}{c|}{Thread4}\tabularnewline
            \hline 
            Costo & Tempo (s) & \multicolumn{1}{c|}{Costo} & Tempo (s) & Costo & Tempo (s) & \multicolumn{1}{c|}{Costo} & \multicolumn{1}{c|}{Tempo (s)}\tabularnewline
            \hline 
            36157 & 0,0 & 36026 & 0,0 & \multicolumn{1}{c|}{36021} & \multicolumn{1}{c|}{0,0} & \multicolumn{1}{c|}{36069} & \multicolumn{1}{c|}{0,0}\tabularnewline
            \hline 
            36129 & 0,6 & 35954 & 0,8 & \multicolumn{1}{c|}{36019} & \multicolumn{1}{c|}{2,0} & \multicolumn{1}{c|}{36049} & \multicolumn{1}{c|}{1,0}\tabularnewline
            \hline 
            36128 & 1,5 & 35953 & 1,2 & \multicolumn{1}{c|}{36003} & \multicolumn{1}{c|}{2,7} & \multicolumn{1}{c|}{36033} & \multicolumn{1}{c|}{1,7}\tabularnewline
            \hline 
            36120 & 2,7 & 35911 & 3,0 & \multicolumn{1}{c|}{35988} & \multicolumn{1}{c|}{6,5} & \multicolumn{1}{c|}{36021} & \multicolumn{1}{c|}{208,5}\tabularnewline
            \hline 
            36113 & 12,6 & 35904 & 5,5 & \multicolumn{1}{c|}{35987} & \multicolumn{1}{c|}{1409,0} & \multicolumn{1}{c|}{36019} & \multicolumn{1}{c|}{209,2}\tabularnewline
            \hline 
            36091 & 14,7 & 35900 & 8,1 & \multicolumn{1}{c|}{35985} & \multicolumn{1}{c|}{1409,8} & \multicolumn{1}{c|}{36018} & \multicolumn{1}{c|}{401,4}\tabularnewline
            \hline 
            36071 & 17,8 & 35897 & 200,7 & \multicolumn{1}{c|}{35985} & \multicolumn{1}{c|}{1800,0} & \multicolumn{1}{c|}{36016} & \multicolumn{1}{c|}{402,1}\tabularnewline
            \hline 
            36063 & 18,7 & 35894 & 201,4 &  & \multicolumn{1}{c|}{} & \multicolumn{1}{c|}{36006} & \multicolumn{1}{c|}{477,1}\tabularnewline
            \cline{1-4} \cline{7-8} 
            36026 & 23,8 & 35891 & 202,3 &  & \multicolumn{1}{c|}{} & \multicolumn{1}{c|}{36004} & \multicolumn{1}{c|}{598,8}\tabularnewline
            \cline{1-4} \cline{7-8} 
            36024 & 24,7 & 35890 & 203,1 &  & \multicolumn{1}{c|}{} & \multicolumn{1}{c|}{36002} & \multicolumn{1}{c|}{599,6}\tabularnewline
            \cline{1-4} \cline{7-8} 
            36022 & 25,6 & 35885 & 204,5 &  & \multicolumn{1}{c|}{} & \multicolumn{1}{c|}{36001} & \multicolumn{1}{c|}{600,3}\tabularnewline
            \cline{1-4} \cline{7-8} 
            36021 & 320,8 & 35882 & 205,2 &  & \multicolumn{1}{c|}{} & \multicolumn{1}{c|}{35998} & \multicolumn{1}{c|}{666,1}\tabularnewline
            \cline{1-4} \cline{7-8} 
            36009 & 322,6 & 35881 & 256,0 &  & \multicolumn{1}{c|}{} & \multicolumn{1}{c|}{35996} & \multicolumn{1}{c|}{666,8}\tabularnewline
            \cline{1-4} \cline{7-8} 
            36002 & 324,5 & 35876 & 257,6 &  & \multicolumn{1}{c|}{} & \multicolumn{1}{c|}{35994} & \multicolumn{1}{c|}{667,6}\tabularnewline
            \cline{1-4} \cline{7-8} 
            35999 & 325,3 & 35873 & 258,4 &  & \multicolumn{1}{c|}{} & \multicolumn{1}{c|}{35993} & \multicolumn{1}{c|}{862,3}\tabularnewline
            \cline{1-4} \cline{7-8} 
            35998 & 326,2 & 35868 & 259,9 &  & \multicolumn{1}{c|}{} & \multicolumn{1}{c|}{35986} & \multicolumn{1}{c|}{863,1}\tabularnewline
            \cline{1-4} \cline{7-8} 
            35992 & 327,8 & 35865 & 260,6 &  & \multicolumn{1}{c|}{} & \multicolumn{1}{c|}{35967} & \multicolumn{1}{c|}{863,8}\tabularnewline
            \cline{1-4} \cline{7-8} 
            35991 & 328,6 & 35862 & 261,4 &  & \multicolumn{1}{c|}{} & \multicolumn{1}{c|}{35962} & \multicolumn{1}{c|}{864,7}\tabularnewline
            \cline{1-4} \cline{7-8} 
            35983 & 330,1 & 35859 & 262,1 &  & \multicolumn{1}{c|}{} & \multicolumn{1}{c|}{35959} & \multicolumn{1}{c|}{865,4}\tabularnewline
            \cline{1-4} \cline{7-8} 
            35976 & 330,8 & 35845 & 262,8 &  & \multicolumn{1}{c|}{} & \multicolumn{1}{c|}{35956} & \multicolumn{1}{c|}{866,2}\tabularnewline
            \cline{1-4} \cline{7-8} 
            35969 & 331,5 & 35842 & 325,6 &  & \multicolumn{1}{c|}{} & \multicolumn{1}{c|}{35954} & \multicolumn{1}{c|}{867,0}\tabularnewline
            \cline{1-4} \cline{7-8} 
            35966 & 332,2 & 35837 & 327,0 &  & \multicolumn{1}{c|}{} & \multicolumn{1}{c|}{35953} & \multicolumn{1}{c|}{867,7}\tabularnewline
            \cline{1-4} \cline{7-8} 
            35964 & 1082,4 & 35830 & 327,7 &  & \multicolumn{1}{c|}{} & \multicolumn{1}{c|}{35953} & \multicolumn{1}{c|}{1800,0}\tabularnewline
            \cline{1-4} \cline{7-8} 
            35955 & 1468,0 & 35827 & 328,5 &  &  &  & \tabularnewline
            \cline{1-4} 
            35955 & 1800,0 & 35826 & 908,5 &  &  &  & \tabularnewline
            \cline{1-4} 
            \multicolumn{1}{c}{} &  & 35823 & 1016,8 &  &  &  & \tabularnewline
            \cline{3-4} 
            \multicolumn{1}{c}{} &  & 35820 & 1017,6 &  &  &  & \tabularnewline
            \cline{3-4} 
            \multicolumn{1}{c}{} &  & 35817 & 1018,4 &  &  &  & \tabularnewline
            \cline{3-4} 
            \multicolumn{1}{c}{} &  & 35812 & 1021,4 &  &  &  & \tabularnewline
            \cline{3-4} 
            \multicolumn{1}{c}{} &  & 35810 & 1080,7 &  &  &  & \tabularnewline
            \cline{3-4} 
            \multicolumn{1}{c}{} &  & 35807 & 1081,3 &  &  &  & \tabularnewline
            \cline{3-4} 
            \multicolumn{1}{c}{} &  & 35805 & 1082,0 &  &  &  & \tabularnewline
            \cline{3-4} 
            \multicolumn{1}{c}{} &  & 35803 & 1082,7 &  &  &  & \tabularnewline
            \cline{3-4} 
            \multicolumn{1}{c}{} &  & 35801 & 1209,3 &  &  &  & \tabularnewline
            \cline{3-4} 
            \multicolumn{1}{c}{} &  & 35798 & 1210,0 &  &  &  & \tabularnewline
            \cline{3-4} 
            \multicolumn{1}{c}{} &  & 35795 & 1210,7 &  &  &  & \tabularnewline
            \cline{3-4} 
            \multicolumn{1}{c}{} &  & 35778 & 1392,0 &  &  &  & \tabularnewline
            \cline{3-4} 
            \multicolumn{1}{c}{} &  & 35775 & 1392,8 &  &  &  & \tabularnewline
            \cline{3-4} 
            \multicolumn{1}{c}{} &  & 35772 & 1393,6 &  &  &  & \tabularnewline
            \cline{3-4} 
            \multicolumn{1}{c}{} &  & 35771 & 1394,4 &  &  &  & \tabularnewline
            \cline{3-4} 
            \multicolumn{1}{c}{} &  & 35769 & 1395,8 &  &  &  & \tabularnewline
            \cline{3-4} 
            \multicolumn{1}{c}{} &  & 35766 & 1396,6 &  &  &  & \tabularnewline
            \cline{3-4} 
            \multicolumn{1}{c}{} &  & 35765 & 1397,3 &  &  &  & \tabularnewline
            \cline{3-4} 
            \multicolumn{1}{c}{} &  & 35755 & 1588,0 &  &  &  & \tabularnewline
            \cline{3-4} 
            \multicolumn{1}{c}{} &  & 35746 & 1588,8 &  &  &  & \tabularnewline
            \cline{3-4} 
            \multicolumn{1}{c}{} &  & 35743 & 1589,5 &  &  &  & \tabularnewline
            \cline{3-4} 
            \multicolumn{1}{c}{} &  & 35742 & 1590,2 &  &  &  & \tabularnewline
            \cline{3-4} 
            \multicolumn{1}{c}{} &  & 35742 & 1800,0 &  &  &  & \tabularnewline
            \cline{3-4} 
            \multicolumn{1}{c}{} & \multicolumn{1}{c}{} & \multicolumn{1}{c}{} & \multicolumn{1}{c}{} &  &  &  & \tabularnewline
        \end{tabular}
    }
    \captionof{table}{Tabella risultati instanze con numero di nodi compreso tra \textbf{$200$} e \textbf{$299$} $+$ algoritmi esatti}
}
\vspace*{\fill}

\vspace*{\fill}
{
    \centering
    \centerline{
        \begin{tabular}{|c|c|cccccc}
            \hline 
            \multicolumn{8}{|c|}{rat575 TABU SEARCH (costo ottimo 6773)}\tabularnewline
            \hline 
            \hline 
            \multicolumn{2}{|c|}{Thread1} & \multicolumn{2}{c|}{Thread2} & \multicolumn{2}{c|}{Thread3} & \multicolumn{2}{c|}{Thread4}\tabularnewline
            \hline 
            Costo & Tempo (s) & \multicolumn{1}{c|}{Costo} & Tempo (s) & Costo & Tempo (s) & \multicolumn{1}{c|}{Costo} & \multicolumn{1}{c|}{Tempo (s)}\tabularnewline
            \hline 
            7077 & 0,0 & \multicolumn{1}{c|}{7084} & \multicolumn{1}{c|}{0} & \multicolumn{1}{c|}{7079} & \multicolumn{1}{c|}{0} & \multicolumn{1}{c|}{7055} & \multicolumn{1}{c|}{0}\tabularnewline
            \hline 
            7075 & 0,8 & \multicolumn{1}{c|}{7084} & \multicolumn{1}{c|}{1800} & \multicolumn{1}{c|}{7078} & \multicolumn{1}{c|}{455,9} & \multicolumn{1}{c|}{7049} & \multicolumn{1}{c|}{0,8}\tabularnewline
            \hline 
            7062 & 2,1 &  & \multicolumn{1}{c|}{} & \multicolumn{1}{c|}{7077} & \multicolumn{1}{c|}{546,6} & \multicolumn{1}{c|}{7034} & \multicolumn{1}{c|}{1,2}\tabularnewline
            \cline{1-2} \cline{5-8} 
            7055 & 3,2 &  & \multicolumn{1}{c|}{} & \multicolumn{1}{c|}{7075} & \multicolumn{1}{c|}{547,7} & \multicolumn{1}{c|}{7033} & \multicolumn{1}{c|}{4,6}\tabularnewline
            \cline{1-2} \cline{5-8} 
            7046 & 5,2 &  & \multicolumn{1}{c|}{} & \multicolumn{1}{c|}{7074} & \multicolumn{1}{c|}{548,9} & \multicolumn{1}{c|}{7032} & \multicolumn{1}{c|}{5,2}\tabularnewline
            \cline{1-2} \cline{5-8} 
            7045 & 729,4 &  & \multicolumn{1}{c|}{} & \multicolumn{1}{c|}{7073} & \multicolumn{1}{c|}{647,4} & \multicolumn{1}{c|}{7021} & \multicolumn{1}{c|}{7,0}\tabularnewline
            \cline{1-2} \cline{5-8} 
            7043 & 730,5 &  & \multicolumn{1}{c|}{} & \multicolumn{1}{c|}{7072} & \multicolumn{1}{c|}{648,4} & \multicolumn{1}{c|}{7019} & \multicolumn{1}{c|}{9,0}\tabularnewline
            \cline{1-2} \cline{5-8} 
            7042 & 731,6 &  & \multicolumn{1}{c|}{} & \multicolumn{1}{c|}{7071} & \multicolumn{1}{c|}{1369,5} & \multicolumn{1}{c|}{7018} & \multicolumn{1}{c|}{561,0}\tabularnewline
            \cline{1-2} \cline{5-8} 
            7040 & 733,6 &  & \multicolumn{1}{c|}{} & \multicolumn{1}{c|}{7070} & \multicolumn{1}{c|}{1370,5} & \multicolumn{1}{c|}{7014} & \multicolumn{1}{c|}{562,3}\tabularnewline
            \cline{1-2} \cline{5-8} 
            7038 & 734,5 &  & \multicolumn{1}{c|}{} & \multicolumn{1}{c|}{7069} & \multicolumn{1}{c|}{1372,5} & \multicolumn{1}{c|}{7013} & \multicolumn{1}{c|}{563,5}\tabularnewline
            \cline{1-2} \cline{5-8} 
            7036 & 735,5 &  & \multicolumn{1}{c|}{} & \multicolumn{1}{c|}{7067} & \multicolumn{1}{c|}{1373,4} & \multicolumn{1}{c|}{7012} & \multicolumn{1}{c|}{565,8}\tabularnewline
            \cline{1-2} \cline{5-8} 
            7034 & 736,5 &  & \multicolumn{1}{c|}{} & \multicolumn{1}{c|}{7066} & \multicolumn{1}{c|}{1374,4} & \multicolumn{1}{c|}{7010} & \multicolumn{1}{c|}{566,9}\tabularnewline
            \cline{1-2} \cline{5-8} 
            7033 & 737,4 &  & \multicolumn{1}{c|}{} & \multicolumn{1}{c|}{7066} & \multicolumn{1}{c|}{1800} & \multicolumn{1}{c|}{7009} & \multicolumn{1}{c|}{568,0}\tabularnewline
            \cline{1-2} \cline{5-8} 
            7032 & 917,4 &  &  &  & \multicolumn{1}{c|}{} & \multicolumn{1}{c|}{7008} & \multicolumn{1}{c|}{569,1}\tabularnewline
            \cline{1-2} \cline{7-8} 
            7031 & 918,5 &  &  &  & \multicolumn{1}{c|}{} & \multicolumn{1}{c|}{7008} & \multicolumn{1}{c|}{1800}\tabularnewline
            \cline{1-2} \cline{7-8} 
            7030 & 920,4 &  &  &  &  &  & \tabularnewline
            \cline{1-2} 
            7028 & 921,3 &  &  &  &  &  & \tabularnewline
            \cline{1-2} 
            7027 & 922,3 &  &  &  &  &  & \tabularnewline
            \cline{1-2} 
            7026 & 923,4 &  &  &  &  &  & \tabularnewline
            \cline{1-2} 
            7025 & 1006,1 &  &  &  &  &  & \tabularnewline
            \cline{1-2} 
            7023 & 1007,2 &  &  &  &  &  & \tabularnewline
            \cline{1-2} 
            7022 & 1008,3 &  &  &  &  &  & \tabularnewline
            \cline{1-2} 
            7021 & 1187,9 &  &  &  &  &  & \tabularnewline
            \cline{1-2} 
            7020 & 1188,9 &  &  &  &  &  & \tabularnewline
            \cline{1-2} 
            7019 & 1189,9 &  &  &  &  &  & \tabularnewline
            \cline{1-2} 
            7018 & 1364,6 &  &  &  &  &  & \tabularnewline
            \cline{1-2} 
            7017 & 1365,5 &  &  &  &  &  & \tabularnewline
            \cline{1-2} 
            7016 & 1366,5 &  &  &  &  &  & \tabularnewline
            \cline{1-2} 
            7013 & 1537,6 &  &  &  &  &  & \tabularnewline
            \cline{1-2} 
            7012 & 1538,9 &  &  &  &  &  & \tabularnewline
            \cline{1-2} 
            7011 & 1541,5 &  &  &  &  &  & \tabularnewline
            \cline{1-2} 
            7010 & 1542,7 &  &  &  &  &  & \tabularnewline
            \cline{1-2} 
            7006 & 1545,0 &  &  &  &  &  & \tabularnewline
            \cline{1-2} 
            7004 & 1546,0 &  &  &  &  &  & \tabularnewline
            \cline{1-2} 
            7003 & 1547,1 &  &  &  &  &  & \tabularnewline
            \cline{1-2} 
            7002 & 1548,2 &  &  &  &  &  & \tabularnewline
            \cline{1-2} 
            7001 & 1550,3 &  &  &  &  &  & \tabularnewline
            \cline{1-2} 
            6999 & 1551,2 &  &  &  &  &  & \tabularnewline
            \cline{1-2} 
            6997 & 1552,2 &  &  &  &  &  & \tabularnewline
            \cline{1-2} 
            6996 & 1553,1 &  &  &  &  &  & \tabularnewline
            \cline{1-2} 
            6995 & 1554,0 &  &  &  &  &  & \tabularnewline
            \cline{1-2} 
            6992 & 1555,0 &  &  &  &  &  & \tabularnewline
            \cline{1-2} 
            6992 & 1800,0 &  &  &  &  &  & \tabularnewline
            \cline{1-2} 
        \end{tabular}
    }
    \captionof{table}{Tabella risultati instanze con numero di nodi compreso tra \textbf{$200$} e \textbf{$299$} $+$ algoritmi esatti}
}
\vspace*{\fill}

\vspace*{\fill}
{
    \centering
    \centerline{
        \begin{tabular}{|c|c|cc|c|c|cc}
            \hline 
            \multicolumn{8}{|c|}{d657 TABU SEARCH (costo ottimo 48912)}\tabularnewline
            \hline 
            \hline 
            \multicolumn{2}{|c|}{Thread1} & \multicolumn{2}{c|}{Thread2} & \multicolumn{2}{c|}{Thread3} & \multicolumn{2}{c|}{Thread4}\tabularnewline
            \hline 
            Costo & Tempo (s) & \multicolumn{1}{c|}{Costo} & Tempo (s) & Costo & Tempo (s) & \multicolumn{1}{c|}{Costo} & \multicolumn{1}{c|}{Tempo (s)}\tabularnewline
            \hline 
            51032 & 0,0 & \multicolumn{1}{c|}{51109} & 0,0 & 51230 & 0,0 & \multicolumn{1}{c|}{50974} & \multicolumn{1}{c|}{0}\tabularnewline
            \hline 
            51020 & 2,3 & \multicolumn{1}{c|}{51097} & 2,3 & 51221 & 2,2 & \multicolumn{1}{c|}{50974} & \multicolumn{1}{c|}{1800}\tabularnewline
            \hline 
            50953 & 12,5 & \multicolumn{1}{c|}{51095} & 7,2 & 51210 & 5,6 &  & \tabularnewline
            \cline{1-6} 
            50945 & 16,7 & \multicolumn{1}{c|}{51092} & 8,8 & 51159 & 7,1 &  & \tabularnewline
            \cline{1-6} 
            50927 & 34,0 & \multicolumn{1}{c|}{51065} & 10,5 & 51150 & 9,4 &  & \tabularnewline
            \cline{1-6} 
            50915 & 39,8 & \multicolumn{1}{c|}{51046} & 11,4 & 51142 & 19,1 &  & \tabularnewline
            \cline{1-6} 
            50904 & 41,3 & \multicolumn{1}{c|}{50999} & 15,9 & 51135 & 112,9 &  & \tabularnewline
            \cline{1-6} 
            50889 & 102,8 & \multicolumn{1}{c|}{50995} & 17,8 & 51124 & 114,5 &  & \tabularnewline
            \cline{1-6} 
            50882 & 107,0 & \multicolumn{1}{c|}{50990} & 22,9 & 51116 & 116,1 &  & \tabularnewline
            \cline{1-6} 
            50879 & 109,0 & \multicolumn{1}{c|}{50990} & 1800,0 & 51111 & 117,6 &  & \tabularnewline
            \cline{1-6} 
            50876 & 112,9 &  &  & 51105 & 120,7 &  & \tabularnewline
            \cline{1-2} \cline{5-6} 
            50872 & 114,7 &  &  & 51099 & 122,1 &  & \tabularnewline
            \cline{1-2} \cline{5-6} 
            50870 & 116,7 &  &  & 51095 & 123,6 &  & \tabularnewline
            \cline{1-2} \cline{5-6} 
            50869 & 118,6 &  &  & 51054 & 126,2 &  & \tabularnewline
            \cline{1-2} \cline{5-6} 
            50868 & 120,5 &  &  & 51048 & 127,6 &  & \tabularnewline
            \cline{1-2} \cline{5-6} 
            50864 & 124,2 &  &  & 51045 & 128,9 &  & \tabularnewline
            \cline{1-2} \cline{5-6} 
            50860 & 125,9 &  &  & 51044 & 576,2 &  & \tabularnewline
            \cline{1-2} \cline{5-6} 
            50859 & 127,7 &  &  & 51037 & 676,6 &  & \tabularnewline
            \cline{1-2} \cline{5-6} 
            50794 & 130,9 &  &  & 51035 & 678,1 &  & \tabularnewline
            \cline{1-2} \cline{5-6} 
            50786 & 132,5 &  &  & 51030 & 679,6 &  & \tabularnewline
            \cline{1-2} \cline{5-6} 
            50778 & 134,0 &  &  & 51029 & 681,1 &  & \tabularnewline
            \cline{1-2} \cline{5-6} 
            50772 & 135,7 &  &  & 51028 & 682,6 &  & \tabularnewline
            \cline{1-2} \cline{5-6} 
            50769 & 137,4 &  &  & 51024 & 685,4 &  & \tabularnewline
            \cline{1-2} \cline{5-6} 
            50761 & 141,6 &  &  & 51020 & 686,8 &  & \tabularnewline
            \cline{1-2} \cline{5-6} 
            50750 & 142,8 &  &  & 51017 & 688,1 &  & \tabularnewline
            \cline{1-2} \cline{5-6} 
            50741 & 144,1 &  &  & 51015 & 689,4 &  & \tabularnewline
            \cline{1-2} \cline{5-6} 
            50734 & 145,3 &  &  & 51014 & 690,8 &  & \tabularnewline
            \cline{1-2} \cline{5-6} 
            50728 & 251,8 &  &  & 51007 & 693,4 &  & \tabularnewline
            \cline{1-2} \cline{5-6} 
            50726 & 253,1 &  &  & 51001 & 694,5 &  & \tabularnewline
            \cline{1-2} \cline{5-6} 
            50725 & 254,3 &  &  & 50995 & 695,8 &  & \tabularnewline
            \cline{1-2} \cline{5-6} 
            50716 & 356,6 &  &  & 50990 & 697,1 &  & \tabularnewline
            \cline{1-2} \cline{5-6} 
            50702 & 358,2 &  &  & 50986 & 698,3 &  & \tabularnewline
            \cline{1-2} \cline{5-6} 
            50701 & 359,8 &  &  & 50986 & 1800,0 &  & \tabularnewline
            \cline{1-2} \cline{5-6} 
            50700 & 361,3 &  & \multicolumn{1}{c}{} & \multicolumn{1}{c}{} & \multicolumn{1}{c}{} &  & \tabularnewline
            \cline{1-2} 
            50696 & 364,2 &  & \multicolumn{1}{c}{} & \multicolumn{1}{c}{} & \multicolumn{1}{c}{} &  & \tabularnewline
            \cline{1-2} 
            50690 & 365,5 &  & \multicolumn{1}{c}{} & \multicolumn{1}{c}{} & \multicolumn{1}{c}{} &  & \tabularnewline
            \cline{1-2} 
            50686 & 366,9 &  & \multicolumn{1}{c}{} & \multicolumn{1}{c}{} & \multicolumn{1}{c}{} &  & \tabularnewline
            \cline{1-2} 
            50674 & 369,5 &  & \multicolumn{1}{c}{} & \multicolumn{1}{c}{} & \multicolumn{1}{c}{} &  & \tabularnewline
            \cline{1-2} 
            50666 & 370,8 &  & \multicolumn{1}{c}{} & \multicolumn{1}{c}{} & \multicolumn{1}{c}{} &  & \tabularnewline
            \cline{1-2} 
            50663 & 371,9 &  & \multicolumn{1}{c}{} & \multicolumn{1}{c}{} & \multicolumn{1}{c}{} &  & \tabularnewline
            \cline{1-2} 
            50659 & 814,1 &  & \multicolumn{1}{c}{} & \multicolumn{1}{c}{} & \multicolumn{1}{c}{} &  & \tabularnewline
            \cline{1-2} 
            50657 & 815,6 &  & \multicolumn{1}{c}{} & \multicolumn{1}{c}{} & \multicolumn{1}{c}{} &  & \tabularnewline
            \cline{1-2} 
            50651 & 818,2 &  & \multicolumn{1}{c}{} & \multicolumn{1}{c}{} & \multicolumn{1}{c}{} &  & \tabularnewline
            \cline{1-2} 
            50648 & 819,5 &  & \multicolumn{1}{c}{} & \multicolumn{1}{c}{} & \multicolumn{1}{c}{} &  & \tabularnewline
            \cline{1-2} 
            50645 & 1152,0 &  & \multicolumn{1}{c}{} & \multicolumn{1}{c}{} & \multicolumn{1}{c}{} &  & \tabularnewline
            \cline{1-2} 
            50641 & 1153,2 &  & \multicolumn{1}{c}{} & \multicolumn{1}{c}{} & \multicolumn{1}{c}{} &  & \tabularnewline
            \cline{1-2} 
            50639 & 1154,5 &  & \multicolumn{1}{c}{} & \multicolumn{1}{c}{} & \multicolumn{1}{c}{} &  & \tabularnewline
            \cline{1-2} 
            50634 & 1155,7 &  & \multicolumn{1}{c}{} & \multicolumn{1}{c}{} & \multicolumn{1}{c}{} &  & \tabularnewline
            \cline{1-2} 
            50626 & 1157,1 &  & \multicolumn{1}{c}{} & \multicolumn{1}{c}{} & \multicolumn{1}{c}{} &  & \tabularnewline
            \cline{1-2} 
            50619 & 1272,1 &  & \multicolumn{1}{c}{} & \multicolumn{1}{c}{} & \multicolumn{1}{c}{} &  & \tabularnewline
            \cline{1-2} 
            50616 & 1273,3 &  & \multicolumn{1}{c}{} & \multicolumn{1}{c}{} & \multicolumn{1}{c}{} &  & \tabularnewline
            \cline{1-2} 
            50616 & 1800,0 &  & \multicolumn{1}{c}{} & \multicolumn{1}{c}{} & \multicolumn{1}{c}{} &  & \tabularnewline
            \cline{1-2} 
        \end{tabular}
    }
    \captionof{table}{Tabella risultati instanze con numero di nodi compreso tra \textbf{$200$} e \textbf{$299$} $+$ algoritmi esatti}
}
\vspace*{\fill}

\vspace*{\fill}
{
    \centering
    \centerline{
        \begin{tabular}{|c|c|c|c|c|c|c|c|}
            \hline 
            \multicolumn{8}{|c|}{u724 TABU SEARCH (costo ottimo 41910)}\tabularnewline
            \hline 
            \hline 
            \multicolumn{2}{|c|}{Thread1} & \multicolumn{2}{c|}{Thread2} & \multicolumn{2}{c|}{Thread3} & \multicolumn{2}{c|}{Thread4}\tabularnewline
            \hline 
            Costo & Tempo (s) & \multicolumn{1}{c|}{Costo} & Tempo (s) & Costo & Tempo (s) & \multicolumn{1}{c|}{Costo} & \multicolumn{1}{c|}{Tempo (s)}\tabularnewline
            \hline 
            43864 & 0 & 43915 & 0 & 43852 & 0 & 43711 & 0\tabularnewline
            \hline 
            43845 & 1,1 & 43910 & 10,2 & 43833 & 1,2 & 43692 & 1,3\tabularnewline
            \hline 
            43842 & 4,2 & 43868 & 12,4 & 43824 & 37,6 & 43680 & 7,8\tabularnewline
            \hline 
            43779 & 13,9 & 43867 & 274,9 & 43819 & 1458,8 & 43677 & 163,1\tabularnewline
            \hline 
            43745 & 15,1 & 43861 & 277,0 & 43818 & 1461,0 & 43670 & 296,0\tabularnewline
            \hline 
            43732 & 22,5 & 43860 & 279,1 & 43811 & 1465,2 & 43661 & 297,8\tabularnewline
            \hline 
            43723 & 138,5 & 43836 & 283,1 & 43803 & 1467,5 & 43659 & 299,6\tabularnewline
            \hline 
            43715 & 140,2 & 43819 & 285,0 & 43796 & 1469,6 & 43657 & 303,0\tabularnewline
            \hline 
            43711 & 141,9 & 43815 & 286,9 & 43794 & 1471,8 & 43653 & 304,6\tabularnewline
            \hline 
            43708 & 145,3 & 43813 & 288,8 & 43793 & 1473,8 & 43650 & 306,2\tabularnewline
            \hline 
            43707 & 277,7 & 43812 & 290,6 & 43789 & 1475,9 & 43649 & 307,7\tabularnewline
            \hline 
            43703 & 279,3 & 43804 & 294,2 & 43783 & 1479,9 & 43648 & 309,2\tabularnewline
            \hline 
            43700 & 281,0 & 43797 & 295,9 & 43778 & 1481,8 & 43647 & 584,7\tabularnewline
            \hline 
            43698 & 282,7 & 43795 & 297,6 & 43774 & 1483,7 & 43643 & 586,2\tabularnewline
            \hline 
            43696 & 284,3 & 43793 & 299,2 & 43768 & 1485,6 & 43639 & 587,8\tabularnewline
            \hline 
            43694 & 287,5 & 43786 & 305,5 & 43765 & 1487,4 & 43635 & 589,3\tabularnewline
            \hline 
            43691 & 289,1 & 43786 & 1800 & 43763 & 1489,2 & 43632 & 591,0\tabularnewline
            \hline 
            43682 & 413,5 & \multicolumn{1}{c}{} &  & 43761 & 1491,2 & 43632 & 1800\tabularnewline
            \cline{1-2} \cline{5-8} 
            43668 & 415,0 & \multicolumn{1}{c}{} &  & 43760 & 1494,8 & \multicolumn{1}{c}{} & \multicolumn{1}{c}{}\tabularnewline
            \cline{1-2} \cline{5-6} 
            43664 & 416,7 & \multicolumn{1}{c}{} &  & 43759 & 1496,4 & \multicolumn{1}{c}{} & \multicolumn{1}{c}{}\tabularnewline
            \cline{1-2} \cline{5-6} 
            43661 & 418,4 & \multicolumn{1}{c}{} &  & 43752 & 1499,6 & \multicolumn{1}{c}{} & \multicolumn{1}{c}{}\tabularnewline
            \cline{1-2} \cline{5-6} 
            43661 & 1800 & \multicolumn{1}{c}{} &  & 43748 & 1501,1 & \multicolumn{1}{c}{} & \multicolumn{1}{c}{}\tabularnewline
            \cline{1-2} \cline{5-6} 
            \multicolumn{1}{c}{} & \multicolumn{1}{c}{} & \multicolumn{1}{c}{} &  & 43745 & 1502,7 & \multicolumn{1}{c}{} & \multicolumn{1}{c}{}\tabularnewline
            \cline{5-6} 
            \multicolumn{1}{c}{} & \multicolumn{1}{c}{} & \multicolumn{1}{c}{} &  & 43740 & 1504,1 & \multicolumn{1}{c}{} & \multicolumn{1}{c}{}\tabularnewline
            \cline{5-6} 
            \multicolumn{1}{c}{} & \multicolumn{1}{c}{} & \multicolumn{1}{c}{} &  & 43740 & 1800 & \multicolumn{1}{c}{} & \multicolumn{1}{c}{}\tabularnewline
            \cline{5-6} 
        \end{tabular}
    }
    \captionof{table}{Tabella risultati instanze con numero di nodi compreso tra \textbf{$200$} e \textbf{$299$} $+$ algoritmi esatti}
}
\vspace*{\fill}

\vspace*{\fill}
{
    \centering
    \centerline{
        \begin{tabular}{|c|c|c|c|c|c|c|c|}
            \hline 
            \multicolumn{8}{|c|}{rat783 TABU SEARCH (costo ottimo 8806)}\tabularnewline
            \hline 
            \hline 
            \multicolumn{2}{|c|}{Thread1} & \multicolumn{2}{c|}{Thread2} & \multicolumn{2}{c|}{Thread3} & \multicolumn{2}{c|}{Thread4}\tabularnewline
            \hline 
            Costo & Tempo (s) & \multicolumn{1}{c|}{Costo} & Tempo (s) & Costo & Tempo (s) & \multicolumn{1}{c|}{Costo} & \multicolumn{1}{c|}{Tempo (s)}\tabularnewline
            \hline 
            9284 & 0 & 9264 & 0 & 9265 & 0 & 9242 & 0\tabularnewline
            \hline 
            9283 & 59,138 & 9263 & 1,598 & 9257 & 1,352 & 9234 & 10,865\tabularnewline
            \hline 
            9280 & 66,021 & 9244 & 36,956 & 9256 & 4,652 & 9227 & 12,11\tabularnewline
            \hline 
            9278 & 73,23 & 9243 & 46,365 & 9255 & 312,909 & 9226 & 319,17\tabularnewline
            \hline 
            9277 & 80,592 & 9237 & 75,046 & 9254 & 315,584 & 9225 & 324,048\tabularnewline
            \hline 
            9276 & 187,071 & 9236 & 77,554 & 9249 & 318,202 & 9223 & 326,39\tabularnewline
            \hline 
            9274 & 188,977 & 9234 & 180,455 & 9248 & 323,082 & 9222 & 328,707\tabularnewline
            \hline 
            9273 & 190,821 & 9233 & 182,178 & 9246 & 325,655 & 9220 & 333,319\tabularnewline
            \hline 
            9267 & 336,05 & 9232 & 817,879 & 9245 & 328,17 & 9218 & 335,499\tabularnewline
            \hline 
            9266 & 337,778 & 9230 & 961,186 & 9244 & 330,618 & 9217 & 337,708\tabularnewline
            \hline 
            9265 & 339,488 & 9226 & 963,541 & 9242 & 335,162 & 9216 & 339,933\tabularnewline
            \hline 
            9264 & 341,182 & 9223 & 965,863 & 9241 & 337,401 & 9215 & 342,214\tabularnewline
            \hline 
            9263 & 342,961 & 9221 & 968,187 & 9240 & 339,715 & 9214 & 346,54\tabularnewline
            \hline 
            9262 & 807,17 & 9220 & 972,935 & 9239 & 341,989 & 9213 & 348,609\tabularnewline
            \hline 
            9261 & 809,36 & 9219 & 975,038 & 9238 & 348,087 & 9212 & 350,655\tabularnewline
            \hline 
            9260 & 811,598 & 9218 & 977,245 & 9237 & 350,126 & 9211 & 352,693\tabularnewline
            \hline 
            9259 & 813,923 & 9208 & 981,403 & 9236 & 352,109 & 9210 & 354,751\tabularnewline
            \hline 
            9258 & 818,163 & 9207 & 983,363 & 9235 & 355,929 & 9209 & 360,401\tabularnewline
            \hline 
            9257 & 820,26 & 9206 & 985,374 & 9234 & 357,908 & 9208 & 362,363\tabularnewline
            \hline 
            9256 & 822,233 & 9205 & 987,334 & 9233 & 359,712 & 9208 & 1800\tabularnewline
            \hline 
            9245 & 826,103 & 9204 & 991,107 & 9232 & 361,496 & \multicolumn{1}{c}{} & \multicolumn{1}{c}{}\tabularnewline
            \cline{1-6} 
            9244 & 827,873 & 9203 & 993,077 & 9231 & 363,321 & \multicolumn{1}{c}{} & \multicolumn{1}{c}{}\tabularnewline
            \cline{1-6} 
            9243 & 829,747 & 9202 & 994,903 & 9230 & 365,161 & \multicolumn{1}{c}{} & \multicolumn{1}{c}{}\tabularnewline
            \cline{1-6} 
            9242 & 831,591 & 9201 & 996,771 & 9229 & 530,5 & \multicolumn{1}{c}{} & \multicolumn{1}{c}{}\tabularnewline
            \cline{1-6} 
            9241 & 833,401 & 9199 & 1122,157 & 9228 & 532,25 & \multicolumn{1}{c}{} & \multicolumn{1}{c}{}\tabularnewline
            \cline{1-6} 
            9240 & 835,158 & 9196 & 1124,374 & 9227 & 677,756 & \multicolumn{1}{c}{} & \multicolumn{1}{c}{}\tabularnewline
            \cline{1-6} 
            9239 & 836,958 & 9195 & 1126,53 & 9226 & 679,984 & \multicolumn{1}{c}{} & \multicolumn{1}{c}{}\tabularnewline
            \cline{1-6} 
            9239 & 1800 & 9194 & 1128,633 & 9225 & 682,174 & \multicolumn{1}{c}{} & \multicolumn{1}{c}{}\tabularnewline
            \cline{1-6} 
            \multicolumn{1}{c}{} &  & 9192 & 1132,679 & 9223 & 686,373 & \multicolumn{1}{c}{} & \multicolumn{1}{c}{}\tabularnewline
            \cline{3-6} 
            \multicolumn{1}{c}{} &  & 9191 & 1134,628 & 9221 & 688,342 & \multicolumn{1}{c}{} & \multicolumn{1}{c}{}\tabularnewline
            \cline{3-6} 
            \multicolumn{1}{c}{} &  & 9190 & 1136,655 & 9220 & 690,422 & \multicolumn{1}{c}{} & \multicolumn{1}{c}{}\tabularnewline
            \cline{3-6} 
            \multicolumn{1}{c}{} &  & 9189 & 1142,332 & 9219 & 692,373 & \multicolumn{1}{c}{} & \multicolumn{1}{c}{}\tabularnewline
            \cline{3-6} 
            \multicolumn{1}{c}{} &  & 9188 & 1144,31 & 9218 & 694,368 & \multicolumn{1}{c}{} & \multicolumn{1}{c}{}\tabularnewline
            \cline{3-6} 
            \multicolumn{1}{c}{} &  & 9187 & 1146,111 & 9217 & 698,042 & \multicolumn{1}{c}{} & \multicolumn{1}{c}{}\tabularnewline
            \cline{3-6} 
            \multicolumn{1}{c}{} &  & 9186 & 1147,922 & 9216 & 699,793 & \multicolumn{1}{c}{} & \multicolumn{1}{c}{}\tabularnewline
            \cline{3-6} 
            \multicolumn{1}{c}{} &  & 9186 & 1800 & 9215 & 701,541 & \multicolumn{1}{c}{} & \multicolumn{1}{c}{}\tabularnewline
            \cline{3-6} 
            \multicolumn{1}{c}{} & \multicolumn{1}{c}{} & \multicolumn{1}{c}{} &  & 9214 & 703,321 & \multicolumn{1}{c}{} & \multicolumn{1}{c}{}\tabularnewline
            \cline{5-6} 
            \multicolumn{1}{c}{} & \multicolumn{1}{c}{} & \multicolumn{1}{c}{} &  & 9214 & 1800 & \multicolumn{1}{c}{} & \multicolumn{1}{c}{}\tabularnewline
            \cline{5-6} 
        \end{tabular}
    }
    \captionof{table}{Tabella risultati instanze con numero di nodi compreso tra \textbf{$200$} e \textbf{$299$} $+$ algoritmi esatti}
}
\vspace*{\fill}

\vspace*{\fill}
{
    \centering
    \centerline{
        \begin{tabular}{|c|c|cc|c|c|cc}
            \hline 
            \multicolumn{8}{|c|}{dsj1000 TABU SEARCH (costo ottimo 18659688)}\tabularnewline
            \hline 
            \hline 
            \multicolumn{2}{|c|}{Thread1} & \multicolumn{2}{c|}{Thread2} & \multicolumn{2}{c|}{Thread3} & \multicolumn{2}{c|}{Thread4}\tabularnewline
            \hline 
            Costo & Tempo (s) & \multicolumn{1}{c|}{Costo} & Tempo (s) & Costo & Tempo (s) & \multicolumn{1}{c|}{Costo} & \multicolumn{1}{c|}{Tempo (s)}\tabularnewline
            \hline 
            19828957 & 0,0 & \multicolumn{1}{c|}{19802820} & 0,0 & 19741229 & 0,0 & \multicolumn{1}{c|}{19750658} & \multicolumn{1}{c|}{0,0}\tabularnewline
            \hline 
            19815585 & 10,0 & \multicolumn{1}{c|}{19799454} & 3,8 & 19737657 & 3,7 & \multicolumn{1}{c|}{19747109} & \multicolumn{1}{c|}{5,2}\tabularnewline
            \hline 
            19815108 & 28,3 & \multicolumn{1}{c|}{19792630} & 11,3 & 19734229 & 44,4 & \multicolumn{1}{c|}{19744713} & \multicolumn{1}{c|}{14,8}\tabularnewline
            \hline 
            19814543 & 37,8 & \multicolumn{1}{c|}{19784187} & 29,1 & 19721723 & 47,1 & \multicolumn{1}{c|}{19741026} & \multicolumn{1}{c|}{31,7}\tabularnewline
            \hline 
            19806185 & 45,7 & \multicolumn{1}{c|}{19781812} & 33,8 & 19721431 & 117,4 & \multicolumn{1}{c|}{19731537} & \multicolumn{1}{c|}{65,7}\tabularnewline
            \hline 
            19802768 & 57,2 & \multicolumn{1}{c|}{19779585} & 52,1 & 19717868 & 121,6 & \multicolumn{1}{c|}{19726726} & \multicolumn{1}{c|}{69,1}\tabularnewline
            \hline 
            19793270 & 60,2 & \multicolumn{1}{c|}{19779059} & 517,1 & 19716480 & 543,5 & \multicolumn{1}{c|}{19723170} & \multicolumn{1}{c|}{72,5}\tabularnewline
            \hline 
            19763782 & 204,9 & \multicolumn{1}{c|}{19777739} & 521,6 & 19714929 & 547,6 & \multicolumn{1}{c|}{19703317} & \multicolumn{1}{c|}{82,7}\tabularnewline
            \hline 
            19735109 & 209,9 & \multicolumn{1}{c|}{19775888} & 530,1 & 19714904 & 551,7 & \multicolumn{1}{c|}{19703270} & \multicolumn{1}{c|}{86,1}\tabularnewline
            \hline 
            19731456 & 215,0 & \multicolumn{1}{c|}{19774017} & 534,2 & 19713214 & 559,4 & \multicolumn{1}{c|}{19700268} & \multicolumn{1}{c|}{100,9}\tabularnewline
            \hline 
            19731152 & 224,7 & \multicolumn{1}{c|}{19772302} & 538,3 & 19711789 & 563,2 & \multicolumn{1}{c|}{19700268} & \multicolumn{1}{c|}{1800,0}\tabularnewline
            \hline 
            19730940 & 229,5 & \multicolumn{1}{c|}{19771185} & 542,4 & 19711668 & 566,9 &  & \tabularnewline
            \cline{1-6} 
            19730819 & 234,2 & \multicolumn{1}{c|}{19769005} & 546,5 & 19711574 & 570,4 &  & \tabularnewline
            \cline{1-6} 
            19730725 & 238,9 & \multicolumn{1}{c|}{19768426} & 550,5 & 19710059 & 577,4 &  & \tabularnewline
            \cline{1-6} 
            19730702 & 243,7 & \multicolumn{1}{c|}{19766651} & 554,6 & 19709646 & 580,8 &  & \tabularnewline
            \cline{1-6} 
            19730312 & 252,8 & \multicolumn{1}{c|}{19766604} & 558,6 & 19709256 & 584,2 &  & \tabularnewline
            \cline{1-6} 
            19729922 & 257,1 & \multicolumn{1}{c|}{19764803} & 566,3 & 19708959 & 587,7 &  & \tabularnewline
            \cline{1-6} 
            19729558 & 265,7 & \multicolumn{1}{c|}{19763166} & 570,1 & 19708912 & 591,2 &  & \tabularnewline
            \cline{1-6} 
            19729161 & 269,7 & \multicolumn{1}{c|}{19762155} & 574,0 & 19708322 & 597,8 &  & \tabularnewline
            \cline{1-6} 
            19728767 & 273,9 & \multicolumn{1}{c|}{19761116} & 577,6 & 19707759 & 601,0 &  & \tabularnewline
            \cline{1-6} 
            19728650 & 277,9 & \multicolumn{1}{c|}{19760999} & 581,2 & 19707258 & 604,2 &  & \tabularnewline
            \cline{1-6} 
            19728023 & 285,7 & \multicolumn{1}{c|}{19760914} & 584,8 & 19706871 & 607,4 &  & \tabularnewline
            \cline{1-6} 
            19727374 & 289,5 & \multicolumn{1}{c|}{19759230} & 591,8 & 19650420 & 924,4 &  & \tabularnewline
            \cline{1-6} 
            19726873 & 293,2 & \multicolumn{1}{c|}{19758836} & 595,2 & 19649525 & 1107,9 &  & \tabularnewline
            \cline{1-6} 
            19726194 & 300,5 & \multicolumn{1}{c|}{19758499} & 598,6 & 19648888 & 1112,5 &  & \tabularnewline
            \cline{1-6} 
            19725501 & 303,8 & \multicolumn{1}{c|}{19758269} & 601,9 & 19647427 & 1121,0 &  & \tabularnewline
            \cline{1-6} 
            19725050 & 307,2 & \multicolumn{1}{c|}{19758172} & 605,3 & 19645696 & 1125,2 &  & \tabularnewline
            \cline{1-6} 
            19724943 & 310,6 & \multicolumn{1}{c|}{19757525} & 611,7 & 19644087 & 1129,4 &  & \tabularnewline
            \cline{1-6} 
            19724021 & 314,0 & \multicolumn{1}{c|}{19756962} & 614,6 & 19642735 & 1133,5 &  & \tabularnewline
            \cline{1-6} 
            19723280 & 320,3 & \multicolumn{1}{c|}{19756504} & 617,5 & 19642341 & 1137,6 &  & \tabularnewline
            \cline{1-6} 
            19722482 & 323,4 & \multicolumn{1}{c|}{19756107} & 620,5 & 19641226 & 1145,5 &  & \tabularnewline
            \cline{1-6} 
            19721767 & 326,4 & \multicolumn{1}{c|}{19755810} & 623,5 & 19640768 & 1149,4 &  & \tabularnewline
            \cline{1-6} 
            19720766 & 329,6 & \multicolumn{1}{c|}{19755465} & 1165,3 & 19640647 & 1153,3 &  & \tabularnewline
            \cline{1-6} 
            19720177 & 332,8 & \multicolumn{1}{c|}{19754597} & 1168,3 & 19640553 & 1157,3 &  & \tabularnewline
            \cline{1-6} 
            19717160 & 464,4 & \multicolumn{1}{c|}{19754261} & 1171,4 & 19638988 & 1164,6 &  & \tabularnewline
            \cline{1-6} 
            19716116 & 469,6 & \multicolumn{1}{c|}{19754261} & 1800,0 & 19638598 & 1168,0 &  & \tabularnewline
            \cline{1-6} 
            19711421 & 474,7 &  &  & 19638301 & 1171,4 &  & \tabularnewline
            \cline{1-2} \cline{5-6} 
            19711406 & 480,0 &  &  & 19638103 & 1174,9 &  & \tabularnewline
            \cline{1-2} \cline{5-6} 
            19710425 & 516,8 &  &  & 19637628 & 1181,5 &  & \tabularnewline
            \cline{1-2} \cline{5-6} 
            19709515 & 521,8 &  &  & 19637065 & 1184,5 &  & \tabularnewline
            \cline{1-2} \cline{5-6} 
            19708784 & 526,9 &  &  & 19636652 & 1187,6 &  & \tabularnewline
            \cline{1-2} \cline{5-6} 
            19708328 & 536,9 &  &  & 19636409 & 1190,6 &  & \tabularnewline
            \cline{1-2} \cline{5-6} 
            19707525 & 541,7 &  &  & 19636292 & 1193,6 &  & \tabularnewline
            \cline{1-2} \cline{5-6} 
            19706950 & 551,1 &  &  & 19636035 & 1712,2 &  & \tabularnewline
            \cline{1-2} \cline{5-6} 
            19705597 & 555,8 &  &  & 19635837 & 1715,8 &  & \tabularnewline
            \cline{1-2} \cline{5-6} 
            19704368 & 560,3 &  &  & 19635743 & 1719,4 &  & \tabularnewline
            \cline{1-2} \cline{5-6} 
            19703208 & 565,0 &  &  & 19634106 & 1726,5 &  & \tabularnewline
            \cline{1-2} \cline{5-6} 
            19702524 & 573,5 &  &  & 19633693 & 1730,0 &  & \tabularnewline
            \cline{1-2} \cline{5-6} 
            19701524 & 577,6 &  &  & 19633306 & 1733,5 &  & \tabularnewline
            \cline{1-2} \cline{5-6} 
            19701501 & 581,7 &  &  & 19633063 & 1736,8 &  & \tabularnewline
            \cline{1-2} \cline{5-6} 
            19700099 & 589,5 &  &  & 19632624 & 1743,1 &  & \tabularnewline
            \cline{1-2} \cline{5-6} 
            19698904 & 593,3 &  &  & 19632023 & 1746,1 &  & \tabularnewline
            \cline{1-2} \cline{5-6} 
            19698140 & 597,1 &  &  & 19631565 & 1749,1 &  & \tabularnewline
            \cline{1-2} \cline{5-6} 
            19697790 & 601,0 &  &  & 19631268 & 1752,2 &  & \tabularnewline
            \cline{1-2} \cline{5-6} 
            19697669 & 604,7 &  &  & 19631243 & 1755,2 &  & \tabularnewline
            \cline{1-2} \cline{5-6} 
            19696418 & 612,0 &  &  & 19631243 & 1800,0 &  & \tabularnewline
            \cline{1-2} \cline{5-6} 
            19695967 & 615,5 &  & \multicolumn{1}{c}{} & \multicolumn{1}{c}{} & \multicolumn{1}{c}{} &  & \tabularnewline
            \cline{1-2} 
            19695577 & 619,0 &  & \multicolumn{1}{c}{} & \multicolumn{1}{c}{} & \multicolumn{1}{c}{} &  & \tabularnewline
            \cline{1-2} 
            19695460 & 622,4 &  & \multicolumn{1}{c}{} & \multicolumn{1}{c}{} & \multicolumn{1}{c}{} &  & \tabularnewline
            \cline{1-2} 
            19692004 & 628,9 &  & \multicolumn{1}{c}{} & \multicolumn{1}{c}{} & \multicolumn{1}{c}{} &  & \tabularnewline
            \cline{1-2} 
            19691367 & 631,9 &  & \multicolumn{1}{c}{} & \multicolumn{1}{c}{} & \multicolumn{1}{c}{} &  & \tabularnewline
            \cline{1-2} 
            19690909 & 634,9 &  & \multicolumn{1}{c}{} & \multicolumn{1}{c}{} & \multicolumn{1}{c}{} &  & \tabularnewline
            \cline{1-2} 
            19690515 & 637,9 &  & \multicolumn{1}{c}{} & \multicolumn{1}{c}{} & \multicolumn{1}{c}{} &  & \tabularnewline
            \cline{1-2} 
            19689161 & 739,2 &  & \multicolumn{1}{c}{} & \multicolumn{1}{c}{} & \multicolumn{1}{c}{} &  & \tabularnewline
            \cline{1-2} 
            19675235 & 744,3 &  & \multicolumn{1}{c}{} & \multicolumn{1}{c}{} & \multicolumn{1}{c}{} &  & \tabularnewline
            \cline{1-2} 
            19674849 & 855,0 &  & \multicolumn{1}{c}{} & \multicolumn{1}{c}{} & \multicolumn{1}{c}{} &  & \tabularnewline
            \cline{1-2} 
            19674407 & 859,0 &  & \multicolumn{1}{c}{} & \multicolumn{1}{c}{} & \multicolumn{1}{c}{} &  & \tabularnewline
            \cline{1-2} 
            19672795 & 866,7 &  & \multicolumn{1}{c}{} & \multicolumn{1}{c}{} & \multicolumn{1}{c}{} &  & \tabularnewline
            \cline{1-2} 
            19671244 & 870,6 &  & \multicolumn{1}{c}{} & \multicolumn{1}{c}{} & \multicolumn{1}{c}{} &  & \tabularnewline
            \cline{1-2} 
            19669888 & 874,4 &  & \multicolumn{1}{c}{} & \multicolumn{1}{c}{} & \multicolumn{1}{c}{} &  & \tabularnewline
            \cline{1-2} 
            19669767 & 878,1 &  & \multicolumn{1}{c}{} & \multicolumn{1}{c}{} & \multicolumn{1}{c}{} &  & \tabularnewline
            \cline{1-2} 
            19669430 & 885,3 &  & \multicolumn{1}{c}{} & \multicolumn{1}{c}{} & \multicolumn{1}{c}{} &  & \tabularnewline
            \cline{1-2} 
            19669033 & 888,7 &  & \multicolumn{1}{c}{} & \multicolumn{1}{c}{} & \multicolumn{1}{c}{} &  & \tabularnewline
            \cline{1-2} 
            19668639 & 892,2 &  & \multicolumn{1}{c}{} & \multicolumn{1}{c}{} & \multicolumn{1}{c}{} &  & \tabularnewline
            \cline{1-2} 
            19668289 & 895,6 &  & \multicolumn{1}{c}{} & \multicolumn{1}{c}{} & \multicolumn{1}{c}{} &  & \tabularnewline
            \cline{1-2} 
            19668266 & 899,0 &  & \multicolumn{1}{c}{} & \multicolumn{1}{c}{} & \multicolumn{1}{c}{} &  & \tabularnewline
            \cline{1-2} 
            19667379 & 905,5 &  & \multicolumn{1}{c}{} & \multicolumn{1}{c}{} & \multicolumn{1}{c}{} &  & \tabularnewline
            \cline{1-2} 
            19666648 & 908,5 &  & \multicolumn{1}{c}{} & \multicolumn{1}{c}{} & \multicolumn{1}{c}{} &  & \tabularnewline
            \cline{1-2} 
            19666258 & 911,6 &  & \multicolumn{1}{c}{} & \multicolumn{1}{c}{} & \multicolumn{1}{c}{} &  & \tabularnewline
            \cline{1-2} 
            19665893 & 1151,8 &  & \multicolumn{1}{c}{} & \multicolumn{1}{c}{} & \multicolumn{1}{c}{} &  & \tabularnewline
            \cline{1-2} 
            19665256 & 1155,2 &  & \multicolumn{1}{c}{} & \multicolumn{1}{c}{} & \multicolumn{1}{c}{} &  & \tabularnewline
            \cline{1-2} 
            19664814 & 1158,7 &  & \multicolumn{1}{c}{} & \multicolumn{1}{c}{} & \multicolumn{1}{c}{} &  & \tabularnewline
            \cline{1-2} 
            19664420 & 1162,1 &  & \multicolumn{1}{c}{} & \multicolumn{1}{c}{} & \multicolumn{1}{c}{} &  & \tabularnewline
            \cline{1-2} 
            19664070 & 1165,6 &  & \multicolumn{1}{c}{} & \multicolumn{1}{c}{} & \multicolumn{1}{c}{} &  & \tabularnewline
            \cline{1-2} 
            19662153 & 1172,1 &  & \multicolumn{1}{c}{} & \multicolumn{1}{c}{} & \multicolumn{1}{c}{} &  & \tabularnewline
            \cline{1-2} 
            19661460 & 1175,1 &  & \multicolumn{1}{c}{} & \multicolumn{1}{c}{} & \multicolumn{1}{c}{} &  & \tabularnewline
            \cline{1-2} 
            19661002 & 1178,1 &  & \multicolumn{1}{c}{} & \multicolumn{1}{c}{} & \multicolumn{1}{c}{} &  & \tabularnewline
            \cline{1-2} 
            19660612 & 1181,1 &  & \multicolumn{1}{c}{} & \multicolumn{1}{c}{} & \multicolumn{1}{c}{} &  & \tabularnewline
            \cline{1-2} 
            19660400 & 1184,2 &  & \multicolumn{1}{c}{} & \multicolumn{1}{c}{} & \multicolumn{1}{c}{} &  & \tabularnewline
            \cline{1-2} 
            19660400 & 1800,0 &  & \multicolumn{1}{c}{} & \multicolumn{1}{c}{} & \multicolumn{1}{c}{} &  & \tabularnewline
            \cline{1-2} 
        \end{tabular}
    }
    \captionof{table}{Tabella risultati instanze con numero di nodi compreso tra \textbf{$200$} e \textbf{$299$} $+$ algoritmi esatti}
}
\vspace*{\fill}

\FloatBarrier

\subsubsection*{GENETICO}

\newcolumntype{C}{>{\centering\arraybackslash}p{11em}}

\FloatBarrier

\begin{table}
    \begin{adjustbox}{center}
        \begin{tabular}{|C|C|}
            \hline 
            \multicolumn{2}{|c|}{lin318 GENETICO GEN\#20 (costo ottimo 42029)}\tabularnewline
            \hline 
            \hline 
            Costo & Tempo(s)\tabularnewline
            \hline 
            67669 & 0,1\tabularnewline
            \hline 
            46016 & 0,9\tabularnewline
            \hline 
            45290 & 2,1\tabularnewline
            \hline 
            44701 & 8,7\tabularnewline
            \hline 
            44481 & 18,2\tabularnewline
            \hline 
            44027 & 27,5\tabularnewline
            \hline 
            43660 & 38,3\tabularnewline
            \hline 
            43570 & 65,2\tabularnewline
            \hline 
            43488 & 65,7\tabularnewline
            \hline 
            43457 & 455,4\tabularnewline
            \hline 
            43347 & 457,1\tabularnewline
            \hline 
            43157 & 458,2\tabularnewline
            \hline 
            43120 & 470,4\tabularnewline
            \hline 
            43053 & 476,9\tabularnewline
            \hline 
            43044 & 534,1\tabularnewline
            \hline 
            42960 & 1674,0\tabularnewline
            \hline 
            42960 & 1800\tabularnewline
            \hline 
        \end{tabular}
    \end{adjustbox}
\caption{Tabella risultati instanze con numero di nodi inferiore a \textbf{$200$} $+$ algoritmi esatti}
\end{table}

\begin{table}
    \begin{adjustbox}{center}
        \begin{tabular}{|C|C|}
            \hline 
            \multicolumn{2}{|c|}{pr439 GENETICO GEN\#20 (costo ottimo 107217)}\tabularnewline
            \hline 
            \hline 
            Costo & Tempo(s)\tabularnewline
            \hline 
            178844 & 0,2\tabularnewline
            \hline 
            172081 & 0,2\tabularnewline
            \hline 
            111632 & 5,5\tabularnewline
            \hline 
            111364 & 150,2\tabularnewline
            \hline 
            111301 & 203,1\tabularnewline
            \hline 
            111265 & 207,7\tabularnewline
            \hline 
            111134 & 235,9\tabularnewline
            \hline 
            110839 & 259,0\tabularnewline
            \hline 
            110667 & 309,8\tabularnewline
            \hline 
            110623 & 875,3\tabularnewline
            \hline 
            110508 & 1319,7\tabularnewline
            \hline 
            110508 & 1800\tabularnewline
        \end{tabular}
    \end{adjustbox}
    \caption{Tabella risultati instanze con numero di nodi inferiore a \textbf{$200$} $+$ algoritmi esatti}
\end{table}

\begin{table}
    \begin{adjustbox}{center}
        \begin{tabular}{|C|C|}
            \hline 
            \multicolumn{2}{|c|}{d493 GENETICO GEN\#20 (costo ottimo 35002)}\tabularnewline
            \hline 
            \hline 
            Costo & Tempo(s)\tabularnewline
            \hline 
            50521 & 0,3\tabularnewline
            \hline 
            37341 & 5,2\tabularnewline
            \hline 
            37107 & 8,0\tabularnewline
            \hline 
            36996 & 17,9\tabularnewline
            \hline 
            36936 & 32,9\tabularnewline
            \hline 
            36855 & 54,3\tabularnewline
            \hline 
            36810 & 115,7\tabularnewline
            \hline 
            36780 & 116,8\tabularnewline
            \hline 
            36771 & 117,6\tabularnewline
            \hline 
            36751 & 130,3\tabularnewline
            \hline 
            36683 & 527,4\tabularnewline
            \hline 
            36525 & 1176,5\tabularnewline
            \hline 
            36525 & 1800\tabularnewline
            \hline 
        \end{tabular}
    \end{adjustbox}
    \caption{Tabella risultati instanze con numero di nodi inferiore a \textbf{$200$} $+$ algoritmi esatti}
\end{table}

\begin{table}
    \begin{adjustbox}{center}
        \begin{tabular}{|C|C|}
            \hline 
            \multicolumn{2}{|c|}{rat575 GENETICO GEN\#20 (costo ottimo 6773)}\tabularnewline
            \hline 
            \hline 
            Costo & Tempo(s)\tabularnewline
            \hline 
            10225 & 0,3\tabularnewline
            \hline 
            9927 & 0,3\tabularnewline
            \hline 
            9858 & 0,3\tabularnewline
            \hline 
            9613 & 0,3\tabularnewline
            \hline 
            7411 & 8,1\tabularnewline
            \hline 
            7332 & 14,6\tabularnewline
            \hline 
            7317 & 112,0\tabularnewline
            \hline 
            7268 & 122,3\tabularnewline
            \hline 
            7263 & 277,5\tabularnewline
            \hline 
            7256 & 278,9\tabularnewline
            \hline 
            7238 & 283,3\tabularnewline
            \hline 
            7228 & 283,9\tabularnewline
            \hline 
            7206 & 338,7\tabularnewline
            \hline 
            7124 & 1175,1\tabularnewline
            \hline 
            7124 & 1800\tabularnewline
            \hline 
        \end{tabular}
    \end{adjustbox}
    \caption{Tabella risultati instanze con numero di nodi inferiore a \textbf{$200$} $+$ algoritmi esatti}
\end{table}

\begin{table}
    \begin{adjustbox}{center}
        \begin{tabular}{|C|C|}
            \hline 
            \multicolumn{2}{|c|}{d657 GENETICO GEN\#20 (costo ottimo 48912)}\tabularnewline
            \hline 
            \hline 
            Costo & Tempo(s)\tabularnewline
            \hline 
            67669 & 0,1\tabularnewline
            \hline 
            46016 & 0,9\tabularnewline
            \hline 
            45290 & 2,1\tabularnewline
            \hline 
            44701 & 8,7\tabularnewline
            \hline 
            44481 & 18,2\tabularnewline
            \hline 
            44027 & 27,5\tabularnewline
            \hline 
            43660 & 38,3\tabularnewline
            \hline 
            43570 & 65,2\tabularnewline
            \hline 
            43488 & 65,7\tabularnewline
            \hline 
            43457 & 455,4\tabularnewline
            \hline 
            43347 & 457,1\tabularnewline
            \hline 
            43157 & 458,2\tabularnewline
            \hline 
            43120 & 470,4\tabularnewline
            \hline 
            43053 & 476,9\tabularnewline
            \hline 
            43044 & 534,1\tabularnewline
            \hline 
            42960 & 1674,0\tabularnewline
            \hline 
            42960 & 1800\tabularnewline
            \hline 
        \end{tabular}
    \end{adjustbox}
    \caption{Tabella risultati instanze con numero di nodi inferiore a \textbf{$200$} $+$ algoritmi esatti}
\end{table}

\begin{table}
    \begin{adjustbox}{center}
        \begin{tabular}{|C|C|}
            \hline 
            \multicolumn{2}{|c|}{u724 GENETICO GEN\#20 (costo ottimo 41910)}\tabularnewline
            \hline 
            \hline 
            Costo & Tempo(s)\tabularnewline
            \hline 
            64140 & 0,5\tabularnewline
            \hline 
            45937 & 18,8\tabularnewline
            \hline 
            44988 & 58,6\tabularnewline
            \hline 
            44351 & 116,3\tabularnewline
            \hline 
            44279 & 1114,6\tabularnewline
            \hline 
            44279 & 1800,0\tabularnewline
            \hline 
        \end{tabular}
    \end{adjustbox}
    \caption{Tabella risultati instanze con numero di nodi inferiore a \textbf{$200$} $+$ algoritmi esatti}
\end{table}



\begin{table}
    \begin{adjustbox}{center}
        \begin{tabular}{|C|C|}
            \hline 
            \multicolumn{2}{|c|}{rat783 GENETICO GEN\#20 (costo ottimo 8806)}\tabularnewline
            \hline 
            \hline 
            Costo & Tempo(s)\tabularnewline
            \hline 
            13519 & 0,4\tabularnewline
            \hline 
            13180 & 0,5\tabularnewline
            \hline 
            9672 & 12,9\tabularnewline
            \hline 
            9562 & 30,1\tabularnewline
            \hline 
            9552 & 57,6\tabularnewline
            \hline 
            9536 & 76,5\tabularnewline
            \hline 
            9516 & 79,8\tabularnewline
            \hline 
            9485 & 89,2\tabularnewline
            \hline 
            9482 & 101,4\tabularnewline
            \hline 
            9426 & 125,2\tabularnewline
            \hline 
            9403 & 656,9\tabularnewline
            \hline 
            9386 & 800,9\tabularnewline
            \hline 
            9380 & 806,2\tabularnewline
            \hline 
            9380 & 1800\tabularnewline
            \hline 
        \end{tabular}
    \end{adjustbox}
    \caption{Tabella risultati instanze con numero di nodi inferiore a \textbf{$200$} $+$ algoritmi esatti}
\end{table}

\begin{table}
\begin{adjustbox}{center}
    \begin{tabular}{|C|C|}
        \hline 
        \multicolumn{2}{|c|}{dsj1000 GENETICO GEN\#20 (costo ottimo 18659688)}\tabularnewline
        \hline 
        \hline 
        Costo & Tempo(s)\tabularnewline
        \hline 
        28530708 & 0,7\tabularnewline
        \hline 
        28339022 & 0,9\tabularnewline
        \hline 
        27940942 & 1,5\tabularnewline
        \hline 
        27704229 & 1,7\tabularnewline
        \hline 
        20180860 & 55,9\tabularnewline
        \hline 
        19914677 & 113,8\tabularnewline
        \hline 
        19899518 & 215,1\tabularnewline
        \hline 
        19886374 & 950,6\tabularnewline
        \hline 
        19885127 & 1241,2\tabularnewline
        \hline 
        19885127 & 1800,0\tabularnewline
        \hline 
    \end{tabular}
\end{adjustbox}
\caption{Tabella risultati instanze con numero di nodi inferiore a \textbf{$200$} $+$ algoritmi esatti}
\end{table}

\FloatBarrier

\subsubsection*{VNS}

\FloatBarrier

\vspace*{\fill}
{
    \centering
    \centerline{\begin{tabular}{|c|c|c|c|c|c|c|c|}
            \hline 
            \multicolumn{8}{|c|}{lin318 VNS (costo ottimo 42029)}\tabularnewline
            \hline 
            \hline 
            \multicolumn{2}{|c|}{Thread1} & \multicolumn{2}{c|}{} & \multicolumn{2}{c|}{Thread2} & \multicolumn{2}{c|}{}\tabularnewline
            \hline 
            Costo & Tempo (s) & Costo & Tempo (s) & Costo & Tempo (s) & Costo & Tempo (s)\tabularnewline
            \hline 
            44328 & 0,0 & 45631 & 0,0 & 45189 & 0,0 & 44805 & 0,0\tabularnewline
            \hline 
            44293 & 0,1 & 44369 & 0,2 & 44674 & 0,2 & 44459 & 1,7\tabularnewline
            \hline 
            44242 & 0,1 & 44163 & 0,3 & 44375 & 0,5 & 44411 & 1,9\tabularnewline
            \hline 
            44128 & 3,0 & 43344 & 0,7 & 44365 & 3,1 & 44251 & 2,1\tabularnewline
            \hline 
            43691 & 3,3 & 43320 & 286,4 & 43945 & 7,2 & 44128 & 2,7\tabularnewline
            \hline 
            43675 & 3,4 & 43310 & 637,1 & 43901 & 7,3 & 44034 & 2,9\tabularnewline
            \hline 
            43639 & 4,8 & 43233 & 637,2 & 43807 & 7,7 & 43968 & 3,0\tabularnewline
            \hline 
            43591 & 6,1 & 43146 & 914,6 & 43685 & 27,2 & 43655 & 3,1\tabularnewline
            \hline 
            43583 & 25,0 & 43142 & 914,6 & 43574 & 27,3 & 43579 & 3,2\tabularnewline
            \hline 
            43369 & 25,0 & 43142 & 1800 & 43447 & 81,3 & 43578 & 38,1\tabularnewline
            \hline 
            43282 & 95,7 & \multicolumn{1}{c}{} &  & 43392 & 81,7 & 43524 & 61,8\tabularnewline
            \cline{1-2} \cline{5-8} 
            43280 & 937,2 & \multicolumn{1}{c}{} &  & 43374 & 124,6 & 43316 & 289,0\tabularnewline
            \cline{1-2} \cline{5-8} 
            43163 & 955,1 & \multicolumn{1}{c}{} &  & 43230 & 137,8 & 43211 & 790,5\tabularnewline
            \cline{1-2} \cline{5-8} 
            43052 & 1483,7 & \multicolumn{1}{c}{} &  & 43168 & 771,2 & 42984 & 1560,1\tabularnewline
            \cline{1-2} \cline{5-8} 
            43052 & 1800 & \multicolumn{1}{c}{} &  & 43129 & 1385,1 & 42984 & 1800\tabularnewline
            \cline{1-2} \cline{5-8} 
            \multicolumn{1}{c}{} & \multicolumn{1}{c}{} & \multicolumn{1}{c}{} &  & 43129 & 1800 & \multicolumn{1}{c}{} & \multicolumn{1}{c}{}\tabularnewline
            \cline{5-6} 
        \end{tabular}
    }
    \captionof{table}{Tabella risultati instanze con numero di nodi compreso tra \textbf{$200$} e \textbf{$299$} $+$ algoritmi esatti}
}
\vspace*{\fill}

\vspace*{\fill}
{
    \centering
    \centerline{\begin{tabular}{|c|c|c|c|c|c|c|c|}
            \hline 
            \multicolumn{8}{|c|}{pr439 VNS (costo ottimo 107217)}\tabularnewline
            \hline 
            \hline 
            \multicolumn{2}{|c|}{Thread1} & \multicolumn{2}{c|}{Thread2} & \multicolumn{2}{c|}{Thread3} & \multicolumn{2}{c|}{Thread4}\tabularnewline
            \hline 
            Costo & Tempo (s) & Costo & Tempo (s) & Costo & Tempo (s) & Costo & Tempo (s)\tabularnewline
            \hline 
            115162 & 0,0 & 115118 & 0,0 & 118329 & 0,0 & 116833 & 0,0\tabularnewline
            \hline 
            114445 & 0,1 & 113644 & 0,6 & 117757 & 0,3 & 116375 & 1,9\tabularnewline
            \hline 
            113959 & 0,9 & 113100 & 6,5 & 117749 & 0,5 & 114772 & 2,9\tabularnewline
            \hline 
            113010 & 1,3 & 112843 & 7,2 & 117469 & 2,8 & 114236 & 6,1\tabularnewline
            \hline 
            112977 & 3,1 & 112083 & 37,8 & 116925 & 4,2 & 114226 & 8,2\tabularnewline
            \hline 
            112431 & 8,9 & 111462 & 66,1 & 113697 & 5,2 & 112902 & 11,1\tabularnewline
            \hline 
            112400 & 9,4 & 111095 & 67,5 & 113573 & 5,3 & 112328 & 18,4\tabularnewline
            \hline 
            111864 & 19,0 & 111013 & 67,9 & 113007 & 15,0 & 111610 & 18,8\tabularnewline
            \hline 
            111378 & 19,2 & 110754 & 203,2 & 112911 & 15,5 & 111182 & 79,3\tabularnewline
            \hline 
            111219 & 19,3 & 110613 & 379,3 & 112512 & 17,0 & 110818 & 139,9\tabularnewline
            \hline 
            111153 & 26,1 & 110219 & 490,3 & 110844 & 18,4 & 110367 & 293,0\tabularnewline
            \hline 
            110824 & 26,5 & 109775 & 568,4 & 110598 & 46,9 & 110297 & 391,1\tabularnewline
            \hline 
            110700 & 26,7 & 109764 & 1219,9 & 110580 & 84,4 & 110229 & 976,1\tabularnewline
            \hline 
            109214 & 28,0 & 109167 & 1220,5 & 110249 & 192,0 & 110138 & 976,3\tabularnewline
            \hline 
            108691 & 1343,0 & 109167 & 1800 & 110132 & 368,4 & 109942 & 1468,1\tabularnewline
            \hline 
            108691 & 1800 & \multicolumn{1}{c}{} &  & 109930 & 994,4 & 109942 & 1800\tabularnewline
            \cline{1-2} \cline{5-8} 
            \multicolumn{1}{c}{} & \multicolumn{1}{c}{} & \multicolumn{1}{c}{} &  & 109563 & 1086,7 & \multicolumn{1}{c}{} & \multicolumn{1}{c}{}\tabularnewline
            \cline{5-6} 
            \multicolumn{1}{c}{} & \multicolumn{1}{c}{} & \multicolumn{1}{c}{} &  & 109287 & 1173,8 & \multicolumn{1}{c}{} & \multicolumn{1}{c}{}\tabularnewline
            \cline{5-6} 
            \multicolumn{1}{c}{} & \multicolumn{1}{c}{} & \multicolumn{1}{c}{} &  & 109287 & 1800 & \multicolumn{1}{c}{} & \multicolumn{1}{c}{}\tabularnewline
            \cline{5-6} 
        \end{tabular}
    }
    \captionof{table}{Tabella risultati instanze con numero di nodi compreso tra \textbf{$200$} e \textbf{$299$} $+$ algoritmi esatti}
}
\vspace*{\fill}

\vspace*{\fill}
{
    \centering
    \centerline{\begin{tabular}{|c|c|cc|c|c|c|c|}
            \hline 
            \multicolumn{8}{|c|}{d493 VNS (costo ottimo 35002)}\tabularnewline
            \hline 
            \hline 
            \multicolumn{2}{|c|}{Thread1} & \multicolumn{2}{c|}{Thread2} & \multicolumn{2}{c|}{Thread3} & \multicolumn{2}{c|}{Thread4}\tabularnewline
            \hline 
            Costo & Tempo (s) & \multicolumn{1}{c|}{Costo} & Tempo (s) & Costo & Tempo (s) & Costo & Tempo (s)\tabularnewline
            \hline 
            37367 & 0,0 & \multicolumn{1}{c|}{37017} & 0,0 & 37712 & 0,0 & 37356 & 0,0\tabularnewline
            \hline 
            37302 & 4,0 & \multicolumn{1}{c|}{36771} & 14,6 & 37561 & 0,8 & 37256 & 0,3\tabularnewline
            \hline 
            37210 & 8,3 & \multicolumn{1}{c|}{36736} & 66,1 & 37519 & 1,5 & 37062 & 1,0\tabularnewline
            \hline 
            37114 & 8,8 & \multicolumn{1}{c|}{36593} & 82,2 & 37328 & 2,3 & 37022 & 105,9\tabularnewline
            \hline 
            37065 & 10,1 & \multicolumn{1}{c|}{36495} & 522,8 & 37285 & 2,6 & 36960 & 106,5\tabularnewline
            \hline 
            36888 & 12,1 & \multicolumn{1}{c|}{36344} & 1170,5 & 37263 & 5,4 & 36902 & 107,2\tabularnewline
            \hline 
            36833 & 20,5 & \multicolumn{1}{c|}{36344} & 1800 & 37177 & 5,9 & 36784 & 107,5\tabularnewline
            \hline 
            36546 & 27,8 &  &  & 36894 & 10,7 & 36579 & 108,0\tabularnewline
            \cline{1-2} \cline{5-8} 
            36532 & 28,2 &  &  & 36813 & 14,2 & 36530 & 108,8\tabularnewline
            \cline{1-2} \cline{5-8} 
            36452 & 31,1 &  &  & 36738 & 94,7 & 36519 & 113,5\tabularnewline
            \cline{1-2} \cline{5-8} 
            36421 & 537,5 &  &  & 36550 & 97,0 & 36388 & 136,3\tabularnewline
            \cline{1-2} \cline{5-8} 
            36276 & 537,9 &  &  & 36517 & 124,4 & 36296 & 136,8\tabularnewline
            \cline{1-2} \cline{5-8} 
            36276 & 1800 &  &  & 36484 & 125,1 & 36296 & 1800\tabularnewline
            \cline{1-2} \cline{5-8} 
            \multicolumn{1}{c}{} & \multicolumn{1}{c}{} &  &  & 36431 & 144,0 & \multicolumn{1}{c}{} & \multicolumn{1}{c}{}\tabularnewline
            \cline{5-6} 
            \multicolumn{1}{c}{} & \multicolumn{1}{c}{} &  &  & 36398 & 144,6 & \multicolumn{1}{c}{} & \multicolumn{1}{c}{}\tabularnewline
            \cline{5-6} 
            \multicolumn{1}{c}{} & \multicolumn{1}{c}{} &  &  & 36260 & 273,5 & \multicolumn{1}{c}{} & \multicolumn{1}{c}{}\tabularnewline
            \cline{5-6} 
            \multicolumn{1}{c}{} & \multicolumn{1}{c}{} &  &  & 36260 & 1800 & \multicolumn{1}{c}{} & \multicolumn{1}{c}{}\tabularnewline
            \cline{5-6} 
            \multicolumn{1}{c}{} & \multicolumn{1}{c}{} &  & \multicolumn{1}{c}{} & \multicolumn{1}{c}{} & \multicolumn{1}{c}{} & \multicolumn{1}{c}{} & \multicolumn{1}{c}{}\tabularnewline
            \multicolumn{1}{c}{} & \multicolumn{1}{c}{} &  & \multicolumn{1}{c}{} & \multicolumn{1}{c}{} & \multicolumn{1}{c}{} & \multicolumn{1}{c}{} & \multicolumn{1}{c}{}\tabularnewline
        \end{tabular}
    }
    \captionof{table}{Tabella risultati instanze con numero di nodi compreso tra \textbf{$200$} e \textbf{$299$} $+$ algoritmi esatti}
}
\vspace*{\fill}

\vspace*{\fill}
{
    \centering
    \centerline{\begin{tabular}{|c|c|c|c|c|c|c|c|}
            \hline 
            \multicolumn{8}{|c|}{rat575 VNS (costo ottimo 6773)}\tabularnewline
            \hline 
            \hline 
            \multicolumn{2}{|c|}{Thread1} & \multicolumn{2}{c|}{Thread2} & \multicolumn{2}{c|}{Thread3} & \multicolumn{2}{c|}{Thread4}\tabularnewline
            \hline 
            Costo & Tempo (s) & Costo & Tempo (s) & Costo & Tempo (s) & Costo & Tempo (s)\tabularnewline
            \hline 
            7239 & 0,0 & 7342 & 0,0 & 7318 & 0,0 & 7369 & 0,0\tabularnewline
            \hline 
            7233 & 34,4 & 7312 & 0,8 & 7294 & 0,4 & 7366 & 0,4\tabularnewline
            \hline 
            7232 & 57,6 & 7287 & 15,7 & 7283 & 5,8 & 7316 & 2,0\tabularnewline
            \hline 
            7229 & 58,2 & 7276 & 16,5 & 7277 & 6,1 & 7270 & 3,0\tabularnewline
            \hline 
            7226 & 71,9 & 7250 & 18,6 & 7274 & 8,4 & 7269 & 4,0\tabularnewline
            \hline 
            7203 & 130,5 & 7248 & 19,1 & 7265 & 20,9 & 7240 & 6,5\tabularnewline
            \hline 
            7192 & 132,2 & 7188 & 127,2 & 7231 & 26,4 & 7229 & 114,3\tabularnewline
            \hline 
            7171 & 300,1 & 7186 & 313,7 & 7200 & 80,6 & 7207 & 115,6\tabularnewline
            \hline 
            7171 & 1800,0 & 7173 & 381,3 & 7189 & 511,7 & 7181 & 520,9\tabularnewline
            \hline 
            \multicolumn{1}{c}{} &  & 7173 & 1800,0 & 7180 & 833,6 & 7157 & 992,6\tabularnewline
            \cline{3-8} 
            \multicolumn{1}{c}{} & \multicolumn{1}{c}{} & \multicolumn{1}{c}{} &  & 7176 & 869,9 & 7157 & 1800,0\tabularnewline
            \cline{5-8} 
            \multicolumn{1}{c}{} & \multicolumn{1}{c}{} & \multicolumn{1}{c}{} &  & 7174 & 871,4 & \multicolumn{1}{c}{} & \multicolumn{1}{c}{}\tabularnewline
            \cline{5-6} 
            \multicolumn{1}{c}{} & \multicolumn{1}{c}{} & \multicolumn{1}{c}{} &  & 7161 & 872,1 & \multicolumn{1}{c}{} & \multicolumn{1}{c}{}\tabularnewline
            \cline{5-6} 
            \multicolumn{1}{c}{} & \multicolumn{1}{c}{} & \multicolumn{1}{c}{} &  & 7129 & 873,5 & \multicolumn{1}{c}{} & \multicolumn{1}{c}{}\tabularnewline
            \cline{5-6} 
            \multicolumn{1}{c}{} & \multicolumn{1}{c}{} & \multicolumn{1}{c}{} &  & 7129 & 1800,0 & \multicolumn{1}{c}{} & \multicolumn{1}{c}{}\tabularnewline
            \cline{5-6} 
            \multicolumn{1}{c}{} & \multicolumn{1}{c}{} & \multicolumn{1}{c}{} & \multicolumn{1}{c}{} & \multicolumn{1}{c}{} & \multicolumn{1}{c}{} & \multicolumn{1}{c}{} & \multicolumn{1}{c}{}\tabularnewline
            \multicolumn{1}{c}{} & \multicolumn{1}{c}{} & \multicolumn{1}{c}{} & \multicolumn{1}{c}{} & \multicolumn{1}{c}{} & \multicolumn{1}{c}{} & \multicolumn{1}{c}{} & \multicolumn{1}{c}{}\tabularnewline
            \multicolumn{1}{c}{} & \multicolumn{1}{c}{} & \multicolumn{1}{c}{} & \multicolumn{1}{c}{} & \multicolumn{1}{c}{} & \multicolumn{1}{c}{} & \multicolumn{1}{c}{} & \multicolumn{1}{c}{}\tabularnewline
            \multicolumn{1}{c}{} & \multicolumn{1}{c}{} & \multicolumn{1}{c}{} & \multicolumn{1}{c}{} & \multicolumn{1}{c}{} & \multicolumn{1}{c}{} & \multicolumn{1}{c}{} & \multicolumn{1}{c}{}\tabularnewline
        \end{tabular}
    }
    \captionof{table}{Tabella risultati instanze con numero di nodi compreso tra \textbf{$200$} e \textbf{$299$} $+$ algoritmi esatti}
}
\vspace*{\fill}

\vspace*{\fill}
{
    \centering
    \centerline{\begin{tabular}{|c|c|c|c|c|c|c|c|}
            \hline 
            \multicolumn{8}{|c|}{d657 VNS (costo ottimo 48912)}\tabularnewline
            \hline 
            \hline 
            \multicolumn{2}{|c|}{Thread1} & \multicolumn{2}{c|}{Thread2} & \multicolumn{2}{c|}{Thread3} & \multicolumn{2}{c|}{Thread4}\tabularnewline
            \hline 
            Costo & Tempo (s) & Costo & Tempo (s) & Costo & Tempo (s) & Costo & Tempo (s)\tabularnewline
            \hline 
            52642 & 0,0 & 52724 & 0,0 & 52852 & 0,0 & 53319 & 0,0\tabularnewline
            \hline 
            52583 & 3,1 & 52629 & 1,6 & 52484 & 1,1 & 53304 & 0,8\tabularnewline
            \hline 
            52443 & 17,5 & 52283 & 29,2 & 52157 & 3,1 & 53274 & 1,7\tabularnewline
            \hline 
            52271 & 19,7 & 52199 & 88,0 & 52053 & 100,1 & 53141 & 2,4\tabularnewline
            \hline 
            52145 & 42,1 & 52166 & 89,0 & 52035 & 100,6 & 52650 & 3,0\tabularnewline
            \hline 
            52072 & 46,4 & 52154 & 120,9 & 51824 & 145,5 & 52037 & 3,4\tabularnewline
            \hline 
            52058 & 51,0 & 51923 & 126,3 & 51789 & 147,6 & 51922 & 136,8\tabularnewline
            \hline 
            51944 & 63,9 & 51874 & 152,2 & 51785 & 148,4 & 51667 & 147,5\tabularnewline
            \hline 
            51767 & 70,8 & 51409 & 153,1 & 51733 & 420,5 & 51352 & 149,8\tabularnewline
            \hline 
            51669 & 214,3 & 51109 & 177,2 & 51703 & 421,5 & 51352 & 1800,0\tabularnewline
            \hline 
            51648 & 319,5 & 50997 & 179,8 & 51620 & 422,3 & \multicolumn{1}{c}{} & \multicolumn{1}{c}{}\tabularnewline
            \cline{1-6} 
            51580 & 325,7 & 50872 & 180,4 & 51602 & 425,0 & \multicolumn{1}{c}{} & \multicolumn{1}{c}{}\tabularnewline
            \cline{1-6} 
            51462 & 326,8 & 50768 & 181,6 & 51550 & 429,8 & \multicolumn{1}{c}{} & \multicolumn{1}{c}{}\tabularnewline
            \cline{1-6} 
            51051 & 388,2 & 50768 & 1800,0 & 51405 & 431,6 & \multicolumn{1}{c}{} & \multicolumn{1}{c}{}\tabularnewline
            \cline{1-6} 
            51051 & 1800,0 & \multicolumn{1}{c}{} &  & 51235 & 433,1 & \multicolumn{1}{c}{} & \multicolumn{1}{c}{}\tabularnewline
            \cline{1-2} \cline{5-6} 
            \multicolumn{1}{c}{} & \multicolumn{1}{c}{} & \multicolumn{1}{c}{} &  & 51224 & 885,6 & \multicolumn{1}{c}{} & \multicolumn{1}{c}{}\tabularnewline
            \cline{5-6} 
            \multicolumn{1}{c}{} & \multicolumn{1}{c}{} & \multicolumn{1}{c}{} &  & 51194 & 889,4 & \multicolumn{1}{c}{} & \multicolumn{1}{c}{}\tabularnewline
            \cline{5-6} 
            \multicolumn{1}{c}{} & \multicolumn{1}{c}{} & \multicolumn{1}{c}{} &  & 50809 & 1194,6 & \multicolumn{1}{c}{} & \multicolumn{1}{c}{}\tabularnewline
            \cline{5-6} 
            \multicolumn{1}{c}{} & \multicolumn{1}{c}{} & \multicolumn{1}{c}{} &  & 50809 & 1800,0 & \multicolumn{1}{c}{} & \multicolumn{1}{c}{}\tabularnewline
            \cline{5-6} 
        \end{tabular}
    }
    \captionof{table}{Tabella risultati instanze con numero di nodi compreso tra \textbf{$200$} e \textbf{$299$} $+$ algoritmi esatti}
}
\vspace*{\fill}

\vspace*{\fill}
{
    \centering
    \centerline{\begin{tabular}{|c|c|c|c|cc|c|c|}
            \hline 
            \multicolumn{8}{|c|}{u724 VNS (costo ottimo 41910)}\tabularnewline
            \hline 
            \hline 
            \multicolumn{2}{|c|}{Thread1} & \multicolumn{2}{c|}{Thread2} & \multicolumn{2}{c|}{Thread3} & \multicolumn{2}{c|}{Thread4}\tabularnewline
            \hline 
            Costo & Tempo (s) & Costo & Tempo (s) & \multicolumn{1}{c|}{Costo} & Tempo (s) & Costo & Tempo (s)\tabularnewline
            \hline 
            45228 & 0,001 & 44729 & 0,001 & \multicolumn{1}{c|}{44683} & 0,001 & 45146 & 0,001\tabularnewline
            \hline 
            45054 & 41,522 & 44669 & 3,564 & \multicolumn{1}{c|}{44505} & 54,886 & 45022 & 0,69\tabularnewline
            \hline 
            44987 & 47,077 & 44636 & 28,622 & \multicolumn{1}{c|}{44441} & 56,136 & 45001 & 25,236\tabularnewline
            \hline 
            44957 & 183,277 & 44496 & 112,255 & \multicolumn{1}{c|}{44390} & 834,913 & 44965 & 26,903\tabularnewline
            \hline 
            44788 & 211,512 & 44459 & 1083,77 & \multicolumn{1}{c|}{44316} & 1118,98 & 44929 & 27,838\tabularnewline
            \hline 
            44786 & 415,122 & 44331 & 1284,459 & \multicolumn{1}{c|}{44316} & 1800,0 & 44846 & 79,558\tabularnewline
            \hline 
            44500 & 438,924 & 44196 & 1708,761 &  &  & 44763 & 117,202\tabularnewline
            \cline{1-4} \cline{7-8} 
            44381 & 479,237 & 44196 & 1800,0 &  &  & 44681 & 124,146\tabularnewline
            \cline{1-4} \cline{7-8} 
            44246 & 483,172 & \multicolumn{1}{c}{} & \multicolumn{1}{c}{} &  &  & 44567 & 345,671\tabularnewline
            \cline{1-2} \cline{7-8} 
            44201 & 744,046 & \multicolumn{1}{c}{} & \multicolumn{1}{c}{} &  &  & 44556 & 458,118\tabularnewline
            \cline{1-2} \cline{7-8} 
            44201 & 1800,0 & \multicolumn{1}{c}{} & \multicolumn{1}{c}{} &  &  & 44416 & 459,21\tabularnewline
            \cline{1-2} \cline{7-8} 
            \multicolumn{1}{c}{} & \multicolumn{1}{c}{} & \multicolumn{1}{c}{} & \multicolumn{1}{c}{} &  &  & 44270 & 1151,915\tabularnewline
            \cline{7-8} 
            \multicolumn{1}{c}{} & \multicolumn{1}{c}{} & \multicolumn{1}{c}{} & \multicolumn{1}{c}{} &  &  & 44202 & 1152,22\tabularnewline
            \cline{7-8} 
            \multicolumn{1}{c}{} & \multicolumn{1}{c}{} & \multicolumn{1}{c}{} & \multicolumn{1}{c}{} &  &  & 44202 & 1800,0\tabularnewline
            \cline{7-8} 
        \end{tabular}
    }
    \captionof{table}{Tabella risultati instanze con numero di nodi compreso tra \textbf{$200$} e \textbf{$299$} $+$ algoritmi esatti}
}
\vspace*{\fill}

\vspace*{\fill}
{
    \centering
    \centerline{\begin{tabular}{cccc|c|c|c|c|}
            \hline 
            \multicolumn{8}{|c|}{rat783 VNS (costo ottimo 8806)}\tabularnewline
            \hline 
            \hline 
            \multicolumn{2}{|c|}{Thread1} & \multicolumn{2}{c|}{Thread2} & \multicolumn{2}{c|}{Thread3} & \multicolumn{2}{c|}{Thread4}\tabularnewline
            \hline 
            \multicolumn{1}{|c|}{Costo} & \multicolumn{1}{c|}{Tempo (s)} & \multicolumn{1}{c|}{Costo} & Tempo (s) & Costo & Tempo (s) & Costo & Tempo (s)\tabularnewline
            \hline 
            \multicolumn{1}{|c|}{9399} & \multicolumn{1}{c|}{0,0} & \multicolumn{1}{c|}{9432} & 0,0 & 9606 & 0,0 & 9459 & 0,0\tabularnewline
            \hline 
            \multicolumn{1}{|c|}{9399} & \multicolumn{1}{c|}{1800} & \multicolumn{1}{c|}{9418} & 24,7 & 9596 & 0,7 & 9453 & 53,0\tabularnewline
            \hline 
            & \multicolumn{1}{c|}{} & \multicolumn{1}{c|}{9393} & 26,2 & 9578 & 2,5 & 9452 & 639,3\tabularnewline
            \cline{3-8} 
            & \multicolumn{1}{c|}{} & \multicolumn{1}{c|}{9371} & 117,1 & 9563 & 43,9 & 9448 & 911,3\tabularnewline
            \cline{3-8} 
            & \multicolumn{1}{c|}{} & \multicolumn{1}{c|}{9366} & 118,0 & 9562 & 44,7 & 9439 & 912,4\tabularnewline
            \cline{3-8} 
            & \multicolumn{1}{c|}{} & \multicolumn{1}{c|}{9356} & 120,2 & 9528 & 47,6 & 9426 & 1000,8\tabularnewline
            \cline{3-8} 
            & \multicolumn{1}{c|}{} & \multicolumn{1}{c|}{9356} & 1800 & 9523 & 63,5 & 9402 & 1660,8\tabularnewline
            \cline{3-8} 
            &  &  &  & 9513 & 78,4 & 9394 & 1677,7\tabularnewline
            \cline{5-8} 
            &  &  &  & 9504 & 84,9 & 9384 & 1683,0\tabularnewline
            \cline{5-8} 
            &  &  &  & 9495 & 95,3 & 9384 & 1800\tabularnewline
            \cline{5-8} 
            &  &  &  & 9488 & 139,0 & \multicolumn{1}{c}{} & \multicolumn{1}{c}{}\tabularnewline
            \cline{5-6} 
            &  &  &  & 9477 & 140,1 & \multicolumn{1}{c}{} & \multicolumn{1}{c}{}\tabularnewline
            \cline{5-6} 
            &  &  &  & 9466 & 163,8 & \multicolumn{1}{c}{} & \multicolumn{1}{c}{}\tabularnewline
            \cline{5-6} 
            &  &  &  & 9445 & 164,0 & \multicolumn{1}{c}{} & \multicolumn{1}{c}{}\tabularnewline
            \cline{5-6} 
            &  &  &  & 9443 & 168,9 & \multicolumn{1}{c}{} & \multicolumn{1}{c}{}\tabularnewline
            \cline{5-6} 
            &  &  &  & 9421 & 169,6 & \multicolumn{1}{c}{} & \multicolumn{1}{c}{}\tabularnewline
            \cline{5-6} 
            &  &  &  & 9420 & 390,7 & \multicolumn{1}{c}{} & \multicolumn{1}{c}{}\tabularnewline
            \cline{5-6} 
            &  &  &  & 9419 & 391,7 & \multicolumn{1}{c}{} & \multicolumn{1}{c}{}\tabularnewline
            \cline{5-6} 
            &  &  &  & 9415 & 392,4 & \multicolumn{1}{c}{} & \multicolumn{1}{c}{}\tabularnewline
            \cline{5-6} 
            &  &  &  & 9406 & 976,6 & \multicolumn{1}{c}{} & \multicolumn{1}{c}{}\tabularnewline
            \cline{5-6} 
            &  &  &  & 9406 & 1800 & \multicolumn{1}{c}{} & \multicolumn{1}{c}{}\tabularnewline
            \cline{5-6} 
        \end{tabular}
    }
    \captionof{table}{Tabella risultati instanze con numero di nodi compreso tra \textbf{$200$} e \textbf{$299$} $+$ algoritmi esatti}
}
\vspace*{\fill}

\vspace*{\fill}
{
    \centering
    \centerline{\begin{tabular}{|c|c|cc|c|c|c|c|}
            \hline 
            \multicolumn{8}{|c|}{dsj1000 VNS (costo ottimo 18659688)}\tabularnewline
            \hline 
            \hline 
            \multicolumn{2}{|c|}{Thread1} & \multicolumn{2}{c|}{Thread2} & \multicolumn{2}{c|}{Thread3} & \multicolumn{2}{c|}{Thread4}\tabularnewline
            \hline 
            Costo & Tempo (s) & \multicolumn{1}{c|}{Costo} & Tempo (s) & Costo & Tempo (s) & Costo & Tempo (s)\tabularnewline
            \hline 
            20388124 & 0,0 & \multicolumn{1}{c|}{19933316} & 0,0 & 20179476 & 0,0 & 20192173 & 0,0\tabularnewline
            \hline 
            20350385 & 0,9 & \multicolumn{1}{c|}{19901388} & 3,9 & 20041915 & 8,3 & 20190252 & 17,1\tabularnewline
            \hline 
            20315711 & 29,2 & \multicolumn{1}{c|}{19819155} & 23,8 & 20037603 & 822,0 & 20161855 & 136,0\tabularnewline
            \hline 
            20273258 & 82,7 & \multicolumn{1}{c|}{19818600} & 849,1 & 20010375 & 889,4 & 20068673 & 149,9\tabularnewline
            \hline 
            20207005 & 90,0 & \multicolumn{1}{c|}{19811979} & 850,2 & 19968630 & 890,5 & 20063392 & 158,9\tabularnewline
            \hline 
            20202303 & 92,6 & \multicolumn{1}{c|}{19811979} & 1800,0 & 19930678 & 1750,2 & 20054192 & 159,4\tabularnewline
            \hline 
            20196943 & 95,1 &  &  & 19918893 & 1755,2 & 19926458 & 160,8\tabularnewline
            \cline{1-2} \cline{5-8} 
            20059067 & 113,1 &  &  & 19918893 & 1800,0 & 19856223 & 638,3\tabularnewline
            \cline{1-2} \cline{5-8} 
            20054941 & 119,3 &  & \multicolumn{1}{c}{} & \multicolumn{1}{c}{} &  & 19834387 & 1649,3\tabularnewline
            \cline{1-2} \cline{7-8} 
            20023983 & 150,4 &  & \multicolumn{1}{c}{} & \multicolumn{1}{c}{} &  & 19834387 & 1800,0\tabularnewline
            \cline{1-2} \cline{7-8} 
            19976516 & 314,6 &  & \multicolumn{1}{c}{} & \multicolumn{1}{c}{} & \multicolumn{1}{c}{} & \multicolumn{1}{c}{} & \multicolumn{1}{c}{}\tabularnewline
            \cline{1-2} 
            19892262 & 324,5 &  & \multicolumn{1}{c}{} & \multicolumn{1}{c}{} & \multicolumn{1}{c}{} & \multicolumn{1}{c}{} & \multicolumn{1}{c}{}\tabularnewline
            \cline{1-2} 
            19861128 & 1082,7 &  & \multicolumn{1}{c}{} & \multicolumn{1}{c}{} & \multicolumn{1}{c}{} & \multicolumn{1}{c}{} & \multicolumn{1}{c}{}\tabularnewline
            \cline{1-2} 
            19861128 & 1800,0 &  & \multicolumn{1}{c}{} & \multicolumn{1}{c}{} & \multicolumn{1}{c}{} & \multicolumn{1}{c}{} & \multicolumn{1}{c}{}\tabularnewline
            \cline{1-2} 
        \end{tabular}
    }
    \captionof{table}{Tabella risultati instanze con numero di nodi compreso tra \textbf{$200$} e \textbf{$299$} $+$ algoritmi esatti}
}
\vspace*{\fill}

\FloatBarrier

\subsubsection*{HARD FIXING \& LOCAL BRANCH}

\FloatBarrier

\vspace*{\fill}
{
    \centering
    \centerline{\begin{tabular}{|c|c|c|c|c|c|c|c|}
            \hline 
            \multicolumn{8}{|c|}{lin318 - HF \& LB - PT1 - (costo ottimo 42029)}\tabularnewline
            \hline 
            \hline 
            \multicolumn{2}{|c|}{Thread1 HF} & \multicolumn{2}{c|}{Thread1 LB} & \multicolumn{2}{c|}{Thread2 HF} & \multicolumn{2}{c|}{Thread2 LB}\tabularnewline
            \hline 
            Costo & Tempo (s) & Costo & Tempo (s) & Costo & Tempo (s) & Costo & Tempo (s)\tabularnewline
            \hline 
            43231 & 0,0 & 43231 & 0,0 & 42935 & 0,0 & 42935 & 0,0\tabularnewline
            \hline 
            43198 & 1,7 & 43132 & 8,1 & 42916 & 1,0 & 42813 & 7,2\tabularnewline
            \hline 
            43009 & 2,1 & 43040 & 15,7 & 42820 & 2,8 & 42721 & 16,2\tabularnewline
            \hline 
            42976 & 2,6 & 42957 & 24,3 & 42803 & 6,0 & 42678 & 25,8\tabularnewline
            \hline 
            42951 & 3,6 & 42875 & 30,6 & 42749 & 6,5 & 42638 & 33,7\tabularnewline
            \hline 
            42939 & 4,8 & 42832 & 39,7 & 42746 & 8,7 & 42601 & 46,8\tabularnewline
            \hline 
            42756 & 11,3 & 42792 & 48,1 & 42724 & 13,3 & 42568 & 55,8\tabularnewline
            \hline 
            42743 & 19,2 & 42743 & 59,2 & 42591 & 14,9 & 42540 & 64,4\tabularnewline
            \hline 
            42724 & 24,2 & 42710 & 70,5 & 42575 & 22,1 & 42524 & 76,4\tabularnewline
            \hline 
            42672 & 25,7 & 42677 & 81,9 & 42483 & 23,0 & 42512 & 89,5\tabularnewline
            \hline 
            42658 & 39,6 & 42644 & 94,1 & 42436 & 27,8 & 42506 & 102,5\tabularnewline
            \hline 
            42592 & 51,1 & 42628 & 107,4 & 42397 & 39,3 & 42425 & 147,1\tabularnewline
            \hline 
            42529 & 163,7 & 42530 & 144,7 & 42357 & 62,7 & 42371 & 189,0\tabularnewline
            \hline 
            42525 & 171,9 & 42455 & 179,9 & 42351 & 72,7 & 42317 & 225,9\tabularnewline
            \hline 
            42480 & 180,2 & 42401 & 234,4 & 42329 & 138,9 & 42293 & 282,4\tabularnewline
            \hline 
            42457 & 197,5 & 42347 & 269,1 & 42294 & 143,8 & 42270 & 327,9\tabularnewline
            \hline 
            42395 & 203,5 & 42314 & 322,3 & 42248 & 187,2 & 42248 & 705,7\tabularnewline
            \hline 
            42381 & 222,3 & 42302 & 374,4 & 42237 & 1415,6 & 42248 & 1800\tabularnewline
            \hline 
            42380 & 231,7 & 42273 & 415,9 & 42237 & 1800 & \multicolumn{1}{c}{} & \multicolumn{1}{c}{}\tabularnewline
            \cline{1-6} 
            42377 & 239,7 & 42265 & 470,7 & \multicolumn{1}{c}{} & \multicolumn{1}{c}{} & \multicolumn{1}{c}{} & \multicolumn{1}{c}{}\tabularnewline
            \cline{1-4} 
            42277 & 249,8 & 42178 & 512,7 & \multicolumn{1}{c}{} & \multicolumn{1}{c}{} & \multicolumn{1}{c}{} & \multicolumn{1}{c}{}\tabularnewline
            \cline{1-4} 
            42218 & 254,6 & 42165 & 559,1 & \multicolumn{1}{c}{} & \multicolumn{1}{c}{} & \multicolumn{1}{c}{} & \multicolumn{1}{c}{}\tabularnewline
            \cline{1-4} 
            42183 & 298,6 & 42143 & 963,8 & \multicolumn{1}{c}{} & \multicolumn{1}{c}{} & \multicolumn{1}{c}{} & \multicolumn{1}{c}{}\tabularnewline
            \cline{1-4} 
            42143 & 327,5 & 42143 & 1800 & \multicolumn{1}{c}{} & \multicolumn{1}{c}{} & \multicolumn{1}{c}{} & \multicolumn{1}{c}{}\tabularnewline
            \cline{1-4} 
            42143 & 1800 & \multicolumn{1}{c}{} & \multicolumn{1}{c}{} & \multicolumn{1}{c}{} & \multicolumn{1}{c}{} & \multicolumn{1}{c}{} & \multicolumn{1}{c}{}\tabularnewline
            \cline{1-2} 
        \end{tabular}
    }
    \captionof{table}{Tabella risultati instanze con numero di nodi compreso tra \textbf{$200$} e \textbf{$299$} $+$ algoritmi esatti}
}
\vspace*{\fill}

\vspace*{\fill}
{
    \centering
    \centerline{\begin{tabular}{|c|c|c|c|c|c|c|c|}
            \hline 
            \multicolumn{8}{|c|}{lin318 - HF \& LB - PT2 - (costo ottimo 42029)}\tabularnewline
            \hline 
            \hline 
            \multicolumn{2}{|c|}{Thread3 HF} & \multicolumn{2}{c|}{Thread3 LB} & \multicolumn{2}{c|}{Thread4 HF} & \multicolumn{2}{c|}{Thread4 LB}\tabularnewline
            \hline 
            Costo & Tempo (s) & Costo & Tempo (s) & Costo & Tempo (s) & Costo & Tempo (s)\tabularnewline
            \hline 
            43094 & 0,0 & 43094 & 0,0 & 43020 & 0,0 & 43020 & 0,0\tabularnewline
            \hline 
            43075 & 0,2 & 43034 & 9,8 & 42704 & 1,9 & 42839 & 7,9\tabularnewline
            \hline 
            43029 & 9,8 & 42987 & 21,1 & 42655 & 6,6 & 42728 & 15,0\tabularnewline
            \hline 
            42964 & 11,1 & 42939 & 30,3 & 42536 & 10,1 & 42691 & 26,0\tabularnewline
            \hline 
            42924 & 19,5 & 42899 & 40,1 & 42512 & 11,8 & 42651 & 37,1\tabularnewline
            \hline 
            42864 & 25,0 & 42850 & 48,4 & 42511 & 18,5 & 42602 & 47,7\tabularnewline
            \hline 
            42815 & 27,8 & 42813 & 58,1 & 42452 & 28,6 & 42573 & 58,7\tabularnewline
            \hline 
            42801 & 33,3 & 42776 & 72,2 & 42340 & 36,5 & 42545 & 68,5\tabularnewline
            \hline 
            42759 & 33,9 & 42743 & 83,2 & 42297 & 44,0 & 42529 & 79,3\tabularnewline
            \hline 
            42707 & 37,5 & 42715 & 94,6 & 42271 & 46,3 & 42518 & 90,2\tabularnewline
            \hline 
            42667 & 44,5 & 42687 & 107,5 & 42234 & 50,6 & 42516 & 102,4\tabularnewline
            \hline 
            42650 & 53,3 & 42671 & 118,8 & 42207 & 68,8 & 42413 & 139,9\tabularnewline
            \hline 
            42640 & 63,3 & 42655 & 130,3 & 42193 & 294,9 & 42311 & 167,0\tabularnewline
            \hline 
            42551 & 68,4 & 42645 & 143,8 & 42160 & 304,4 & 42237 & 195,4\tabularnewline
            \hline 
            42551 & 1800 & 42564 & 208,6 & 42128 & 336,8 & 42186 & 219,4\tabularnewline
            \hline 
            \multicolumn{1}{c}{} &  & 42507 & 259,3 & 42107 & 436,4 & 42135 & 240,3\tabularnewline
            \cline{3-8} 
            \multicolumn{1}{c}{} &  & 42461 & 314,1 & 42107 & 1800 & 42112 & 277,3\tabularnewline
            \cline{3-8} 
            \multicolumn{1}{c}{} &  & 42451 & 395,8 & \multicolumn{1}{c}{} &  & 42104 & 309,7\tabularnewline
            \cline{3-4} \cline{7-8} 
            \multicolumn{1}{c}{} &  & 42451 & 1800 & \multicolumn{1}{c}{} &  & 42091 & 341,5\tabularnewline
            \cline{3-4} \cline{7-8} 
            \multicolumn{1}{c}{} & \multicolumn{1}{c}{} & \multicolumn{1}{c}{} & \multicolumn{1}{c}{} & \multicolumn{1}{c}{} &  & 42091 & 1800\tabularnewline
            \cline{7-8} 
        \end{tabular}
    }
    \captionof{table}{Tabella risultati instanze con numero di nodi compreso tra \textbf{$200$} e \textbf{$299$} $+$ algoritmi esatti}
}
\vspace*{\fill}

\vspace*{\fill}
{
    \centering
    \centerline{\begin{tabular}{|c|c|c|c|c|c|c|c|}
            \hline 
            \multicolumn{8}{|c|}{pr439 - HF \& LB - PT1 - (costo ottimo 107217)}\tabularnewline
            \hline 
            \hline 
            \multicolumn{2}{|c|}{Thread3 HF} & \multicolumn{2}{c|}{Thread3 LB} & \multicolumn{2}{c|}{Thread4 HF} & \multicolumn{2}{c|}{Thread4 LB}\tabularnewline
            \hline 
            109508 & 0,0 & 109508 & 0,0 & 109240 & 0,0 & 109240 & 0,0\tabularnewline
            \hline 
            109354 & 1,0 & 109276 & 32,0 & 109206 & 0,3 & 108998 & 54,1\tabularnewline
            \hline 
            109309 & 4,3 & 109061 & 59,0 & 109038 & 0,6 & 108815 & 93,7\tabularnewline
            \hline 
            109259 & 6,4 & 108867 & 90,3 & 108747 & 3,8 & 108647 & 134,1\tabularnewline
            \hline 
            109236 & 9,6 & 108718 & 124,4 & 108722 & 25,8 & 108540 & 172,5\tabularnewline
            \hline 
            109195 & 10,2 & 108611 & 159,9 & 108496 & 29,1 & 108495 & 220,6\tabularnewline
            \hline 
            109057 & 11,0 & 108510 & 196,9 & 108455 & 52,2 & 108452 & 265,3\tabularnewline
            \hline 
            108693 & 11,7 & 108419 & 234,4 & 108400 & 59,1 & 108427 & 308,6\tabularnewline
            \hline 
            108673 & 12,9 & 108334 & 269,1 & 108361 & 71,2 & 108404 & 356,0\tabularnewline
            \hline 
            108664 & 22,6 & 108283 & 306,3 & 108314 & 105,1 & 108386 & 403,0\tabularnewline
            \hline 
            108573 & 24,2 & 108251 & 345,6 & 108312 & 122,6 & 108370 & 454,8\tabularnewline
            \hline 
            108555 & 37,1 & 108204 & 386,5 & 108214 & 152,4 & 108360 & 505,9\tabularnewline
            \hline 
            108462 & 48,2 & 108178 & 429,3 & 108173 & 179,4 & 108334 & 557,1\tabularnewline
            \hline 
            108411 & 61,2 & 108152 & 476,5 & 108150 & 188,4 & 108202 & 774,1\tabularnewline
            \hline 
            108120 & 67,7 & 108129 & 523,7 & 108147 & 206,6 & 108076 & 945,7\tabularnewline
            \hline 
            108085 & 77,7 & 108120 & 571,0 & 108045 & 230,2 & 108009 & 1087,1\tabularnewline
            \hline 
            108055 & 83,4 & 107935 & 731,8 & 108019 & 300,2 & 107950 & 1229,6\tabularnewline
            \hline 
            108024 & 92,2 & 107794 & 818,5 & 107967 & 342,9 & 107903 & 1357,0\tabularnewline
            \hline 
            107985 & 97,3 & 107764 & 958,1 & 107961 & 864,3 & 107866 & 1529,1\tabularnewline
            \hline 
            107884 & 106,6 & 107759 & 1108,2 & 107960 & 932,5 & 107753 & 1712,5\tabularnewline
            \hline 
            107807 & 128,4 & 107757 & 1231,1 & 107887 & 990,1 & 107753 & 1800,0\tabularnewline
            \hline 
            107807 & 1800,0 & 107673 & 1327,4 & 107867 & 1052,7 & \multicolumn{1}{c}{} & \multicolumn{1}{c}{}\tabularnewline
            \cline{1-6} 
            \multicolumn{1}{c}{} &  & 107587 & 1459,4 & 107862 & 1241,9 & \multicolumn{1}{c}{} & \multicolumn{1}{c}{}\tabularnewline
            \cline{3-6} 
            \multicolumn{1}{c}{} &  & 107587 & 1800,0 & 107471 & 1250,0 & \multicolumn{1}{c}{} & \multicolumn{1}{c}{}\tabularnewline
            \cline{3-6} 
            \multicolumn{1}{c}{} & \multicolumn{1}{c}{} & \multicolumn{1}{c}{} &  & 107400 & 1291,8 & \multicolumn{1}{c}{} & \multicolumn{1}{c}{}\tabularnewline
            \cline{5-6} 
            \multicolumn{1}{c}{} & \multicolumn{1}{c}{} & \multicolumn{1}{c}{} &  & 107400 & 1800,0 & \multicolumn{1}{c}{} & \multicolumn{1}{c}{}\tabularnewline
            \cline{5-6} 
        \end{tabular}
    }
    \captionof{table}{Tabella risultati instanze con numero di nodi compreso tra \textbf{$200$} e \textbf{$299$} $+$ algoritmi esatti}
}
\vspace*{\fill}\vspace*{\fill}
{
    \centering
    \centerline{\begin{tabular}{|c|c|c|c|c|c|c|c|}
            \hline 
            \multicolumn{8}{|c|}{pr439 - HF \& LB - PT2 - (costo ottimo 107217)}\tabularnewline
            \hline 
            \hline 
            \multicolumn{2}{|c|}{Thread1 HF} & \multicolumn{2}{c|}{Thread1 LB} & \multicolumn{2}{c|}{Thread2 HF} & \multicolumn{2}{c|}{Thread2 LB}\tabularnewline
            \hline 
            Costo & Tempo (s) & Costo & Tempo (s) & Costo & Tempo (s) & Costo & Tempo (s)\tabularnewline
            \hline 
            43231 & 0,0 & 43231 & 0,0 & 42935 & 0,0 & 42935 & 0,0\tabularnewline
            \hline 
            43198 & 1,7 & 43132 & 8,1 & 42916 & 1,0 & 42813 & 7,2\tabularnewline
            \hline 
            43009 & 2,1 & 43040 & 15,7 & 42820 & 2,8 & 42721 & 16,2\tabularnewline
            \hline 
            42976 & 2,6 & 42957 & 24,3 & 42803 & 6,0 & 42678 & 25,8\tabularnewline
            \hline 
            42951 & 3,6 & 42875 & 30,6 & 42749 & 6,5 & 42638 & 33,7\tabularnewline
            \hline 
            42939 & 4,8 & 42832 & 39,7 & 42746 & 8,7 & 42601 & 46,8\tabularnewline
            \hline 
            42756 & 11,3 & 42792 & 48,1 & 42724 & 13,3 & 42568 & 55,8\tabularnewline
            \hline 
            42743 & 19,2 & 42743 & 59,2 & 42591 & 14,9 & 42540 & 64,4\tabularnewline
            \hline 
            42724 & 24,2 & 42710 & 70,5 & 42575 & 22,1 & 42524 & 76,4\tabularnewline
            \hline 
            42672 & 25,7 & 42677 & 81,9 & 42483 & 23,0 & 42512 & 89,5\tabularnewline
            \hline 
            42658 & 39,6 & 42644 & 94,1 & 42436 & 27,8 & 42506 & 102,5\tabularnewline
            \hline 
            42592 & 51,1 & 42628 & 107,4 & 42397 & 39,3 & 42425 & 147,1\tabularnewline
            \hline 
            42529 & 163,7 & 42530 & 144,7 & 42357 & 62,7 & 42371 & 189,0\tabularnewline
            \hline 
            42525 & 171,9 & 42455 & 179,9 & 42351 & 72,7 & 42317 & 225,9\tabularnewline
            \hline 
            42480 & 180,2 & 42401 & 234,4 & 42329 & 138,9 & 42293 & 282,4\tabularnewline
            \hline 
            42457 & 197,5 & 42347 & 269,1 & 42294 & 143,8 & 42270 & 327,9\tabularnewline
            \hline 
            42395 & 203,5 & 42314 & 322,3 & 42248 & 187,2 & 42248 & 705,7\tabularnewline
            \hline 
            42381 & 222,3 & 42302 & 374,4 & 42237 & 1415,6 & 42248 & 1800\tabularnewline
            \hline 
            42380 & 231,7 & 42273 & 415,9 & 42237 & 1800 & \multicolumn{1}{c}{} & \multicolumn{1}{c}{}\tabularnewline
            \cline{1-6} 
            42377 & 239,7 & 42265 & 470,7 & \multicolumn{1}{c}{} & \multicolumn{1}{c}{} & \multicolumn{1}{c}{} & \multicolumn{1}{c}{}\tabularnewline
            \cline{1-4} 
            42277 & 249,8 & 42178 & 512,7 & \multicolumn{1}{c}{} & \multicolumn{1}{c}{} & \multicolumn{1}{c}{} & \multicolumn{1}{c}{}\tabularnewline
            \cline{1-4} 
            42218 & 254,6 & 42165 & 559,1 & \multicolumn{1}{c}{} & \multicolumn{1}{c}{} & \multicolumn{1}{c}{} & \multicolumn{1}{c}{}\tabularnewline
            \cline{1-4} 
            42183 & 298,6 & 42143 & 963,8 & \multicolumn{1}{c}{} & \multicolumn{1}{c}{} & \multicolumn{1}{c}{} & \multicolumn{1}{c}{}\tabularnewline
            \cline{1-4} 
            42143 & 327,5 & 42143 & 1800 & \multicolumn{1}{c}{} & \multicolumn{1}{c}{} & \multicolumn{1}{c}{} & \multicolumn{1}{c}{}\tabularnewline
            \cline{1-4} 
            42143 & 1800 & \multicolumn{1}{c}{} & \multicolumn{1}{c}{} & \multicolumn{1}{c}{} & \multicolumn{1}{c}{} & \multicolumn{1}{c}{} & \multicolumn{1}{c}{}\tabularnewline
            \cline{1-2} 
        \end{tabular}
    }
    \captionof{table}{Tabella risultati instanze con numero di nodi compreso tra \textbf{$200$} e \textbf{$299$} $+$ algoritmi esatti}
}
\vspace*{\fill}

\vspace*{\fill}
{
    \centering
    \centerline{\begin{tabular}{|c|c|c|c|c|c|c|c|}
            \hline 
            \multicolumn{8}{|c|}{d493 - HF \& LB - PT1 - (costo ottimo 35002)}\tabularnewline
            \hline 
            \hline 
            \multicolumn{2}{|c|}{Thread3 HF} & \multicolumn{2}{c|}{Thread3 LB} & \multicolumn{2}{c|}{Thread4 HF} & \multicolumn{2}{c|}{Thread4 LB}\tabularnewline
            \hline 
            36157 & 0,0 & 36157 & 0,0 & 36026 & 0,0 & 36026 & 0,0\tabularnewline
            \hline 
            36092 & 3,5 & 36107 & 34,4 & 36017 & 1,7 & 35946 & 18,3\tabularnewline
            \hline 
            36085 & 6,8 & 36070 & 75,7 & 36009 & 3,5 & 35873 & 36,2\tabularnewline
            \hline 
            35951 & 8,2 & 36026 & 111,1 & 36003 & 6,7 & 35828 & 66,5\tabularnewline
            \hline 
            35867 & 10,6 & 35997 & 148,0 & 36002 & 7,8 & 35788 & 104,6\tabularnewline
            \hline 
            35853 & 12,0 & 35969 & 194,2 & 35983 & 9,3 & 35763 & 142,2\tabularnewline
            \hline 
            35817 & 13,1 & 35938 & 236,3 & 35976 & 11,9 & 35702 & 171,2\tabularnewline
            \hline 
            35788 & 16,8 & 35919 & 288,7 & 35961 & 13,4 & 35681 & 208,7\tabularnewline
            \hline 
            35785 & 18,9 & 35900 & 335,0 & 35916 & 14,6 & 35662 & 256,7\tabularnewline
            \hline 
            35780 & 27,5 & 35886 & 384,6 & 35895 & 15,7 & 35644 & 308,5\tabularnewline
            \hline 
            35768 & 28,6 & 35873 & 429,1 & 35809 & 19,3 & 35629 & 358,2\tabularnewline
            \hline 
            35764 & 31,3 & 35861 & 479,0 & 35769 & 23,2 & 35617 & 407,3\tabularnewline
            \hline 
            35704 & 32,6 & 35849 & 531,2 & 35742 & 27,0 & 35609 & 455,9\tabularnewline
            \hline 
            35701 & 34,3 & 35838 & 573,9 & 35711 & 27,6 & 35601 & 508,1\tabularnewline
            \hline 
            35682 & 38,5 & 35828 & 625,5 & 35708 & 30,7 & 35594 & 561,6\tabularnewline
            \hline 
            35669 & 40,8 & 35819 & 679,7 & 35702 & 33,3 & 35588 & 622,3\tabularnewline
            \hline 
            35654 & 48,0 & 35805 & 734,0 & 35687 & 37,6 & 35583 & 675,3\tabularnewline
            \hline 
            35628 & 58,2 & 35796 & 796,9 & 35668 & 40,9 & 35579 & 733,7\tabularnewline
            \hline 
            35607 & 93,5 & 35787 & 849,7 & 35656 & 50,4 & 35560 & 786,4\tabularnewline
            \hline 
            35586 & 106,6 & 35780 & 914,9 & 35609 & 51,8 & 35557 & 838,9\tabularnewline
            \hline 
            35563 & 137,9 & 35775 & 972,3 & 35591 & 52,9 & 35554 & 891,5\tabularnewline
            \hline 
            35559 & 153,7 & 35771 & 1042,4 & 35589 & 60,9 & 35552 & 946,8\tabularnewline
            \hline 
            35557 & 162,2 & 35768 & 1112,0 & 35568 & 75,2 & 35488 & 1078,6\tabularnewline
            \hline 
            35479 & 175,0 & 35765 & 1180,5 & 35564 & 85,5 & 35411 & 1142,1\tabularnewline
            \hline 
            35440 & 190,2 & 35752 & 1247,0 & 35555 & 94,1 & 35385 & 1314,6\tabularnewline
            \hline 
            35437 & 198,2 & 35749 & 1314,5 & 35548 & 102,0 & 35364 & 1471,6\tabularnewline
            \hline 
            35361 & 208,2 & 35747 & 1384,9 & 35531 & 136,0 & 35344 & 1631,3\tabularnewline
            \hline 
            35358 & 219,1 & 35645 & 1525,9 & 35518 & 148,1 & 35299 & 1714,8\tabularnewline
            \hline 
            35312 & 241,5 & 35605 & 1717,0 & 35446 & 165,4 & 35282 & 1800\tabularnewline
            \hline 
            35272 & 252,4 & 35605 & 1800 & 35434 & 171,3 & \multicolumn{1}{c}{} & \multicolumn{1}{c}{}\tabularnewline
            \cline{1-6} 
            35260 & 336,8 & \multicolumn{1}{c}{} &  & 35423 & 184,6 & \multicolumn{1}{c}{} & \multicolumn{1}{c}{}\tabularnewline
            \cline{1-2} \cline{5-6} 
            35248 & 354,0 & \multicolumn{1}{c}{} &  & 35373 & 191,3 & \multicolumn{1}{c}{} & \multicolumn{1}{c}{}\tabularnewline
            \cline{1-2} \cline{5-6} 
            35244 & 369,7 & \multicolumn{1}{c}{} &  & 35369 & 232,0 & \multicolumn{1}{c}{} & \multicolumn{1}{c}{}\tabularnewline
            \cline{1-2} \cline{5-6} 
            35149 & 389,5 & \multicolumn{1}{c}{} &  & 35357 & 246,1 & \multicolumn{1}{c}{} & \multicolumn{1}{c}{}\tabularnewline
            \cline{1-2} \cline{5-6} 
            35090 & 416,0 & \multicolumn{1}{c}{} &  & 35356 & 258,6 & \multicolumn{1}{c}{} & \multicolumn{1}{c}{}\tabularnewline
            \cline{1-2} \cline{5-6} 
            35084 & 429,7 & \multicolumn{1}{c}{} &  & 35346 & 284,5 & \multicolumn{1}{c}{} & \multicolumn{1}{c}{}\tabularnewline
            \cline{1-2} \cline{5-6} 
            35053 & 539,0 & \multicolumn{1}{c}{} &  & 35343 & 305,7 & \multicolumn{1}{c}{} & \multicolumn{1}{c}{}\tabularnewline
            \cline{1-2} \cline{5-6} 
            35047 & 592,6 & \multicolumn{1}{c}{} &  & 35338 & 321,4 & \multicolumn{1}{c}{} & \multicolumn{1}{c}{}\tabularnewline
            \cline{1-2} \cline{5-6} 
            35046 & 605,7 & \multicolumn{1}{c}{} &  & 35335 & 338,7 & \multicolumn{1}{c}{} & \multicolumn{1}{c}{}\tabularnewline
            \cline{1-2} \cline{5-6} 
            35043 & 624,5 & \multicolumn{1}{c}{} &  & 35329 & 361,1 & \multicolumn{1}{c}{} & \multicolumn{1}{c}{}\tabularnewline
            \cline{1-2} \cline{5-6} 
            35030 & 1128,2 & \multicolumn{1}{c}{} &  & 35274 & 374,2 & \multicolumn{1}{c}{} & \multicolumn{1}{c}{}\tabularnewline
            \cline{1-2} \cline{5-6} 
            35028 & 1189,2 & \multicolumn{1}{c}{} &  & 35191 & 381,8 & \multicolumn{1}{c}{} & \multicolumn{1}{c}{}\tabularnewline
            \cline{1-2} \cline{5-6} 
            35028 & 1800 & \multicolumn{1}{c}{} &  & 35189 & 395,1 & \multicolumn{1}{c}{} & \multicolumn{1}{c}{}\tabularnewline
            \cline{1-2} \cline{5-6} 
            \multicolumn{1}{c}{} & \multicolumn{1}{c}{} & \multicolumn{1}{c}{} &  & 35185 & 413,1 & \multicolumn{1}{c}{} & \multicolumn{1}{c}{}\tabularnewline
            \cline{5-6} 
            \multicolumn{1}{c}{} & \multicolumn{1}{c}{} & \multicolumn{1}{c}{} &  & 35152 & 448,8 & \multicolumn{1}{c}{} & \multicolumn{1}{c}{}\tabularnewline
            \cline{5-6} 
            \multicolumn{1}{c}{} & \multicolumn{1}{c}{} & \multicolumn{1}{c}{} &  & 35145 & 454,1 & \multicolumn{1}{c}{} & \multicolumn{1}{c}{}\tabularnewline
            \cline{5-6} 
            \multicolumn{1}{c}{} & \multicolumn{1}{c}{} & \multicolumn{1}{c}{} &  & 35133 & 474,8 & \multicolumn{1}{c}{} & \multicolumn{1}{c}{}\tabularnewline
            \cline{5-6} 
            \multicolumn{1}{c}{} & \multicolumn{1}{c}{} & \multicolumn{1}{c}{} &  & 35133 & 1800 & \multicolumn{1}{c}{} & \multicolumn{1}{c}{}\tabularnewline
            \cline{5-6} 
        \end{tabular}
    }
    \captionof{table}{Tabella risultati instanze con numero di nodi compreso tra \textbf{$200$} e \textbf{$299$} $+$ algoritmi esatti}
}
\vspace*{\fill}\vspace*{\fill}
{
    \centering
    \centerline{\begin{tabular}{|c|c|c|c|c|c|c|c|}
            \hline 
            \multicolumn{8}{|c|}{d493 - HF \& LB - PT2 - (costo ottimo 35002)}\tabularnewline
            \hline 
            \hline 
            \multicolumn{2}{|c|}{Thread1 HF} & \multicolumn{2}{c|}{Thread1 LB} & \multicolumn{2}{c|}{Thread2 HF} & \multicolumn{2}{c|}{Thread2 LB}\tabularnewline
            \hline 
            36021 & 0,0 & 36021 & 0,0 & 36069 & 0,0 & \multicolumn{1}{c|}{36069} & \multicolumn{1}{c|}{0,0}\tabularnewline
            \hline 
            35991 & 1,0 & 35928 & 14,9 & 36059 & 3,8 & \multicolumn{1}{c|}{36000} & \multicolumn{1}{c|}{25,9}\tabularnewline
            \hline 
            35948 & 6,7 & 35857 & 47,2 & 36054 & 9,6 & \multicolumn{1}{c|}{35960} & \multicolumn{1}{c|}{63,9}\tabularnewline
            \hline 
            35935 & 8,8 & 35808 & 76,5 & 36030 & 14,6 & \multicolumn{1}{c|}{35934} & \multicolumn{1}{c|}{98,9}\tabularnewline
            \hline 
            35914 & 10,0 & 35765 & 105,8 & 36002 & 17,4 & \multicolumn{1}{c|}{35906} & \multicolumn{1}{c|}{133,2}\tabularnewline
            \hline 
            35898 & 19,9 & 35736 & 142,3 & 35962 & 19,8 & \multicolumn{1}{c|}{35885} & \multicolumn{1}{c|}{178,1}\tabularnewline
            \hline 
            35869 & 20,8 & 35712 & 178,4 & 35934 & 22,3 & \multicolumn{1}{c|}{35866} & \multicolumn{1}{c|}{225,2}\tabularnewline
            \hline 
            35868 & 22,3 & 35691 & 215,1 & 35931 & 26,5 & \multicolumn{1}{c|}{35847} & \multicolumn{1}{c|}{270,1}\tabularnewline
            \hline 
            35725 & 33,7 & 35671 & 260,2 & 35902 & 30,3 & \multicolumn{1}{c|}{35828} & \multicolumn{1}{c|}{314,8}\tabularnewline
            \hline 
            35690 & 36,3 & 35652 & 303,5 & 35880 & 34,7 & \multicolumn{1}{c|}{35809} & \multicolumn{1}{c|}{358,5}\tabularnewline
            \hline 
            35680 & 44,0 & 35633 & 347,5 & 35850 & 36,5 & \multicolumn{1}{c|}{35791} & \multicolumn{1}{c|}{401,8}\tabularnewline
            \hline 
            35653 & 58,7 & 35617 & 393,7 & 35824 & 40,9 & \multicolumn{1}{c|}{35775} & \multicolumn{1}{c|}{443,2}\tabularnewline
            \hline 
            35630 & 65,8 & 35601 & 439,5 & 35776 & 43,8 & \multicolumn{1}{c|}{35764} & \multicolumn{1}{c|}{495,3}\tabularnewline
            \hline 
            35584 & 72,9 & 35587 & 484,2 & 35774 & 46,1 & \multicolumn{1}{c|}{35757} & \multicolumn{1}{c|}{537,6}\tabularnewline
            \hline 
            35567 & 77,7 & 35574 & 528,2 & 35772 & 51,0 & \multicolumn{1}{c|}{35752} & \multicolumn{1}{c|}{614,4}\tabularnewline
            \hline 
            35541 & 86,1 & 35562 & 570,4 & 35718 & 51,7 & \multicolumn{1}{c|}{35748} & \multicolumn{1}{c|}{692,6}\tabularnewline
            \hline 
            35535 & 94,3 & 35551 & 614,9 & 35697 & 56,4 & \multicolumn{1}{c|}{35745} & \multicolumn{1}{c|}{764,9}\tabularnewline
            \hline 
            35491 & 104,0 & 35541 & 657,9 & 35682 & 58,6 & \multicolumn{1}{c|}{35744} & \multicolumn{1}{c|}{836,7}\tabularnewline
            \hline 
            35488 & 110,7 & 35476 & 686,0 & 35621 & 65,4 & \multicolumn{1}{c|}{35706} & \multicolumn{1}{c|}{1060,6}\tabularnewline
            \hline 
            35449 & 113,1 & 35468 & 726,2 & 35618 & 72,7 & \multicolumn{1}{c|}{35677} & \multicolumn{1}{c|}{1262,0}\tabularnewline
            \hline 
            35417 & 118,3 & 35460 & 770,3 & 35614 & 76,4 & \multicolumn{1}{c|}{35658} & \multicolumn{1}{c|}{1451,5}\tabularnewline
            \hline 
            35416 & 125,5 & 35452 & 813,0 & 35609 & 92,0 & \multicolumn{1}{c|}{35645} & \multicolumn{1}{c|}{1686,9}\tabularnewline
            \hline 
            35397 & 130,7 & 35447 & 852,6 & 35606 & 99,6 & \multicolumn{1}{c|}{35645} & \multicolumn{1}{c|}{1800}\tabularnewline
            \hline 
            35379 & 133,2 & 35413 & 876,5 & 35600 & 104,6 &  & \tabularnewline
            \cline{1-6} 
            35328 & 141,0 & 35409 & 913,9 & 35597 & 112,7 &  & \tabularnewline
            \cline{1-6} 
            35287 & 146,0 & 35406 & 956,2 & 35596 & 123,1 &  & \tabularnewline
            \cline{1-6} 
            35283 & 147,3 & 35404 & 1011,7 & 35562 & 129,0 &  & \tabularnewline
            \cline{1-6} 
            35276 & 155,7 & 35402 & 1052,8 & 35537 & 137,6 &  & \tabularnewline
            \cline{1-6} 
            35269 & 158,7 & 35401 & 1102,6 & 35481 & 151,6 &  & \tabularnewline
            \cline{1-6} 
            35261 & 170,7 & 35392 & 1356,3 & 35473 & 160,4 &  & \tabularnewline
            \cline{1-6} 
            35231 & 175,7 & 35384 & 1690,5 & 35468 & 165,9 &  & \tabularnewline
            \cline{1-6} 
            35230 & 204,2 & 35378 & 1800 & 35456 & 173,5 &  & \tabularnewline
            \cline{1-6} 
            35219 & 215,1 & \multicolumn{1}{c}{} &  & 35451 & 181,3 &  & \tabularnewline
            \cline{1-2} \cline{5-6} 
            35213 & 246,5 & \multicolumn{1}{c}{} &  & 35442 & 215,7 &  & \tabularnewline
            \cline{1-2} \cline{5-6} 
            35212 & 277,9 & \multicolumn{1}{c}{} &  & 35433 & 249,3 &  & \tabularnewline
            \cline{1-2} \cline{5-6} 
            35195 & 292,4 & \multicolumn{1}{c}{} &  & 35430 & 259,5 &  & \tabularnewline
            \cline{1-2} \cline{5-6} 
            35194 & 305,5 & \multicolumn{1}{c}{} &  & 35400 & 271,3 &  & \tabularnewline
            \cline{1-2} \cline{5-6} 
            35155 & 312,7 & \multicolumn{1}{c}{} &  & 35378 & 289,9 &  & \tabularnewline
            \cline{1-2} \cline{5-6} 
            35149 & 329,1 & \multicolumn{1}{c}{} &  & 35365 & 330,7 &  & \tabularnewline
            \cline{1-2} \cline{5-6} 
            35131 & 341,7 & \multicolumn{1}{c}{} &  & 35345 & 335,5 &  & \tabularnewline
            \cline{1-2} \cline{5-6} 
            35128 & 351,0 & \multicolumn{1}{c}{} &  & 35340 & 380,9 &  & \tabularnewline
            \cline{1-2} \cline{5-6} 
            35127 & 386,1 & \multicolumn{1}{c}{} &  & 35338 & 412,9 &  & \tabularnewline
            \cline{1-2} \cline{5-6} 
            35121 & 396,7 & \multicolumn{1}{c}{} &  & 35331 & 427,5 &  & \tabularnewline
            \cline{1-2} \cline{5-6} 
            35121 & 1800 & \multicolumn{1}{c}{} &  & 35315 & 452,8 &  & \tabularnewline
            \cline{1-2} \cline{5-6} 
            \multicolumn{1}{c}{} & \multicolumn{1}{c}{} & \multicolumn{1}{c}{} &  & 35305 & 473,6 &  & \tabularnewline
            \cline{5-6} 
            \multicolumn{1}{c}{} & \multicolumn{1}{c}{} & \multicolumn{1}{c}{} &  & 35276 & 498,6 &  & \tabularnewline
            \cline{5-6} 
            \multicolumn{1}{c}{} & \multicolumn{1}{c}{} & \multicolumn{1}{c}{} &  & 35255 & 520,6 &  & \tabularnewline
            \cline{5-6} 
            \multicolumn{1}{c}{} & \multicolumn{1}{c}{} & \multicolumn{1}{c}{} &  & 35224 & 565,2 &  & \tabularnewline
            \cline{5-6} 
            \multicolumn{1}{c}{} & \multicolumn{1}{c}{} & \multicolumn{1}{c}{} &  & 35219 & 592,1 &  & \tabularnewline
            \cline{5-6} 
            \multicolumn{1}{c}{} & \multicolumn{1}{c}{} & \multicolumn{1}{c}{} &  & 35212 & 618,5 &  & \tabularnewline
            \cline{5-6} 
            \multicolumn{1}{c}{} & \multicolumn{1}{c}{} & \multicolumn{1}{c}{} &  & 35209 & 635,4 &  & \tabularnewline
            \cline{5-6} 
            \multicolumn{1}{c}{} & \multicolumn{1}{c}{} & \multicolumn{1}{c}{} &  & 35209 & 1800 &  & \tabularnewline
            \cline{5-6} 
        \end{tabular}
    }
    \captionof{table}{Tabella risultati instanze con numero di nodi compreso tra \textbf{$200$} e \textbf{$299$} $+$ algoritmi esatti}
}
\vspace*{\fill}

\vspace*{\fill}
{
    \centering
    \centerline{\begin{tabular}{|c|c|c|c|c|c|c|c|}
            \hline 
            \multicolumn{8}{|c|}{rat575 - HF \& LB - PT1 - (costo ottimo 6773)}\tabularnewline
            \hline 
            \hline 
            \multicolumn{2}{|c|}{Thread3 HF} & \multicolumn{2}{c|}{Thread3 LB} & \multicolumn{2}{c|}{Thread4 HF} & \multicolumn{2}{c|}{Thread4 LB}\tabularnewline
            \hline 
            7077 & 0,0 & 7077 & 0,0 & 7084 & 0,0 & 7084 & 0,0\tabularnewline
            \hline 
            7072 & 2,7 & 7063 & 25,2 & 7080 & 1,7 & 7070 & 21,4\tabularnewline
            \hline 
            7063 & 3,8 & 7050 & 52,5 & 7077 & 7,5 & 7057 & 45,9\tabularnewline
            \hline 
            7061 & 8,4 & 7039 & 76,1 & 7075 & 8,9 & 7046 & 67,1\tabularnewline
            \hline 
            7055 & 9,7 & 7028 & 105,9 & 7068 & 10,5 & 7035 & 93,6\tabularnewline
            \hline 
            7050 & 11,9 & 7019 & 134,0 & 7047 & 13,6 & 7025 & 122,4\tabularnewline
            \hline 
            7040 & 15,2 & 7011 & 169,3 & 7044 & 17,6 & 7016 & 150,7\tabularnewline
            \hline 
            7031 & 21,3 & 7003 & 199,4 & 7027 & 19,1 & 7009 & 184,9\tabularnewline
            \hline 
            7030 & 22,9 & 6995 & 233,0 & 7016 & 20,6 & 7003 & 216,7\tabularnewline
            \hline 
            7025 & 23,8 & 6988 & 267,7 & 7006 & 25,8 & 6998 & 250,9\tabularnewline
            \hline 
            7011 & 25,6 & 6981 & 305,8 & 7000 & 29,3 & 6993 & 279,1\tabularnewline
            \hline 
            7004 & 27,0 & 6975 & 342,2 & 6988 & 32,6 & 6988 & 307,8\tabularnewline
            \hline 
            7003 & 28,9 & 6969 & 372,5 & 6984 & 37,9 & 6983 & 331,0\tabularnewline
            \hline 
            6978 & 33,7 & 6963 & 403,1 & 6959 & 39,1 & 6978 & 367,1\tabularnewline
            \hline 
            6974 & 36,0 & 6958 & 430,1 & 6955 & 42,5 & 6974 & 399,4\tabularnewline
            \hline 
            6973 & 39,2 & 6954 & 458,5 & 6951 & 43,4 & 6970 & 428,5\tabularnewline
            \hline 
            6971 & 41,7 & 6950 & 489,1 & 6946 & 55,9 & 6967 & 458,9\tabularnewline
            \hline 
            6970 & 42,5 & 6946 & 515,9 & 6938 & 63,6 & 6965 & 488,4\tabularnewline
            \hline 
            6966 & 43,9 & 6939 & 544,3 & 6933 & 67,5 & 6963 & 523,4\tabularnewline
            \hline 
            6965 & 45,5 & 6935 & 572,8 & 6931 & 73,8 & 6961 & 555,9\tabularnewline
            \hline 
            6958 & 59,3 & 6932 & 605,5 & 6925 & 77,4 & 6959 & 586,1\tabularnewline
            \hline 
            6940 & 70,8 & 6929 & 635,2 & 6923 & 88,1 & 6957 & 616,9\tabularnewline
            \hline 
            6931 & 76,2 & 6926 & 662,9 & 6909 & 96,7 & 6956 & 651,7\tabularnewline
            \hline 
            6923 & 82,9 & 6924 & 694,4 & 6908 & 102,8 & 6955 & 679,6\tabularnewline
            \hline 
            6914 & 85,9 & 6922 & 732,9 & 6904 & 104,8 & 6954 & 716,7\tabularnewline
            \hline 
            6906 & 87,9 & 6920 & 772,7 & 6894 & 116,0 & 6953 & 743,8\tabularnewline
            \hline 
            6902 & 91,8 & 6916 & 806,2 & 6893 & 121,8 & 6934 & 785,3\tabularnewline
            \hline 
            6899 & 96,1 & 6913 & 836,9 & 6891 & 131,9 & 6924 & 811,8\tabularnewline
            \hline 
            6891 & 126,1 & 6912 & 872,0 & 6889 & 140,1 & 6915 & 841,5\tabularnewline
            \hline 
            6890 & 141,8 & 6911 & 908,5 & 6886 & 142,8 & 6907 & 891,8\tabularnewline
            \hline 
            6889 & 153,2 & 6910 & 942,9 & 6884 & 144,6 & 6901 & 941,9\tabularnewline
            \hline 
            6880 & 164,6 & 6909 & 980,1 & 6873 & 176,3 & 6894 & 966,5\tabularnewline
            \hline 
            6876 & 177,6 & 6908 & 1021,8 & 6872 & 182,9 & 6889 & 1024,8\tabularnewline
            \hline 
            6875 & 278,6 & 6903 & 1102,9 & 6858 & 204,2 & 6885 & 1073,3\tabularnewline
            \hline 
            6872 & 307,2 & 6900 & 1155,0 & 6853 & 209,3 & 6882 & 1112,6\tabularnewline
            \hline 
            6869 & 320,3 & 6894 & 1198,4 & 6851 & 212,9 & 6879 & 1181,9\tabularnewline
            \hline 
            6852 & 334,1 & 6891 & 1274,8 & 6849 & 245,3 & 6877 & 1250,6\tabularnewline
            \hline 
            6839 & 370,4 & 6888 & 1335,1 & 6842 & 250,1 & 6872 & 1488,3\tabularnewline
            \hline 
            6837 & 378,1 & 6885 & 1423,7 & 6841 & 261,0 & 6868 & 1675,2\tabularnewline
            \hline 
            6828 & 382,5 & 6881 & 1481,2 & 6835 & 267,0 & 6860 & 1748,3\tabularnewline
            \hline 
            6826 & 391,0 & 6879 & 1579,7 & 6834 & 272,1 & 6858 & 1800\tabularnewline
            \hline 
            6817 & 395,1 & 6878 & 1670,1 & 6830 & 289,7 & \multicolumn{1}{c}{} & \multicolumn{1}{c}{}\tabularnewline
            \cline{1-6} 
            6814 & 557,7 & 6876 & 1738,7 & 6828 & 301,6 & \multicolumn{1}{c}{} & \multicolumn{1}{c}{}\tabularnewline
            \cline{1-6} 
            6813 & 641,4 & 6876 & 1800 & 6822 & 350,8 & \multicolumn{1}{c}{} & \multicolumn{1}{c}{}\tabularnewline
            \cline{1-6} 
            6811 & 676,1 & \multicolumn{1}{c}{} &  & 6819 & 385,2 & \multicolumn{1}{c}{} & \multicolumn{1}{c}{}\tabularnewline
            \cline{1-2} \cline{5-6} 
            6810 & 731,6 & \multicolumn{1}{c}{} &  & 6818 & 460,0 & \multicolumn{1}{c}{} & \multicolumn{1}{c}{}\tabularnewline
            \cline{1-2} \cline{5-6} 
            6809 & 771,5 & \multicolumn{1}{c}{} &  & 6814 & 486,1 & \multicolumn{1}{c}{} & \multicolumn{1}{c}{}\tabularnewline
            \cline{1-2} \cline{5-6} 
            6806 & 795,7 & \multicolumn{1}{c}{} &  & 6813 & 550,8 & \multicolumn{1}{c}{} & \multicolumn{1}{c}{}\tabularnewline
            \cline{1-2} \cline{5-6} 
            6804 & 1066,0 & \multicolumn{1}{c}{} &  & 6808 & 582,0 & \multicolumn{1}{c}{} & \multicolumn{1}{c}{}\tabularnewline
            \cline{1-2} \cline{5-6} 
            6803 & 1087,0 & \multicolumn{1}{c}{} &  & 6804 & 600,2 & \multicolumn{1}{c}{} & \multicolumn{1}{c}{}\tabularnewline
            \cline{1-2} \cline{5-6} 
            6802 & 1114,8 & \multicolumn{1}{c}{} &  & 6798 & 626,8 & \multicolumn{1}{c}{} & \multicolumn{1}{c}{}\tabularnewline
            \cline{1-2} \cline{5-6} 
            6801 & 1147,4 & \multicolumn{1}{c}{} &  & 6793 & 649,3 & \multicolumn{1}{c}{} & \multicolumn{1}{c}{}\tabularnewline
            \cline{1-2} \cline{5-6} 
            6800 & 1252,5 & \multicolumn{1}{c}{} &  & 6788 & 678,9 & \multicolumn{1}{c}{} & \multicolumn{1}{c}{}\tabularnewline
            \cline{1-2} \cline{5-6} 
            6799 & 1285,1 & \multicolumn{1}{c}{} &  & 6786 & 734,2 & \multicolumn{1}{c}{} & \multicolumn{1}{c}{}\tabularnewline
            \cline{1-2} \cline{5-6} 
            6797 & 1372,5 & \multicolumn{1}{c}{} &  & 6784 & 754,7 & \multicolumn{1}{c}{} & \multicolumn{1}{c}{}\tabularnewline
            \cline{1-2} \cline{5-6} 
            6796 & 1413,7 & \multicolumn{1}{c}{} &  & 6783 & 784,3 & \multicolumn{1}{c}{} & \multicolumn{1}{c}{}\tabularnewline
            \cline{1-2} \cline{5-6} 
            6795 & 1439,0 & \multicolumn{1}{c}{} &  & 6781 & 844,7 & \multicolumn{1}{c}{} & \multicolumn{1}{c}{}\tabularnewline
            \cline{1-2} \cline{5-6} 
            6793 & 1484,4 & \multicolumn{1}{c}{} &  & 6781 & 1800 & \multicolumn{1}{c}{} & \multicolumn{1}{c}{}\tabularnewline
            \cline{1-2} \cline{5-6} 
            6793 & 1800,0 & \multicolumn{1}{c}{} & \multicolumn{1}{c}{} & \multicolumn{1}{c}{} & \multicolumn{1}{c}{} & \multicolumn{1}{c}{} & \multicolumn{1}{c}{}\tabularnewline
            \cline{1-2} 
        \end{tabular}
    }
    \captionof{table}{Tabella risultati instanze con numero di nodi compreso tra \textbf{$200$} e \textbf{$299$} $+$ algoritmi esatti}
}
\vspace*{\fill}\vspace*{\fill}
{
    \centering
    \centerline{\begin{tabular}{|c|c|c|c|c|c|c|c|}
            \hline 
            \multicolumn{8}{|c|}{rat575 - HF \& LB - PT2 - (costo ottimo 6773)}\tabularnewline
            \hline 
            \hline 
            \multicolumn{2}{|c|}{Thread1 HF} & \multicolumn{2}{c|}{Thread1 LB} & \multicolumn{2}{c|}{Thread2 HF} & \multicolumn{2}{c|}{Thread2 LB}\tabularnewline
            \hline 
            7079 & 0,0 & 7079 & 0,0 & 7055 & 0,0 & 7055 & 0,0\tabularnewline
            \hline 
            7069 & 1,5 & 7059 & 5,9 & 7048 & 0,8 & 7041 & 28,4\tabularnewline
            \hline 
            7056 & 5,5 & 7043 & 25,2 & 7043 & 2,0 & 7029 & 56,8\tabularnewline
            \hline 
            7048 & 6,9 & 7034 & 52,1 & 7042 & 5,9 & 7017 & 81,2\tabularnewline
            \hline 
            7044 & 9,5 & 7026 & 73,7 & 7021 & 12,2 & 7009 & 122,1\tabularnewline
            \hline 
            7043 & 12,0 & 7018 & 98,2 & 7008 & 15,9 & 7001 & 163,2\tabularnewline
            \hline 
            7041 & 14,1 & 7010 & 126,0 & 7002 & 19,5 & 6994 & 192,2\tabularnewline
            \hline 
            7040 & 16,8 & 7003 & 158,2 & 6971 & 27,9 & 6988 & 228,2\tabularnewline
            \hline 
            7034 & 19,4 & 6997 & 190,2 & 6965 & 30,6 & 6982 & 263,5\tabularnewline
            \hline 
            7019 & 22,7 & 6993 & 223,3 & 6956 & 34,0 & 6977 & 302,0\tabularnewline
            \hline 
            7013 & 26,2 & 6989 & 253,1 & 6948 & 37,2 & 6972 & 345,4\tabularnewline
            \hline 
            7012 & 28,4 & 6985 & 282,9 & 6941 & 44,5 & 6967 & 381,8\tabularnewline
            \hline 
            7008 & 31,3 & 6981 & 315,5 & 6939 & 49,3 & 6962 & 413,7\tabularnewline
            \hline 
            7000 & 34,6 & 6977 & 348,9 & 6938 & 52,2 & 6958 & 450,3\tabularnewline
            \hline 
            6989 & 45,6 & 6975 & 381,9 & 6937 & 63,7 & 6954 & 491,9\tabularnewline
            \hline 
            6981 & 52,0 & 6973 & 414,4 & 6925 & 68,4 & 6950 & 521,8\tabularnewline
            \hline 
            6946 & 80,2 & 6971 & 452,4 & 6917 & 70,5 & 6946 & 554,3\tabularnewline
            \hline 
            6925 & 91,4 & 6967 & 491,3 & 6907 & 72,2 & 6943 & 584,1\tabularnewline
            \hline 
            6920 & 115,7 & 6965 & 526,4 & 6893 & 82,0 & 6934 & 625,1\tabularnewline
            \hline 
            6914 & 130,7 & 6963 & 564,5 & 6891 & 98,9 & 6922 & 660,7\tabularnewline
            \hline 
            6911 & 141,8 & 6961 & 602,2 & 6889 & 100,2 & 6919 & 695,5\tabularnewline
            \hline 
            6910 & 158,1 & 6959 & 647,5 & 6878 & 103,3 & 6916 & 727,8\tabularnewline
            \hline 
            6907 & 162,5 & 6958 & 684,7 & 6876 & 107,1 & 6912 & 768,2\tabularnewline
            \hline 
            6895 & 178,2 & 6957 & 727,4 & 6873 & 111,3 & 6910 & 802,7\tabularnewline
            \hline 
            6893 & 188,9 & 6956 & 768,0 & 6871 & 116,9 & 6908 & 837,4\tabularnewline
            \hline 
            6889 & 194,6 & 6955 & 807,1 & 6865 & 124,1 & 6903 & 866,2\tabularnewline
            \hline 
            6883 & 201,2 & 6945 & 884,6 & 6852 & 127,7 & 6901 & 903,5\tabularnewline
            \hline 
            6870 & 210,4 & 6937 & 925,3 & 6851 & 135,0 & 6899 & 937,9\tabularnewline
            \hline 
            6868 & 217,3 & 6926 & 970,3 & 6848 & 164,3 & 6898 & 974,5\tabularnewline
            \hline 
            6863 & 220,1 & 6918 & 1040,9 & 6845 & 177,1 & 6897 & 1010,9\tabularnewline
            \hline 
            6862 & 225,5 & 6910 & 1096,0 & 6826 & 181,2 & 6896 & 1044,9\tabularnewline
            \hline 
            6853 & 233,5 & 6902 & 1130,3 & 6820 & 190,9 & 6895 & 1086,0\tabularnewline
            \hline 
            6844 & 248,1 & 6897 & 1191,2 & 6815 & 200,4 & 6894 & 1122,1\tabularnewline
            \hline 
            6843 & 278,0 & 6892 & 1250,1 & 6813 & 235,0 & 6883 & 1186,1\tabularnewline
            \hline 
            6837 & 295,1 & 6887 & 1305,3 & 6809 & 249,0 & 6872 & 1212,0\tabularnewline
            \hline 
            6836 & 379,5 & 6884 & 1368,8 & 6804 & 259,2 & 6861 & 1238,1\tabularnewline
            \hline 
            6831 & 410,1 & 6876 & 1409,0 & 6802 & 273,2 & 6851 & 1260,2\tabularnewline
            \hline 
            6830 & 524,8 & 6873 & 1472,0 & 6801 & 595,5 & 6834 & 1275,9\tabularnewline
            \hline 
            6826 & 841,5 & 6870 & 1552,8 & 6800 & 622,8 & 6826 & 1307,8\tabularnewline
            \hline 
            6819 & 859,2 & 6864 & 1601,0 & 6796 & 679,5 & 6824 & 1369,5\tabularnewline
            \hline 
            6815 & 900,7 & 6863 & 1694,6 & 6794 & 715,4 & 6820 & 1570,6\tabularnewline
            \hline 
            6799 & 931,3 & 6863 & 1800 & 6792 & 738,4 & 6816 & 1659,3\tabularnewline
            \hline 
            6797 & 944,6 & \multicolumn{1}{c}{} &  & 6791 & 765,5 & 6814 & 1765,1\tabularnewline
            \cline{1-2} \cline{5-8} 
            6796 & 1353,9 & \multicolumn{1}{c}{} &  & 6788 & 971,4 & 6814 & 1800\tabularnewline
            \cline{1-2} \cline{5-8} 
            6792 & 1430,2 & \multicolumn{1}{c}{} &  & 6787 & 1012,6 & \multicolumn{1}{c}{} & \multicolumn{1}{c}{}\tabularnewline
            \cline{1-2} \cline{5-6} 
            6791 & 1762,1 & \multicolumn{1}{c}{} &  & 6786 & 1054,2 & \multicolumn{1}{c}{} & \multicolumn{1}{c}{}\tabularnewline
            \cline{1-2} \cline{5-6} 
            6791 & 1800 & \multicolumn{1}{c}{} &  & 6784 & 1078,4 & \multicolumn{1}{c}{} & \multicolumn{1}{c}{}\tabularnewline
            \cline{1-2} \cline{5-6} 
            \multicolumn{1}{c}{} & \multicolumn{1}{c}{} & \multicolumn{1}{c}{} &  & 6783 & 1104,9 & \multicolumn{1}{c}{} & \multicolumn{1}{c}{}\tabularnewline
            \cline{5-6} 
            \multicolumn{1}{c}{} & \multicolumn{1}{c}{} & \multicolumn{1}{c}{} &  & 6782 & 1135,6 & \multicolumn{1}{c}{} & \multicolumn{1}{c}{}\tabularnewline
            \cline{5-6} 
            \multicolumn{1}{c}{} & \multicolumn{1}{c}{} & \multicolumn{1}{c}{} &  & 6782 & 1800 & \multicolumn{1}{c}{} & \multicolumn{1}{c}{}\tabularnewline
            \cline{5-6} 
        \end{tabular}
    }
    \captionof{table}{Tabella risultati instanze con numero di nodi compreso tra \textbf{$200$} e \textbf{$299$} $+$ algoritmi esatti}
}
\vspace*{\fill}

\vspace*{\fill}
{
    \centering
    \centerline{\begin{tabular}{|c|c|c|c|c|c|c|c|}
            \hline 
            \multicolumn{8}{|c|}{lin318 - HF \& LB - PT2 - (costo ottimo 42029)}\tabularnewline
            \hline 
            \hline 
            \multicolumn{2}{|c|}{Thread3 HF} & \multicolumn{2}{c|}{Thread3 LB} & \multicolumn{2}{c|}{Thread4 HF} & \multicolumn{2}{c|}{Thread4 LB}\tabularnewline
            \hline 
            Costo & Tempo (s) & Costo & Tempo (s) & Costo & Tempo (s) & Costo & Tempo (s)\tabularnewline
            \hline 
            51032 & 0,0 & \multicolumn{1}{c|}{51032} & 0,0 & 51230 & 0,0 & \multicolumn{1}{c|}{51230} & \multicolumn{1}{c|}{0,0}\tabularnewline
            \hline 
            51008 & 10,8 & \multicolumn{1}{c|}{50911} & 40,6 & 51218 & 24,6 & \multicolumn{1}{c|}{51101} & \multicolumn{1}{c|}{39,4}\tabularnewline
            \hline 
            51002 & 13,4 & \multicolumn{1}{c|}{50804} & 76,9 & 51131 & 43,1 & \multicolumn{1}{c|}{50983} & \multicolumn{1}{c|}{99,1}\tabularnewline
            \hline 
            50979 & 16,7 & \multicolumn{1}{c|}{50702} & 103,2 & 51040 & 56,8 & \multicolumn{1}{c|}{50904} & \multicolumn{1}{c|}{161,7}\tabularnewline
            \hline 
            50916 & 18,5 & \multicolumn{1}{c|}{50635} & 154,4 & 51028 & 59,2 & \multicolumn{1}{c|}{50832} & \multicolumn{1}{c|}{227,6}\tabularnewline
            \hline 
            50902 & 26,3 & \multicolumn{1}{c|}{50575} & 195,3 & 50952 & 61,7 & \multicolumn{1}{c|}{50763} & \multicolumn{1}{c|}{313,9}\tabularnewline
            \hline 
            50889 & 32,6 & \multicolumn{1}{c|}{50515} & 230,5 & 50914 & 65,1 & \multicolumn{1}{c|}{50696} & \multicolumn{1}{c|}{380,4}\tabularnewline
            \hline 
            50884 & 38,3 & \multicolumn{1}{c|}{50463} & 284,2 & 50913 & 76,2 & \multicolumn{1}{c|}{50631} & \multicolumn{1}{c|}{455,6}\tabularnewline
            \hline 
            50766 & 40,2 & \multicolumn{1}{c|}{50416} & 332,2 & 50848 & 83,1 & \multicolumn{1}{c|}{50580} & \multicolumn{1}{c|}{531,5}\tabularnewline
            \hline 
            50761 & 42,9 & \multicolumn{1}{c|}{50371} & 380,6 & 50785 & 88,7 & \multicolumn{1}{c|}{50538} & \multicolumn{1}{c|}{606,7}\tabularnewline
            \hline 
            50723 & 46,3 & \multicolumn{1}{c|}{50340} & 442,5 & 50778 & 90,3 & \multicolumn{1}{c|}{50460} & \multicolumn{1}{c|}{663,3}\tabularnewline
            \hline 
            50610 & 49,2 & \multicolumn{1}{c|}{50312} & 514,6 & 50742 & 95,5 & \multicolumn{1}{c|}{50418} & \multicolumn{1}{c|}{739,6}\tabularnewline
            \hline 
            50601 & 58,3 & \multicolumn{1}{c|}{50290} & 593,8 & 50722 & 101,2 & \multicolumn{1}{c|}{50381} & \multicolumn{1}{c|}{804,6}\tabularnewline
            \hline 
            50574 & 61,2 & \multicolumn{1}{c|}{50269} & 663,7 & 50701 & 105,5 & \multicolumn{1}{c|}{50345} & \multicolumn{1}{c|}{870,4}\tabularnewline
            \hline 
            50475 & 72,4 & \multicolumn{1}{c|}{50250} & 754,7 & 50680 & 109,5 & \multicolumn{1}{c|}{50314} & \multicolumn{1}{c|}{943,4}\tabularnewline
            \hline 
            50452 & 81,7 & \multicolumn{1}{c|}{50231} & 843,5 & 50592 & 111,4 & \multicolumn{1}{c|}{50284} & \multicolumn{1}{c|}{1005,3}\tabularnewline
            \hline 
            50426 & 87,6 & \multicolumn{1}{c|}{50201} & 908,6 & 50491 & 115,9 & \multicolumn{1}{c|}{50254} & \multicolumn{1}{c|}{1074,6}\tabularnewline
            \hline 
            50418 & 88,7 & \multicolumn{1}{c|}{50183} & 993,0 & 50469 & 121,4 & \multicolumn{1}{c|}{50228} & \multicolumn{1}{c|}{1158,1}\tabularnewline
            \hline 
            50387 & 93,7 & \multicolumn{1}{c|}{50165} & 1061,1 & 50354 & 131,7 & \multicolumn{1}{c|}{50203} & \multicolumn{1}{c|}{1255,8}\tabularnewline
            \hline 
            50325 & 98,2 & \multicolumn{1}{c|}{50138} & 1151,0 & 50323 & 134,3 & \multicolumn{1}{c|}{50179} & \multicolumn{1}{c|}{1330,3}\tabularnewline
            \hline 
            50307 & 105,3 & \multicolumn{1}{c|}{50121} & 1238,1 & 50297 & 150,2 & \multicolumn{1}{c|}{50158} & \multicolumn{1}{c|}{1417,5}\tabularnewline
            \hline 
            50300 & 108,9 & \multicolumn{1}{c|}{50107} & 1325,2 & 50276 & 153,3 & \multicolumn{1}{c|}{50123} & \multicolumn{1}{c|}{1494,7}\tabularnewline
            \hline 
            50268 & 113,0 & \multicolumn{1}{c|}{50093} & 1421,7 & 50275 & 155,1 & \multicolumn{1}{c|}{50102} & \multicolumn{1}{c|}{1572,0}\tabularnewline
            \hline 
            50222 & 118,2 & \multicolumn{1}{c|}{50081} & 1507,1 & 50193 & 160,0 & \multicolumn{1}{c|}{50084} & \multicolumn{1}{c|}{1683,1}\tabularnewline
            \hline 
            50194 & 123,2 & \multicolumn{1}{c|}{50056} & 1578,7 & 50192 & 167,4 & \multicolumn{1}{c|}{50067} & \multicolumn{1}{c|}{1766,2}\tabularnewline
            \hline 
            50133 & 132,7 & \multicolumn{1}{c|}{50004} & 1649,6 & 50191 & 174,3 & \multicolumn{1}{c|}{50051} & \multicolumn{1}{c|}{1800}\tabularnewline
            \hline 
            50063 & 146,9 & \multicolumn{1}{c|}{49977} & 1724,2 & 50189 & 177,7 &  & \tabularnewline
            \cline{1-6} 
            50054 & 162,1 & \multicolumn{1}{c|}{49969} & 1800 & 50184 & 181,1 &  & \tabularnewline
            \cline{1-6} 
            50037 & 167,5 &  &  & 50181 & 183,1 &  & \tabularnewline
            \cline{1-2} \cline{5-6} 
            49954 & 189,3 &  &  & 50177 & 190,3 &  & \tabularnewline
            \cline{1-2} \cline{5-6} 
            49934 & 204,8 &  &  & 50173 & 201,2 &  & \tabularnewline
            \cline{1-2} \cline{5-6} 
            49930 & 208,8 &  &  & 50149 & 215,3 &  & \tabularnewline
            \cline{1-2} \cline{5-6} 
            49914 & 212,3 &  &  & 50142 & 217,9 &  & \tabularnewline
            \cline{1-2} \cline{5-6} 
            49897 & 243,9 &  &  & 50137 & 236,2 &  & \tabularnewline
            \cline{1-2} \cline{5-6} 
            49869 & 269,2 &  &  & 50132 & 327,3 &  & \tabularnewline
            \cline{1-2} \cline{5-6} 
            49863 & 299,4 &  &  & 50111 & 345,2 &  & \tabularnewline
            \cline{1-2} \cline{5-6} 
            49791 & 336,5 &  &  & 50107 & 404,6 &  & \tabularnewline
            \cline{1-2} \cline{5-6} 
            49766 & 343,5 &  &  & 50094 & 433,0 &  & \tabularnewline
            \cline{1-2} \cline{5-6} 
            49753 & 362,9 &  &  & 49953 & 453,6 &  & \tabularnewline
            \cline{1-2} \cline{5-6} 
            49697 & 380,6 &  &  & 49877 & 470,3 &  & \tabularnewline
            \cline{1-2} \cline{5-6} 
            49656 & 391,3 &  &  & 49804 & 484,7 &  & \tabularnewline
            \cline{1-2} \cline{5-6} 
            49635 & 430,2 &  &  & 49752 & 504,2 &  & \tabularnewline
            \cline{1-2} \cline{5-6} 
            49630 & 438,4 &  &  & 49720 & 550,3 &  & \tabularnewline
            \cline{1-2} \cline{5-6} 
            49606 & 462,9 &  &  & 49648 & 567,4 &  & \tabularnewline
            \cline{1-2} \cline{5-6} 
            49595 & 498,8 &  &  & 49640 & 591,0 &  & \tabularnewline
            \cline{1-2} \cline{5-6} 
            49435 & 524,3 &  &  & 49542 & 624,4 &  & \tabularnewline
            \cline{1-2} \cline{5-6} 
            49410 & 557,8 &  &  & 49501 & 641,6 &  & \tabularnewline
            \cline{1-2} \cline{5-6} 
            49406 & 621,9 &  &  & 49465 & 662,8 &  & \tabularnewline
            \cline{1-2} \cline{5-6} 
            49398 & 648,9 &  &  & 49431 & 675,4 &  & \tabularnewline
            \cline{1-2} \cline{5-6} 
            49376 & 664,2 &  &  & 49423 & 718,3 &  & \tabularnewline
            \cline{1-2} \cline{5-6} 
            49375 & 678,8 &  &  & 49365 & 750,1 &  & \tabularnewline
            \cline{1-2} \cline{5-6} 
            49360 & 709,8 &  &  & 49365 & 1800 &  & \tabularnewline
            \cline{1-2} \cline{5-6} 
            49353 & 715,7 &  & \multicolumn{1}{c}{} & \multicolumn{1}{c}{} & \multicolumn{1}{c}{} &  & \tabularnewline
            \cline{1-2} 
            49339 & 863,4 &  & \multicolumn{1}{c}{} & \multicolumn{1}{c}{} & \multicolumn{1}{c}{} &  & \tabularnewline
            \cline{1-2} 
            49332 & 876,5 &  & \multicolumn{1}{c}{} & \multicolumn{1}{c}{} & \multicolumn{1}{c}{} &  & \tabularnewline
            \cline{1-2} 
            49326 & 889,8 &  & \multicolumn{1}{c}{} & \multicolumn{1}{c}{} & \multicolumn{1}{c}{} &  & \tabularnewline
            \cline{1-2} 
            49320 & 977,7 &  & \multicolumn{1}{c}{} & \multicolumn{1}{c}{} & \multicolumn{1}{c}{} &  & \tabularnewline
            \cline{1-2} 
            49317 & 1032,8 &  & \multicolumn{1}{c}{} & \multicolumn{1}{c}{} & \multicolumn{1}{c}{} &  & \tabularnewline
            \cline{1-2} 
            49297 & 1065,7 &  & \multicolumn{1}{c}{} & \multicolumn{1}{c}{} & \multicolumn{1}{c}{} &  & \tabularnewline
            \cline{1-2} 
            49266 & 1235,0 &  & \multicolumn{1}{c}{} & \multicolumn{1}{c}{} & \multicolumn{1}{c}{} &  & \tabularnewline
            \cline{1-2} 
            49219 & 1327,9 &  & \multicolumn{1}{c}{} & \multicolumn{1}{c}{} & \multicolumn{1}{c}{} &  & \tabularnewline
            \cline{1-2} 
            49176 & 1475,7 &  & \multicolumn{1}{c}{} & \multicolumn{1}{c}{} & \multicolumn{1}{c}{} &  & \tabularnewline
            \cline{1-2} 
            49137 & 1593,2 &  & \multicolumn{1}{c}{} & \multicolumn{1}{c}{} & \multicolumn{1}{c}{} &  & \tabularnewline
            \cline{1-2} 
            49137 & 1800 &  & \multicolumn{1}{c}{} & \multicolumn{1}{c}{} & \multicolumn{1}{c}{} &  & \tabularnewline
            \cline{1-2} 
        \end{tabular}
    }
    \captionof{table}{Tabella risultati instanze con numero di nodi compreso tra \textbf{$200$} e \textbf{$299$} $+$ algoritmi esatti}
}
\vspace*{\fill}\vspace*{\fill}
{
    \centering
    \centerline{\begin{tabular}{|c|c|c|c|c|c|c|c|}
            \hline 
            \multicolumn{8}{|c|}{lin318 - HARD FIXING E LOCAL BRANCH - PT1 - (costo ottimo 42029)}\tabularnewline
            \hline 
            \hline 
            \multicolumn{2}{|c|}{Thread1 HF} & \multicolumn{2}{c|}{Thread1 LB} & \multicolumn{2}{c|}{Thread2 HF} & \multicolumn{2}{c|}{Thread2 LB}\tabularnewline
            \hline 
            Costo & Tempo (s) & Costo & Tempo (s) & Costo & Tempo (s) & Costo & Tempo (s)\tabularnewline
            \hline 
            51109 & 0,0 & \multicolumn{1}{c|}{51109} & 0,0 & 50974 & 0,0 & \multicolumn{1}{c|}{50974} & \multicolumn{1}{c|}{0,0}\tabularnewline
            \hline 
            51095 & 1,2 & \multicolumn{1}{c|}{51033} & 66,1 & 50953 & 8,6 & \multicolumn{1}{c|}{50889} & \multicolumn{1}{c|}{45,9}\tabularnewline
            \hline 
            51088 & 2,9 & \multicolumn{1}{c|}{50981} & 132,7 & 50821 & 12,3 & \multicolumn{1}{c|}{50807} & \multicolumn{1}{c|}{100,3}\tabularnewline
            \hline 
            51082 & 4,1 & \multicolumn{1}{c|}{50934} & 212,1 & 50781 & 15,7 & \multicolumn{1}{c|}{50708} & \multicolumn{1}{c|}{143,6}\tabularnewline
            \hline 
            51050 & 5,3 & \multicolumn{1}{c|}{50891} & 293,2 & 50733 & 21,5 & \multicolumn{1}{c|}{50627} & \multicolumn{1}{c|}{201,4}\tabularnewline
            \hline 
            51046 & 8,2 & \multicolumn{1}{c|}{50853} & 353,6 & 50538 & 26,6 & \multicolumn{1}{c|}{50572} & \multicolumn{1}{c|}{254,0}\tabularnewline
            \hline 
            51003 & 9,7 & \multicolumn{1}{c|}{50815} & 411,7 & 50464 & 31,5 & \multicolumn{1}{c|}{50521} & \multicolumn{1}{c|}{340,5}\tabularnewline
            \hline 
            50997 & 10,8 & \multicolumn{1}{c|}{50777} & 490,0 & 50394 & 36,3 & \multicolumn{1}{c|}{50482} & \multicolumn{1}{c|}{402,1}\tabularnewline
            \hline 
            50983 & 15,9 & \multicolumn{1}{c|}{50744} & 564,9 & 50355 & 38,3 & \multicolumn{1}{c|}{50448} & \multicolumn{1}{c|}{476,1}\tabularnewline
            \hline 
            50936 & 20,6 & \multicolumn{1}{c|}{50706} & 641,6 & 50346 & 47,9 & \multicolumn{1}{c|}{50416} & \multicolumn{1}{c|}{535,6}\tabularnewline
            \hline 
            50848 & 27,9 & \multicolumn{1}{c|}{50674} & 715,7 & 50327 & 82,9 & \multicolumn{1}{c|}{50388} & \multicolumn{1}{c|}{598,2}\tabularnewline
            \hline 
            50816 & 32,8 & \multicolumn{1}{c|}{50646} & 806,4 & 50303 & 93,4 & \multicolumn{1}{c|}{50365} & \multicolumn{1}{c|}{715,1}\tabularnewline
            \hline 
            50713 & 35,7 & \multicolumn{1}{c|}{50619} & 873,9 & 50276 & 116,3 & \multicolumn{1}{c|}{50344} & \multicolumn{1}{c|}{834,1}\tabularnewline
            \hline 
            50604 & 49,9 & \multicolumn{1}{c|}{50594} & 942,7 & 50037 & 127,6 & \multicolumn{1}{c|}{50323} & \multicolumn{1}{c|}{934,3}\tabularnewline
            \hline 
            50577 & 52,3 & \multicolumn{1}{c|}{50571} & 1017,2 & 50027 & 132,8 & \multicolumn{1}{c|}{50302} & \multicolumn{1}{c|}{1036,1}\tabularnewline
            \hline 
            50534 & 54,9 & \multicolumn{1}{c|}{50549} & 1088,8 & 49900 & 145,5 & \multicolumn{1}{c|}{50281} & \multicolumn{1}{c|}{1134,6}\tabularnewline
            \hline 
            50509 & 61,8 & \multicolumn{1}{c|}{50528} & 1164,4 & 49853 & 154,9 & \multicolumn{1}{c|}{50261} & \multicolumn{1}{c|}{1240,5}\tabularnewline
            \hline 
            50376 & 65,3 & \multicolumn{1}{c|}{50509} & 1246,2 & 49806 & 159,6 & \multicolumn{1}{c|}{50230} & \multicolumn{1}{c|}{1318,5}\tabularnewline
            \hline 
            50338 & 69,9 & \multicolumn{1}{c|}{50492} & 1320,4 & 49759 & 166,3 & \multicolumn{1}{c|}{50210} & \multicolumn{1}{c|}{1403,9}\tabularnewline
            \hline 
            50299 & 83,8 & \multicolumn{1}{c|}{50451} & 1392,0 & 49689 & 171,8 & \multicolumn{1}{c|}{50191} & \multicolumn{1}{c|}{1493,4}\tabularnewline
            \hline 
            50287 & 86,5 & \multicolumn{1}{c|}{50432} & 1477,2 & 49679 & 215,9 & \multicolumn{1}{c|}{50177} & \multicolumn{1}{c|}{1586,8}\tabularnewline
            \hline 
            50286 & 87,7 & \multicolumn{1}{c|}{50415} & 1550,7 & 49672 & 244,2 & \multicolumn{1}{c|}{50171} & \multicolumn{1}{c|}{1712,8}\tabularnewline
            \hline 
            50270 & 90,4 & \multicolumn{1}{c|}{50398} & 1615,3 & 49619 & 251,5 & \multicolumn{1}{c|}{50170} & \multicolumn{1}{c|}{1800}\tabularnewline
            \hline 
            50224 & 93,6 & \multicolumn{1}{c|}{50383} & 1686,7 & 49600 & 259,1 &  & \tabularnewline
            \cline{1-6} 
            50202 & 96,5 & \multicolumn{1}{c|}{50365} & 1746,9 & 49533 & 292,7 &  & \tabularnewline
            \cline{1-6} 
            50189 & 102,7 & \multicolumn{1}{c|}{50352} & 1800 & 49478 & 302,7 &  & \tabularnewline
            \cline{1-6} 
            50151 & 106,8 &  &  & 49433 & 323,0 &  & \tabularnewline
            \cline{1-2} \cline{5-6} 
            50142 & 110,9 &  &  & 49432 & 328,2 &  & \tabularnewline
            \cline{1-2} \cline{5-6} 
            50137 & 113,0 &  &  & 49423 & 360,8 &  & \tabularnewline
            \cline{1-2} \cline{5-6} 
            50122 & 122,9 &  &  & 49412 & 374,3 &  & \tabularnewline
            \cline{1-2} \cline{5-6} 
            50121 & 129,3 &  &  & 49372 & 422,6 &  & \tabularnewline
            \cline{1-2} \cline{5-6} 
            50116 & 133,2 &  &  & 49332 & 438,0 &  & \tabularnewline
            \cline{1-2} \cline{5-6} 
            49992 & 140,0 &  &  & 49298 & 469,3 &  & \tabularnewline
            \cline{1-2} \cline{5-6} 
            49971 & 145,5 &  &  & 49283 & 501,4 &  & \tabularnewline
            \cline{1-2} \cline{5-6} 
            49952 & 149,7 &  &  & 49268 & 541,4 &  & \tabularnewline
            \cline{1-2} \cline{5-6} 
            49926 & 169,4 &  &  & 49261 & 805,8 &  & \tabularnewline
            \cline{1-2} \cline{5-6} 
            49881 & 189,0 &  &  & 49256 & 824,5 &  & \tabularnewline
            \cline{1-2} \cline{5-6} 
            49762 & 201,5 &  &  & 49194 & 853,8 &  & \tabularnewline
            \cline{1-2} \cline{5-6} 
            49736 & 218,4 &  &  & 49185 & 878,7 &  & \tabularnewline
            \cline{1-2} \cline{5-6} 
            49717 & 234,9 &  &  & 49178 & 927,9 &  & \tabularnewline
            \cline{1-2} \cline{5-6} 
            49708 & 241,5 &  &  & 49159 & 1013,8 &  & \tabularnewline
            \cline{1-2} \cline{5-6} 
            49691 & 272,6 &  &  & 49140 & 1052,2 &  & \tabularnewline
            \cline{1-2} \cline{5-6} 
            49684 & 275,3 &  &  & 49120 & 1271,2 &  & \tabularnewline
            \cline{1-2} \cline{5-6} 
            49648 & 281,1 &  &  & 49077 & 1665,4 &  & \tabularnewline
            \cline{1-2} \cline{5-6} 
            49641 & 283,4 &  &  & 49075 & 1755,6 &  & \tabularnewline
            \cline{1-2} \cline{5-6} 
            49613 & 302,1 &  &  & 49075 & 1800 &  & \tabularnewline
            \cline{1-2} \cline{5-6} 
            49601 & 313,6 &  & \multicolumn{1}{c}{} & \multicolumn{1}{c}{} & \multicolumn{1}{c}{} &  & \tabularnewline
            \cline{1-2} 
            49578 & 317,6 &  & \multicolumn{1}{c}{} & \multicolumn{1}{c}{} & \multicolumn{1}{c}{} &  & \tabularnewline
            \cline{1-2} 
            49546 & 362,6 &  & \multicolumn{1}{c}{} & \multicolumn{1}{c}{} & \multicolumn{1}{c}{} &  & \tabularnewline
            \cline{1-2} 
            49536 & 399,0 &  & \multicolumn{1}{c}{} & \multicolumn{1}{c}{} & \multicolumn{1}{c}{} &  & \tabularnewline
            \cline{1-2} 
            49530 & 420,6 &  & \multicolumn{1}{c}{} & \multicolumn{1}{c}{} & \multicolumn{1}{c}{} &  & \tabularnewline
            \cline{1-2} 
            49487 & 437,5 &  & \multicolumn{1}{c}{} & \multicolumn{1}{c}{} & \multicolumn{1}{c}{} &  & \tabularnewline
            \cline{1-2} 
            49482 & 472,4 &  & \multicolumn{1}{c}{} & \multicolumn{1}{c}{} & \multicolumn{1}{c}{} &  & \tabularnewline
            \cline{1-2} 
            49478 & 491,2 &  & \multicolumn{1}{c}{} & \multicolumn{1}{c}{} & \multicolumn{1}{c}{} &  & \tabularnewline
            \cline{1-2} 
            49460 & 508,4 &  & \multicolumn{1}{c}{} & \multicolumn{1}{c}{} & \multicolumn{1}{c}{} &  & \tabularnewline
            \cline{1-2} 
            49448 & 546,3 &  & \multicolumn{1}{c}{} & \multicolumn{1}{c}{} & \multicolumn{1}{c}{} &  & \tabularnewline
            \cline{1-2} 
            49424 & 568,8 &  & \multicolumn{1}{c}{} & \multicolumn{1}{c}{} & \multicolumn{1}{c}{} &  & \tabularnewline
            \cline{1-2} 
            49394 & 605,4 &  & \multicolumn{1}{c}{} & \multicolumn{1}{c}{} & \multicolumn{1}{c}{} &  & \tabularnewline
            \cline{1-2} 
            49381 & 638,5 &  & \multicolumn{1}{c}{} & \multicolumn{1}{c}{} & \multicolumn{1}{c}{} &  & \tabularnewline
            \cline{1-2} 
            49380 & 717,0 &  & \multicolumn{1}{c}{} & \multicolumn{1}{c}{} & \multicolumn{1}{c}{} &  & \tabularnewline
            \cline{1-2} 
            49368 & 838,5 &  & \multicolumn{1}{c}{} & \multicolumn{1}{c}{} & \multicolumn{1}{c}{} &  & \tabularnewline
            \cline{1-2} 
            49368 & 1800 &  & \multicolumn{1}{c}{} & \multicolumn{1}{c}{} & \multicolumn{1}{c}{} &  & \tabularnewline
            \cline{1-2} 
        \end{tabular}
    }
    \captionof{table}{Tabella risultati instanze con numero di nodi compreso tra \textbf{$200$} e \textbf{$299$} $+$ algoritmi esatti}
}
\vspace*{\fill}

\vspace*{\fill}
{
    \centering
    \centerline{\begin{tabular}{|c|c|c|c|c|c|c|c|}
            \hline 
            \multicolumn{8}{|c|}{u724 - HF \& LB - PT1 - (costo ottimo 41910)}\tabularnewline
            \hline 
            \hline 
            \multicolumn{2}{|c|}{Thread3 HF} & \multicolumn{2}{c|}{Thread3 LB} & \multicolumn{2}{c|}{Thread4 HF} & \multicolumn{2}{c|}{Thread4 LB}\tabularnewline
            \hline 
            Costo & Tempo (s) & Costo & Tempo (s) & Costo & Tempo (s) & Costo & Tempo (s)\tabularnewline
            \hline 
            43864 & 0,0 & \multicolumn{1}{c|}{43864} & 0,0 & 43915 & 0,0 & \multicolumn{1}{c|}{43915} & \multicolumn{1}{c|}{0,0}\tabularnewline
            \hline 
            43793 & 9,0 & \multicolumn{1}{c|}{43796} & 66,7 & 43897 & 4,0 & \multicolumn{1}{c|}{43848} & \multicolumn{1}{c|}{53,2}\tabularnewline
            \hline 
            43736 & 19,3 & \multicolumn{1}{c|}{43756} & 129,2 & 43881 & 8,4 & \multicolumn{1}{c|}{43781} & \multicolumn{1}{c|}{105,9}\tabularnewline
            \hline 
            43699 & 23,2 & \multicolumn{1}{c|}{43719} & 198,2 & 43878 & 10,0 & \multicolumn{1}{c|}{43733} & \multicolumn{1}{c|}{163,5}\tabularnewline
            \hline 
            43678 & 32,1 & \multicolumn{1}{c|}{43684} & 270,6 & 43862 & 12,0 & \multicolumn{1}{c|}{43691} & \multicolumn{1}{c|}{243,8}\tabularnewline
            \hline 
            43611 & 34,8 & \multicolumn{1}{c|}{43648} & 342,9 & 43814 & 26,6 & \multicolumn{1}{c|}{43652} & \multicolumn{1}{c|}{304,8}\tabularnewline
            \hline 
            43519 & 37,9 & \multicolumn{1}{c|}{43613} & 423,6 & 43784 & 30,3 & \multicolumn{1}{c|}{43613} & \multicolumn{1}{c|}{360,7}\tabularnewline
            \hline 
            43447 & 44,9 & \multicolumn{1}{c|}{43583} & 510,6 & 43759 & 47,3 & \multicolumn{1}{c|}{43578} & \multicolumn{1}{c|}{426,6}\tabularnewline
            \hline 
            43441 & 47,0 & \multicolumn{1}{c|}{43556} & 612,3 & 43756 & 55,4 & \multicolumn{1}{c|}{43544} & \multicolumn{1}{c|}{502,3}\tabularnewline
            \hline 
            43429 & 49,3 & \multicolumn{1}{c|}{43516} & 671,6 & 43699 & 60,1 & \multicolumn{1}{c|}{43491} & \multicolumn{1}{c|}{562,3}\tabularnewline
            \hline 
            43328 & 57,6 & \multicolumn{1}{c|}{43469} & 725,9 & 43689 & 64,8 & \multicolumn{1}{c|}{43459} & \multicolumn{1}{c|}{625,2}\tabularnewline
            \hline 
            43307 & 61,3 & \multicolumn{1}{c|}{43405} & 760,3 & 43675 & 67,6 & \multicolumn{1}{c|}{43427} & \multicolumn{1}{c|}{696,5}\tabularnewline
            \hline 
            43286 & 66,5 & \multicolumn{1}{c|}{43357} & 805,4 & 43582 & 100,6 & \multicolumn{1}{c|}{43387} & \multicolumn{1}{c|}{768,2}\tabularnewline
            \hline 
            43214 & 72,6 & \multicolumn{1}{c|}{43316} & 853,7 & 43579 & 210,8 & \multicolumn{1}{c|}{43314} & \multicolumn{1}{c|}{790,0}\tabularnewline
            \hline 
            43151 & 75,6 & \multicolumn{1}{c|}{43287} & 919,5 & 43502 & 224,6 & \multicolumn{1}{c|}{43289} & \multicolumn{1}{c|}{898,2}\tabularnewline
            \hline 
            43105 & 77,2 & \multicolumn{1}{c|}{43261} & 985,9 & 43457 & 231,9 & \multicolumn{1}{c|}{43272} & \multicolumn{1}{c|}{1016,0}\tabularnewline
            \hline 
            43045 & 79,4 & \multicolumn{1}{c|}{43235} & 1061,2 & 43293 & 259,4 & \multicolumn{1}{c|}{43245} & \multicolumn{1}{c|}{1072,4}\tabularnewline
            \hline 
            43021 & 86,6 & \multicolumn{1}{c|}{43209} & 1128,0 & 43248 & 272,4 & \multicolumn{1}{c|}{43228} & \multicolumn{1}{c|}{1162,3}\tabularnewline
            \hline 
            43016 & 88,7 & \multicolumn{1}{c|}{43185} & 1208,4 & 43218 & 290,6 & \multicolumn{1}{c|}{43212} & \multicolumn{1}{c|}{1255,7}\tabularnewline
            \hline 
            42976 & 93,0 & \multicolumn{1}{c|}{43122} & 1243,0 & 43140 & 301,2 & \multicolumn{1}{c|}{43200} & \multicolumn{1}{c|}{1361,4}\tabularnewline
            \hline 
            42971 & 97,4 & \multicolumn{1}{c|}{43102} & 1315,1 & 43132 & 304,5 & \multicolumn{1}{c|}{43189} & \multicolumn{1}{c|}{1466,2}\tabularnewline
            \hline 
            42948 & 101,6 & \multicolumn{1}{c|}{43083} & 1380,3 & 43086 & 312,8 & \multicolumn{1}{c|}{43181} & \multicolumn{1}{c|}{1577,7}\tabularnewline
            \hline 
            42930 & 104,4 & \multicolumn{1}{c|}{43064} & 1444,7 & 43025 & 321,6 & \multicolumn{1}{c|}{43171} & \multicolumn{1}{c|}{1730,6}\tabularnewline
            \hline 
            42907 & 119,0 & \multicolumn{1}{c|}{43047} & 1514,2 & 42987 & 331,0 & \multicolumn{1}{c|}{43166} & \multicolumn{1}{c|}{1800,0}\tabularnewline
            \hline 
            42747 & 130,2 & \multicolumn{1}{c|}{43030} & 1585,4 & 42942 & 349,3 &  & \tabularnewline
            \cline{1-6} 
            42745 & 140,9 & \multicolumn{1}{c|}{43014} & 1657,0 & 42939 & 363,2 &  & \tabularnewline
            \cline{1-6} 
            42723 & 149,5 & \multicolumn{1}{c|}{43000} & 1744,6 & 42872 & 373,8 &  & \tabularnewline
            \cline{1-6} 
            42707 & 159,1 & \multicolumn{1}{c|}{42993} & 1800,0 & 42860 & 405,7 &  & \tabularnewline
            \cline{1-6} 
            42698 & 171,1 &  &  & 42821 & 459,9 &  & \tabularnewline
            \cline{1-2} \cline{5-6} 
            42682 & 189,1 &  &  & 42813 & 493,5 &  & \tabularnewline
            \cline{1-2} \cline{5-6} 
            42664 & 198,5 &  &  & 42760 & 522,5 &  & \tabularnewline
            \cline{1-2} \cline{5-6} 
            42648 & 251,2 &  &  & 42744 & 547,6 &  & \tabularnewline
            \cline{1-2} \cline{5-6} 
            42618 & 276,0 &  &  & 42726 & 729,0 &  & \tabularnewline
            \cline{1-2} \cline{5-6} 
            42614 & 292,6 &  &  & 42707 & 773,8 &  & \tabularnewline
            \cline{1-2} \cline{5-6} 
            42586 & 322,1 &  &  & 42686 & 857,5 &  & \tabularnewline
            \cline{1-2} \cline{5-6} 
            42571 & 345,4 &  &  & 42652 & 912,3 &  & \tabularnewline
            \cline{1-2} \cline{5-6} 
            42496 & 384,8 &  &  & 42617 & 951,7 &  & \tabularnewline
            \cline{1-2} \cline{5-6} 
            42482 & 403,7 &  &  & 42598 & 1000,9 &  & \tabularnewline
            \cline{1-2} \cline{5-6} 
            42478 & 456,5 &  &  & 42566 & 1047,8 &  & \tabularnewline
            \cline{1-2} \cline{5-6} 
            42448 & 474,5 &  &  & 42566 & 1800,0 &  & \tabularnewline
            \cline{1-2} \cline{5-6} 
            42428 & 500,1 &  & \multicolumn{1}{c}{} & \multicolumn{1}{c}{} & \multicolumn{1}{c}{} &  & \tabularnewline
            \cline{1-2} 
            42397 & 517,3 &  & \multicolumn{1}{c}{} & \multicolumn{1}{c}{} & \multicolumn{1}{c}{} &  & \tabularnewline
            \cline{1-2} 
            42396 & 531,1 &  & \multicolumn{1}{c}{} & \multicolumn{1}{c}{} & \multicolumn{1}{c}{} &  & \tabularnewline
            \cline{1-2} 
            42356 & 609,3 &  & \multicolumn{1}{c}{} & \multicolumn{1}{c}{} & \multicolumn{1}{c}{} &  & \tabularnewline
            \cline{1-2} 
            42353 & 703,2 &  & \multicolumn{1}{c}{} & \multicolumn{1}{c}{} & \multicolumn{1}{c}{} &  & \tabularnewline
            \cline{1-2} 
            42352 & 840,5 &  & \multicolumn{1}{c}{} & \multicolumn{1}{c}{} & \multicolumn{1}{c}{} &  & \tabularnewline
            \cline{1-2} 
            42340 & 1085,0 &  & \multicolumn{1}{c}{} & \multicolumn{1}{c}{} & \multicolumn{1}{c}{} &  & \tabularnewline
            \cline{1-2} 
            42325 & 1137,8 &  & \multicolumn{1}{c}{} & \multicolumn{1}{c}{} & \multicolumn{1}{c}{} &  & \tabularnewline
            \cline{1-2} 
            42307 & 1164,2 &  & \multicolumn{1}{c}{} & \multicolumn{1}{c}{} & \multicolumn{1}{c}{} &  & \tabularnewline
            \cline{1-2} 
            42276 & 1210,0 &  & \multicolumn{1}{c}{} & \multicolumn{1}{c}{} & \multicolumn{1}{c}{} &  & \tabularnewline
            \cline{1-2} 
            42274 & 1260,3 &  & \multicolumn{1}{c}{} & \multicolumn{1}{c}{} & \multicolumn{1}{c}{} &  & \tabularnewline
            \cline{1-2} 
            42249 & 1323,8 &  & \multicolumn{1}{c}{} & \multicolumn{1}{c}{} & \multicolumn{1}{c}{} &  & \tabularnewline
            \cline{1-2} 
            42243 & 1368,3 &  & \multicolumn{1}{c}{} & \multicolumn{1}{c}{} & \multicolumn{1}{c}{} &  & \tabularnewline
            \cline{1-2} 
            42235 & 1741,4 &  & \multicolumn{1}{c}{} & \multicolumn{1}{c}{} & \multicolumn{1}{c}{} &  & \tabularnewline
            \cline{1-2} 
            42235 & 1800,0 &  & \multicolumn{1}{c}{} & \multicolumn{1}{c}{} & \multicolumn{1}{c}{} &  & \tabularnewline
            \cline{1-2} 
            49448 & 546,3 &  & \multicolumn{1}{c}{} & \multicolumn{1}{c}{} & \multicolumn{1}{c}{} &  & \tabularnewline
            \cline{1-2} 
            49424 & 568,8 &  & \multicolumn{1}{c}{} & \multicolumn{1}{c}{} & \multicolumn{1}{c}{} &  & \tabularnewline
            \cline{1-2} 
            49394 & 605,4 &  & \multicolumn{1}{c}{} & \multicolumn{1}{c}{} & \multicolumn{1}{c}{} &  & \tabularnewline
            \cline{1-2} 
            49381 & 638,5 &  & \multicolumn{1}{c}{} & \multicolumn{1}{c}{} & \multicolumn{1}{c}{} &  & \tabularnewline
            \cline{1-2} 
            49380 & 717,0 &  & \multicolumn{1}{c}{} & \multicolumn{1}{c}{} & \multicolumn{1}{c}{} &  & \tabularnewline
            \cline{1-2} 
            49368 & 838,5 &  & \multicolumn{1}{c}{} & \multicolumn{1}{c}{} & \multicolumn{1}{c}{} &  & \tabularnewline
            \cline{1-2} 
            49368 & 1800 &  & \multicolumn{1}{c}{} & \multicolumn{1}{c}{} & \multicolumn{1}{c}{} &  & \tabularnewline
            \cline{1-2} 
        \end{tabular}
    }
    \captionof{table}{Tabella risultati instanze con numero di nodi compreso tra \textbf{$200$} e \textbf{$299$} $+$ algoritmi esatti}
}
\vspace*{\fill}\vspace*{\fill}
{
    \centering
    \centerline{\begin{tabular}{|c|c|c|c|c|c|c|c|}
            \hline 
            \multicolumn{8}{|c|}{u724 - HF \& LB - PT2 - (costo ottimo 41910)}\tabularnewline
            \hline 
            \hline 
            \multicolumn{2}{|c|}{Thread1 HF} & \multicolumn{2}{c|}{Thread1 LB} & \multicolumn{2}{c|}{Thread2 HF} & \multicolumn{2}{c|}{Thread2 LB}\tabularnewline
            \hline 
            Costo & Tempo (s) & Costo & Tempo (s) & Costo & Tempo (s) & Costo & Tempo (s)\tabularnewline
            \hline 
            43852 & 0,0 & \multicolumn{1}{c|}{43852} & 0,0 & 43711 & 0,0 & \multicolumn{1}{c|}{43711} & \multicolumn{1}{c|}{0,0}\tabularnewline
            \hline 
            43833 & 5,7 & \multicolumn{1}{c|}{43746} & 23,7 & 43690 & 0,8 & \multicolumn{1}{c|}{43633} & \multicolumn{1}{c|}{47,0}\tabularnewline
            \hline 
            43790 & 8,1 & \multicolumn{1}{c|}{43668} & 60,0 & 43686 & 3,6 & \multicolumn{1}{c|}{43563} & \multicolumn{1}{c|}{102,5}\tabularnewline
            \hline 
            43726 & 10,6 & \multicolumn{1}{c|}{43595} & 102,3 & 43616 & 10,4 & \multicolumn{1}{c|}{43511} & \multicolumn{1}{c|}{151,9}\tabularnewline
            \hline 
            43717 & 12,2 & \multicolumn{1}{c|}{43525} & 147,9 & 43570 & 13,3 & \multicolumn{1}{c|}{43459} & \multicolumn{1}{c|}{202,5}\tabularnewline
            \hline 
            43537 & 20,6 & \multicolumn{1}{c|}{43474} & 204,8 & 43561 & 18,0 & \multicolumn{1}{c|}{43409} & \multicolumn{1}{c|}{257,4}\tabularnewline
            \hline 
            43515 & 27,0 & \multicolumn{1}{c|}{43426} & 272,8 & 43543 & 19,8 & \multicolumn{1}{c|}{43362} & \multicolumn{1}{c|}{294,9}\tabularnewline
            \hline 
            43496 & 40,5 & \multicolumn{1}{c|}{43383} & 327,5 & 43531 & 21,8 & \multicolumn{1}{c|}{43322} & \multicolumn{1}{c|}{342,9}\tabularnewline
            \hline 
            43394 & 45,0 & \multicolumn{1}{c|}{43342} & 399,5 & 43484 & 27,9 & \multicolumn{1}{c|}{43288} & \multicolumn{1}{c|}{406,1}\tabularnewline
            \hline 
            43363 & 47,9 & \multicolumn{1}{c|}{43311} & 465,0 & 43470 & 29,3 & \multicolumn{1}{c|}{43253} & \multicolumn{1}{c|}{460,2}\tabularnewline
            \hline 
            43256 & 52,1 & \multicolumn{1}{c|}{43281} & 535,2 & 43445 & 38,3 & \multicolumn{1}{c|}{43219} & \multicolumn{1}{c|}{514,1}\tabularnewline
            \hline 
            43175 & 59,1 & \multicolumn{1}{c|}{43256} & 633,5 & 43440 & 40,8 & \multicolumn{1}{c|}{43187} & \multicolumn{1}{c|}{568,3}\tabularnewline
            \hline 
            43159 & 63,4 & \multicolumn{1}{c|}{43237} & 709,7 & 43365 & 45,5 & \multicolumn{1}{c|}{43156} & \multicolumn{1}{c|}{623,5}\tabularnewline
            \hline 
            43136 & 67,2 & \multicolumn{1}{c|}{43218} & 804,4 & 43355 & 52,9 & \multicolumn{1}{c|}{43129} & \multicolumn{1}{c|}{693,9}\tabularnewline
            \hline 
            43129 & 69,4 & \multicolumn{1}{c|}{43201} & 885,6 & 43336 & 57,9 & \multicolumn{1}{c|}{43103} & \multicolumn{1}{c|}{738,5}\tabularnewline
            \hline 
            43125 & 76,3 & \multicolumn{1}{c|}{43188} & 990,6 & 43271 & 99,2 & \multicolumn{1}{c|}{43080} & \multicolumn{1}{c|}{796,0}\tabularnewline
            \hline 
            43120 & 84,6 & \multicolumn{1}{c|}{43154} & 1058,4 & 43234 & 109,9 & \multicolumn{1}{c|}{43059} & \multicolumn{1}{c|}{858,1}\tabularnewline
            \hline 
            43066 & 91,1 & \multicolumn{1}{c|}{43133} & 1152,4 & 43220 & 122,2 & \multicolumn{1}{c|}{43039} & \multicolumn{1}{c|}{916,8}\tabularnewline
            \hline 
            43052 & 97,4 & \multicolumn{1}{c|}{43121} & 1225,5 & 43098 & 142,2 & \multicolumn{1}{c|}{43020} & \multicolumn{1}{c|}{987,6}\tabularnewline
            \hline 
            43051 & 112,7 & \multicolumn{1}{c|}{43109} & 1300,7 & 43026 & 149,6 & \multicolumn{1}{c|}{43001} & \multicolumn{1}{c|}{1051,4}\tabularnewline
            \hline 
            43021 & 130,2 & \multicolumn{1}{c|}{43073} & 1388,0 & 42964 & 163,2 & \multicolumn{1}{c|}{42984} & \multicolumn{1}{c|}{1127,6}\tabularnewline
            \hline 
            42982 & 137,7 & \multicolumn{1}{c|}{43059} & 1492,8 & 42931 & 167,6 & \multicolumn{1}{c|}{42970} & \multicolumn{1}{c|}{1206,0}\tabularnewline
            \hline 
            42967 & 144,7 & \multicolumn{1}{c|}{43047} & 1603,8 & 42912 & 191,8 & \multicolumn{1}{c|}{42958} & \multicolumn{1}{c|}{1280,5}\tabularnewline
            \hline 
            42917 & 162,4 & \multicolumn{1}{c|}{43041} & 1709,6 & 42903 & 204,3 & \multicolumn{1}{c|}{42946} & \multicolumn{1}{c|}{1361,2}\tabularnewline
            \hline 
            42911 & 169,2 & \multicolumn{1}{c|}{43036} & 1800,0 & 42800 & 225,2 & \multicolumn{1}{c|}{42934} & \multicolumn{1}{c|}{1441,5}\tabularnewline
            \hline 
            42880 & 201,9 &  &  & 42794 & 246,4 & \multicolumn{1}{c|}{42924} & \multicolumn{1}{c|}{1529,3}\tabularnewline
            \cline{1-2} \cline{5-8} 
            42876 & 216,7 &  &  & 42725 & 280,9 & \multicolumn{1}{c|}{42914} & \multicolumn{1}{c|}{1625,5}\tabularnewline
            \cline{1-2} \cline{5-8} 
            42867 & 225,3 &  &  & 42675 & 293,1 & \multicolumn{1}{c|}{42906} & \multicolumn{1}{c|}{1706,7}\tabularnewline
            \cline{1-2} \cline{5-8} 
            42836 & 238,3 &  &  & 42643 & 309,7 & \multicolumn{1}{c|}{42896} & \multicolumn{1}{c|}{1786,7}\tabularnewline
            \cline{1-2} \cline{5-8} 
            42827 & 254,1 &  &  & 42628 & 314,1 & \multicolumn{1}{c|}{42896} & \multicolumn{1}{c|}{1800,0}\tabularnewline
            \cline{1-2} \cline{5-8} 
            42816 & 273,4 &  &  & 42600 & 325,2 &  & \tabularnewline
            \cline{1-2} \cline{5-6} 
            42800 & 285,0 &  &  & 42586 & 334,5 &  & \tabularnewline
            \cline{1-2} \cline{5-6} 
            42751 & 292,4 &  &  & 42564 & 347,5 &  & \tabularnewline
            \cline{1-2} \cline{5-6} 
            42731 & 308,8 &  &  & 42554 & 368,6 &  & \tabularnewline
            \cline{1-2} \cline{5-6} 
            42722 & 327,3 &  &  & 42549 & 402,0 &  & \tabularnewline
            \cline{1-2} \cline{5-6} 
            42680 & 349,0 &  &  & 42537 & 434,7 &  & \tabularnewline
            \cline{1-2} \cline{5-6} 
            42632 & 369,4 &  &  & 42518 & 467,6 &  & \tabularnewline
            \cline{1-2} \cline{5-6} 
            42621 & 391,6 &  &  & 42468 & 493,3 &  & \tabularnewline
            \cline{1-2} \cline{5-6} 
            42604 & 460,7 &  &  & 42440 & 513,8 &  & \tabularnewline
            \cline{1-2} \cline{5-6} 
            42536 & 483,7 &  &  & 42421 & 568,8 &  & \tabularnewline
            \cline{1-2} \cline{5-6} 
            42520 & 511,7 &  &  & 42417 & 592,8 &  & \tabularnewline
            \cline{1-2} \cline{5-6} 
            42510 & 526,1 &  &  & 42396 & 631,6 &  & \tabularnewline
            \cline{1-2} \cline{5-6} 
            42457 & 546,4 &  &  & 42379 & 664,0 &  & \tabularnewline
            \cline{1-2} \cline{5-6} 
            42446 & 570,0 &  &  & 42319 & 673,6 &  & \tabularnewline
            \cline{1-2} \cline{5-6} 
            42437 & 606,7 &  &  & 42296 & 681,4 &  & \tabularnewline
            \cline{1-2} \cline{5-6} 
            42392 & 644,8 &  &  & 42254 & 706,4 &  & \tabularnewline
            \cline{1-2} \cline{5-6} 
            42389 & 658,9 &  &  & 42253 & 737,5 &  & \tabularnewline
            \cline{1-2} \cline{5-6} 
            42367 & 689,8 &  &  & 42230 & 755,7 &  & \tabularnewline
            \cline{1-2} \cline{5-6} 
            42365 & 792,7 &  &  & 42214 & 806,3 &  & \tabularnewline
            \cline{1-2} \cline{5-6} 
            42363 & 842,6 &  &  & 42163 & 825,9 &  & \tabularnewline
            \cline{1-2} \cline{5-6} 
            42318 & 913,5 &  &  & 42155 & 859,4 &  & \tabularnewline
            \cline{1-2} \cline{5-6} 
            42296 & 943,1 &  &  & 42145 & 904,2 &  & \tabularnewline
            \cline{1-2} \cline{5-6} 
            42290 & 994,1 &  &  & 42117 & 949,0 &  & \tabularnewline
            \cline{1-2} \cline{5-6} 
            42282 & 1033,4 &  &  & 42103 & 970,1 &  & \tabularnewline
            \cline{1-2} \cline{5-6} 
            42273 & 1067,0 &  &  & 42100 & 1148,5 &  & \tabularnewline
            \cline{1-2} \cline{5-6} 
            42270 & 1099,0 &  &  & 42094 & 1239,4 &  & \tabularnewline
            \cline{1-2} \cline{5-6} 
            42267 & 1142,1 &  &  & 42086 & 1415,7 &  & \tabularnewline
            \cline{1-2} \cline{5-6} 
            42265 & 1205,3 &  &  & 42086 & 1800,0 &  & \tabularnewline
            \cline{1-2} \cline{5-6} 
            42260 & 1250,9 &  & \multicolumn{1}{c}{} & \multicolumn{1}{c}{} & \multicolumn{1}{c}{} &  & \tabularnewline
            \cline{1-2} 
            42258 & 1282,3 &  & \multicolumn{1}{c}{} & \multicolumn{1}{c}{} & \multicolumn{1}{c}{} &  & \tabularnewline
            \cline{1-2} 
            42251 & 1318,3 &  & \multicolumn{1}{c}{} & \multicolumn{1}{c}{} & \multicolumn{1}{c}{} &  & \tabularnewline
            \cline{1-2} 
            42237 & 1358,8 &  & \multicolumn{1}{c}{} & \multicolumn{1}{c}{} & \multicolumn{1}{c}{} &  & \tabularnewline
            \cline{1-2} 
            42211 & 1392,6 &  & \multicolumn{1}{c}{} & \multicolumn{1}{c}{} & \multicolumn{1}{c}{} &  & \tabularnewline
            \cline{1-2} 
            42206 & 1480,7 &  & \multicolumn{1}{c}{} & \multicolumn{1}{c}{} & \multicolumn{1}{c}{} &  & \tabularnewline
            \cline{1-2} 
            42205 & 1557,6 &  & \multicolumn{1}{c}{} & \multicolumn{1}{c}{} & \multicolumn{1}{c}{} &  & \tabularnewline
            \cline{1-2} 
            42189 & 1635,3 &  & \multicolumn{1}{c}{} & \multicolumn{1}{c}{} & \multicolumn{1}{c}{} &  & \tabularnewline
            \cline{1-2} 
            42177 & 1726,0 &  & \multicolumn{1}{c}{} & \multicolumn{1}{c}{} & \multicolumn{1}{c}{} &  & \tabularnewline
            \cline{1-2} 
            42177 & 1800,0 &  & \multicolumn{1}{c}{} & \multicolumn{1}{c}{} & \multicolumn{1}{c}{} &  & \tabularnewline
            \cline{1-2} 
        \end{tabular}
    }
    \captionof{table}{Tabella risultati instanze con numero di nodi compreso tra \textbf{$200$} e \textbf{$299$} $+$ algoritmi esatti}
}
\vspace*{\fill}

\vspace*{\fill}
{
    \centering
    \centerline{\begin{tabular}{|c|c|c|c|c|c|c|c|}
            \hline 
            \multicolumn{8}{|c|}{rat783 - HF \& LB - PT1 - (costo ottimo 8806)}\tabularnewline
            \hline 
            \hline 
            \multicolumn{2}{|c|}{Thread3 HF} & \multicolumn{2}{c|}{Thread3 LB} & \multicolumn{2}{c|}{Thread4 HF} & \multicolumn{2}{c|}{Thread4 LB}\tabularnewline
            \hline 
            Costo & Tempo (s) & Costo & Tempo (s) & Costo & Tempo (s) & Costo & Tempo (s)\tabularnewline
            \hline 
            9284 & 0,0 & 9284 & 0,0 & 9264 & 0,0 & \multicolumn{1}{c|}{9264} & \multicolumn{1}{c|}{0,0}\tabularnewline
            \hline 
            9283 & 4,2 & 9260 & 21,9 & 9260 & 1,4 & \multicolumn{1}{c|}{9237} & \multicolumn{1}{c|}{14,2}\tabularnewline
            \hline 
            9276 & 8,2 & 9238 & 54,4 & 9252 & 3,4 & \multicolumn{1}{c|}{9222} & \multicolumn{1}{c|}{51,0}\tabularnewline
            \hline 
            9269 & 11,4 & 9220 & 81,6 & 9227 & 6,4 & \multicolumn{1}{c|}{9210} & \multicolumn{1}{c|}{97,4}\tabularnewline
            \hline 
            9247 & 15,8 & 9205 & 113,9 & 9222 & 8,2 & \multicolumn{1}{c|}{9200} & \multicolumn{1}{c|}{153,8}\tabularnewline
            \hline 
            9242 & 18,5 & 9193 & 165,2 & 9217 & 11,5 & \multicolumn{1}{c|}{9190} & \multicolumn{1}{c|}{198,7}\tabularnewline
            \hline 
            9237 & 27,2 & 9181 & 201,7 & 9216 & 12,9 & \multicolumn{1}{c|}{9181} & \multicolumn{1}{c|}{259,5}\tabularnewline
            \hline 
            9236 & 30,8 & 9171 & 252,6 & 9206 & 15,4 & \multicolumn{1}{c|}{9172} & \multicolumn{1}{c|}{327,2}\tabularnewline
            \hline 
            9216 & 36,5 & 9161 & 307,7 & 9203 & 20,9 & \multicolumn{1}{c|}{9166} & \multicolumn{1}{c|}{385,8}\tabularnewline
            \hline 
            9201 & 41,7 & 9152 & 349,7 & 9200 & 22,2 & \multicolumn{1}{c|}{9161} & \multicolumn{1}{c|}{457,8}\tabularnewline
            \hline 
            9189 & 46,7 & 9143 & 400,5 & 9193 & 26,7 & \multicolumn{1}{c|}{9156} & \multicolumn{1}{c|}{533,5}\tabularnewline
            \hline 
            9184 & 50,3 & 9134 & 449,3 & 9182 & 37,2 & \multicolumn{1}{c|}{9151} & \multicolumn{1}{c|}{601,9}\tabularnewline
            \hline 
            9183 & 52,4 & 9126 & 500,5 & 9181 & 40,1 & \multicolumn{1}{c|}{9147} & \multicolumn{1}{c|}{669,8}\tabularnewline
            \hline 
            9180 & 55,6 & 9119 & 545,5 & 9177 & 47,9 & \multicolumn{1}{c|}{9143} & \multicolumn{1}{c|}{761,1}\tabularnewline
            \hline 
            9164 & 66,0 & 9112 & 598,4 & 9156 & 52,9 & \multicolumn{1}{c|}{9137} & \multicolumn{1}{c|}{827,5}\tabularnewline
            \hline 
            9156 & 68,7 & 9105 & 669,4 & 9145 & 56,9 & \multicolumn{1}{c|}{9134} & \multicolumn{1}{c|}{909,3}\tabularnewline
            \hline 
            9148 & 74,0 & 9099 & 740,4 & 9139 & 66,0 & \multicolumn{1}{c|}{9129} & \multicolumn{1}{c|}{963,8}\tabularnewline
            \hline 
            9146 & 83,9 & 9093 & 780,8 & 9138 & 69,2 & \multicolumn{1}{c|}{9126} & \multicolumn{1}{c|}{1033,5}\tabularnewline
            \hline 
            9141 & 93,0 & 9088 & 836,1 & 9137 & 71,9 & \multicolumn{1}{c|}{9123} & \multicolumn{1}{c|}{1088,8}\tabularnewline
            \hline 
            9138 & 96,9 & 9083 & 899,9 & 9125 & 77,6 & \multicolumn{1}{c|}{9120} & \multicolumn{1}{c|}{1160,8}\tabularnewline
            \hline 
            9129 & 99,7 & 9078 & 971,1 & 9120 & 81,4 & \multicolumn{1}{c|}{9107} & \multicolumn{1}{c|}{1204,5}\tabularnewline
            \hline 
            9128 & 105,1 & 9073 & 1032,8 & 9112 & 89,7 & \multicolumn{1}{c|}{9104} & \multicolumn{1}{c|}{1273,9}\tabularnewline
            \hline 
            9111 & 115,5 & 9069 & 1087,4 & 9108 & 94,8 & \multicolumn{1}{c|}{9101} & \multicolumn{1}{c|}{1357,3}\tabularnewline
            \hline 
            9110 & 117,1 & 9065 & 1145,5 & 9101 & 101,2 & \multicolumn{1}{c|}{9098} & \multicolumn{1}{c|}{1454,6}\tabularnewline
            \hline 
            9098 & 120,1 & 9061 & 1215,7 & 9099 & 107,5 & \multicolumn{1}{c|}{9095} & \multicolumn{1}{c|}{1559,2}\tabularnewline
            \hline 
            9092 & 122,8 & 9057 & 1265,1 & 9096 & 149,3 & \multicolumn{1}{c|}{9088} & \multicolumn{1}{c|}{1608,6}\tabularnewline
            \hline 
            9090 & 130,3 & 9054 & 1344,3 & 9092 & 165,9 & \multicolumn{1}{c|}{9085} & \multicolumn{1}{c|}{1693,6}\tabularnewline
            \hline 
            9089 & 133,2 & 9051 & 1407,2 & 9075 & 200,4 & \multicolumn{1}{c|}{9083} & \multicolumn{1}{c|}{1793,5}\tabularnewline
            \hline 
            9076 & 142,3 & 9048 & 1480,7 & 9071 & 220,5 & \multicolumn{1}{c|}{9083} & \multicolumn{1}{c|}{1800,0}\tabularnewline
            \hline 
            9053 & 147,5 & 9045 & 1545,1 & 9067 & 230,5 &  & \tabularnewline
            \cline{1-6} 
            9045 & 151,8 & 9042 & 1620,0 & 9064 & 260,6 &  & \tabularnewline
            \cline{1-6} 
            9040 & 156,8 & 9040 & 1691,9 & 9063 & 274,7 &  & \tabularnewline
            \cline{1-6} 
            9028 & 159,6 & 9038 & 1792,9 & 9056 & 303,9 &  & \tabularnewline
            \cline{1-6} 
            9025 & 164,0 & 9038 & 1800,0 & 9016 & 315,9 &  & \tabularnewline
            \cline{1-6} 
            9019 & 174,9 & \multicolumn{1}{c}{} &  & 9015 & 326,2 &  & \tabularnewline
            \cline{1-2} \cline{5-6} 
            9011 & 205,4 & \multicolumn{1}{c}{} &  & 9013 & 360,2 &  & \tabularnewline
            \cline{1-2} \cline{5-6} 
            9006 & 219,6 & \multicolumn{1}{c}{} &  & 9004 & 366,7 &  & \tabularnewline
            \cline{1-2} \cline{5-6} 
            8995 & 237,2 & \multicolumn{1}{c}{} &  & 9002 & 379,8 &  & \tabularnewline
            \cline{1-2} \cline{5-6} 
            8993 & 258,5 & \multicolumn{1}{c}{} &  & 9000 & 399,7 &  & \tabularnewline
            \cline{1-2} \cline{5-6} 
            8986 & 280,6 & \multicolumn{1}{c}{} &  & 8998 & 425,7 &  & \tabularnewline
            \cline{1-2} \cline{5-6} 
            8985 & 285,6 & \multicolumn{1}{c}{} &  & 8994 & 439,0 &  & \tabularnewline
            \cline{1-2} \cline{5-6} 
            8976 & 302,8 & \multicolumn{1}{c}{} &  & 8993 & 470,4 &  & \tabularnewline
            \cline{1-2} \cline{5-6} 
            8972 & 312,5 & \multicolumn{1}{c}{} &  & 8988 & 485,4 &  & \tabularnewline
            \cline{1-2} \cline{5-6} 
            8963 & 354,1 & \multicolumn{1}{c}{} &  & 8983 & 490,6 &  & \tabularnewline
            \cline{1-2} \cline{5-6} 
            8956 & 382,8 & \multicolumn{1}{c}{} &  & 8978 & 502,7 &  & \tabularnewline
            \cline{1-2} \cline{5-6} 
            8955 & 396,1 & \multicolumn{1}{c}{} &  & 8973 & 526,2 &  & \tabularnewline
            \cline{1-2} \cline{5-6} 
            8954 & 415,1 & \multicolumn{1}{c}{} &  & 8959 & 536,6 &  & \tabularnewline
            \cline{1-2} \cline{5-6} 
            8948 & 437,0 & \multicolumn{1}{c}{} &  & 8947 & 546,8 &  & \tabularnewline
            \cline{1-2} \cline{5-6} 
            8947 & 466,5 & \multicolumn{1}{c}{} &  & 8942 & 569,1 &  & \tabularnewline
            \cline{1-2} \cline{5-6} 
            8945 & 510,0 & \multicolumn{1}{c}{} &  & 8935 & 574,5 &  & \tabularnewline
            \cline{1-2} \cline{5-6} 
            8941 & 537,0 & \multicolumn{1}{c}{} &  & 8926 & 579,6 &  & \tabularnewline
            \cline{1-2} \cline{5-6} 
            8940 & 611,8 & \multicolumn{1}{c}{} &  & 8924 & 592,4 &  & \tabularnewline
            \cline{1-2} \cline{5-6} 
            8938 & 677,9 & \multicolumn{1}{c}{} &  & 8923 & 609,2 &  & \tabularnewline
            \cline{1-2} \cline{5-6} 
            8937 & 711,2 & \multicolumn{1}{c}{} &  & 8915 & 685,5 &  & \tabularnewline
            \cline{1-2} \cline{5-6} 
            8925 & 735,6 & \multicolumn{1}{c}{} &  & 8912 & 727,5 &  & \tabularnewline
            \cline{1-2} \cline{5-6} 
            8920 & 758,4 & \multicolumn{1}{c}{} &  & 8910 & 824,0 &  & \tabularnewline
            \cline{1-2} \cline{5-6} 
            8919 & 809,7 & \multicolumn{1}{c}{} &  & 8910 & 1800,0 &  & \tabularnewline
            \cline{1-2} \cline{5-6} 
            8911 & 828,6 & \multicolumn{1}{c}{} & \multicolumn{1}{c}{} & \multicolumn{1}{c}{} & \multicolumn{1}{c}{} &  & \tabularnewline
            \cline{1-2} 
            8908 & 868,1 & \multicolumn{1}{c}{} & \multicolumn{1}{c}{} & \multicolumn{1}{c}{} & \multicolumn{1}{c}{} &  & \tabularnewline
            \cline{1-2} 
            8904 & 989,8 & \multicolumn{1}{c}{} & \multicolumn{1}{c}{} & \multicolumn{1}{c}{} & \multicolumn{1}{c}{} &  & \tabularnewline
            \cline{1-2} 
            8900 & 1241,1 & \multicolumn{1}{c}{} & \multicolumn{1}{c}{} & \multicolumn{1}{c}{} & \multicolumn{1}{c}{} &  & \tabularnewline
            \cline{1-2} 
            8897 & 1288,3 & \multicolumn{1}{c}{} & \multicolumn{1}{c}{} & \multicolumn{1}{c}{} & \multicolumn{1}{c}{} &  & \tabularnewline
            \cline{1-2} 
            8897 & 1800,0 & \multicolumn{1}{c}{} & \multicolumn{1}{c}{} & \multicolumn{1}{c}{} & \multicolumn{1}{c}{} &  & \tabularnewline
            \cline{1-2} 
        \end{tabular}
    }
    \captionof{table}{Tabella risultati instanze con numero di nodi compreso tra \textbf{$200$} e \textbf{$299$} $+$ algoritmi esatti}
}
\vspace*{\fill}\vspace*{\fill}
{
    \centering
    \centerline{\begin{tabular}{|c|c|c|c|c|c|c|c|}
            \hline 
            \multicolumn{8}{|c|}{rat783 - HF \& LB - PT2 - (costo ottimo 8806)}\tabularnewline
            \hline 
            \hline 
            \multicolumn{2}{|c|}{Thread1 HF} & \multicolumn{2}{c|}{Thread1 LB} & \multicolumn{2}{c|}{Thread2 HF} & \multicolumn{2}{c|}{Thread2 LB}\tabularnewline
            \hline 
            Costo & Tempo (s) & Costo & Tempo (s) & Costo & Tempo (s) & Costo & Tempo (s)\tabularnewline
            \hline 
            9265 & 0,0 & \multicolumn{1}{c|}{9265} & 0,0 & 9242 & 0,0 & \multicolumn{1}{c|}{9242} & \multicolumn{1}{c|}{0,0}\tabularnewline
            \hline 
            9254 & 5,0 & \multicolumn{1}{c|}{9250} & 41,7 & 9228 & 8,6 & \multicolumn{1}{c|}{9227} & \multicolumn{1}{c|}{47,4}\tabularnewline
            \hline 
            9248 & 23,2 & \multicolumn{1}{c|}{9238} & 85,8 & 9221 & 12,0 & \multicolumn{1}{c|}{9216} & \multicolumn{1}{c|}{94,5}\tabularnewline
            \hline 
            9231 & 27,4 & \multicolumn{1}{c|}{9226} & 144,9 & 9218 & 15,3 & \multicolumn{1}{c|}{9206} & \multicolumn{1}{c|}{174,4}\tabularnewline
            \hline 
            9228 & 32,1 & \multicolumn{1}{c|}{9215} & 200,6 & 9203 & 20,4 & \multicolumn{1}{c|}{9199} & \multicolumn{1}{c|}{232,2}\tabularnewline
            \hline 
            9227 & 35,4 & \multicolumn{1}{c|}{9199} & 243,0 & 9202 & 26,3 & \multicolumn{1}{c|}{9192} & \multicolumn{1}{c|}{295,9}\tabularnewline
            \hline 
            9226 & 40,3 & \multicolumn{1}{c|}{9189} & 321,3 & 9198 & 31,7 & \multicolumn{1}{c|}{9184} & \multicolumn{1}{c|}{347,9}\tabularnewline
            \hline 
            9222 & 47,4 & \multicolumn{1}{c|}{9180} & 389,8 & 9197 & 34,3 & \multicolumn{1}{c|}{9178} & \multicolumn{1}{c|}{418,2}\tabularnewline
            \hline 
            9220 & 50,2 & \multicolumn{1}{c|}{9171} & 458,1 & 9184 & 47,0 & \multicolumn{1}{c|}{9172} & \multicolumn{1}{c|}{478,6}\tabularnewline
            \hline 
            9214 & 55,0 & \multicolumn{1}{c|}{9163} & 517,6 & 9160 & 51,6 & \multicolumn{1}{c|}{9166} & \multicolumn{1}{c|}{557,8}\tabularnewline
            \hline 
            9203 & 65,2 & \multicolumn{1}{c|}{9155} & 591,7 & 9159 & 59,4 & \multicolumn{1}{c|}{9160} & \multicolumn{1}{c|}{620,4}\tabularnewline
            \hline 
            9198 & 68,9 & \multicolumn{1}{c|}{9147} & 652,4 & 9153 & 68,9 & \multicolumn{1}{c|}{9154} & \multicolumn{1}{c|}{691,0}\tabularnewline
            \hline 
            9196 & 75,4 & \multicolumn{1}{c|}{9140} & 718,9 & 9152 & 75,5 & \multicolumn{1}{c|}{9149} & \multicolumn{1}{c|}{774,1}\tabularnewline
            \hline 
            9187 & 86,9 & \multicolumn{1}{c|}{9133} & 782,1 & 9149 & 78,4 & \multicolumn{1}{c|}{9145} & \multicolumn{1}{c|}{888,5}\tabularnewline
            \hline 
            9185 & 91,3 & \multicolumn{1}{c|}{9127} & 855,2 & 9142 & 81,2 & \multicolumn{1}{c|}{9141} & \multicolumn{1}{c|}{972,9}\tabularnewline
            \hline 
            9182 & 100,3 & \multicolumn{1}{c|}{9122} & 931,2 & 9138 & 82,7 & \multicolumn{1}{c|}{9137} & \multicolumn{1}{c|}{1062,1}\tabularnewline
            \hline 
            9162 & 111,8 & \multicolumn{1}{c|}{9117} & 1002,5 & 9134 & 85,7 & \multicolumn{1}{c|}{9133} & \multicolumn{1}{c|}{1169,8}\tabularnewline
            \hline 
            9158 & 115,9 & \multicolumn{1}{c|}{9108} & 1062,3 & 9127 & 89,4 & \multicolumn{1}{c|}{9130} & \multicolumn{1}{c|}{1240,0}\tabularnewline
            \hline 
            9148 & 118,7 & \multicolumn{1}{c|}{9104} & 1131,1 & 9115 & 105,9 & \multicolumn{1}{c|}{9127} & \multicolumn{1}{c|}{1342,9}\tabularnewline
            \hline 
            9128 & 128,9 & \multicolumn{1}{c|}{9100} & 1253,4 & 9111 & 109,4 & \multicolumn{1}{c|}{9122} & \multicolumn{1}{c|}{1438,8}\tabularnewline
            \hline 
            9122 & 133,3 & \multicolumn{1}{c|}{9097} & 1350,6 & 9100 & 112,0 & \multicolumn{1}{c|}{9119} & \multicolumn{1}{c|}{1525,5}\tabularnewline
            \hline 
            9113 & 137,8 & \multicolumn{1}{c|}{9094} & 1433,2 & 9095 & 114,4 & \multicolumn{1}{c|}{9117} & \multicolumn{1}{c|}{1633,4}\tabularnewline
            \hline 
            9112 & 139,9 & \multicolumn{1}{c|}{9091} & 1509,5 & 9094 & 116,3 & \multicolumn{1}{c|}{9115} & \multicolumn{1}{c|}{1736,7}\tabularnewline
            \hline 
            9108 & 147,7 & \multicolumn{1}{c|}{9088} & 1593,0 & 9089 & 121,1 & \multicolumn{1}{c|}{9114} & \multicolumn{1}{c|}{1800,0}\tabularnewline
            \hline 
            9106 & 153,3 & \multicolumn{1}{c|}{9084} & 1711,6 & 9081 & 128,0 &  & \tabularnewline
            \cline{1-6} 
            9099 & 159,3 & \multicolumn{1}{c|}{9081} & 1800,0 & 9079 & 135,6 &  & \tabularnewline
            \cline{1-6} 
            9090 & 175,0 &  &  & 9068 & 151,6 &  & \tabularnewline
            \cline{1-2} \cline{5-6} 
            9089 & 183,0 &  &  & 9065 & 159,8 &  & \tabularnewline
            \cline{1-2} \cline{5-6} 
            9087 & 188,9 &  &  & 9063 & 169,2 &  & \tabularnewline
            \cline{1-2} \cline{5-6} 
            9082 & 203,4 &  &  & 9058 & 178,4 &  & \tabularnewline
            \cline{1-2} \cline{5-6} 
            9080 & 222,9 &  &  & 9057 & 193,6 &  & \tabularnewline
            \cline{1-2} \cline{5-6} 
            9075 & 245,8 &  &  & 9055 & 199,0 &  & \tabularnewline
            \cline{1-2} \cline{5-6} 
            9064 & 266,4 &  &  & 9054 & 203,6 &  & \tabularnewline
            \cline{1-2} \cline{5-6} 
            9041 & 273,6 &  &  & 9047 & 207,4 &  & \tabularnewline
            \cline{1-2} \cline{5-6} 
            9026 & 278,7 &  &  & 9043 & 232,0 &  & \tabularnewline
            \cline{1-2} \cline{5-6} 
            9023 & 296,3 &  &  & 9037 & 244,1 &  & \tabularnewline
            \cline{1-2} \cline{5-6} 
            9017 & 314,8 &  &  & 9036 & 259,8 &  & \tabularnewline
            \cline{1-2} \cline{5-6} 
            9008 & 320,3 &  &  & 9032 & 276,7 &  & \tabularnewline
            \cline{1-2} \cline{5-6} 
            9002 & 328,4 &  &  & 9025 & 287,9 &  & \tabularnewline
            \cline{1-2} \cline{5-6} 
            8992 & 343,6 &  &  & 9019 & 305,7 &  & \tabularnewline
            \cline{1-2} \cline{5-6} 
            8991 & 359,8 &  &  & 9013 & 328,0 &  & \tabularnewline
            \cline{1-2} \cline{5-6} 
            8988 & 369,6 &  &  & 9010 & 354,9 &  & \tabularnewline
            \cline{1-2} \cline{5-6} 
            8986 & 419,5 &  &  & 9006 & 370,6 &  & \tabularnewline
            \cline{1-2} \cline{5-6} 
            8982 & 423,6 &  &  & 9002 & 389,6 &  & \tabularnewline
            \cline{1-2} \cline{5-6} 
            8981 & 444,1 &  &  & 9001 & 414,5 &  & \tabularnewline
            \cline{1-2} \cline{5-6} 
            8975 & 452,1 &  &  & 9000 & 421,9 &  & \tabularnewline
            \cline{1-2} \cline{5-6} 
            8970 & 462,9 &  &  & 8993 & 442,5 &  & \tabularnewline
            \cline{1-2} \cline{5-6} 
            8969 & 484,7 &  &  & 8988 & 468,3 &  & \tabularnewline
            \cline{1-2} \cline{5-6} 
            8967 & 503,3 &  &  & 8985 & 475,3 &  & \tabularnewline
            \cline{1-2} \cline{5-6} 
            8960 & 524,9 &  &  & 8979 & 489,4 &  & \tabularnewline
            \cline{1-2} \cline{5-6} 
            8958 & 545,2 &  &  & 8977 & 508,4 &  & \tabularnewline
            \cline{1-2} \cline{5-6} 
            8952 & 559,9 &  &  & 8976 & 578,5 &  & \tabularnewline
            \cline{1-2} \cline{5-6} 
            8946 & 584,5 &  &  & 8974 & 646,4 &  & \tabularnewline
            \cline{1-2} \cline{5-6} 
            8945 & 600,0 &  &  & 8973 & 684,7 &  & \tabularnewline
            \cline{1-2} \cline{5-6} 
            8940 & 611,1 &  &  & 8966 & 712,6 &  & \tabularnewline
            \cline{1-2} \cline{5-6} 
            8939 & 623,6 &  &  & 8965 & 744,3 &  & \tabularnewline
            \cline{1-2} \cline{5-6} 
            8931 & 638,7 &  &  & 8960 & 792,0 &  & \tabularnewline
            \cline{1-2} \cline{5-6} 
            8930 & 646,4 &  &  & 8949 & 830,6 &  & \tabularnewline
            \cline{1-2} \cline{5-6} 
            8927 & 655,4 &  &  & 8948 & 838,3 &  & \tabularnewline
            \cline{1-2} \cline{5-6} 
            8917 & 663,0 &  &  & 8945 & 853,2 &  & \tabularnewline
            \cline{1-2} \cline{5-6} 
            8915 & 667,7 &  &  & 8944 & 906,5 &  & \tabularnewline
            \cline{1-2} \cline{5-6} 
            8912 & 719,1 &  &  & 8939 & 982,1 &  & \tabularnewline
            \cline{1-2} \cline{5-6} 
            8911 & 762,0 &  &  & 8937 & 1090,4 &  & \tabularnewline
            \cline{1-2} \cline{5-6} 
            8910 & 789,3 &  &  & 8927 & 1139,3 &  & \tabularnewline
            \cline{1-2} \cline{5-6} 
            8908 & 831,9 &  &  & 8925 & 1175,4 &  & \tabularnewline
            \cline{1-2} \cline{5-6} 
            8902 & 850,3 &  &  & 8923 & 1231,3 &  & \tabularnewline
            \cline{1-2} \cline{5-6} 
            8901 & 965,9 &  &  & 8917 & 1561,2 &  & \tabularnewline
            \cline{1-2} \cline{5-6} 
            8898 & 1008,1 &  &  & 8913 & 1593,6 &  & \tabularnewline
            \cline{1-2} \cline{5-6} 
            8895 & 1214,3 &  &  & 8910 & 1664,7 &  & \tabularnewline
            \cline{1-2} \cline{5-6} 
            8894 & 1355,8 &  &  & 8895 & 1724,4 &  & \tabularnewline
            \cline{1-2} \cline{5-6} 
            8886 & 1389,7 &  &  & 8894 & 1776,9 &  & \tabularnewline
            \cline{1-2} \cline{5-6} 
            8884 & 1418,1 &  &  & 8894 & 1800,0 &  & \tabularnewline
            \cline{1-2} \cline{5-6} 
            8883 & 1749,7 &  & \multicolumn{1}{c}{} & \multicolumn{1}{c}{} & \multicolumn{1}{c}{} &  & \tabularnewline
            \cline{1-2} 
            8883 & 1800,0 &  & \multicolumn{1}{c}{} & \multicolumn{1}{c}{} & \multicolumn{1}{c}{} &  & \tabularnewline
            \cline{1-2} 
        \end{tabular}
    }
    \captionof{table}{Tabella risultati instanze con numero di nodi compreso tra \textbf{$200$} e \textbf{$299$} $+$ algoritmi esatti}
}
\vspace*{\fill}

\vspace*{\fill}
{
    \centering
    \centerline{\begin{tabular}{|c|c|c|c|c|c|c|c|}
            \hline 
            \multicolumn{8}{|c|}{dsj1000 - HF \& LB - PT1 - (costo ottimo 18659688)}\tabularnewline
            \hline 
            \hline 
            \multicolumn{2}{|c|}{Thread3 HF} & \multicolumn{2}{c|}{Thread3 LB} & \multicolumn{2}{c|}{Thread4 HF} & \multicolumn{2}{c|}{Thread4 LB}\tabularnewline
            \hline 
            Costo & Tempo (s) & Costo & Tempo (s) & Costo & Tempo (s) & Costo & Tempo (s)\tabularnewline
            \hline 
            19828957 & 0,0 & \multicolumn{1}{c|}{19828957} & \multicolumn{1}{c|}{0,0} & \multicolumn{1}{c|}{19802820} & \multicolumn{1}{c|}{0,0} & \multicolumn{1}{c|}{19802820} & \multicolumn{1}{c|}{0,0}\tabularnewline
            \hline 
            19827606 & 35,2 & \multicolumn{1}{c|}{19785306} & \multicolumn{1}{c|}{334,9} & \multicolumn{1}{c|}{19802624} & \multicolumn{1}{c|}{12,7} & \multicolumn{1}{c|}{19766778} & \multicolumn{1}{c|}{1800,0}\tabularnewline
            \hline 
            19800821 & 42,4 & \multicolumn{1}{c|}{19757645} & \multicolumn{1}{c|}{574,7} & \multicolumn{1}{c|}{19797273} & \multicolumn{1}{c|}{21,1} &  & \tabularnewline
            \cline{1-6} 
            19800407 & 46,1 & \multicolumn{1}{c|}{19730261} & \multicolumn{1}{c|}{861,5} & \multicolumn{1}{c|}{19786996} & \multicolumn{1}{c|}{250,5} &  & \tabularnewline
            \cline{1-6} 
            19796390 & 55,4 & \multicolumn{1}{c|}{19705557} & \multicolumn{1}{c|}{1148,6} & \multicolumn{1}{c|}{19764462} & \multicolumn{1}{c|}{264,9} &  & \tabularnewline
            \cline{1-6} 
            19790196 & 61,5 & \multicolumn{1}{c|}{19686452} & \multicolumn{1}{c|}{1642,8} & \multicolumn{1}{c|}{19762089} & \multicolumn{1}{c|}{278,0} &  & \tabularnewline
            \cline{1-6} 
            19760026 & 80,4 & \multicolumn{1}{c|}{19686452} & \multicolumn{1}{c|}{1800,0} & \multicolumn{1}{c|}{19762089} & \multicolumn{1}{c|}{1800,0} &  & \tabularnewline
            \cline{1-6} 
            19745769 & 84,3 &  &  &  &  &  & \tabularnewline
            \cline{1-2} 
            19744871 & 106,9 &  &  &  &  &  & \tabularnewline
            \cline{1-2} 
            19717210 & 129,6 &  &  &  &  &  & \tabularnewline
            \cline{1-2} 
            19711613 & 151,9 &  &  &  &  &  & \tabularnewline
            \cline{1-2} 
            19698196 & 178,0 &  &  &  &  &  & \tabularnewline
            \cline{1-2} 
            19688308 & 192,7 &  &  &  &  &  & \tabularnewline
            \cline{1-2} 
            19679234 & 214,6 &  &  &  &  &  & \tabularnewline
            \cline{1-2} 
            19647964 & 221,3 &  &  &  &  &  & \tabularnewline
            \cline{1-2} 
            19644534 & 228,3 &  &  &  &  &  & \tabularnewline
            \cline{1-2} 
            19642590 & 265,8 &  &  &  &  &  & \tabularnewline
            \cline{1-2} 
            19637587 & 272,1 &  &  &  &  &  & \tabularnewline
            \cline{1-2} 
            19623073 & 275,4 &  &  &  &  &  & \tabularnewline
            \cline{1-2} 
            19608692 & 336,3 &  &  &  &  &  & \tabularnewline
            \cline{1-2} 
            19605498 & 354,5 &  &  &  &  &  & \tabularnewline
            \cline{1-2} 
            19588944 & 380,5 &  &  &  &  &  & \tabularnewline
            \cline{1-2} 
            19586950 & 384,0 &  &  &  &  &  & \tabularnewline
            \cline{1-2} 
            19578103 & 395,5 &  &  &  &  &  & \tabularnewline
            \cline{1-2} 
            19565140 & 429,3 &  &  &  &  &  & \tabularnewline
            \cline{1-2} 
            19545338 & 453,7 &  &  &  &  &  & \tabularnewline
            \cline{1-2} 
            19542075 & 462,2 &  &  &  &  &  & \tabularnewline
            \cline{1-2} 
            19537854 & 465,7 &  &  &  &  &  & \tabularnewline
            \cline{1-2} 
            19529469 & 471,1 &  &  &  &  &  & \tabularnewline
            \cline{1-2} 
            19529469 & 1800,0 &  &  &  &  &  & \tabularnewline
            \cline{1-2} 
        \end{tabular}
    }
    \captionof{table}{Tabella risultati instanze con numero di nodi compreso tra \textbf{$200$} e \textbf{$299$} $+$ algoritmi esatti}
}
\vspace*{\fill}

\vspace*{\fill}
{
    \centering
    \centerline{\begin{tabular}{|c|c|c|c|c|c|c|c|}
            \hline 
            \multicolumn{8}{|c|}{dsj1000 - HF \& LB - PT2 - (costo ottimo 18659688)}\tabularnewline
            \hline 
            \hline 
            \multicolumn{2}{|c|}{Thread3 HF} & \multicolumn{2}{c|}{Thread3 LB} & \multicolumn{2}{c|}{Thread4 HF} & \multicolumn{2}{c|}{Thread4 LB}\tabularnewline
            \hline 
            Costo & Tempo (s) & Costo & Tempo (s) & Costo & Tempo (s) & Costo & Tempo (s)\tabularnewline
            \hline 
            19741229 & 0,0 & 19741229 & 0,0 & 19750658 & 0,0 & \multicolumn{1}{c|}{19750658} & 0,0\tabularnewline
            \hline 
            19734873 & 3,4 & 19686323 & 209,5 & 19707751 & 59,8 & \multicolumn{1}{c|}{19728370} & 1800,0\tabularnewline
            \hline 
            19724110 & 19,2 & 19664867 & 496,3 & 19707751 & 1800,0 &  & \multicolumn{1}{c}{}\tabularnewline
            \cline{1-6} 
            19707498 & 28,4 & 19646947 & 856,9 & \multicolumn{1}{c}{} & \multicolumn{1}{c}{} &  & \multicolumn{1}{c}{}\tabularnewline
            \cline{1-4} 
            19697656 & 67,3 & 19631054 & 1232,5 & \multicolumn{1}{c}{} & \multicolumn{1}{c}{} &  & \multicolumn{1}{c}{}\tabularnewline
            \cline{1-4} 
            19697656 & 1800,0 & 19619359 & 1800,0 & \multicolumn{1}{c}{} & \multicolumn{1}{c}{} &  & \multicolumn{1}{c}{}\tabularnewline
            \cline{1-4} 
        \end{tabular}
    }
    \captionof{table}{Tabella risultati instanze con numero di nodi compreso tra \textbf{$200$} e \textbf{$299$} $+$ algoritmi esatti}
}
\vspace*{\fill}

\FloatBarrier

\subsubsection*{POLISHING OLD}

\FloatBarrier

\begin{table}
    \begin{adjustbox}{center}
        \begin{tabular}{|C|C|}
            \hline 
            \multicolumn{2}{|c|}{lin318 POLISHING GEN\#20 (costo ottimo 42029)}\tabularnewline
            \hline 
            \hline 
            Costo & Tempo(s)\tabularnewline
            \hline 
            53491 & 0,0\tabularnewline
            \hline 
            43973 & 139,5\tabularnewline
            \hline 
            43850 & 218,8\tabularnewline
            \hline 
            43108 & 266,1\tabularnewline
            \hline 
            42747 & 280,5\tabularnewline
            \hline 
            42575 & 312,0\tabularnewline
            \hline 
            42446 & 317,2\tabularnewline
            \hline 
            42366 & 320,7\tabularnewline
            \hline 
            42311 & 322,7\tabularnewline
            \hline 
            42246 & 508,5\tabularnewline
            \hline 
            42216 & 716,3\tabularnewline
            \hline 
            42199 & 893,1\tabularnewline
            \hline 
            42149 & 923,9\tabularnewline
            \hline 
            42136 & 1089,7\tabularnewline
            \hline 
            42125 & 1166,8\tabularnewline
            \hline 
            42040 & 1183,8\tabularnewline
            \hline 
            42029 & 1228,0\tabularnewline
            \hline 
        \end{tabular}
    \end{adjustbox}
    \caption{Tabella risultati instanze con numero di nodi inferiore a \textbf{$200$} $+$ algoritmi esatti}
\end{table}

\begin{table}
    \begin{adjustbox}{center}
        \begin{tabular}{|C|C|}
            \hline 
            \multicolumn{2}{|c|}{pr439 POLISHING GEN\#20 (costo ottimo 107217)}\tabularnewline
            \hline 
            \hline 
            \multicolumn{2}{|c|}{Thread1}\tabularnewline
            \hline 
            Costo & Tempo (s)\tabularnewline
            \hline 
            134016 & 0,0\tabularnewline
            \hline 
            114211 & 111,6\tabularnewline
            \hline 
            112134 & 140,6\tabularnewline
            \hline 
            110751 & 147,6\tabularnewline
            \hline 
            110626 & 153,5\tabularnewline
            \hline 
            109026 & 164,0\tabularnewline
            \hline 
            108898 & 174,6\tabularnewline
            \hline 
            108876 & 183,3\tabularnewline
            \hline 
            108472 & 288,2\tabularnewline
            \hline 
            108294 & 311,4\tabularnewline
            \hline 
            108267 & 376,3\tabularnewline
            \hline 
            108022 & 391,2\tabularnewline
            \hline 
            108002 & 469,7\tabularnewline
            \hline 
            107785 & 565,0\tabularnewline
            \hline 
            107783 & 1036,6\tabularnewline
            \hline 
            107762 & 1147,4\tabularnewline
            \hline 
            107734 & 1166,4\tabularnewline
            \hline 
            107715 & 1172,5\tabularnewline
            \hline 
            107692 & 1175,2\tabularnewline
            \hline 
            107674 & 1244,1\tabularnewline
            \hline 
            107671 & 1432,1\tabularnewline
            \hline 
            107653 & 1732,3\tabularnewline
            \hline 
            107653 & 1800\tabularnewline
            \hline 
        \end{tabular}
    \end{adjustbox}
    \caption{Tabella risultati instanze con numero di nodi inferiore a \textbf{$200$} $+$ algoritmi esatti}
\end{table}

\begin{table}
    \begin{adjustbox}{center}
        \begin{tabular}{|C|C|}
            \hline 
            \multicolumn{2}{|c|}{d493 POLISHING GEN\#20 (costo ottimo 35002)}\tabularnewline
            \hline 
            \hline 
            Costo & Tempo(s)\tabularnewline
            
            35919 & 357,9\tabularnewline
            \hline 
            35864 & 375,5\tabularnewline
            \hline 
            35798 & 418,2\tabularnewline
            \hline 
            35760 & 440,9\tabularnewline
            \hline 
            35592 & 506,4\tabularnewline
            \hline 
            35591 & 538,8\tabularnewline
            \hline 
            35535 & 565,3\tabularnewline
            \hline 
            35503 & 721,3\tabularnewline
            \hline 
            35477 & 766,2\tabularnewline
            \hline 
            35475 & 773,0\tabularnewline
            \hline 
            35471 & 787,8\tabularnewline
            \hline 
            35349 & 1007,6\tabularnewline
            \hline 
            35335 & 1079,2\tabularnewline
            \hline 
            35332 & 1128,4\tabularnewline
            \hline 
            35291 & 1504,2\tabularnewline
            \hline 
            35279 & 1712,8\tabularnewline
            \hline 
            35279 & 1800,0\tabularnewline
            \hline 
        \end{tabular}
    \end{adjustbox}
    \caption{Tabella risultati instanze con numero di nodi inferiore a \textbf{$200$} $+$ algoritmi esatti}
\end{table}

\begin{table}
    \begin{adjustbox}{center}
        \begin{tabular}{|C|C|}
            \hline 
            \multicolumn{2}{|c|}{rat575 POLISHING GEN\#20 (costo ottimo 6773)}\tabularnewline
            \hline 
            \hline 
            Costo & Tempo(s)\tabularnewline
            \hline 
            8445 & 0,0\tabularnewline
            \hline 
            7572 & 12,1\tabularnewline
            \hline 
            7555 & 38,7\tabularnewline
            \hline 
            7499 & 47,2\tabularnewline
            \hline 
            7412 & 58,0\tabularnewline
            \hline 
            7294 & 196,9\tabularnewline
            \hline 
            7270 & 205,1\tabularnewline
            \hline 
            7264 & 236,1\tabularnewline
            \hline 
            7251 & 241,4\tabularnewline
            \hline 
            7247 & 245,0\tabularnewline
            \hline 
            7245 & 251,8\tabularnewline
            \hline 
            7244 & 254,2\tabularnewline
            \hline 
            7238 & 256,6\tabularnewline
            \hline 
            7235 & 264,4\tabularnewline
            \hline 
            7179 & 315,4\tabularnewline
            \hline 
            7157 & 345,1\tabularnewline
            \hline 
            7122 & 393,7\tabularnewline
            \hline 
            7109 & 408,4\tabularnewline
            \hline 
            7092 & 436,2\tabularnewline
            \hline 
            7061 & 478,4\tabularnewline
            \hline 
            7054 & 485,2\tabularnewline
            \hline 
            7044 & 489,1\tabularnewline
            \hline 
            7037 & 499,8\tabularnewline
            \hline 
            7034 & 533,3\tabularnewline
            \hline 
            7033 & 538,4\tabularnewline
            \hline 
            7032 & 542,9\tabularnewline
            \hline 
            7024 & 573,5\tabularnewline
            \hline 
            7021 & 588,2\tabularnewline
            \hline 
            7013 & 662,4\tabularnewline
            \hline 
            7006 & 724,1\tabularnewline
            \hline 
            6991 & 781,5\tabularnewline
            \hline 
            6979 & 808,0\tabularnewline
            \hline 
            6934 & 838,7\tabularnewline
            \hline 
            6931 & 858,7\tabularnewline
            \hline 
            6923 & 938,6\tabularnewline
            \hline 
            6920 & 978,1\tabularnewline
            \hline 
            6913 & 1028,3\tabularnewline
            \hline 
            6906 & 1036,3\tabularnewline
            \hline 
            6898 & 1069,7\tabularnewline
            \hline 
            6895 & 1086,2\tabularnewline
            \hline 
            6894 & 1093,6\tabularnewline
            \hline 
            6893 & 1167,7\tabularnewline
            \hline 
            6879 & 1261,6\tabularnewline
            \hline 
            6875 & 1268,7\tabularnewline
            \hline 
            6872 & 1274,7\tabularnewline
            \hline 
            6868 & 1282,3\tabularnewline
            \hline 
            6866 & 1307,5\tabularnewline
            \hline 
            6858 & 1370,4\tabularnewline
            \hline 
            6849 & 1470,9\tabularnewline
            \hline 
            6846 & 1478,3\tabularnewline
            \hline 
            6845 & 1504,2\tabularnewline
            \hline 
            6833 & 1657,3\tabularnewline
            \hline 
            6832 & 1676,9\tabularnewline
            \hline 
            6830 & 1680,9\tabularnewline
            \hline 
            6829 & 1689,6\tabularnewline
            \hline 
            6827 & 1782,1\tabularnewline
            \hline 
            6827 & 1800,0\tabularnewline
            \hline 
        \end{tabular}
    \end{adjustbox}
    \caption{Tabella risultati instanze con numero di nodi inferiore a \textbf{$200$} $+$ algoritmi esatti}
\end{table}

\begin{table}
    \begin{adjustbox}{center}
        \begin{tabular}{|C|C|}
            \hline 
            \multicolumn{2}{|c|}{d657 POLISHING GEN\#20 (costo ottimo 48912)}\tabularnewline
            \hline 
            \hline 
            Costo & Tempo(s)\tabularnewline
            \hline 
            58910 & 0,1\tabularnewline
            \hline 
            54332 & 23,1\tabularnewline
            \hline 
            54028 & 48,9\tabularnewline
            \hline 
            53136 & 69,7\tabularnewline
            \hline 
            53115 & 91,0\tabularnewline
            \hline 
            52659 & 99,3\tabularnewline
            \hline 
            52409 & 120,3\tabularnewline
            \hline 
            52359 & 136,2\tabularnewline
            \hline 
            51692 & 208,5\tabularnewline
            \hline 
            51666 & 224,4\tabularnewline
            \hline 
            51637 & 293,0\tabularnewline
            \hline 
            51618 & 323,7\tabularnewline
            \hline 
            51478 & 388,9\tabularnewline
            \hline 
            51427 & 412,7\tabularnewline
            \hline 
            51300 & 416,2\tabularnewline
            \hline 
            51143 & 483,8\tabularnewline
            \hline 
            51097 & 505,6\tabularnewline
            \hline 
            51060 & 584,6\tabularnewline
            \hline 
            51006 & 594,5\tabularnewline
            \hline 
            50969 & 600,9\tabularnewline
            \hline 
            50667 & 747,0\tabularnewline
            \hline 
            50533 & 844,0\tabularnewline
            \hline 
            50490 & 913,7\tabularnewline
            \hline 
            50489 & 926,6\tabularnewline
            \hline 
            50472 & 956,8\tabularnewline
            \hline 
            50449 & 1015,2\tabularnewline
            \hline 
            50421 & 1082,7\tabularnewline
            \hline 
            50363 & 1255,6\tabularnewline
            \hline 
            50360 & 1359,2\tabularnewline
            \hline 
            50203 & 1463,2\tabularnewline
            \hline 
            50197 & 1741,5\tabularnewline
            \hline 
            50174 & 1764,8\tabularnewline
            \hline 
            50137 & 1784,3\tabularnewline
            \hline 
            50137 & 1800,0\tabularnewline
            \hline 
        \end{tabular}
    \end{adjustbox}
    \caption{Tabella risultati instanze con numero di nodi inferiore a \textbf{$200$} $+$ algoritmi esatti}
\end{table}

\begin{table}
    \begin{adjustbox}{center}
        \begin{tabular}{|C|C|}
            \hline 
            \multicolumn{2}{|c|}{u724 POLISHING GEN\#20 (costo ottimo 41910)}\tabularnewline
            \hline 
            \hline 
            Costo & Tempo(s)\tabularnewline
            \hline 
            52713 & 0,1\tabularnewline
            \hline 
            46497 & 28,4\tabularnewline
            \hline 
            46449 & 49,4\tabularnewline
            \hline 
            45327 & 143,6\tabularnewline
            \hline 
            45144 & 236,9\tabularnewline
            \hline 
            44385 & 257,1\tabularnewline
            \hline 
            44335 & 282,4\tabularnewline
            \hline 
            44114 & 448,6\tabularnewline
            \hline 
            44102 & 457,5\tabularnewline
            \hline 
            43965 & 560,8\tabularnewline
            \hline 
            43835 & 675,4\tabularnewline
            \hline 
            43802 & 707,7\tabularnewline
            \hline 
            43765 & 723,1\tabularnewline
            \hline 
            43753 & 747,5\tabularnewline
            \hline 
            43734 & 1144,8\tabularnewline
            \hline 
            43707 & 1455,2\tabularnewline
            \hline 
            43695 & 1477,9\tabularnewline
            \hline 
            43642 & 1687,1\tabularnewline
            \hline 
            43642 & 1800,0\tabularnewline
            \hline 
        \end{tabular}
    \end{adjustbox}
    \caption{Tabella risultati instanze con numero di nodi inferiore a \textbf{$200$} $+$ algoritmi esatti}
\end{table}



\begin{table}
    \begin{adjustbox}{center}
        \begin{tabular}{|C|C|}
            \hline 
            \multicolumn{2}{|c|}{rat783 POLISHING GEN\#20 (costo ottimo 8806)}\tabularnewline
            \hline 
            \hline 
            Costo & Tempo(s)\tabularnewline
            \hline 
            10569 & 0,1\tabularnewline
            \hline 
            9845 & 452,9\tabularnewline
            \hline 
            9766 & 487,3\tabularnewline
            \hline 
            9660 & 598,2\tabularnewline
            \hline 
            9592 & 620,3\tabularnewline
            \hline 
            9503 & 645,5\tabularnewline
            \hline 
            9484 & 675,4\tabularnewline
            \hline 
            9362 & 695,1\tabularnewline
            \hline 
            9360 & 720,3\tabularnewline
            \hline 
            9321 & 753,3\tabularnewline
            \hline 
            9312 & 783,8\tabularnewline
            \hline 
            9177 & 927,7\tabularnewline
            \hline 
            9118 & 967,6\tabularnewline
            \hline 
            9077 & 1049,7\tabularnewline
            \hline 
            9058 & 1108,4\tabularnewline
            \hline 
            9052 & 1134,1\tabularnewline
            \hline 
            9037 & 1218,8\tabularnewline
            \hline 
            9034 & 1232,5\tabularnewline
            \hline 
            9023 & 1261,6\tabularnewline
            \hline 
            9018 & 1340,7\tabularnewline
            \hline 
            9011 & 1530,9\tabularnewline
            \hline 
            9010 & 1554,4\tabularnewline
            \hline 
            9003 & 1638,3\tabularnewline
            \hline 
            8999 & 1733,4\tabularnewline
            \hline 
            8980 & 1745,3\tabularnewline
            \hline 
            8980 & 1800,0\tabularnewline
            \hline 
        \end{tabular}
    \end{adjustbox}
    \caption{Tabella risultati instanze con numero di nodi inferiore a \textbf{$200$} $+$ algoritmi esatti}
\end{table}

\begin{table}
    \begin{adjustbox}{center}
        \begin{tabular}{|C|C|}
            \hline 
            \multicolumn{2}{|c|}{dsj1000 POLISHING GEN\#20 (costo ottimo 18659688)}\tabularnewline
            \hline 
            \hline 
            Costo & Tempo(s)\tabularnewline
            \hline 
            24020026 & 0,2\tabularnewline
            \hline 
            21338705 & 36,2\tabularnewline
            \hline 
            21285117 & 177,8\tabularnewline
            \hline 
            21118513 & 207,0\tabularnewline
            \hline 
            20915692 & 257,8\tabularnewline
            \hline 
            20544991 & 309,2\tabularnewline
            \hline 
            20527792 & 415,3\tabularnewline
            \hline 
            20391806 & 454,1\tabularnewline
            \hline 
            19949773 & 1965,0\tabularnewline
            \hline 
            19949773 & 1800,0\tabularnewline
            \hline 
        \end{tabular}
    \end{adjustbox}
    \caption{Tabella risultati instanze con numero di nodi inferiore a \textbf{$200$} $+$ algoritmi esatti}
\end{table}

\FloatBarrier

\subsubsection*{POLISHING NEW}

\FloatBarrier

\begin{table}
    \begin{adjustbox}{center}
        \begin{tabular}{|C|C|}
            \hline 
            \multicolumn{2}{|c|}{lin318 POLISHING GEN\#20 (costo ottimo 42029)}\tabularnewline
            \hline 
            \hline 
            Costo & Tempo(s)\tabularnewline
            \hline 
            54981 & 0,1\tabularnewline
            \hline 
            51105 & 6,2\tabularnewline
            \hline 
            49247 & 13,5\tabularnewline
            \hline 
            47680 & 20,1\tabularnewline
            \hline 
            46976 & 26,5\tabularnewline
            \hline 
            46911 & 32,4\tabularnewline
            \hline 
            46388 & 37,5\tabularnewline
            \hline 
            46111 & 42,1\tabularnewline
            \hline 
            45970 & 46,7\tabularnewline
            \hline 
            45650 & 51,7\tabularnewline
            \hline 
            45049 & 57,4\tabularnewline
            \hline 
            44750 & 63,3\tabularnewline
            \hline 
            44551 & 69,0\tabularnewline
            \hline 
            44090 & 79,1\tabularnewline
            \hline 
            43998 & 83,4\tabularnewline
            \hline 
            43927 & 87,8\tabularnewline
            \hline 
            43835 & 91,9\tabularnewline
            \hline 
            43596 & 96,3\tabularnewline
            \hline 
            43143 & 109,2\tabularnewline
            \hline 
            43089 & 117,4\tabularnewline
            \hline 
            42921 & 137,7\tabularnewline
            \hline 
            42878 & 181,9\tabularnewline
            \hline 
            42855 & 225,8\tabularnewline
            \hline 
            42832 & 271,1\tabularnewline
            \hline 
            42820 & 275,0\tabularnewline
            \hline 
            42797 & 286,0\tabularnewline
            \hline 
            42778 & 293,5\tabularnewline
            \hline 
            42769 & 335,2\tabularnewline
            \hline 
            42752 & 380,8\tabularnewline
            \hline 
            42711 & 729,0\tabularnewline
            \hline 
            42692 & 777,4\tabularnewline
            \hline 
            42667 & 874,4\tabularnewline
            \hline 
            42642 & 958,9\tabularnewline
            \hline 
            42629 & 1024,6\tabularnewline
            \hline 
            42605 & 1272,4\tabularnewline
            \hline 
            42575 & 1604,5\tabularnewline
            \hline 
            42550 & 1652,1\tabularnewline
            \hline 
            42550 & 1800\tabularnewline
            \hline 
        \end{tabular}
    \end{adjustbox}
    \caption{Tabella risultati instanze con numero di nodi inferiore a \textbf{$200$} $+$ algoritmi esatti}
\end{table}

\begin{table}
    \begin{adjustbox}{center}
        \begin{tabular}{|C|C|}
            \hline 
            \multicolumn{2}{|c|}{pr439 POLISHING GEN\#20 (costo ottimo 107217)}\tabularnewline
            \hline 
            \hline 
            \multicolumn{2}{|c|}{Thread1}\tabularnewline
            \hline 
            Costo & Tempo (s)\tabularnewline
            \hline 
            134289 & 0,1\tabularnewline
            \hline 
            124339 & 9,8\tabularnewline
            \hline 
            122260 & 18,1\tabularnewline
            \hline 
            120612 & 33,9\tabularnewline
            \hline 
            118650 & 41,1\tabularnewline
            \hline 
            118050 & 47,9\tabularnewline
            \hline 
            116941 & 55,1\tabularnewline
            \hline 
            116030 & 62,0\tabularnewline
            \hline 
            114880 & 75,5\tabularnewline
            \hline 
            113888 & 88,1\tabularnewline
            \hline 
            113629 & 101,2\tabularnewline
            \hline 
            112983 & 107,3\tabularnewline
            \hline 
            112872 & 113,4\tabularnewline
            \hline 
            112828 & 119,9\tabularnewline
            \hline 
            112785 & 137,8\tabularnewline
            \hline 
            112782 & 161,5\tabularnewline
            \hline 
            112733 & 184,9\tabularnewline
            \hline 
            112461 & 249,7\tabularnewline
            \hline 
            112458 & 256,0\tabularnewline
            \hline 
            112409 & 262,2\tabularnewline
            \hline 
            112401 & 268,8\tabularnewline
            \hline 
            112260 & 286,8\tabularnewline
            \hline 
            112173 & 299,0\tabularnewline
            \hline 
            112165 & 317,2\tabularnewline
            \hline 
            112129 & 380,4\tabularnewline
            \hline 
            111704 & 440,0\tabularnewline
            \hline 
            111668 & 450,7\tabularnewline
            \hline 
            111640 & 472,3\tabularnewline
            \hline 
            111631 & 550,8\tabularnewline
            \hline 
            111484 & 663,4\tabularnewline
            \hline 
            111456 & 723,5\tabularnewline
            \hline 
            111309 & 729,4\tabularnewline
            \hline 
            111289 & 762,8\tabularnewline
            \hline 
            111252 & 882,4\tabularnewline
            \hline 
            111180 & 949,2\tabularnewline
            \hline 
            111040 & 1073,3\tabularnewline
            \hline 
            110852 & 1263,8\tabularnewline
            \hline 
            110781 & 1342,3\tabularnewline
            \hline 
            110376 & 1537,1\tabularnewline
            \hline 
            110376 & 1800\tabularnewline
            \hline 
        \end{tabular}
    \end{adjustbox}
    \caption{Tabella risultati instanze con numero di nodi inferiore a \textbf{$200$} $+$ algoritmi esatti}
\end{table}

\begin{table}
    \begin{adjustbox}{center}
        \begin{tabular}{|C|C|}
            \hline 
            \multicolumn{2}{|c|}{d493 POLISHING GEN\#20 (costo ottimo 35002)}\tabularnewline
            \hline 
            \hline 
            Costo & Tempo(s)\tabularnewline
            \hline 
            41715 & 0,1\tabularnewline
            \hline 
            39768 & 20,8\tabularnewline
            \hline 
            38964 & 40,2\tabularnewline
            \hline 
            38593 & 53,0\tabularnewline
            \hline 
            38148 & 66,0\tabularnewline
            \hline 
            37575 & 77,2\tabularnewline
            \hline 
            37232 & 87,0\tabularnewline
            \hline 
            37203 & 96,8\tabularnewline
            \hline 
            36875 & 107,5\tabularnewline
            \hline 
            36768 & 116,7\tabularnewline
            \hline 
            36574 & 125,2\tabularnewline
            \hline 
            36469 & 133,6\tabularnewline
            \hline 
            36428 & 142,1\tabularnewline
            \hline 
            36329 & 158,9\tabularnewline
            \hline 
            36277 & 167,4\tabularnewline
            \hline 
            36227 & 175,9\tabularnewline
            \hline 
            36096 & 184,2\tabularnewline
            \hline 
            36064 & 192,4\tabularnewline
            \hline 
            36053 & 215,4\tabularnewline
            \hline 
            36045 & 223,3\tabularnewline
            \hline 
            36042 & 239,1\tabularnewline
            \hline 
            36039 & 245,9\tabularnewline
            \hline 
            36032 & 253,5\tabularnewline
            \hline 
            36006 & 261,0\tabularnewline
            \hline 
            35947 & 268,1\tabularnewline
            \hline 
            35882 & 289,3\tabularnewline
            \hline 
            35864 & 443,1\tabularnewline
            \hline 
            35862 & 450,7\tabularnewline
            \hline 
            35861 & 540,7\tabularnewline
            \hline 
            35775 & 614,4\tabularnewline
            \hline 
            35726 & 974,8\tabularnewline
            \hline 
            35720 & 1443,6\tabularnewline
            \hline 
            35714 & 1464,8\tabularnewline
            \hline 
            35695 & 1544,0\tabularnewline
            \hline 
            35690 & 1565,0\tabularnewline
            \hline 
            35690 & 1800,0\tabularnewline
            \hline 
        \end{tabular}
    \end{adjustbox}
    \caption{Tabella risultati instanze con numero di nodi inferiore a \textbf{$200$} $+$ algoritmi esatti}
\end{table}

\begin{table}
    \begin{adjustbox}{center}
        \begin{tabular}{|C|C|}
            \hline 
            \multicolumn{2}{|c|}{rat575 POLISHING GEN\#20 (costo ottimo 6773)}\tabularnewline
            \hline 
            \hline 
            Costo & Tempo(s)\tabularnewline
            \hline 
            8540 & 0,2\tabularnewline
            \hline 
            7800 & 36,7\tabularnewline
            \hline 
            7767 & 60,9\tabularnewline
            \hline 
            7449 & 79,8\tabularnewline
            \hline 
            7405 & 99,8\tabularnewline
            \hline 
            7369 & 116,3\tabularnewline
            \hline 
            7332 & 129,9\tabularnewline
            \hline 
            7276 & 144,1\tabularnewline
            \hline 
            7245 & 173,7\tabularnewline
            \hline 
            7163 & 187,7\tabularnewline
            \hline 
            7133 & 201,4\tabularnewline
            \hline 
            7128 & 214,2\tabularnewline
            \hline 
            7095 & 228,0\tabularnewline
            \hline 
            7070 & 240,7\tabularnewline
            \hline 
            7014 & 267,5\tabularnewline
            \hline 
            7005 & 280,8\tabularnewline
            \hline 
            7002 & 293,5\tabularnewline
            \hline 
            6948 & 307,2\tabularnewline
            \hline 
            6938 & 320,5\tabularnewline
            \hline 
            6922 & 333,8\tabularnewline
            \hline 
            6917 & 346,6\tabularnewline
            \hline 
            6896 & 359,1\tabularnewline
            \hline 
            6895 & 369,8\tabularnewline
            \hline 
            6887 & 382,3\tabularnewline
            \hline 
            6884 & 394,6\tabularnewline
            \hline 
            6883 & 407,5\tabularnewline
            \hline 
            6879 & 420,1\tabularnewline
            \hline 
            6877 & 432,5\tabularnewline
            \hline 
            6875 & 454,9\tabularnewline
            \hline 
            6873 & 480,8\tabularnewline
            \hline 
            6871 & 612,6\tabularnewline
            \hline 
            6870 & 643,2\tabularnewline
            \hline 
            6867 & 759,4\tabularnewline
            \hline 
            6866 & 881,4\tabularnewline
            \hline 
            6864 & 893,1\tabularnewline
            \hline 
            6860 & 1006,8\tabularnewline
            \hline 
            6855 & 1645,6\tabularnewline
            \hline 
            6855 & 1800,0\tabularnewline
            \hline 
        \end{tabular}
    \end{adjustbox}
    \caption{Tabella risultati instanze con numero di nodi inferiore a \textbf{$200$} $+$ algoritmi esatti}
\end{table}

\begin{table}
    \begin{adjustbox}{center}
        \begin{tabular}{|C|C|}
            \hline 
            \multicolumn{2}{|c|}{d657 POLISHING GEN\#20 (costo ottimo 48912)}\tabularnewline
            \hline 
            \hline 
            Costo & Tempo(s)\tabularnewline
            \hline 
            61347 & 0,2\tabularnewline
            \hline 
            56669 & 21,9\tabularnewline
            \hline 
            55895 & 41,1\tabularnewline
            \hline 
            55054 & 78,1\tabularnewline
            \hline 
            54228 & 95,5\tabularnewline
            \hline 
            52536 & 109,2\tabularnewline
            \hline 
            52506 & 123,2\tabularnewline
            \hline 
            52452 & 149,4\tabularnewline
            \hline 
            52097 & 161,7\tabularnewline
            \hline 
            51585 & 174,7\tabularnewline
            \hline 
            51433 & 186,9\tabularnewline
            \hline 
            51114 & 199,8\tabularnewline
            \hline 
            50643 & 224,1\tabularnewline
            \hline 
            50485 & 235,9\tabularnewline
            \hline 
            50401 & 259,4\tabularnewline
            \hline 
            50367 & 304,2\tabularnewline
            \hline 
            50283 & 326,4\tabularnewline
            \hline 
            50261 & 337,4\tabularnewline
            \hline 
            50252 & 409,5\tabularnewline
            \hline 
            50221 & 429,2\tabularnewline
            \hline 
            50177 & 532,3\tabularnewline
            \hline 
            50176 & 542,7\tabularnewline
            \hline 
            50132 & 589,7\tabularnewline
            \hline 
            50131 & 617,9\tabularnewline
            \hline 
            50090 & 721,2\tabularnewline
            \hline 
            50069 & 759,6\tabularnewline
            \hline 
            50063 & 769,1\tabularnewline
            \hline 
            50021 & 815,2\tabularnewline
            \hline 
            50001 & 919,2\tabularnewline
            \hline 
            49993 & 928,9\tabularnewline
            \hline 
            49973 & 949,5\tabularnewline
            \hline 
            49945 & 967,5\tabularnewline
            \hline 
            49942 & 1004,3\tabularnewline
            \hline 
            49883 & 1106,3\tabularnewline
            \hline 
            49865 & 1135,3\tabularnewline
            \hline 
            49822 & 1144,8\tabularnewline
            \hline 
            49777 & 1257,1\tabularnewline
            \hline 
            49766 & 1367,3\tabularnewline
            \hline 
            49721 & 1377,2\tabularnewline
            \hline 
            49715 & 1480,8\tabularnewline
            \hline 
            49677 & 1679,7\tabularnewline
            \hline 
            49664 & 1784,9\tabularnewline
            \hline 
            49664 & 1800\tabularnewline
            \hline 
        \end{tabular}
    \end{adjustbox}
    \caption{Tabella risultati instanze con numero di nodi inferiore a \textbf{$200$} $+$ algoritmi esatti}
\end{table}

\begin{table}
    \begin{adjustbox}{center}
        \begin{tabular}{|C|C|}
            \hline 
            \multicolumn{2}{|c|}{u724 POLISHING GEN\#20 (costo ottimo 41910)}\tabularnewline
            \hline 
            \hline 
            Costo & Tempo(s)\tabularnewline
            \hline 
            51626 & 0,2\tabularnewline
            \hline 
            49285 & 21,4\tabularnewline
            \hline 
            48591 & 39,7\tabularnewline
            \hline 
            47940 & 56,6\tabularnewline
            \hline 
            46975 & 73,8\tabularnewline
            \hline 
            46558 & 89,4\tabularnewline
            \hline 
            46129 & 105,9\tabularnewline
            \hline 
            45741 & 121,6\tabularnewline
            \hline 
            45662 & 137,3\tabularnewline
            \hline 
            45308 & 153,1\tabularnewline
            \hline 
            45163 & 168,4\tabularnewline
            \hline 
            44871 & 184,2\tabularnewline
            \hline 
            44321 & 198,7\tabularnewline
            \hline 
            44181 & 213,9\tabularnewline
            \hline 
            44006 & 243,1\tabularnewline
            \hline 
            43871 & 273,1\tabularnewline
            \hline 
            43801 & 302,7\tabularnewline
            \hline 
            43767 & 317,5\tabularnewline
            \hline 
            43647 & 331,9\tabularnewline
            \hline 
            43560 & 346,1\tabularnewline
            \hline 
            43450 & 359,8\tabularnewline
            \hline 
            43440 & 387,3\tabularnewline
            \hline 
            43432 & 415,1\tabularnewline
            \hline 
            43408 & 457,0\tabularnewline
            \hline 
            43314 & 600,7\tabularnewline
            \hline 
            43305 & 664,4\tabularnewline
            \hline 
            43271 & 677,6\tabularnewline
            \hline 
            43256 & 718,5\tabularnewline
            \hline 
            43200 & 865,4\tabularnewline
            \hline 
            43170 & 913,9\tabularnewline
            \hline 
            43127 & 1046,7\tabularnewline
            \hline 
            43095 & 1059,2\tabularnewline
            \hline 
            43064 & 1083,9\tabularnewline
            \hline 
            43033 & 1130,7\tabularnewline
            \hline 
            43002 & 1176,0\tabularnewline
            \hline 
            42988 & 1198,6\tabularnewline
            \hline 
            42978 & 1327,2\tabularnewline
            \hline 
            42917 & 1586,8\tabularnewline
            \hline 
            42916 & 1599,0\tabularnewline
            \hline 
            42855 & 1611,4\tabularnewline
            \hline 
            42845 & 1748,9\tabularnewline
            \hline 
            42826 & 1800\tabularnewline
            \hline 
        \end{tabular}
    \end{adjustbox}
    \caption{Tabella risultati instanze con numero di nodi inferiore a \textbf{$200$} $+$ algoritmi esatti}
\end{table}



\begin{table}
    \begin{adjustbox}{center}
        \begin{tabular}{|C|C|}
            \hline 
            \multicolumn{2}{|c|}{rat783 POLISHING GEN\#20 (costo ottimo 8806)}\tabularnewline
            \hline 
            \hline 
            Costo & Tempo(s)\tabularnewline
            \hline 
            11112 & 0,3\tabularnewline
            \hline 
            10506 & 47,1\tabularnewline
            \hline 
            10299 & 81,3\tabularnewline
            \hline 
            10120 & 116,8\tabularnewline
            \hline 
            10021 & 145,6\tabularnewline
            \hline 
            9894 & 174,6\tabularnewline
            \hline 
            9769 & 208,7\tabularnewline
            \hline 
            9544 & 233,3\tabularnewline
            \hline 
            9521 & 261,9\tabularnewline
            \hline 
            9494 & 287,6\tabularnewline
            \hline 
            9400 & 310,7\tabularnewline
            \hline 
            9381 & 334,0\tabularnewline
            \hline 
            9350 & 354,2\tabularnewline
            \hline 
            9295 & 375,8\tabularnewline
            \hline 
            9273 & 399,0\tabularnewline
            \hline 
            9238 & 435,1\tabularnewline
            \hline 
            9220 & 454,7\tabularnewline
            \hline 
            9212 & 473,1\tabularnewline
            \hline 
            9209 & 536,1\tabularnewline
            \hline 
            9204 & 554,2\tabularnewline
            \hline 
            9189 & 572,2\tabularnewline
            \hline 
            9185 & 593,1\tabularnewline
            \hline 
            9178 & 624,8\tabularnewline
            \hline 
            9175 & 666,3\tabularnewline
            \hline 
            9170 & 685,6\tabularnewline
            \hline 
            9168 & 726,6\tabularnewline
            \hline 
            9167 & 923,6\tabularnewline
            \hline 
            9165 & 957,9\tabularnewline
            \hline 
            9160 & 1128,5\tabularnewline
            \hline 
            9157 & 1145,2\tabularnewline
            \hline 
            9152 & 1240,2\tabularnewline
            \hline 
            9150 & 1271,4\tabularnewline
            \hline 
            9147 & 1436,0\tabularnewline
            \hline 
            9146 & 1634,5\tabularnewline
            \hline 
            9146 & 1800,0\tabularnewline
            \hline 
        \end{tabular}
    \end{adjustbox}
    \caption{Tabella risultati instanze con numero di nodi inferiore a \textbf{$200$} $+$ algoritmi esatti}
\end{table}

\begin{table}
    \begin{adjustbox}{center}
        \begin{tabular}{|C|C|}
            \hline 
            \multicolumn{2}{|c|}{dsj1000 POLISHING GEN\#20 (costo ottimo 18659688)}\tabularnewline
            \hline 
            \hline 
            Costo & Tempo(s)\tabularnewline
            \hline 
            24494420 & 0,4\tabularnewline
            \hline 
            22674664 & 76,3\tabularnewline
            \hline 
            21691770 & 158,5\tabularnewline
            \hline 
            21155611 & 256,3\tabularnewline
            \hline 
            20987474 & 325,0\tabularnewline
            \hline 
            20616148 & 356,1\tabularnewline
            \hline 
            20422249 & 388,6\tabularnewline
            \hline 
            20213051 & 418,5\tabularnewline
            \hline 
            20178946 & 472,5\tabularnewline
            \hline 
            20143854 & 496,5\tabularnewline
            \hline 
            20011270 & 523,3\tabularnewline
            \hline 
            20007438 & 549,7\tabularnewline
            \hline 
            19845787 & 577,3\tabularnewline
            \hline 
            19787330 & 601,7\tabularnewline
            \hline 
            19784020 & 627,0\tabularnewline
            \hline 
            19738981 & 674,3\tabularnewline
            \hline 
            19703485 & 721,7\tabularnewline
            \hline 
            19701881 & 746,1\tabularnewline
            \hline 
            19697274 & 865,4\tabularnewline
            \hline 
            19693950 & 907,0\tabularnewline
            \hline 
            19642970 & 1138,3\tabularnewline
            \hline 
            19629331 & 1159,7\tabularnewline
            \hline 
            19626007 & 1180,0\tabularnewline
            \hline 
            19614747 & 1198,8\tabularnewline
            \hline 
            19605271 & 1237,8\tabularnewline
            \hline 
            19600250 & 1297,9\tabularnewline
            \hline 
            19592437 & 1341,1\tabularnewline
            \hline 
            19559338 & 1555,4\tabularnewline
            \hline 
            19553706 & 1595,0\tabularnewline
            \hline 
            19551042 & 1634,7\tabularnewline
            \hline 
            19535350 & 1654,9\tabularnewline
            \hline 
            19525508 & 1674,1\tabularnewline
            \hline 
            19502251 & 1732,7\tabularnewline
            \hline 
            19502251 & 1800,0\tabularnewline
            \hline 
        \end{tabular}
    \end{adjustbox}
    \caption{Tabella risultati instanze con numero di nodi inferiore a \textbf{$200$} $+$ algoritmi esatti}
\end{table}

\end{document}
