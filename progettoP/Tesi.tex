\documentclass[11pt]{article}
\usepackage{fullpage}


\begin{document}

ABSTRACT

Il presente lavoro riguarda la progettazione di un software in grado di risolvere istanze del problema del Commesso Viaggiatore applicando differenti algoritmi risolutori i quali saranno confrontati tra loro in termini di efficienza e bont\`a della soluzione prodotta. Verr\`a fornita una descrizione degli strumenti utilizzati e sar\`a analizzato il codice di programmazione realizzato; non mancheranno paragrafi dedicati ad approfondire concetti teorici senza i quali la comprensione del codice potrebbe risultare meno chiara. 

INTRODUZZIONE 

Questa capitolo introduttivo \`e dedicato alla storia, alle applicazioni e alle correnti sfide riguardanti uno dei pi\`u importanti problemi che la disciplina di Ricerca Operativa si trova ad affrontare, ossia il problema del commesso viaggiatore(Travelling Salesman Problem -TSP). 
Il nome deriva dalla sua pi\`u tipica rappresentazione: data una rete di citt\`a, connesse tramite delle strade, si vuole trovare il percorso di minore distanza che un commesso viaggiatore deve seguire per visitare tutte le citt\`a una ed una sola volta e ritornare alla citt\`a di partenza. Per quanto detto, risulta naturale modellare il TSP come un grafo pesato i cui nodi modellizzano le citt\`a relative al problema in questione mentre i possibili collegamenti tra le localit\`a sono modellati con gli archi del grafo i cui pesi possono rappresentare,per esempio, la distanza esistente fra la coppia di nodi collegati dall’ arco.
Il problema del commesso viaggiatore risulta essere NP-hard: questo significa che non è presente in letteratura un algoritmo che risolva questo problema in tempo polinomiale e se $P$ è diverso da $NP$ come si ritiene non esiste un siffatto algoritmo. Poich\`e esiste sempre una istanza per cui il tempo di risoluzione cresce esponenzialmente non è sempre possibile utilizzare algoritmi esatti per risolvere il TPS. Questo significa che l’utilizzo di algoritmi euristici, in grado di risolvere in modo efficace istanze con un numero elevato di nodi in tempi ragionevoli, risulta fondamentale.



Il problema del commesso viaggiatore riveste un ruolo notevole nell' ambito di problemi di logistica distributiva, detti anche problemi di routing. Questi riguardano l’organizzazione di sistemi di distribuzione di beni e servizi. Esempi di problemi di questo genere sono la movimentazione di pezzi o semilavorati tra reparti di produzione, la raccolta e distribuzione di materiali, la distribuzione di merci da centri di produzione a centri di distribuzione.
Sebbene le applicazioni nel contesto dei trasporti siano le più naturali per il TSP, la semplicit\`a del modello ha portato a molte applicazioni interessanti in altre aree. Un esempio può essere la programmazione di una macchina per eseguire fori in un circuito. In questo caso i fori da forare sono le città e il costo del viaggio è il tempo necessario per spostare la testa del trapano da un foro all'altro. 


Problemi matematici riconducibili al TSP furono trattati nell' Ottocento dal matematico irlandese Sir William Rowan Hamilton e dal matematico Britannico Thomas Penyngton.  Nel 1857, a Dublino, Rowan Hamilton descrisse un gioco, detto Icosian game, a una riunione della British Association for the Advancement of Science. Il gioco consisteva nel trovare un percorso che toccasse tutti i vertici di un icosaedro, passando lungo gli spigoli, ma senza mai percorrere due volte lo stesso spigolo. L'icosaedro ha 12 vertici, 30 spigoli e 20 facce identiche a forma di triangolo equilatero.
Il gioco, venduto alla ditta J. Jacques and Sons per 25 sterline, fu brevettato a Londra nel 1859, ma vendette pochissimo. Questo problema \`e un TSP nel quale gli archi che collegano vertici adiacenti, e quindi corrispondono a spigoli dell'icosaedro, sono consentiti e gli altri no (si può pensare che richiedano moltissimo tempo e quindi vadano sicuramente scartati), per tale ragione si tratta di un caso molto particolare di TSP. La forma generale del TSP fu invece studiata solo negli anni Venti e Trenta del ventesimo secolo dal matematico ed economista Karl Menger. Tuttavia, per molto tempo non si ebbe altra idea che quella di generare e valutare tutte le soluzioni, il che mantenne il problema praticamente insolubile. Il numero totale dei differenti percorsi possibili attraverso le $n$ citt\`a \`e facile da calcolare: data una citt\`a di partenza, ci sono a disposizione $(n - 1)$ scelte per la seconda citt\`a, $(n - 2)$ per la terza e così via. Il totale delle possibili scelte tra le quali cercare il percorso migliore in termini di costo \`e dunque $(n - 1)!$, ma dato che il problema ha simmetria, questo numero va diviso a metà. Insomma, date n citt\`a, ci sono $\frac{(n-1)!}{2}$ percorsi che le collegano.

Solo nel 1954, George Dantzig, Ray Fulkerson e Selmer Johnson proposero un metodo più raffinato per risolvere il TSP  su un campione di $n = 49$ citt\`a: queste rappresentavano le capitali degli Stati Uniti e il costo del percorso era calcolato in base alle distanze stradali. 

Nel 1962, Procter and Gamble bandì un concorso per 33 citt\`a, nel 1977 fu bandito un concorso che collegasse le 120 principali citt\`a della Germania Federale e la vittoria andò a Martin Gr\"otschel oggi Presidente del Konrad-Zuse-Zentrum f\"ur Informarionstechnik Berlin(ZIB) e docente presso la Technische Universit\"at Berlin(TUB).

Nel 1987  Padberg e Rinaldi riuscirono a completare il giro degli Stati Uniti attraverso 532 citt\`a. Nello stesso periodo Groetschel e Holland trovarono il TSP ottimale per il giro del mondo che passava per 666 mete importanti. 
Nel 2001, Applegate, Bixby, Chvátal, and Cook trovarono la soluzione esatta a un problema di 15.112 citt\`a tedesche, usando il metodo cutting plane, originariamente proposto nel 1954 da George Dantzig, Delbert Ray Fulkerson e Selmer Johnson. Il calcolo fu eseguito da una rete di 110 processori della Rice University e della Princeton University. Il tempo di elaborazione totale fu equivalente a 22,6 anni su un singolo processore Alpha a 500 MHz.
Sempre Applegate, Bixby, Chv\a`tal, Cook, e Helsgaun trovarono nel Maggio del 2004 il percorso ottimale di 24,978 citt\`a della Svezia. 
Nel marzo 2005, il TSP riguardante la visita di tutti i 33.810 punti in una scheda di circuito fu risolto usando CONCORDE: fu trovato un percorso di 66.048.945 unit\`a, e provato che non poteva esisterne uno migliore. L'esecuzione richiese approssimativamente 15,7 anni CPU. 
Ai giorni nostri il risolutore Concorde per il problema del commesso viaggiatore è utilizzato per ottenere soluzioni ottime su tutte le 110 istanze della libreria TSPLIB; l' istanza con più nodi in assoluto ha 85,900 citt\`a. 





\end{document}

ewew

\end{document}