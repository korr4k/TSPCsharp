\chapter{Tabu search}
\section{Cenni storici}
La parola \textit{tabu} indica “una proibizione imposta da regole sociali come misura protettiva” oppure come qualcosa “proibito perchè costituisce un rischio”.
Tra i più famosi algoritmi euristici migliorativi troviamo la tecnica del \textbf{Tabu Search} al giorno d'oggi sviluppata in multiple formulazioni a testimonianza della sua efficacia e semplicità concettuale. La più famosa si deve a \textbf{Fred Glover} e risale alla fine degli anni '80, anche se idee simili furono abbozzate in contemporanea da \textbf{P. Hansen}. D'altro canto il concetto fondamentale di usare proibizioni per incoraggiare una diversificazione della ricerca era già apparso in vari lavori precedenti, ad esempio nell'algoritmo di \textbf{Kernighan–Lin} per il partizionamento dei grafi.

Numerosi esperimenti computazionali effettuati, dimostrano che la Tabu Search è ormai una tecnica di ottimizzazione affermata e grazie alla sua flessibilità risulta competitiva con gran parte degli altri algoritmi euristici più famosi.

Insieme al Simulated Annealing e agli algoritmi genetici, Tabu Search è stata indicata dal Committee on the Next Decade of Operations Research 


IN CHE ANNO STA ROBA? 


come tecnica molto promettente per il trattamento futuro di applicazioni pratiche ed inoltre è stata formalmente dimostrata la convergenza asintotica verso la soluzione di questa tecnica.
In seguito, U. Faigle e W. Kern hanno proposto una versione probabilistica della Tabu
Search che converge quasi sicuramente ad un ottimo globale. Tali risultati teorici, sebbene
interessanti, sono piuttosto irrilevanti per le applicazioni pratiche. L'obiettivo della Tabu Search, come di altre tecniche euristiche, è di trovare rapidamente soluzioni buone a
problemi che spesso non permettono di determinare soluzioni ottime in tempi ragionevoli.
Il successo diffuso nelle applicazioni pratiche di ottimizzazione ha stimolato negli ultimi anni una rapida crescita della ricerca con tabù.
Tabu Search si basa sulla premessa che la soluzione dei problemi deve incorporare una memoria adattabile e reagire durante l'esplorazione. L'uso della memoria adattabile contrasta con strutture "senza memoria", come quelle ispirate a fenomeni della fisica e della biologia, con quelle a "memoria rigida" esemplificate dal "branch and bound". 
L'enfasi sull'esplorazione attiva (e quindi finalizzata) nella Tabu Search, sia in condizione deterministica che probabilistica, deriva dal presupposto che una cattiva scelta strategica può dare più informazioni rispetto a una buona scelta casuale.
\newpage

\section{Applicazioni della Tabu Search}
La Ricerca Tabu può essere impiegata in vari campi, tra cui:
\begin{itemize}
\item Instradamento (problema del commesso viaggiatore, instradamento di veicoli, capacità dell'instradamento, tempo di finestra per l'instradamento, instradamento diversificato, instradamento a flotta
mista, viaggio dell'acquirente, programmazione del convoglio);
\item Programmazione (tempo di attraversamento delle celle di produzione, elaborazione di una pianificazione, pianificazione della forza lavoro, programmazione Aula, programmazione di macchine, programmazione di processi produttivi flow-shop o job shop, sequenziamento e dosaggio);
\item Telecomunicazioni (instradamento delle chiamate, imballaggio della larghezza di banda, posizionamento dell'"Hub" della struttura, percorso di assegnazione, struttura della rete per i servizi, pianificazione dello sconto al cliente, la non immunità architetturale, reti sincrone ottiche);
\item Progettazione (progettazione a computer, reti con tolleranza di errore, struttura della rete di trasporto, spazio di progettazione architettonica, schema di coerenza, struttura di reti a canone fisso, problemi di taglio irregolare, pianificazione della disposizione aziendale); 
\item Produzione, Magazzino e Investimenti (produzione flessibile, produzione Just-in-Time, parte di selezione della capacità MRP, pianificazione di un inventario multi-voce, sconto per volume di acquisto, mescolamento degli gli investimenti fissi);
\item Localizzazione e allocazione (posizionamento ed assegnazione delle diverse materie prime, assegnazione quadratica, assegnazione semi-quadratica, assegnazione multilivello generalizzata) ;
\item Logica ed Intelligenza Artificiale (massima soddisfacibilità, logica probabilistica, raggruppamento, modello di riconoscimento o classificazione, integrità dei dati, formazione di reti neurali, struttura di reti neurali); 
\item Ottimizzazione di grafi (il grafo di partizionamento, partizionamento di “clique” o cricca, problemi di cricca massima, pianificazione di massimizzazione grafica, problemi di P-medio, grafica dei colori);
\item Tecnologia (inversione sismica, distribuzione di energia elettrica, progettazione di ingegneria strutturale, volume minimo di ellissoidi, costruzione della stazione spaziale, posizionamento di celle di circuito, ricerca Off-Shore di idrocarburi); 
\item Ottimizzazione Generale Combinatoria (programmazione in zero-uno, ottimizzazione della carica fissa, programmazione non lineare non convessa, strutture tutto o niente, programmazione a due livelli);
\end{itemize}


\section{Tabu Search}
Data una funzione $f$ da ottimizare attraverso la ricerca del suo valore minimo all'interno di un insierme $S$, la tecnica più affermata prevede la visita del "neighbourhood" di una qualsiasi soluzione di partenza secondo il "metodo di discesa" descritto dai seguenti passi:
\begin{enumerate}
\item Scegli una soluzione di partenza $s \in S$ e definisci un intorno di ricerca $N(s)$;
\item Trova il migliore $s' \in N(s)$ (ad es. tale che $f(s')<=f(k)$ $\forall k \in N(s)$);
\item Se $f(s')>=f(s)$ allora stop. Altrimenti imposta $s=s'$ e torna al punto 2.
\end{enumerate}
Questo metodo può fermarsi ad un minimo locale di $f$ ma non ad uno globale PERCHE'??????? AL MASSIMO BISOGNA DIRE CHE ASSICURA, NON CHE PUO'. In generale, $N(s)$ non è definito esplicitamente, bisogna cercare $s'$ esplorando diverse direzioni rispetto ad $s$.

Un primo passo verso la definizione di Tabu Search sta nel ridefinire il metodo appena esposto come segue:
\begin{enumerate}
\item Scegli una soluzione di partenza $s$ in $S$;
\item Genera un sottoinsieme $V^*$ di soluzioni in un intorno $N(s)$;
\item Trova un ottimo $s'$ in $V^*$ (per es. tale che $f(s')<=f(k) \forall$ $k \in V^*$) ed imposta $s=s'$;
\item Se $f(s')>=f(s)$ allora stop. Altrimenti torna al punto 2.
\end{enumerate}
Nel primo metodo introdotto si era posto $V^* = N(s??? o i????)$, questo spesso risulta essere troppo costoso in termini di tempo ed un appropriata scelta di $V^*$ può portare ad un sostanziale miglioramento.\\

Con alcuni miglioramenti a quanto già esposto è possibile avvicinarsi ulteriormente alla definizione vera e propria dell'algoritmo Tabu Search.\\
Indichiamo con:\\ 
$s^*$ := la migliore soluzione trovata;\\
$k$ := contatore di iterazioni;
\begin{enumerate}
\item Scegli una soluzione di partenza $s \in S$. Imposta $s^*=s$ e $k=0$;
\item Poni $k = k+1$ e genera un sottoinsieme $V^*$ di soluzioni in $N(s,k)$;
\item Scegli un $s'$ ottimo in $V^*$ (rispetto ad $f$ o ad una funzione modificata $\widetilde{f}$) ed imposta $s=j$ COSA E' j??????;
\item Se $f(s')<f(s^*)$ allora imposta $s^*=s'$;
\item Se è incontrata una condizione di arresto, allora stop. Altrimenti torna al punto 2.
\end{enumerate}

Si può osservare che in questa ultima formulazione viene incluso il classico metodo di discesa.\\
Ora siamo pronti per descrivere la Tabu Search, definiamo:\\
$s$ soluzione;\\
$m$ numero delle mosse;\\
$tr(s,m) \in Tr (r=1,...,t)$ come insieme di condizioni tabù;\\
$ar(s,m) \in Ar(s,m) (r=1,...,a)$ come condizioni di aspirazione (rilassamento dalla condizione di tabù);\\
$\widetilde{f} = f$ + Intensificazione + Diversificazione;

\begin{enumerate}
\item Scegli una soluzione di partenza $s \in S$. Imposta $s^*=s$ e $k=0$;
\item Poni $k=k+1$ e genera un sottoinsieme $V^*$ di soluzioni in $N(s,k)$ tale che nessuna delle condizioni tabù $tr(s,m) \in Tr$ con $(r=1,...,t)$ sia violata a meno di essere evasa con le condizioni di aspirazione $ar(s,m) \in Ar(i,m)$ con $(r=1,...,a)$;
\item Scegli un ottimo $s'=s+m \in V^*$ (rispetto ad f od alla funzione $\widetilde{f}$) ed imposta $s=s'$;
\item Se $f(s') < f(s^*)$ allora imposta $s^*=s'$;
\item Aggiorna le condizioni di tabù e di aspirazione;
\item Se si incontra una condizione di arresto, allora stop. Altrimenti torna al punto 2.
\end{enumerate}

BISOGNA FARE UN CENNO SU COSA SONO LE CONDIZIONI DI ARRESTO, UN ESEMPIO TIPO. IMMAGINO SI INTENDA TIPO UN GAP DALL'OTTIMO, k HA RAGGIUNTO UN CERTO VALORE OPPURE SI E' SUPERATO UN CERTO TEMPO LIMITE

\section{Caratteristiche}
Traduciamo i passi esposti in una descrizione più dettagliata e discorsiva così da completare il quadro teorica sulla tecnica Tabu Search: data una funzione $f(x)$ da ottimizzare su un insieme $X$, Tabu Search esegue una ricerca locale particolare, procedendo iterativamente, passando da una soluzione ad un'altra, fino a quando il criterio di terminazione scelto viene soddisfatto. Ad ogni $x \in X$ viene associato un intorno di $N(x) \subset X$ così che ogni soluzione $x'$ di $N(x)$ sia raggiungibile da $x$ con un'operazione chiamata "mossa". Tabu Search va oltre la normale ricerca locale modificando $N(x)$, sostituendolo con un altro intorno $N^*(x)$, rendendo la ricerca più efficace. \'E proprio il metodo di definizione di tale nuovo intorno che definisce fondamentalmente il Tabu Search: vengono utilizzate particolari strutture di memoria per memorizzare dinamicamente delle restrizioni da applicare ad ogni iterazione dell'algoritmo.

Esistono molteplici tipi di restrizioni, tra i più noti si trovano tecniche per identificare preventivamente il ripresentarsi di situazioni passate vietando quindi loro di appartenere a $N^*(x)$. Molto banalmente, ad esempio, è possibile definire come regola che soluzioni ottime locali passate non possano nuovamente essere selezionate.

In generale le restizioni definiscono delle operazioni come non attuabili e quindi tabù, da qui il nome dell'algoritmo.

Il termine tabù supera quindi il suo stretto significato etimologico in quanto questa restrizione non è in genere definitiva: gli algoritmi prevedono casi così detti di garazia, per cui un loro utilizzo può permettere l'uscita dalla condizione tabù. Un modo utile per la visualizzazione e l'attuazione del presente processo è quello di immaginare di sostituire le iniziali classificazioni tabù delle soluzioni, cioè quelle escluse dalla scelta di $N*(x)$, secondo la loro dipendenza dagli elementi che compongono lo stato tabù che introduce sanzioni per scoraggiarne in maniera significativa la scelta. Inoltre, le valutazioni tabù includono periodicamente anche incentivi per incoraggiare la scelta di altri tipi di soluzioni, a causa di livelli di aspirazione ed influenze a lungo termine. Va sottolineato che il concetto di un intorno di TS è diverso da quello utilizzato nella ricerca locale, abbracciando i tipi di mosse impiegate nei processi costruttivi e distruttivi (dove le basi per tali spostamenti sono quindi chiamati intorni costruttivi ed intorni distruttivi). Tali usi allargati del concetto di intorno rafforzano una prospettiva fondamentale di Tabu search, che è quella di definire gli intorni in modo dinamico, che può includere o tenere conto simultaneamente di diversi tipi di mosse da meccanismi successivamente identificati.

TUTTO QUESTO TEORICAMENTE PUO' ANCHE ANDARE MA BISOGNA FARE UNA DISTINZIONE CHIARA CHE SONO ALGORITMI MOLTO PIU' COMPLESSI DI QUELLO FATTO DA NOI PERCHE' SENNO' CI SI ASPETTA DI TROVARLE NEL CODICE. NOI SEMPLICEMENTE DOPO CHE IL TABU' RAGGIUNGE UNA CERTA AMPIEZZA LO PULIAMO.

\section{Implementazione Tabu Search}
Viene presentato ora il codice commentato che implementa l'algoritmo Tabu Search per risolvere un \textbf{TSP}, partendo dalla classe \textbf{Tabu}, la quale contiene un insieme di utility per l'esecuzione dell'algoritmo implementato nel metodo \textbf{TabuSearch}.

QUA VA PRIMA CHIARITO BENE. BISOGNA DIRE CHE L'OPERAZIONE TABU E' IMPEDIRE AD UN LATO DI ESSERE RIUTILIZZATO, QUANDO IL TABU ARRIVA AD UNA CERTA SOGLIA THRESHOLD (DI SOLITO SULLA CENTINIA) SI INTRODUCONO LE AZIONI DI GARANZIA COME LE HAI CHIAMATE: SI CAMBIA MODALITA' E AD OGNI NUOVO INSERIMENTO SI PULISCONO LE 10 PIU' VECCHIE OPERAZIONI TABU FINO A CHE NON SI SCENDE SOTTO LE 50 RIMANENTI. A QUEL PUNTO LE GARANZIE VENGONO TOLTE E SI RICAMBIA MODALITA'

\subsection{Classe Tabu}
\begin{lstlisting}
namespace TSPCsharp
{
    public class Tabu
    {
        string mode;
        string originalMode;
        List<string> tabuVector;
        Instance inst;
        int threshold;

        public Tabu(string mode, Instance inst, int threshold)
        {
            this.mode = mode;
            this.originalMode = mode;
            this.inst = inst;
            this.threshold = threshold;
            this.tabuVector = new List<string>();
        }

        //Metodo per controllare se il lato definito dai nodi di indice 'a' e 'b' è tabù
        public bool IsTabu(int a, int b)
        {
            if (tabuVector.Contains(a + ";" + b) ||
                tabuVector.Contains(b + ";" + a))
                return true;
            else
                return false;
        }
		
		//Metodo per inserire una coppia di lati nella lista tabu
        public void AddTabu(int a, int b, int c, int d)
        {
            if (mode == "A" && tabuVector.Count >= (threshold - 1))
            {
                //passo in modalità B
                mode = "B";
            }
            else if (mode == "B")
            {
                if (tabuVector.Count >= 50)
                {
                    //rimuovo i primi 10
                    for (int i = 0; i < 10; i++)
                    {
                        tabuVector.RemoveAt(i);
                    }
                }
                else
                    mode = "A";
            }
            //Controllo se i lati sono già presenti nella lista
            if (!tabuVector.Contains(a + ";" + b))
                tabuVector.Add(a + ";" + b);
            if (!tabuVector.Contains(c + ";" + d))
                tabuVector.Add(c + ";" + d);
        }

        public void Clear()
        {
            mode = originalMode;
            tabuVector.Clear();
        }
		
		//Metodo che ritorna la lunghezza della lista delle operazioni tabu
        public int TabuLenght()
        {
            return tabuVector.Count;
        }
    }
}
\end{lstlisting}

\subsection{Implementazione Tabu Search}

Dopo aver presentato la classe Tabu entriamo ora nei dettagli implementativi dell'algoritmo

DOVE SI SPIEGA CHE VIENE SEMPRE ATTUATO LO SCAMBIO MIGLIORE ANCHE SE PEGGIORATIVO? IN TAL CASO L'OPERAZIONE INVERSA DIVENTA TABU MENTRE IN CASO CONTRARIO E' UN SEMPLICE DUO OPT?
SERVE COLLEGARE CON LA TEORIA SPIEGATA PERCHE' LA SI DICE SOLO CHE SI DEFINISCE UNA MOSSA. BISOGNA SPIEGARE QUALE E' LA MOSSA

\begin{lstlisting}
static void TabuSearch(Instance instance, Process process, Tabu tabu, PathStandard currentSol, PathStandard incumbentPath, Stopwatch clock)
{
    //Variabile utilizzata per la stampa con GNUPlot
    int typeSol;
    //Indice del nodo da cui inizia la ricerca
    int indexStart = 0;
    //Variabile che contiene lo scambio migliore
    string nextBestMove = "";
    //Guadagno migliore possibile
    double bestGain = double.MaxValue;
    //Indici dei nodi coinvolti nel potenziale scambio
    int a, b, c, d;
    //Variabili che memorizzano le distanze tra i nodi
    double distAC, distBD, distTotABCD;
    
    //fino al timelimit
    do
    {
        //Viene eseguita un operazione 2-opt e controllato se l'operazione è tabu
        for (int j = 0; j < instance.NNodes; j++, indexStart = b)
        {
            a = indexStart;
            b = currentSol.path[a];
            c = currentSol.path[b];
            d = currentSol.path[c];
            
            for (int i = 0; i < instance.NNodes - 3; i++, c = d, d = currentSol.path[c])
            {
                //Se l'operazione non è Tabu è possibile eseguire lo scambio
                if (!tabu.IsTabu(a, c) && !tabu.IsTabu(b, d))
                {
                    distAC = Point.Distance(instance.Coord[a], instance.Coord[c], instance.EdgeType);
                    distBD = Point.Distance(instance.Coord[b], instance.Coord[d], instance.EdgeType);
                    
                    distTotABCD = Point.Distance(instance.Coord[a], instance.Coord[b], instance.EdgeType) +
                    Point.Distance(instance.Coord[c], instance.Coord[d], instance.EdgeType);
                    //Se ho un best gain più piccolo faccio lo scambio
                    if ((distAC + distBD) - distTotABCD < bestGain && (distAC + distBD) != distTotABCD)
                    {
                        //Nodi coinvolti nello scambio
                        nextBestMove = a + ";" + b + ";" + c + ";" + d;
                        bestGain = (distAC + distBD) - distTotABCD;
                    }
                }
            }
        }
        
        string[] currentElements = nextBestMove.Split(';');
        a = int.Parse(currentElements[0]);
        b = int.Parse(currentElements[1]);
        c = int.Parse(currentElements[2]);
        d = int.Parse(currentElements[3]);
        
        //Eseguo lo scmabio nel percorso (da b a c) a (da c a d)
        Utility.SwapRoute(c, b, currentSol);
        
        currentSol.path[a] = c;
        currentSol.path[b] = d;
        
        currentSol.cost += bestGain;
        
        //Aggiorno l'incumbent
        if (incumbentPath.cost > currentSol.cost)
        {
            incumbentPath = (PathGenetic)currentSol.Clone();
        }
        
        if (bestGain < 0)
            typeSol = 0;
        else
        {
            //Lo scambio è peggiorativo, di conseguenza viene memorizzato come operazione tabu
            tabu.AddTabu(a, b, c, d);
            typeSol = 1;
        }
        
        Utility.PrintHeuristicSolution(instance, process, currentSol, incumbentPath.cost, typeSol);
        bestGain = double.MaxValue;
    
    } while (clock.ElapsedMilliseconds / 1000.0 < instance.TimeLimit);
}
\end{lstlisting}

\subsection{Metodo TabuSearch}

L'intestazione del metodo risulta essere:

\begin{lstlisting}
static void TabuSearch(Instance instance, Process process, Random rnd, Stopwatch clock, string choice)
\end{lstlisting}

Dove:

\begin{itemize}
    \item \textbf{instance}: oggetto della classe \textit{Instance} contenente tutti i dati che descrivono l'istanza del problema del Commesso Viaggiatore fornita in ingresso dall'utente;
    \item \textbf{process}: 
    \item \textbf{rnd}: istanza della classe \textit{Random} precedentemente inizializzato con un seme random diverso per ogni iterazione del programma;
    \item \textbf{clock}: 
    \item \textbf{choice}:
\end{itemize}

\begin{lstlisting}
static void TabuSearch(Instance instance, Process process, Random rnd, Stopwatch clock, string choice)
{
    //Settaggio per GNUPlot
    int typeSol = 0;
    //listArray[i][y] = z; significa che il y-esimo nodo più vicino a quello di indice i è proprio quello di indice z
    //Questa informazione è necessaria per trovare la prima soluzione attraverso il Nearest Neighbor
    List<int>[] listArray = Utility.BuildSLComplete(instance);
    //Variabili in cui si salva la migliore soluzione trovata
    PathStandard incumbentSol;
    //Variabili in cui si salva la soluzione attuale sulla quale agisce l'algoritmo TABU
    PathStandard solHeuristic = Utility.NearestNeightbor(instance, rnd, listArray);
    //Semplice assegnazione della prima soluzione incumbent
    incumbentSol = (PathStandard)solHeuristic.Clone();
    
    //Stampa della soluzione incumbent iniziale (sia su file che a video attraverso GNUPlot)
    Utility.PrintHeuristicSolution(instance, process, incumbentSol, incumbentSol.cost, typeSol);

    //Viene selezionata la modalità di partenza e impostata la soglia pari a 100
    Tabu tabu = new Tabu("A", instance, 100);
    //Il metodo TS esegue l'effettiva operazione TABU a partire dalla soluzione solHeuristic
    TS(instance, process, tabu, solHeuristic, incumbentSol, clock);
    
    //Stampa della soluzione incumbet finale (sia su file che a video attraverso GNUPlot)
    Utility.PrintHeuristicSolution(instance, process, incumbentSol, incumbentSol.cost, typeSol);
}
\end{lstlisting}


\chapter{VNS}
La Variable Neighborhood Search (VNS) è uno dei più noti algoritmi di ricerca metaeuristici e viene utilizzata per trovare una soluzione ottima spostandosi da un neighborhood all'altro.
Indichiamo con $N_k$ con $k=1,...,k_{max}$ l'insieme finito di neighborhood preselezionati, e con $N_k(x)$ l'iniseme delle soluzioni di x nel k-esimo neighborhood. L'agoritmo base del VNS può essere descritto come segue:\\\\

Inizializzazione: 
\begin{itemize}
\item selezionare l'insieme di neighborhood $N_k$, $K=1,...,k_{max}$ che verranno utilizzati nella ricerca;
\item trovare una soluzione iniziale;
\item selezionare una stopping condition. Nella nostra implementazione come stopping condition utilizziamo un timelimit.
\end{itemize} 

Ripetere i seguenti passi fino a che la stopping condition viene soddisfatta: 
\begin{enumerate}
\item $k \leftarrow 1 $
\item Ripetere i seguenti passaggi fino a che $k=k_{max}$;
\begin{itemize}
\item \textbf{Shaking}: generare un $x'$ random dal k-esimo neighborhood di x $(x' \in N_k(x))$;
\item \textbf{Local search}: eseguire la ricerca di una soluzione ottima locale che indichiamo con $x''$ utilizzando $x'$ come soluzione di partenza.
\item \textbf{Move or not}: se la soluzione trovata è migliore dell'incumbent, $x' \leftarrow x''$ e continuare la ricerca in $N_1 (k \leftarrow 1)$; altrimenti $k \leftarrow k+1$;
\end{itemize}
\end{enumerate}

OK NON SONO MOLTO CONVINTO DI UNA COSA, SEMBRA CHE GLI INTORNI DA VISITARE SIANO DECISI PRIMA DI VISITARLI (NOI LI CREIAMO RUNTIME QUANDO SERVONO). DIREI PIUTTOSTO CHE SI RIPETONO TANTE RICERCHE DI OTTIMO LOCALE APPLICANDO UNA MOSSA (NEL NOSTRO CASO 2-OPT). AL SUO TERMINE SI PASSA AD UN INTORNO DIVERSO DOVE APPLICARE LA MOSSA (DA NOI SI APPLICA FINTO 3-OPT) IL TUTTO CICLICAMENTE FINCHE' LA CONDIZIONE DI STOPPING (TIME LIMIT) SOPRAGGIUNGE

Un'osservazione molto importante riguarda il fatto che la soluzione $x'$ al passo 2 viene generata in modo casuale in modo da eliminare la ciclicità che si verificherebbe se essa venisse scelta in modo deterministico.\\
Dato che l'ottimo locale all'interno di un neighborhood non è necessariamente lo stesso all'interno di un altro, il passaggio da un neighborhood ad un altro può essere fatto che all'interno della fase di local search.

\section{Implementazione VNS}

BISOGNA RIFARSI A QUANTO DETTO SOPRA, DIRE CHE LA MOSSA E' IL 2-OPT, MENTRE DIRE CHE LA FUNZIONE (HO CAMBIATO NOME) ChangeVNS CAMBIA INTORNO(?) E CHE VIENE VISTA NEL DETTAGLIO DOPO

Per risolvere il problema del commesso viaggiatore con l'impiego del VNS come neighborhood abbiamo utilizzato lo spazio delle soluzioni generate attraverso il 2-opt e quelle ottenute attraverso uno scambio 3-opt.

\subsection{Metodo VNS}

L' intestazione del metodo risulta essere:

\begin{lstlisting}
static void VNS(Instance instance, Process process, Random rnd, Stopwatch clock, string choice)
\end{lstlisting}

Dove:

\begin{itemize}
    \item \textbf{instance}: oggetto della classe \textit{Instance} contenente tutti i dati che descrivono l'istanza del problema del Commesso Viaggiatore fornita in ingresso dall'utente;
    \item \textbf{rnd}: istanza della classe \textit{Random} precedentemente inizializzato con un seme random diverso per ogni iterazione del programma;
    \item \textbf{clock}:
    \item \textbf{choice}:
\end{itemize}

\begin{lstlisting}
static void VNS(Instance instance, Process process, Random rnd, Stopwatch clock, string choice)
        {
            int typeSol = 0;
            List<int>[] listArray = Utility.BuildSLComplete(instance);
            PathStandard incumbentSol;
            PathStandard solHeuristic = Utility.NearestNeightbor(instance, rnd, listArray);
            incumbentSol = (PathStandard)solHeuristic.Clone();
            Utility.PrintHeuristicSolution(instance, process, incumbentSol, incumbentSol.cost, typeSol);

            do
            {
            	//Eseguo un'operazione 2-opt 
                TwoOpt(instance, solHeuristic);
				
				//se la soluzione migliora aggiorno l'incumbent
                if (incumbentSol.cost > solHeuristic.cost)
                {
                    incumbentSol = (PathStandard)solHeuristic.Clone();

                    Utility.PrintHeuristicSolution(instance, process, incumbentSol, incumbentSol.cost, typeSol);

                    Console.WriteLine("Incumbed changed");
                }
                //se la soluzione non migliora proseguo
                else
                    Console.WriteLine("Incumbed not changed");
				//chiamo il metodo VNS
                ChangeVNS(instance, solHeuristic, rnd);

            } while (clock.ElapsedMilliseconds / 1000.0 < instance.TimeLimit);

            Console.WriteLine("Best distance found within the timelit is: " + incumbentSol.cost);
        }
\end{lstlisting}

\subsection{Implementazione ChangeVNS}
Come visto nell'introduzione al metodo VNS, esso comporta l'esecuzione di un'operazione di 3-opt la quale cerca comporta la variazione di 3 lati nel circuito, nella speranza di trovare, con le successive operazioni di 2-opt, una soluzione migliore di quella corrente. QUI VANNO USATI I TERMINI USATI NELLA TEORIA. LA 3-OPT DEFINISCE UN NUOVO INTORNO SU CUI APPLICARE LA MOSSA..CREDO..

L' intestazione del metodo risulta essere:

\begin{lstlisting}
static void ChangeVNS(Instance instance, Process process, Random rnd, Stopwatch clock, string choice)
\end{lstlisting}

Dove:

\begin{itemize}
    \item \textbf{instance}: oggetto della classe \textit{Instance} contenente tutti i dati che descrivono l'istanza del problema del Commesso Viaggiatore fornita in ingresso dall'utente;
    \item \textbf{currentSol}: istanza della classe \textit{PathStandard} contenente la soluzione corrente;
    \item \textbf{rnd}: istanza della classe \textit{Random} precedentemente inizializzato con un seme random diverso per ogni iterazione del programma;.
\end{itemize}

\begin{lstlisting}
static void ChangeVNS(Instance instance, PathStandard currentSol, Random rnd)
        {
        	//Indici dei nodi coinvolti nello scambio 3-opt
            int a, b, c, d, e, f;

            a = rnd.Next(currentSol.path.Length);
            b = currentSol.path[a];

            do
            {
                c = rnd.Next(currentSol.path.Length);
                d = currentSol.path[c];
            } while ((a == c && b == d) || a == d || b == c);

            do
            {
                e = rnd.Next(currentSol.path.Length);
                f = currentSol.path[e];
            } while ((e == a && f == b) || e == b || f == a || (e == c && f == d) || e == d || f == c);

            List<int> order = new List<int>();
			
			//Eseguo lo scambio 3-opt
            for (int i = 0, index = 0; i < currentSol.path.Length && order.Count != 4; i++, index = currentSol.path[index])
            {
                if (a == index)
                {
                    order.Add(a);
                    order.Add(b);
                    i++;
                    index = currentSol.path[index];
                }
                else if (c == index)
                {
                    order.Add(c);
                    order.Add(d);
                    i++;
                    index = currentSol.path[index];
                }
                else if (e == index)
                {
                    order.Add(e);
                    order.Add(f);
                    i++;
                    index = currentSol.path[index];
                }
            }

            if (order[0] != a && order[2] != a)
            {
                order.Add(a);
                order.Add(b);
            }
            else if (order[0] != c && order[2] != c)
            {
                order.Add(c);
                order.Add(d);
            }
            else
            {
                order.Add(e);
                order.Add(f);
            }

            Utility.SwapRoute(order[2], order[1], currentSol);

            currentSol.path[order[0]] = order[2];

            Utility.SwapRoute(order[4], order[3], currentSol);

            currentSol.path[order[1]] = order[4];

            currentSol.path[order[3]] = order[5];

            currentSol.cost += Point.Distance(instance.Coord[order[0]], instance.Coord[order[2]], instance.EdgeType) +
                Point.Distance(instance.Coord[order[1]], instance.Coord[order[4]], instance.EdgeType) +
                Point.Distance(instance.Coord[order[3]], instance.Coord[order[5]], instance.EdgeType) -
                Point.Distance(instance.Coord[a], instance.Coord[b], instance.EdgeType) -
                Point.Distance(instance.Coord[c], instance.Coord[d], instance.EdgeType) -
                Point.Distance(instance.Coord[e], instance.Coord[f], instance.EdgeType);

        }
\end{lstlisting}
\end{document}